%! TeX program = lualatex
\documentclass[../main.tex]{subfiles}
\begin{document}
\begin{lesson}{Calculating Volumes by slicing}
  To calculate the area under the curve, we \emph{slice} the region under the curve into tiny rectangles and take the limit. The \emph{idea of slicing} can be adapted to calculate the volume of a solid.

  % bring a loaf of bread.
  \begin{mdframed}[style=withref]
    Let \(S\) be a solid that lies between \(x=a\) and \(x=b\). If \(A(x)\) is the \emph{cross-sectional area} of \(S\) in the plane \(P_{x}\), through \(x\) and perpendicular to the \(x\)-axis, 
    \begin{equation} \label{eq:volume}
      (\text{volume of } S) = \lim_{n \to \infty} \sum_{i=1}^{n} A(x_{i}^{*}) \Delta x = \int_{a}^{b} A(x) \;dx.
    \end{equation}

    The so-called disk method (textbook page 645) and washer method (textbook page 648) are special cases of Equation~\eqref{eq:volume}.
  \end{mdframed}
  \vspace{2in}
  % use the word "measure"

  \begin{example}
    Set up, but do not evaluate, a definite integral representing the volume of sphere centred at the origin with radius $1$. \\
    {\footnotesize Explore \url{https://www.geogebra.org/m/yse5hfrf} (GeoGebra)}
  \end{example}
  \clearpage


  A \emph{solid of revolution} is obtained by rotating a region in a plane about a line in space. \\
  {\footnotesize Explore \url{https://www.geogebra.org/m/th54w7h9} (GeoGebra).}


  The cross-section (perpendicular to the line of rotation) of a solid of revolution is always a \emph{disk}. 
  \vspace{2in}
  % \begin{align*} \label{eq:volume-cross-sectional-area}
  %   (\text{cross-sectional area of a solid of revolution}) &= (\text{area of the outer circle}) - (\text{area of the inner circle}) \\
  %                                 &= \pi (\text{outer radius} - \text{inner radius})^{2}.
  % \end{align*}


  \begin{example}
    Suppose a solid of revolution $S$ is obtained by rotating the finite region enclosed between $y = x^{2}$ and $y = x$ about the $x$-axis. Find the volume of $S$.
    % poll which is the outer curve, and which is the inner curve?

    % we need to measure the cross-sectional area.
    % talk about the meaing of the limits of integration.
    % look for a verticle slice

    \includegraphics[scale=0.8]{../standalones/build/plot_solid_of_revolution_cross_section.pdf}
  \end{example}
  \clearpage

  \begin{example}
    Suppose a solid of revolution $S$ is obtained by rotating the finite region enclosed between $y = x^{2}$ and $y = x$ about the $y$-axis.  Set up, but do not evaluate, an integral representing the volume of $S$.
    % poll which is the outer curve, and which is the inner curve?

    % we need to measure the cross-sectional area.
    % talk about the meaing of the limits of integration.
    % poll for expressions of outer and inner radii.
    % look for a horizontal slice
    \begin{center}
      \includegraphics[scale=0.8]{../standalones/build/plot_solid_of_revolution_cross_section.pdf}
    \end{center}
  \end{example}
  \clearpage

  \begin{example}
    Suppose a solid of revolution $S$ is obtained by rotating the finite region enclosed between $y = x^{2}$ and $y = x$ about the line $x = -1$.  Set up, but do not evaluate, an integral representing the volume of $S$.

    \begin{center}
      \includegraphics{../standalones/build/plot_solid_of_revolution_cross_section_offset.pdf}
    \end{center}
  \end{example}
\end{lesson}
\end{document}
