%! TeX program = lualatex
\documentclass[../main.tex]{subfiles}
\begin{document} \section{The Fundamental Theorem of Calculus, Part 1}
  The Fundamental Theorem of Calculus has two parts.

  \begin{mdframed}[style=withref-compact]
    If \(f(x)\) is continuous over \([a,b]\), then the function \(F(x)\) defined by
    \[
      F(x) = \int_{a}^{x} f(t) \;dt
    \]
    is differentiable on \((a,b)\) and \(F'(x) = f(x)\).

    \textbook{Theorem: 5.4: Fundamental Theorem of Calculus Part 1 on page 552}
  \end{mdframed}

  In a more compact notation, FTC part 1 can be restated as 
  \blanklines{10}

  Using properties of definite integrals, we can differentiate functions defined as integrals where variables appears in both the upper and the lower bound of the integral.
  \begin{example} \label{ex:ftc-one}
    Let \(F(x) = \int_{x^{3}}^{x^{2} + 1} e^{t^{2}} \;dt\). Find \(F'(x)\).

    Which of the following are good start points of our solutions?
    \begin{enumerate}[label=(\alph*)]
      \item Write \(F(x) = \int_{0}^{x^{3}} e^{t^{2}} \;dt + \int_{0}^{x^{2}+1} e^{t^{2}} \;dt \).
      \item Write \(F(x) = \int_{x^{3}}^{0} e^{t^{2}} \;dt + \int_{0}^{x^{2}+1} e^{t^{2}} \;dt \).
      \item Write \(F(x) = \int_{0}^{x^{3}} e^{t^{2}} \;dt + \int_{0}^{1} e^{t^{2}} \;dt + \int_{1}^{x^{2}} e^{t^{2}} \;dt\).
      \item Write \(F(x) = \int_{x^{3}}^{1} e^{t^{2}} \;dt + \int_{1}^{x^{2}+1} e^{t^{2}} \;dt \).
    \end{enumerate}

    {\footnotesize (There is more space on the next page)}
    \blanklines{50}
  \end{example}

\end{document}
