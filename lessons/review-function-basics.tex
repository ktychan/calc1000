%! TeX program = lualatex
\documentclass[../main.tex]{subfiles}
\begin{document} \section{Review of functions}
\begin{example}
  Which of the following is \hlwarn{not} a function?

  \begin{multicols}{2}
    \begin{enumerate}[label=(\alph*)]
      \item \(y = x^{2} - 1\).
      \item \(x^{2} + y^{2} = 1\).
      \item \(x^{x}\) for \(x > 0\).
      \item \(g(x) = \begin{cases} x^{2} + 1 & x > 3\\ \sqrt{x} & x \le 3 \end{cases}\).
    \end{enumerate}
  \end{multicols}
\end{example} 


A \hlmain{function}, often denoted abstractly as \(y = f(x)\) or just \(f(x)\), expresses a certain dependence between two quantities: the \underline{\hspace{1in}} variable \(x\) and the \underline{\hspace{1in}} variable \(y\).

\bigskip
\begin{definition}[function]
  A \hlmain{function} \(f\) consists of a set of inputs, a set of output, and a rule for assigning each input to \underline{\hspace{2in}} output.
  \begin{enumerate}
    \item The \underline{\hspace{1in}} of \(f\) is \underline{\hspace{3in}}.
    \item The \underline{\hspace{1in}} of \(f\) is \underline{\hspace{3in}}.
  \end{enumerate}

  \textbook{Page 8}
\end{definition}
To \hlmain{evaluate} a function \(f(x)\) at \(x = a\) means to \underline{\hspace{3in}}

\begin{example}
  Do these two functions have the same domain?
  \[
    \sqrt{x - 4}\sqrt{x+3}, \hspace{1in} \sqrt{(x-4)(x+3)}.
  \]

  \blanklines{25}
\end{example}

\clearpage

Lines show up a lot in later chapters (e.g., tangent lines) of \thiscourse{} and every science.

\begin{example}
  Describe the given line \(L\) \hlinfo{in words} in two different ways.

  \includestandalone[page=1]{../standalones/plot-line}
\end{example}

The \hlmain{point-slope form} is a recipe for lines and calls for \underline{\hspace{3in}}.

\blanklines{10}

The \hlmain{slope-intercept form} is the other recipe for lines and calls for \underline{\hspace{2in}}.

\blanklines{10}

The \hlmain{standard form} of a line is \(ax + by = c\) where \(a, b, c\) are constants and \(a,b\) cannot both be zero.
%(Reflection) If a line is expressed in its standard form, is \(a\) always its slope?

\blanklines{10}

\clearpage

Let's talk a little bit about study skills. 

\begin{mdframed}[style=simple]
  \begin{enumerate}
    \item Remember \underline{\hspace{4in}}
    \item Ask \underline{\hspace{4.5in}}
    \item Practise \underline{\hspace{4.2in}} 
  \end{enumerate}
\end{mdframed}

\begin{example}
  Let's practise these study skills by playing with the concept of \emph{increasing} functions.

  \faComments{} Which function(s) is (are) increasing on the interval \([1,3]\)?

  \begin{center}
    \includestandalone[page=1]{../standalones/plot-increasing}
    \quad
    \includestandalone[page=2]{../standalones/plot-increasing}
    \quad
    \includestandalone[page=3]{../standalones/plot-increasing}
  \end{center}
  \blanklines{28}
\end{example}

\clearpage

We review some common functions (nouns) and operations (verbs) because functions that describe natural phenomena are often created by ``putting together'' basic functions.

\begin{example}
  For each class of basic functions below, write down one or two examples and its general form.

  \begin{center}
    \begin{tabular}{p{1.25in}p{3.5in}|p{1.5in}}
      & Examples & General Form \\\midrule
      Power \newline functions & & \\[0.5in]\midrule
      Polynomials & & \\[0.5in]\midrule
      Trigonometric \newline functions & & \\[0.5in]\midrule
      Inverse \newline trigonometric \newline functions & & \\[0.5in]\midrule
      Exponential \newline functions & & \\[0.5in]\midrule
      Inverse \newline exponential \newline functions & & \\[0.5in]
    \end{tabular}
  \end{center}
\end{example}

\faComments{} Why go through the trouble of grouping functions together?

\blanklines{20}
\clearpage

Common operations on functions are
\[
  + \hspace{0.75in}
  - \hspace{0.75in}
  \times \hspace{0.75in}
  \div \hspace{0.75in}
  \circ
\]

\blanklines{12}

\faExclamationTriangle{}
Note \(f \cdot g\) and \(f \circ g\) are \underline{\hlwarn{not}} (always) the same function!

\blanklines{5}

\begin{example} \label{ex:review-circ}
  Suppose \(f(x) = \sqrt{x}\) and \(g(x) = \frac{x^{2}-1}{\sqrt{x-1}}\). Write down \(f \circ g\) and \(g \circ f\).

  \blanklines{20}

  \faComments{} What does Example~\ref{ex:review-circ} teach us about function composition?

  \blanklines{3}
\end{example}
\clearpage

Review of some terminologies. 
\begin{itemize}[wide, noitemsep]
  \item A \hlmain{root function} is a power function of the form \(x^{1/n}\) or \(\sqrt[n]{x}\) where \(n\) is a \hlsupp{positive} integer. 

    \blanklines{5}

  \item A \hlmain{rational power} is a power function of the form \(x^{p/q}\) where \(p,q\) are integers but \(q \ne 0\).

    \blanklines{5}

  \item Functions that involve \(+, -, \times, \div\), roots and \hlwarn{rational powers} are called \hlmain{algebraic functions}. 

    \blanklines{5}

  \item If a function is \hlsupp{not} algebraic, then it is called a \hlmain{transcendental function}.

    \blanklines{5}

  \item A \hlmain{zero} of a function \(f(x)\) is a \hlsupp{number \(r\)} (in its domain) so that \(f(r) = 0\).  Graphically, zeros of a function are also called its \hlmain{\underline{\hspace{1in}} intercept} or \hlmain{\underline{\hspace{1in}}}.

    \blanklines{5}
\end{itemize}

Let's take a short break and chat a bit about philosophy. \faComments{} Why learn mathematics when computers seem capable of all sorts of computations? \faComments{} What separates human understanding of mathematics from software computations?

\blanklines{15}
\clearpage

Translation (aka shift) and scaling are two basic transformation on functions.

\hlmain{Translation}: The operation \(f(x - a) + b\) shifts \(f(x)\) horizontally by \(a\) and vertically by \(b\). 

\blanklines{15}

\hlmain{Scaling}: The operation \(B f(A x)\) scales \(f(x)\) horizontally by \(A\) and vertically by \(B\).

\blanklines{15}

We can also perform translation and scaling by relabelling the axes. 

\blanklines{15}

\clearpage
\begin{example}
  Sketch \(f(x) = \arctan(-2x) + \pi/4\) without using any software or calculators.


  \includestandalone[page=1]{../standalones/plot-exercise-sketch}

  \blanklines{10}

  \includestandalone[page=2]{../standalones/plot-exercise-sketch}

\end{example}

Why is sketching ``manually'' useful? Consider evaluating \(\lim_{x \to \infty} \left( \arctan(-2x) + \pi/4 \right)\) (which is a common question). The sketch above tells you the answer immediately!
\clearpage

\begin{exercise}
  Sketch \(f(x) = -2\left(\frac{x}{2} + 3\right)^{2}\) without using any software or calculators.

  The grading standard for sketching functions, aside from multiple choice questions, is that students are \hlwarn{required to show step-by-step sketches}. The solution needs to clearly sketch one starting function, call it \(f(x)\). Each subsequent step shows exactly one of the four transformations (horizontal translation \(g(x - a)\), vertical translation \(g(x) + c\), horizontal scaling \(g(ax)\) and vertical scaling \(Ag(x)\)) of the function in the previous step.

  \blanklines{45}
\end{exercise}

\end{document}
