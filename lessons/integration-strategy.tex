%! TeX program = lualatex
\documentclass[../main.tex]{subfiles}
\begin{document} \section{Problem-solving and integration}
  The general problem-solving strategy for integration problems is very similar to those we talked about in the midterm review.

  \begin{mdframed}[style=simple]
    Express a problem in the language of integration and calculate the desired answer. 
  \end{mdframed}

  Most, not all, integration problems boil down to two parts.
  \begin{enumerate}[wide, label=\textbf{Part~\arabic*}.]
    \item Figure out the \hlmain{exact} expression to evaluate. Sometimes you need an equation as demonstrated in Assignment 4 Question 3. Sometimes, you have to be creative.

      \begin{itemize}[wide]
        \item Find \(F(x)\) given an expression for \(F'\) and \(F(a) = b\)?
          \blanklines{8}

        \item Approximation?  \hlinfo{DRAW THE PICTURE and LABEL everything!}
          \blanklines{8}

        \item Area between curves?  \hlinfo{DRAW THE PICTURE and LABEL everything!}
          \blanklines{8}

        \item Volume?  \hlinfo{DRAW THE PICTURE and LABEL everything!}
          \blanklines{8}

      \end{itemize}

    \item \hlmain{Evaluate} the expression you wrote down. 

      \hlmain{\faStar{} Exam Tip}: Practise performing technical calculations quickly and accurately, so you have time during the exam to figure out a problem-solving strategy.

      \begin{itemize}
        \item We learned three\footnote{If you take Calc~1301 or Calc~1501, then you will learn a bunch more integration techniques.} methods to evaluate indefinite integrals.

          \begin{enumerate}
            \item Guess and check, a.k.a., use the table of integration. 
              \blanklines{5}
            \item Use properties of indefinite integrals to break up a larger integral into simpler integrals. 
            \item Use the substitution rule to \emph{transform} a complicated integral into something we can readily evaluate.
          \end{enumerate}

        \item We can evaluate integrals using three different approaches as demonstrated in Assignment 4 Question 1.

          \begin{enumerate}
            \item The approximation approach. 
              \[
                \int_{a}^{b} f(x) \;dx = \hspace{3in}
              \]
              \blanklines{5}

            \item The geometric approach. 
              \[
                \int_{a}^{b} f(x) \;dx = \hspace{3in}
              \]
              \blanklines{5}

            \item The FTC approach. 
              \[
                \int_{a}^{b} f(x) \;dx = \hspace{3in}
              \]
          \end{enumerate}
      \end{itemize}
  \end{enumerate}

  \clearpage
  A bit more more on the substitutions rule. 

  \begin{example}
    Find two different substitutions to evaluate \(\int e^{6x} \;dx\).
    \blanklines{15}
  \end{example}

  \begin{example}
    Evaluate \(\int \frac{\ln(x)^{2} + 1}{x} \;dx\).
    \blanklines{10}
  \end{example}

  \begin{example}
    Evaluate \(\int \cos^{2}(x) \;dx\).
    \blanklines{15}
  \end{example}

  \clearpage
  An integration shortcut for simple integrals.

  Recall sometimes, but not always, that linear functions are good candidates for substitution in an integration problem. If we do enough problems, the following pattern emerges.
  \begin{mdframed}[style=simple]
    \begin{equation} \label{eq:integration-affine-substitution}
      \int_{a}^{b} f(cx+ d) \;dx = \hspace{2in} \text{ by substituting } u = cx + d.
    \end{equation}
  For indefinite integrals, simply do nothing for the bounds. 
  \end{mdframed}
  
  \begin{example}
    Evaluate the following simple integrals, quickly!

    \begin{enumerate}[wide]
      \item \(\int_{1}^{-1} \sqrt[3]{3x + 1} \;dx = \)
        \blanklines{10}
      \item \(\int \frac{1}{\sqrt{1 - (x/2 + 1)^2}} \;dx = \)
        \blanklines{10}
      \item \(\int \frac{1}{3x-2} \;dx = \)
        \blanklines{10}
    \end{enumerate}
  \end{example}

\end{document}
