%! TeX program = lualatex
\documentclass[../main.tex]{subfiles}
\begin{document} \section{Continuity and non-differentiability}
\begin{mdframed}[style=withref-compact] \label{thm:differentiability-implies-continuity}
  \textbf{Theorem}. If \(f\) is {differentiable} at \(a\), then \(f\) is {continuous} at \(a\).

  \textbook{Theorem 3.1 on page 237}
\end{mdframed}

\faComment{} If we know \(f(x)\) is continuous at \(a\), can we conclude that \(f(x)\) is also differentiable at \(a\)?

\blanklines{5}

\faComment{} What is the significance of Theorem~3.1? How do we use it?

\blanklines{5}

\begin{example}
  Explain why the following functions are not differentiable at \(0\).

  \begin{enumerate}[wide]
    \item \(g_{1}(x) = \frac{x - 1}{x}\).

      \blanklines{2}

    \item \(g_{2}(x) = \begin{cases} 1 &\text{if } x \ge 0 \\ 0 &\text{if } x < 0 \end{cases}\).

      \blanklines{2}

    \item \(g_{3}(x) = |x|\).

      \blanklines{3}

    \item \(g_{4}(x) = \sqrt[3]{x}\). 

      \includestandalone{../standalones/plot-vertical-tangent}
  \end{enumerate}
\end{example}
\end{document}

