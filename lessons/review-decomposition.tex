%! TeX program = lualatex
\documentclass[../main.tex]{subfiles}
\pagestyle{draft}

\begin{document}
\begin{lesson}{Decomposition of functions}
  Algebraic operations \[+ \qquad - \qquad \times \qquad \div \qquad \circ\] are ``verbs'' of the mathematical language. By \emph{legally} combining verbs (operations) with nouns (functions) we can build sophisticated mathematical functions to describe natural phenomena. 

  We review algebraic operations, so we can \hlmain{parse} expressions and \hlmain{apply formulas} correctly.

  % \begin{mdframed}[style=simple-compact]
  %   Algebraic operations must be applied in the following order
  %   \[ \underbrace{( \cdots )}_{\text{first}} \quad\longrightarrow\quad \times, \div \quad\longrightarrow\quad \underbrace{+, -}_{\text{last}}. \]
  % \end{mdframed}

  % \faExclamationTriangle{} Applying operations out of order is a major but avoidable mistake. 

  \begin{example}
    Let's do the silly simple thing and evaluate \((2 + 3) \times 5 + 2^{3}\) using a parse tree.
    \blanklines{15}
  \end{example}

  \begin{example} 
    Draw the parse tree of the function \(\frac{1 - x}{\sin^{2}(x) (1+x)}\) to help us visualize the structure of the function.
    \blanklines{16}
  \end{example}
  \clearpage

  Parse trees help us apply formulas \hlmain{in the correct order} and break up a complicated problem into smaller and, hopefully simpler, problems.

  We will learn the following differentiation formulas.  Treat them like ``symbol soup'' for now. 
  \begin{align*}
    (f+g)' &= f' + g' \\
    (fg)' &= f'g + f \cdot g' \\
    (f/g)' &= \frac{f'g - f \cdot g'}{g^{2}} \\
    (f \circ g)' &= f'(g(x)) g'(x)
  \end{align*}

  \begin{example}
    Apply the above differentiation formulas (without actually evaluating the derivatives) in the correct order to calculate the derivative of \(\frac{1-x}{\sin^{2}(x)}\).  \newline
    {\scriptsize If you are feeling adventurous, try to differentiate the function in the previous example. Admittedly, it's a bit painful.}
  \end{example}
  \blanklines{38}

\end{lesson}
\end{document}
