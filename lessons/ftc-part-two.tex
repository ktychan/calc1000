%! TeX program = lualatex
\documentclass[../main.tex]{subfiles}
\begin{document} \section{The Fundamental Theorem of Calculus, part 2}
The Fundamental Theorem of Calculus also makes a connection between definite and indefinite integrals.

\begin{mdframed}[style=withref-compact]
  If \(f(x)\) is continuous over \([a,b]\) and \(F(x)\) is any antiderivative of \(f(x)\), then 
  \[
    \int_{a}^{b} f(x) \;dx = F(b) - F(a).
  \]

  \textbook{Theorem 5.5: The Fundamental Theorem of Calculus Part 2 on page 555}
\end{mdframed}

FTC part 2 is our \hlmain{workhorse theorem} to evaluate definite integrals. \hlmain{Indefinite integrals and all associated integration formulas} can now be used to evaluate definite integrals.

Sometimes, we use the notation \([F(x)]_{a}^{b}\) to mean \(F(b) - F(a)\). Using this notation, FTC part 2 can be succinctly expressed as:

\blanklines{10}

\begin{example}
  Evaluate \(\int_{0}^{1} x^{2} - \pi \;dx\).

  \blanklines{20}
\end{example}
\clearpage

\begin{example}
  Evaluate \(\int_{-\pi}^{\pi/2} 2\cos(x) \;dx\).

  \blanklines{15}
\end{example}

\begin{example}
  Evaluate \(\int_{1}^{2.3} 3 \sqrt{x} - \frac{1}{x} \;dx\).

  \blanklines{15}
\end{example}

\end{document}
