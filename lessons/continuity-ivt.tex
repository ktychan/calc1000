%! TeX program = lualatex
\documentclass[../main.tex]{subfiles}
\begin{document}
\begin{lesson}{The Intermediate Value Theorem}
  \faComments{} Suppose you go to Montreal for a day trip. If your thermometer at home reads \(5^{\circ}\)C at \(7 \text{ am}\) and \({10}^{\circ}\)C at \(5\text{ pm}\), does the thermometer \underline{have to} read \({8}^{\circ}\)C \emph{sometime} between 7 am and 5 pm?

  \blanklines{4}

  Imagine and draw the graph of a temperature function \(H(t)\) that fits the given scenario. Compare graphs with your neighbours. Use your graph to support your conclusion.

  \begin{figure}[!h]  % [h] for here, [ht] for here top, [hb] for here bottom
    \centering
    \includegraphics[width=\textwidth]{standalones/build/plot_ivt_motivation}
    \label{fig:label}
  \end{figure}

  What \underline{mathematical property} of the temperature function \(H(t)\) \emph{guides} you to think this way?
  \blanklines{5}

  Can you tell \underline{exactly} when (down to seconds) the temperature reached \(8^{\circ}\)C?
  \blanklines{5}

  \blanklines{2}
  \begin{mdframed}[style=withref]
    Let \(f\) be a \underline{\hspace{1in}} function over a closed, bounded interval \([a,b]\). If \(z\) is any real number between \(f(a)\) and \(f(b)\), then \underline{\hspace{1in}} a number \(c\) in \([a,b]\) satisfying \(f(c) = z\).

    \textbook{Theorem 2.11: The Intermediate Value Theorem on page 188}
  \end{mdframed}
  \blanklines{4}

  When is IVT useful? You wish to 
  \blanklines{2} % solve an equation \(f(z) = c\)
  \large\hlwarn{BUT}
  \blanklines{3} % don't care exactly what \(z\) is

  How do we use IVT? We must successfully complete all three tasks: 
  \begin{enumerate}[label=(IVT \arabic*)]
    \item Find, guess, or be given {a \emph{closed and bounded} interval \([a,b]\)}
    \item Verify {\(f\) is continuous on \([a,b]\)}
    \item Verify {\(z\) is between \(f(a)\) and \(f(b)\)}
  \end{enumerate}
  If we cannot complete one or more of these tasks, then IVT \hlwarn{does not apply}.

  \begin{example}
    Show the function \(f(t) = t^{5} - t^{2} + 1\) has a root between \(-1\) and \(1\).
  \end{example}
  \blanklines{13}
\end{lesson}
\end{document}

