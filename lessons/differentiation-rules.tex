%! TeX program = lualatex
\documentclass[../main.tex]{subfiles}
\begin{document}
\begin{lesson}{Differentiation Rules (with respect to \(+,-,\times,\div\))}
  Our (academic or practical) lives will be extremely tedious if we had to calculate derivative using its \(f'(x) = \lim_{h \to 0} {\textstyle\frac{f(x+h) - f(x)}{h}}\) every single time.  The good news is that patterns emerge if we actually try to calculate every derivative this way. 

  Differentiation rules are such patterns and express the interactions between differentiation (as an operation) and algebraic operations \(+, -, \times, \div, \circ\). We focus on \(+, -, \times, \div\) this week and study composition next week.

  \begin{mdframed}[style=simple] \label{thm:diff-rules-1}
    \textbf{Differentiation Rules} \hfill {\footnotesize (part \(1\) of \(4\))}
    \begin{align*}
      \frac{d}{dx} (c)
    &= 0, \text{ for any constant \(c\)}
    && \text{(constant)} \\
    \frac{d}{dx} (x^{n})
    &= n x^{n-1}, \text{ for any number \(n \ne 1\), not just integers} 
    && \text{(power)} \\
    \frac{d}{dx} \sin(x)
    &= \cos(x) \\
    \frac{d}{dx} \cos(x)
    &= -\sin(x) \\
      \frac{d}{dx} e^{x} &= e^{x} \text{ and } \frac{d}{dx}b^{x} = \ln(b)b^{x} \text{ for any \hlwarn{constant} } b > 0 \text{ and } b \ne 1
    \end{align*}
  \end{mdframed}

  Part 1 of the differentiation rules tells us how to find the derivatives of very basic functions. Let's make sure we can apply them correctly.

  \begin{example}
    Evaluate \((x^{3})'\) and \(\frac{d}{dx}\frac{1}{\sqrt[3]{x}^{5}}\).

    \blanklines{10}
  \end{example}

  % \faComment{} Is the calculation \(\textstyle{}\frac{d}{dx} (x + 1)^{3} = 3 (x + 1)^{2}\) correct? Why \hlinfo{and} why not?
  %
  % \blanklines{5}
  %
  % \faExclamationTriangle{} It is \emph{very important} that the power rule is applied to power functions, \emph{not} composite functions that look like power functions. Differentiating composite functions requires the chain rule! 

  \begin{example}
    Can we \emph{directly} apply the power rule to differentiate the following functions?
    \[
      \sqrt{-x^{2} + 1}, \qquad (4x^{2} + x - 1)^{3/2}, \qquad e^{-x}, \qquad x^{x}
    \]

    \blanklines{5}
  \end{example}

  \begin{mdframed}[style=simple]
    \textbf{Differentiation Rules} \hfill {\footnotesize (part \(2\) of \(4\))}

    If \(f\) and \(g\) are differentiable functions, then
    \begin{align*}
      \frac{d}{dx} [f(x) + g(x)] 
    &= \frac{d}{dx}f(x) + \frac{d}{dx}g(x)
    && \text{(sum)} \\[1em]
    \frac{d}{dx} [f(x) - g(x)] 
    &= \frac{d}{dx}f(x) - \frac{d}{dx}g(x)
    && \text{(difference)} \\[1em]
    \frac{d}{dx} [c f(x)] 
    &= c \, \frac{d}{dx} f(x), \text{ for any constant \(c\)}
    && \text{(constant multiple)}
    \end{align*}
  \end{mdframed}
  Let's make sure that we can apply the sum, difference and constant multiple rules correctly. The examples on this page are just about reading and using the formulas correctly. 

  \begin{example}
    Differentiate \(\frac{x^{2}}{2} - \sqrt{x} + e\).

    \blanklines{10}
  \end{example}

  \begin{example}
    Let \(f(x) = -\cos(x) + \sqrt{3} e^{x}\).  Find \(f'(x)\).

    \blanklines{10}
  \end{example}
  \clearpage

  \begin{mdframed}[style=simple]
    \textbf{Differentiation Rules} \hfill {\footnotesize (part \(3\) of \(4\))}

    If \(f\) and \(g\) are differentiable functions, then
    \begin{align*}
      \frac{d}{dx} [f(x)g(x)] 
    &= { \frac{d}{dx} [f(x)] \cdot g(x)+ f(x) \cdot \frac{d}{dx}[g(x)]}
    && \text{(product)} \\
    &= f' \cdot g + f \cdot g' \\[1em]
    \frac{d}{dx} \frac{f(x)}{g(x)}
    &= {\frac{\frac{d}{dx} [f(x)] \cdot g(x) - f(x) \cdot \frac{d}{dx}[g(x)] }{ [g(x)]^{2} }}
    && \text{(quotient)} \\
    &= \frac{f' \cdot g - f \cdot g'}{g^{2}}
    \end{align*}
  \end{mdframed}
  Applying the product and the quotient rule tend to result in somewhat messy write-up which can (and statistically does) lead to preventable mistakes. \hlmain{Keeping your writing well organized} is the pro-tip to prevent mistakes.

  \begin{example}
    Evaluate \(\frac{d}{dx} x e^{x}\).

    \blanklines{8}
  \end{example}

  \begin{example}
    Evaluate \(\frac{d}{dx} \frac{\sqrt[3]{x}}{e^{x}}\).

    \blanklines{20}
  \end{example}
  \clearpage

  Let's differentiate more complicated functions. The functions are more complicated, but the problem-solving principle is the same. 
  \begin{mdframed}[style=simple]
    \color{main}
    \centering
    Keep it simple! 

    Apply one, just one, rule at a time and keep your writing well-organized.
  \end{mdframed}

  \begin{example}
    Suppose \(f(x) = \frac{3 \sqrt[5]{x}}{x^{-1} + 2}\). Simplify \(f'(1)\). 
    \newline
    {\scriptsize Sometimes, we say \emph{simplify} to mean evaluate.}

    \blanklines{20}
  \end{example}


  \begin{example}
    Differentiate \(\frac{3/(\sqrt[3]{x})^{2} - \sin(x)}{x^{2} - 1}\).

    \blanklines{20}
  \end{example}
  \clearpage
\end{lesson}
\end{document}
