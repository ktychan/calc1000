%! TeX program = lualatex
\documentclass[../main.tex]{subfiles}
\begin{document} \section{Sigma Notation}
  The Sigma notation is a compact way to express the sum of a bunch of numbers. It appears frequently in science literature and, for \thecoursesubject~\thecoursenumb, Riemann sums and its approximations. The course could also use Sigma notation with other types of problems to promote basic science literacy skill development.
  \blanklines{5}
  {\huge \[ \sum_{i=m}^{n} a_{i} = \hspace{3in}\]}
  \blanklines{5}

  There are two common tasks involving Sigma notation: Interpreting a summation and using it to express a summation.
  \begin{example}[Reading Sigma notations] \label{ex:sigma-notation-read)}
    Write the following summations in expanded form.
    \begin{align*}
      \sum_{i=3}^{5} (2i^{3} + 5^{i} + 1)
      &= \underline{\hspace{5in}} \\[3ex]
      \sum_{k=10}^{15} x
      &= \underline{\hspace{5in}}.
    \end{align*}
  \end{example}
  \faComment{} Assume \(a\) is a constant. Do \(\sum_{i=1}^{3}f\left(a + \frac{i}{2}\right)\) and \(\sum_{i=0}^{2}f\left(a + \frac{i+1}{2}\right)\) represent different objects? 
    \blanklines{3}
  \clearpage
  When we try to express a sum using the Sigma notation, we need to remember that the index of summation, say \(i\), \hlwarn{must be an integer and increases by exactly \(1\)}. Look for a pattern of change consistent across the summands and express such pattern as a multiple of \(i\).  Lastly, choose the smallest (or the largest) term in a sum to be the first term in the Sigma notation and express such first term using \(i\).
  \blanklines{10}

  \begin{example}
    Express the following sum using the Sigma notation. There are infinitely many correct expressions. Check your answer by asking a neighbour to expand your sum.
    \blanklines{2}
    \[
      A = \frac{9}{2} + 3 + \frac{3}{2} + 6 = \hspace{3in}
    \]
    \blanklines{10}
  \end{example}

  \begin{example}
    Express the following sum using the Sigma notation. 
    \[
      A = \frac{1}{2} \bigg( -\frac{3}{2} \bigg)^{2} + \frac{1}{2} \bigg( -1 \bigg)^{2} + \frac{1}{2} \bigg( -\frac{1}{2} \bigg)^{2} + \frac{1}{2} \bigg( 0 \bigg)^{2} + \frac{1}{2} \bigg( \frac{1}{2} \bigg)^{2} + \frac{1}{2} \bigg( 1 \bigg)^{2}
    \]
    \blanklines{10}
  \end{example}
  \vfill

  \clearpage
  \begin{mdframed}[style=withref-compact]
    \label{page:summation-properties}
    Let \(c\) be a constant. Let \(a_{1}, \dots, a_{n}, b_{1}, \dots, b_{n}\) be two lists of numbers. 
    \begin{align*}
      \sum_{i=1}^{n} {\color{attn} \mathbf{c}} \; a_{i} 
      &= {\color{attn} \mathbf{c}} \sum_{i=1}^{n} a_{i} 
      &&\text{(constant multiple)} \\[6ex]
      \sum_{i=1}^{n} a_{i} {\color{attn}\,\mathbf{+}\,} \sum_{i=1}^{n} b_{i} 
      &= \sum_{i=1}^{n} (a_{i} {\color{attn}\,\mathbf{+}\,}\, b_{i}) 
      &&\text{(sum)} \\[6ex]
      \sum_{i=1}^{n} a_{i} {\color{attn}\,\mathbf{-}\,} \sum_{i=1}^{n} b_{i} 
      &= \sum_{i=1}^{n} (a_{i} {\color{attn}\,\mathbf{-}\,} b_{i})
      && \text{(difference)} \\[6ex]
      \sum_{i=1}^{{\color{attn}\mathbf{n}}} a_{i} 
      &= 
      \sum_{i=1}^{{\color{attn}\mathbf{m}}} a_{i} + 
      \sum_{i={\color{attn}\mathbf{m+1}}}^{{\color{attn}\mathbf{n}}} a_{i} 
      && \text{(split/merge a sum)} \\[6ex]
    \end{align*}

    \textbook{Rule: Properties of Sigma Notation, page 509}
  \end{mdframed}

  To \hlmain{simplify} or to \hlmain{evaluate} a summation written in Sigma notation means to perform the specified addition (``get rid of the Sigma notation'').  These formulas are useful and \hlwarn{examinable}.
  \begin{equation} \label{formula:summations}
    \sum_{i=1}^{n} i = {\color{main} \frac{n(n+1)}{2}}, \qquad \sum_{i=1}^{n} i^{2} = {\color{main}\frac{n(n+1)}{2}} \cdot \frac{2n+1}{3}, \qquad \sum_{i=1}^{n} i^{3} = {\color{main}\frac{n(n+1)}{2}} \cdot \frac{n(n+1)}{2}.
  \end{equation}
  \blanklines{5}

  \begin{example}
    Simplify \(\sum_{i=1}^{n}(2i + 1)\).
    \blanklines{5}
  \end{example}
  \clearpage

  \begin{example}[Expressing an area using the Sigma notation]
    Express the \hlmain{total area \(A\)} of the rectangles shown below in Sigma notation. The height of each rectangle is labelled on their top edge. The width of each rectangle is \(1/2\).

    \blanklines{4}
    \begin{center}
      \includegraphics{../standalones/build/plot_rectangles}
    \end{center}
    \blanklines{30}
  \end{example}
  \clearpage

  Sometimes, we have to manipulate the Sigma notation before applying summation formulas.
  \begin{equation} \label{eq:summation-reindex}
    \sum_{i=m}^{n} a_{i} = \sum_{j = 1}^{n - m + 1} a_{m+j-1}.
  \end{equation}
  This ``reindexing'' formula is obtained by performing a change of variable \(j = i - m + 1\).
  \blanklines{5}
  \begin{example}
    Rewrite \(\sum_{i=10}^{100} i^{3}\) so that the index of the summation starts at \(1\).
    \blanklines{15}
  \end{example}

  \begin{example} 
    Simplify \(\sum_{i=50}^{300} 6 i^{2}\).  {\footnotesize Hint: Apply one of the summations formulas but not directly.}
    \blanklines{15}
  \end{example}
  \clearpage

  Example~\ref{ex:limit-riemann} is an example of routine calculations when we evaluate definite integrals (a major topic to be discussed starting no later than next week) from their definitions.
  \begin{example} \label{ex:limit-riemann}
    Use one of the summation formulas on page~\pageref{formula:summations} on to evaluate \(\lim_{n \to \infty} \sum_{i=1}^{n} \frac{1}{n} \left( \frac{i}{n} \right)^{2}\).  

    {\footnotesize Hint: The symbol \(n\) has two roles: It is a constant with respect to \(i\) inside the summation, but a variable outside the summation. Simplify the summation first and evaluate the resulting function in \(n\). Pretend \(n\) is a real-valued variable. The answer is \(1/3\).}
  \end{example}


\end{document}
