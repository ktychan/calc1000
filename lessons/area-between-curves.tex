%! TeX program = lualatex
\documentclass[../main.tex]{subfiles}
\begin{document} \section{Area between curves}
We return to the geometric interpretation of a definite integral to compute area between two curves.

\begin{example}
  Two curves are given in the plots below. 
  \begin{figure}[H]
    \centering
    \includestandalone[page=1]{../standalones/plot-area-between-curves}
    \hfill{}
    \includestandalone[page=2]{../standalones/plot-area-between-curves}
    \hfill{}
    \includestandalone[page=3]{../standalones/plot-area-between-curves}
  \end{figure}

  Set up a definite integral representing the area of the region between two curves \hlmain{\(y = f(x)\)} and \hlsupp{\(y = g(x)\)} and between lines \(x = -2\) and \(x = 1\).

  \blanklines{35}
\end{example}

What if two curves cross each other?
\begin{example}
  Find the area bounded between \(\sin(x)\) and \(\cos(x)\) over the interval \([0, \pi]\).

  \includestandalone[page=4]{../standalones/plot-area-between-curves}

  \blanklines{35}
\end{example}
\clearpage

What if the given curves are not functions? What if the horizontal bounds are not given?
\begin{example}
  Find the area of the finite region enclosed by the curves \(y = x - 1\) and \(y^{2} = 2x + 6\).

    % how can we transform Example~2 to a simpler problem?
    % talk about how to decide when to integrate with respect to y (vertical line test)
  
  \includestandalone[page=5]{../standalones/plot-area-between-curves}

  \blanklines{35}
\end{example}

\end{document}
