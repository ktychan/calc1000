%! TeX program = lualatex
\documentclass[../main.tex]{subfiles}
\begin{document}
\begin{lesson}{Area between curves}
  We return to the geometric interpretation of a definite integral to compute area between two curves.

  \begin{example}
    Two curves are given in the plots below. 
    \begin{figure}[H]
      \centering
      \includegraphics[width=2in]{../standalones/build/plot_area_between_curves_motivation.pdf}
      \hspace{1em}
      \includegraphics[width=2in]{../standalones/build/plot_area_between_curves_motivation_above.pdf}
      \hspace{2em}
      \includegraphics[width=2in]{../standalones/build/plot_area_between_curves_motivation_below.pdf}
    \end{figure}

    Set up a definite integral representing the area of the region between two curves \hlmain{\(y = f(x)\)} and \hlsupp{\(y = g(x)\)} and between lines \(x = -2\) and \(x = 1\).


    \blanklines{35}
  \end{example}

  What if two curves cross each other?
  \begin{example}
    Find the area bounded between \(\sin(x)\) and \(\cos(x)\) over the interval \([0, \pi]\).

    \includegraphics{../standalones/build/plot_area_between_curves_cross}

    \blanklines{38}
  \end{example}

  What if the given curves are not functions? What if the horizontal bounds are not given?
  \begin{example}
    Set up, but do not evaluate, a definite integral representing the area of the finite region enclosed by the curves \(y = x - 1\) and \(y^{2} = 2x + 6\).

    % how can we transform Example~2 to a simpler problem?
    % talk about how to decide when to integrate with respect to y (vertical line test)
    \begin{tikzpicture}
      \begin{axis}[
        axis lines = middle, % boxed, middle
        axis on top,
        axis equal image,
        %
        % domain and range
        %
        % xmin={}, 
        % xmax={},
        % ymin={},
        % ymax={},
        enlargelimits=true,
        %
        % axis labels
        %
        xlabel={\(x\)},
        ylabel={\(y\)},
        xlabel style={anchor=west},
        ylabel style={anchor=south},
        label style={at={(ticklabel* cs:1)}, font=\footnotesize},
        %
        % ticks
        %
        xtick={},
        xticklabels={},
        ytick={},
        yticklabels={},
        ticklabel style={font=\footnotesize},
        %
        % grid
        %
        grid=none, % none, major, minor, both
        % 
        % plot parameters
        %
        smooth,
        samples=100,
        no markers,
        %
        % titles and what not
        %
        title={},
        ]
        \addplot [domain=-3:6] {sqrt(2*x+6)};
        \addplot [domain=-3:2] {-sqrt(2*x+6)};
        \addplot [domain=-2:6] {x - 1};
      \end{axis}
    \end{tikzpicture}
    \blanklines{35}
  \end{example}
\end{lesson}
\end{document}
