%! TeX program = lualatex
\documentclass[../main.tex]{subfiles}
\begin{document} \section{Definite integrals}
  Everything we did in approximating areas using rectangles leads to a major concept in integration: The definite integral which is defined for all functions, not just the nonnegative ones.
  \begin{mdframed}[style=withref-compact]
    The \hlmain{definite integral of a function \(f(x)\) \hlsupp{from \(\alpha\) to \(\beta\)}} is defined by 
    \begin{equation} \label{eq:definite-integral} 
      \phantom{\int_{\alpha}^{\beta} f(x) \;dx} = \hspace{3in}
    \end{equation}
    \hlsupp{provided} that the limit exists. If this limit exists, then we say \hlmain{\(f\) is integrable on \([\alpha,\beta]\)} or is an integrable function. 

    Moreover, the number \(\int_{\alpha}^{\beta} f(x) \;dx\), if exists, represents the \hlmain{net area between \(y = f(x)\) and \(y = 0\)} \hlsupp{if \(\alpha \le \beta\)}. Some literature calls this the net area ``under'' the curve.
  \end{mdframed}
  \[
    \text{net area} = \bigg(\text{area below \(y = f(x)\) and above \(y = 0\)}\bigg) - \bigg(\text{area above \(y = f(x)\) and below \(y = 0\)}\bigg).
  \]
  
  \includegraphics{../standalones/build/plot_definite_integral_intro}

  \url{https://www.geogebra.org/calculator/gnfmamnj}
  
  % \begin{example} \label{ex:definite-integral-area}
  %   The function \(f(x)\) is sketched below. Evaluate \(\int_{-1}^{5} f(x) \;dx\).
  %
  %   \includegraphics{../standalones/build/plot_definite_integral_area}
  % \end{example}

  \bigskip
  We now study properties of definite integrals. Because the definite integral is defined in terms of a summation, all but one of its properties mimic that of summation properties on page~\pageref{page:summation-properties}.

  We have the usual constant multiple, sum and difference properties.
  \begin{align*}
    \int_{a}^{b} {\color{attn} \mathbf{C}} f(x) \;dx &= {\color{attn} \mathbf{C}} \int_{a}^{b} f(x) \;dx && \text{(constant multiple)}\\
    \int_{a}^{b} f(x) \;dx + \int_{a}^{b} g(x) \;dx &= \int_{a}^{b} f(x) + g(x) \;dx && \text{(sum)} \\
    \int_{a}^{b} f(x) \;dx - \int_{a}^{b} g(x) \;dx &= \int_{a}^{b} f(x) - g(x) \;dx && \text{(difference)}
  \end{align*}
  \clearpage
    
  Recall summations have a split/merge property \(\sum_{i=1}^{n} a_{i} = \sum_{i=1}^{m} a_{i} + \sum_{i=m+1}^{n} a_{i}\) \hlattn{restricted} to \(1 \le m \le n\).
  Definite integrals also have a split/merge property \hlattn{but with no restriction at all}.
  \begin{equation} \label{eq:definite-integral-split}
    \int_{a}^{b} f(x) \;dx = \int_{a}^{c} f(x) \;dx + \int_{c}^{b} f(x) \;dx, \text{ for any } c \text{ (can be outside \([a,b]\)}).
  \end{equation}
  \blanklines{5}

  Here is a typical problem involving the split/merge property.
  \begin{example} \label{ex:definite-integral-split}
    Suppose \(\int_{0}^{10} f(x) \;dx = 5\) and \(\int_{\pi}^{10} f(x) \;dx = 1\).  Evaluate \(\int_{0}^{\pi} f(x) \;dx\).
    \blanklines{10}
  \end{example}

  The split/merge property is helpful (in fact, necessary) for evaluating definite integrals of piecewise functions.
  \begin{example}
    Find an expression for \(\int_{-3}^{1} |(x+2)(x-1)| \;dx \) without the absolute value. 

    \blanklines{20}
  \end{example}
  \vfill{}\clearpage

  Definite integrals have a ``flip'' property (non-standard terminology) which \hlattn{does not} have a counterpart in summations.
  \begin{equation} \label{eq:definite-integral-flip}
    \int_{a}^{b} f(x) \;dx = \hspace{3in}
  \end{equation}
  Write down the limit definitions of \(\int_{a}^{b} f(x) \;dx\) and \(\int_{b}^{a} f(x) \;dx\). Notice the bounds are flipped.
  Compare the two expressions (focus on the summands) and decide how these two integrals are related. Complete the right-hand side of Equation~\eqref{eq:definite-integral-flip}.
  \blanklines{10}

  % \vfill{}\clearpage
  \begin{example}
    Let \(A = \int_{a}^{b} f(x) \;dx\). Which of the following is a true statement?
    \begin{enumerate}[label=(\alph*)]
      \item \(A = \int_{0}^{a} f(x) \;dx + \int_{0}^{b} f(x) \;dx\).
      \item \(A = - \int_{0}^{a} f(x) \;dx + \int_{0}^{b} f(x) \;dx\).
    \end{enumerate}
    
    \blanklines{20}
  \end{example}

  % \begin{example}
  %   The two endpoints of an interval are \(a,b\), and we don't know if \(a < b\) or \(a > b\). However we know that \(\int_{a}^{b} f(x) \;dx < 0\) and \(f\) is nonnegative.  
  %
  %   Which is larger? \(a\) or \(b\)?
  %   \blanklines{10}
  % \end{example}
  \clearpage

  The net area interpretation of a definite integral yields two comparison properties. For each part below, fill in the blanks with one of \(=, \le , \ge\) by interpreting definite integrals as net area.
  \begin{enumerate}[wide, noitemsep]
    \item If \(f(x) \le g(x)\) over \([a,b]\), then
      \[
        \int_{a}^{b} f(x) \;dx \underline{\hspace{2cm}} \int_{a}^{b} g(x) \;dx.
      \]
      \blanklines{8}

    \item If \(m\) and \(M\) are constants such that \(m \le f(x) \le M\) on \([a,b]\), then
      \[
        m(b-a) \underline{\hspace{2cm}} \int_{a}^{b} f(x) \underline{\hspace{2cm}} M(b-a).
      \]
      \blanklines{8}
  \end{enumerate}

  \begin{example}
    A function \(f(x)\) defined on \([a,b]\) has the following properties. 
    \begin{enumerate}
      \item \(f\) is continuous on \([a,b]\)
      \item \(f\) has local extrema \(-1, 3, -2\). 
      \item \(f(a) = 1\) and \(f(b) = 0\).
      \item \(b - a = 10\).
    \end{enumerate}

    Find numbers \(A,B\) such that \(A \le \int_{a}^{b} f(x) \;dx \le B\). Justify your answer. 
    \blanklines{10}
  \end{example}
  \clearpage
 
  We now consider the \emph{total} area ``under'' the curve which is the area (in its usual interpretation) enclosed between \(y = f(x)\) and \(y = 0\).

  \blanklines{10}

  \begin{example}
    Consider the following function \(f(x)\).  It is known that the only zeros of \(f\) in \([\alpha, \beta]\) are \(x = 3\) and \(x = 2\). Express the \emph{total} area enclosed between \(y = f(x)\) and \(y = 0\) as a sum of one or more integrals.

    \includegraphics{../standalones/build/plot_definite_integral_intro}

    \blanklines{20}
  \end{example}
  \clearpage


  We wrap up the discussion of the geometric interpretation of definite integrals by defining the average value of a function. 
  \blanklines{10}

  The average value of a continuous \(f(x)\) on \([a,b]\) is
  \[
    f_{\text{avg}} = \frac{1}{b-a} \int_{a}^{b} f(x) \;dx.
  \]

  \begin{example}
    Find the average value of \(f(x) = x^{3} + \sqrt{x}\) over \((-4,0)\).

    \blanklines{10}
  \end{example}

\end{document}
