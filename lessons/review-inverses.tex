%! TeX program = lualatex
\documentclass[../main.tex]{subfiles}
\begin{document} \section{Review of inverse functions}
Intuitively, the inverse of a function \(f(x)\) is another function \(g(x)\) that undoes whatever \(f(x)\) does to its input. This intuition is not precise enough to be useful. Let's start with the definition.

\begin{mdframed}[style=withref-compact]
  Given a function \(f\) whose domain is \(D\) and range is \(R\), its \hlmain{inverse function} (if it exists) is \hlsupp{the} function \(f^{-1}(x)\) such that
  \[
    f^{-1}(f(x)) = x \text{ on } D \quad\text{and}\quad f(f^{-1}(x)) = x \text{ on } R.
  \]

  \textbook{Page 78}
\end{mdframed}

{\faExclamationTriangle{} \(f^{-1}(x)\) is NOT \(\tfrac{1}{f(x)}\). To write \(\tfrac{1}{f(x)}\) as a power, write \(\tfrac{1}{f(x)} = f(x)^{-1}\) or \(\tfrac{1}{f(x)} = \big( f(x) \big)^{-1}\).}
\bigskip

This definition is \hlinfo{actually very useful}. This definition \emph{alone} tells us how to
\begin{enumerate}[noitemsep]
  \item check whether two functions are inverses of each other, 
  \item determine whether a function has an inverse graphically, 
  \item sketch the inverse function, 
  \item find inverses (graphically and algebraically) if possible,
  \item find inverses (graphically and algebraically) by restricting the domain, and
  \item avoid subtle mistakes.
\end{enumerate}

We skip the trial-and-error part of the story and go straight to the \hlinfo{insight}: Let's pretend \(f(x)\) has an inverse and denote \underline{\hspace{2in}}. The definition now gives us an useful equation.

\blanklines{25}
\clearpage

\begin{example}
  Are \(x^{2}\) and \(\sqrt{x}\) inverses of each other? Assume \(x^{2}\) is defined for all real numbers.

  \blanklines{10}
\end{example}

\begin{example}
  Find the inverse of \(\frac{2x}{1 + x}\).

  \blanklines{15}
\end{example}

\begin{exercise}
  Find the inverse of \(\frac{(x - 3)^{4}}{2}\), restrict the domain if necessary.  Determine the domain and range of its inverse. 

  \blanklines{10}
\end{exercise}

\begin{exercise}
  Are \(f(x) = (x+1)^{3} - 2\) and \(g(x) = (x+2)^{1/3} - 1\) inverses of each other?

  \blanklines{10}
\end{exercise}
\clearpage

\begin{exercise}
  Sketch the inverse of the given function if it exists.

  \includestandalone[page=2]{../standalones/plot-exercise-sketch-inverse}
  \quad
  \includestandalone[page=1]{../standalones/plot-exercise-sketch-inverse}
\end{exercise}

\begin{exercise}
  Sketch the inverse of the given function if it exists.

  \includestandalone[page=3]{../standalones/plot-exercise-sketch-inverse}
  \quad
  \includestandalone[page=1]{../standalones/plot-exercise-sketch-inverse}
\end{exercise}

\begin{exercise}
  The inverse of \(x^{2}\) restricted to the domain \((-\infty, 0]\) is \(-\sqrt{x}\). One can explain this phenomenon using a graph, the definition and an appropriate equivalent equation. 

  Find at least two out of the three explanations for yourself.

  \blanklines{19}
\end{exercise}
\clearpage

We now turn to inverses of trig functions which have \hlwarn{restrictions on their range} because all trig functions fail the horizontal line test.

\begin{tabular}{l|p{1in}|p{1in}||l|p{1in}|p{1in}}
  \toprule
                       & domain & range &
                       & domain & range \\\midrule
  \(y = \sin^{-1}(x)\) & \(-1 \le x \le 1\) & &
  \(y = \csc^{-1}(x)\) &  & \\[3ex]\midrule
  \(y = \cos^{-1}(x)\) & \(-1 \le x \le 1\) & &
  \(y = \sec^{-1}(x)\) &  & \\[3ex]\midrule
  \(y = \tan^{-1}(x)\) & all \(x\)          & &
  \(y = \cot^{-1}(x)\) &  & \\[3ex]\bottomrule
\end{tabular}

Remember, equivalent equations apply to inverse trig functions.
\begin{example} \label{eg:arccos-cos-caution}
  Evaluate \(\cos^{-1}(\cos(2\pi))\). 

  \blanklines{10}
\end{example}

\begin{exercise}
  Evaluate \(\tan^{-1}\left(\frac{\sqrt{3}}{3}\right)\).

  \blanklines{25}
\end{exercise}
\clearpage

Learn the thinking process in Example~\ref{eg:trig-sec-arctan}, and remember its \emph{starting} step.  The idea applies to all problems in Exercise~\ref{ex:trig-f-finverse}.
\begin{example} \label{eg:trig-sec-arctan}
  Simplify \(\sec(\tan^{-1}(x))\).   

  \blanklines{25}
\end{example}

\begin{exercise} \label{eg:trig-cos-arcsin}
  Simplify \(\frac{1}{\cos(\sin^{-1}(x))}\).

  \blanklines{25}
\end{exercise}
\clearpage

\begin{exercise} \label{ex:trig-f-finverse}
  Practise using triangles to simplify the following expressions. 

  \begin{multicols}{4}
    \begin{enumerate}
      \item \(\sin(\sin^{-1}(x))\)
      \item \(\cos(\sin^{-1}(x))\)
      \item \(\tan(\sin^{-1}(x))\)
      \item \(\sec(\sin^{-1}(x))\)
      \item \(\sin(\cos^{-1}(x))\)
      \item \(\cos(\cos^{-1}(x))\)
      \item \(\tan(\cos^{-1}(x))\)
      \item \(\sec(\cos^{-1}(x))\)
      \item \(\sin(\tan^{-1}(x))\)
      \item \(\cos(\tan^{-1}(x))\)
      \item \(\tan(\tan^{-1}(x))\)
      \item \(\sec(\tan^{-1}(x))\)
      \item \(\sin(\sec^{-1}(x))\)
      \item \(\cos(\sec^{-1}(x))\)
      \item \(\tan(\sec^{-1}(x))\)
      \item \(\sec(\sec^{-1}(x))\)
    \end{enumerate}
  \end{multicols}

  \blanklines{45}
\end{exercise}
\clearpage

\begin{exercise}
  Label the graphs with corresponding inverse trig functions.

  \begin{center}
    \includestandalone[page=15]{../standalones/plot-trigs}

    \includestandalone[page=16]{../standalones/plot-trigs}

    \includestandalone[page=17]{../standalones/plot-trigs}
  \end{center}
\end{exercise}
\clearpage
\end{document}

