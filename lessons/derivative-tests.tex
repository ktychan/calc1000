%! TeX program = lualatex
\documentclass[../main.tex]{subfiles}
\begin{document} \section{Derivatives and the shape of a graph}

\begin{center}
  \hfill{}
  \includestandalone[page=1]{../standalones/plot-derivative-test}
  \hfill{}
  \includestandalone[page=2]{../standalones/plot-derivative-test}
  \hfill{}
  \includestandalone[page=3]{../standalones/plot-derivative-test}
  \hfill{}
  \includestandalone[page=4]{../standalones/plot-derivative-test}
  \hfill{}
\end{center}

\begin{mdframed}[style=simple-compact]
  \begin{enumerate}[label=(\alph*)]
    \item If \underline{\hspace{1.5in}} for every \(x\) in \(I\), then \(f(x)\) is increasing on \(I\).
    \item If \underline{\hspace{1.5in}} for every \(x\) in \(I\), then \(f(x)\) is decreasing on \(I\).
  \end{enumerate}
\end{mdframed}

\faStar{} If \(f'(c) = 0\) for some constant \(c\), then \(f(x)\) is \underline{\hspace{3.5in}}.

\begin{example}
  Consider the graph of \(f'(x)\) below. On the horizontal axis, label the interval on which \(f(x)\) is increasing. 

  \begin{center}
    \includestandalone[page=5]{../standalones/plot-derivative-test}
  \end{center}
\end{example}

We learned that critical points are candidates for local extrema. Let's learn two tests to determine if a critical point is a local extrema or not.

The idea: If \(f(x)\) has a local extrema at \(c\), then \((c,f(c))\) must be the ``peak or valley of a mountain.''  We simply look left and right.
\begin{center}
  \includestandalone[page=6]{../standalones/plot-derivative-test}
  \hfill
  \includestandalone[page=7]{../standalones/plot-derivative-test}
  \hfill
  \includestandalone[page=8]{../standalones/plot-derivative-test}

  \url{https://www.geogebra.org/calculator/atcbwgvm}
\end{center}
\clearpage

\begin{mdframed}[style=withref-compact]
  Suppose that \(f\) is a continuous function over an interval \(I\) containing a critical number \(c\). 

  If \(f\) is differentiable over \(I\), except possibly at \(c\), then \(f(c)\) satisfies one of the following descriptions:
  \begin{enumerate}
    \item If \(f'\) \emph{changes} sign from positive when \(x < c\) to negative when \(x > c\), then \(f\) has a \underline{\hspace{2in}} at \(c\).
    \item If \(f'\) \emph{changes} sign from negative when \(x < c\) to positive when \(x > c\), then \(f\) has a \underline{\hspace{2in}} at \(c\).
    \item If \(f'\) has the same sign for \(x < c\) and \(x > c\), then \(f(c)\) has neither a local maximum nor a local minimum of \(f\).
  \end{enumerate}

  \textbook{Theorem~4.9 First Derivative Test on page 392}
\end{mdframed}

\begin{example}[Textbook Example~4.18] \label{ex:first-derivative}
  Use the first derivative test to find the location of all local extrema for \(f(x) = 5x^{1/3} - x^{5/3}\). Graph: \url{https://www.geogebra.org/calculator/nzwzefct}

  \blanklines{25}
\end{example}
\clearpage

We can dig up more information about a function by considering concavity.
\begin{mdframed}[style=withref-compact]
  Let \(f\) be a differentiable function over an \emph{open} interval \(I\).
  \begin{itemize}
    \item If \(f'\) is \underline{\hspace{1in}} over \(I\), then we say \(f\) is \hlmain{concave up}.
    \item If \(f'\) is \underline{\hspace{1in}} over \(I\), then we say \(f\) is \hlmain{concave down}.
    \item If \(f\) is continuous at \(c\) and changes concavity at \(c\), then \((a,f(a))\) is an \hlmain{inflection point}. 
  \end{itemize}

  \textbook{Definition on page 395 and 397}
\end{mdframed}

\begin{center}
  \includestandalone[page=9]{../standalones/plot-derivative-test}
  \hfill
  \includestandalone[page=10]{../standalones/plot-derivative-test}
  \hfill
  \includestandalone[page=11]{../standalones/plot-derivative-test}

  \url{https://www.geogebra.org/calculator/atcbwgvm}
\end{center}

\begin{mdframed}[style=withref-compact]
  Let \(f\) be a function whose second derivative exists over an interval \(I\).
  \begin{itemize}
    \item If \underline{\hspace{1in}} on an interval \(I\), then the graph of \(f\) is \hlmain{concave up} on \(I\).
    \item If \underline{\hspace{1in}} on an interval \(I\), then the graph of \(f\) is \hlmain{concave down} on \(I\).
  \end{itemize}

  \textbook{Theorem 4.0 Test for Concavity on page 396}
\end{mdframed}

\begin{example} \label{ex:concavity}
  Let \(f = x^{4} - 4x^{3}\). Find all intervals on which \(f\) is concave up and all intervals on which \(f\) is concave down. List all inflection points of \(f\).

  \blanklines{15}
\end{example}

\clearpage

When a function is twice differentiable, the Second Derivative Test \hlmain{could be a faster method} for finding local extrema.

\begin{mdframed}[style=withref-compact]
  Suppose \(f'(c) = 0\) and \(f''\) is continuous near \(c\).
  \begin{itemize}
    \item If \(f''(c) > 0\) (meaning \(f\) is \underline{\phantom{concave down}} near \(c\)), then \(f\) has a \underline{\hspace{1in}}.
    \item If \(f''(c) < 0\) (meaning \(f\) is \underline{\phantom{concave down}} near \(c\)), then \(f\) has a \underline{\hspace{1in}}.
    \item[\faExclamationTriangle{}] If \(f''(c) = 0\) or \(f''(c)\) does not exists, then the Second Derivative Test is \hlwarn{inconclusive} meaning that we need to \hlsupp{explore other methods} to determine if \(f\) has a local extrema at \(c\).
  \end{itemize}

  \textbook{Theorem~4.11 Second Derivative Test on page 400}
\end{mdframed}

\begin{example}
  Use the Second Derivative Test to find all local extrema for \(f(x) = x^{4} - 4x^{3}\) from Example~\ref{ex:concavity}.
  \blanklines{35}
\end{example}
\vfill

\end{document}
