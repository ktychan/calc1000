%! TeX program = lualatex
\documentclass[../main.tex]{subfiles}
\begin{document} \section{Review of trigonometric functions}
\faComments{} What mathematical knowledge do you associate with trigonometry?

\blanklines{5}

\begin{mdframed}[style=simple-compact]
  \textbf{Fundamentals of trigonometry}.

  \begin{minipage}{0.45\textwidth}
    \includestandalone[page=1]{../standalones/triangle}
  \end{minipage}
  \begin{minipage}{0.55\textwidth}
    \begin{align} 
      \label{eq:sin}
      \sin(\theta) = \hspace{2in} \\[5ex]
      \label{eq:cos}
      \cos(\theta) = \hspace{2in} \\[5ex]
      \label{eq:trig-pythegorean}
      \hspace{1.5in} = \hspace{1cm}
    \end{align}
  \end{minipage}

  \bigskip
  We measure angles in radians unless specified otherwise.
  \begin{center}
    \includestandalone[page=3]{../standalones/triangle}
    \includestandalone[page=4]{../standalones/triangle}
  \end{center}

  \bigskip
  All trigonometric functions are periodic with period \(2\pi\).
  \begin{center}
    \includestandalone[page=5]{../standalones/triangle}

    \includestandalone[page=6]{../standalones/triangle}
  \end{center}
\end{mdframed}
\clearpage

\faExclamationTriangle{}
Midterm and final exams are closed-book with no formula sheets, no calculators and no aids of any kind.  Trying to memorize too much might result in false memory.  Trying to deduce everything might take up a lot of time.  \hlmain{Find the right balance for yourself through practice.}

\begin{example}
  Evaluate \(\cos(11\pi/3)\).

  \blanklines{12}
\end{example}

We need to remember trigonometric equations often, but not always, have more than one solutions.
If a question does not state a restriction for \(\theta\), then we need to find as many solutions as possible.

Moreover, Example~\ref{ex:trig-problem-solving-1} and \ref{ex:trig-problem-solving-2} together demonstrate the idea of \hlmain{\faStar{} change of variable \faStar{}} which is a key problem-solving technique to be used throughout the course.
\begin{example} \label{ex:trig-problem-solving-1}
  Find all solutions of \(\sin(u) = \frac{\sqrt{3}}{2}\), i.e., solve for \(u\).

  \blanklines{10}
\end{example}

\begin{example} \label{ex:trig-problem-solving-2}
  Find all solutions of \(\sin(\theta + \pi/3) = \frac{\sqrt{3}}{2}\), i.e., solve for \(\theta\).

  {\footnotesize Make a change of variable \(u = (\cdots)\) so that you can \hlmain{reuse} the solution of Example~\ref{ex:trig-problem-solving-1}.}

  \blanklines{15}

  % Common misconception: Why not just apply \(\arcsin\) to both sides?
\end{example}
\clearpage



All trigonometric functions build on \(\sin(x)\) and \(\cos(x)\).  Knowing the relations among trigonometric functions helps us remember their properties.

\begin{mdframed}[style=simple-compact]
  \textbf{Relations among trigonometric functions}.
  \begin{align}
    \tan(\theta) &= \hspace{1.5in} & \cot(\theta) &= \hspace{1.5in}& \\[3ex]
    \sec(\theta) &= \hspace{1.5in} & \csc(\theta) &= \hspace{1.5in}& 
  \end{align}

  Pythagorean Theorem gives us two \emph{more} fundamental trig identities.
  \vspace{1.5in}
\end{mdframed}

\begin{exercise}
  Complete Table~\ref{table:trig-values}.  Write \hlsupp{VA} for vertical asymptotes.



  \begin{table}[H]
    \begin{tabular}{r|p{1cm}|p{1cm}|p{1cm}|p{1cm}}
      \(\theta\) & \(\sin(\theta)\) & \(\cos(\theta)\) & \(\tan(\theta)\) & \(\cot(\theta)\) \\\toprule 
      \(    0 + 2k\pi\) &&& \\[2ex]\midrule\midrule
      \(\pi/6 + 2k\pi\) &&& \\[2ex]\midrule
      \(\pi/4 + 2k\pi\) &&& \\[2ex]\midrule
      \(\pi/3 + 2k\pi\) &&& \\[2ex]\midrule\midrule
      \(  \pi/2 + 2k\pi\) &&& \\[2ex]\midrule\midrule
      \(2\pi/3 + 2k\pi\) &&& \\[2ex]\midrule
      \(3\pi/4 + 2k\pi\) &&& \\[2ex]\midrule
      \(5\pi/6 + 2k\pi\) &&& \\[2ex]\midrule\midrule
      \(   \pi + 2k\pi\) &&& \\[2ex]
    \end{tabular}
    \quad
    \begin{tabular}{r|p{1cm}|p{1cm}|p{1cm}|p{1cm}}
      \(\theta\) & \(\sin(\theta)\) & \(\cos(\theta)\) & \(\tan(\theta)\) & \(\cot(\theta)\) \\\toprule 
      \(     0 + 2k\pi\) &&& \\[2ex]\midrule\midrule
      \(-\pi/6 + 2k\pi\) &&& \\[2ex]\midrule
      \(-\pi/4 + 2k\pi\) &&& \\[2ex]\midrule
      \(-\pi/3 + 2k\pi\) &&& \\[2ex]\midrule\midrule
      \( -\pi/2 + 2k\pi\) &&& \\[2ex]\midrule\midrule
      \(-2\pi/3 + 2k\pi\) &&& \\[2ex]\midrule
      \(-3\pi/4 + 2k\pi\) &&& \\[2ex]\midrule
      \(-5\pi/6 + 2k\pi\) &&& \\[2ex]\midrule
      \midrule
      \(   -\pi + 2k\pi\) &&& \\[2ex]
    \end{tabular}

    \caption{Special values for trigonometric function}
    \label{table:trig-values}
  \end{table}
\end{exercise}


\clearpage

Let's turn to algebra skills.
\begin{example}[Textbook Example~1.25a on page 68]
  Find all solutions of \(1 + \cos(2 \theta) = \cos(\theta)\).

  {\scriptsize The textbook straight up tells us to use trig identities. Why? What's a \hlmain{general} problem-solving idea that helps us in the long run?}

  \blanklines{45}
\end{example}

\faExclamationTriangle{} Most of the suggested exercises in this section challenge your algebra skill --- a key requirement for success in calculus. Please do all of them and discuss your results with others. Talk about how you break up a problem into smaller ones, the purpose of your calculations and problem-solving strategies. 
\clearpage

\begin{exercise}
  Label each graph with their corresponding trig function. Which ones are even functions? Which ones are odd functions?  \newline
  {\footnotesize Recall even functions satisfy \(f(-x) = f(x)\) and has symmetry about the \(y\)-axis. Odd functions satisfy \(f(-x) = -f(x)\) and has symmetric about the origin.}
  \bigskip

  \begin{center}
    \includestandalone[page=1]{../standalones/plot-trigs}
    \hfill\vline\hfill
    \includestandalone[page=4]{../standalones/plot-trigs}
  \end{center}
  \vfill{}

  \begin{center}
    \includestandalone[page=7]{../standalones/plot-trigs}
    \hfill\vline\hfill
    \includestandalone[page=10]{../standalones/plot-trigs}
  \end{center}
  \vfill{}

  \begin{center}
    \includestandalone[page=13]{../standalones/plot-trigs}
    \hfill\vline\hfill
    \includestandalone[page=14]{../standalones/plot-trigs}
  \end{center}
  \vfill{}
\end{exercise}
\end{document}
