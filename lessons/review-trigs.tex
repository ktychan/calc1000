%! TeX program = lualatex
\documentclass[../main.tex]{subfiles}
\pagestyle{draft}

\begin{document}
\begin{lesson}{Review of Trigonometric Functions}
  \faComments{} What mathematical knowledge do you associate to trigonometry?
  \blanklines{5}

  \begin{example}
    Convert radian to degree and vice versa. 

    \begin{minipage}{.4\textwidth}
      \begin{tabular}{p{1in} | p{1in}}
        Radian & Degree \\\midrule
        \(2\pi\) & \\[2ex] \midrule
                 & \(45^{\circ}\) \\[2ex] \midrule
                 & \(-135^{\circ}\) \\[2ex] \midrule
        \(11\pi/3\) & \\[2ex] \midrule
      \end{tabular}
    \end{minipage}
    \begin{minipage}{.59\textwidth}
      \includegraphics{../standalones/build/plot_unit_circle.pdf}
    \end{minipage}
  \end{example}

  Two fundamental trigonometric functions \(\sin(\theta)\) and \(\cos(\theta)\) are \hlinfo{defined} using the unit circle.

  \begin{minipage}{2.5in}
    \includegraphics{../standalones/build/plot_unit_circle.pdf}
  \end{minipage}
  \begin{minipage}{4.5in}
    \blanklines{12}
  \end{minipage}

  \begin{example}
    Evaluate \(\cos(11\pi/3)\).

    \begin{minipage}{2.5in}
      \includegraphics{../standalones/build/plot_unit_circle.pdf}
    \end{minipage}
    \begin{minipage}{4.5in}
      \blanklines{12}
    \end{minipage}
  \end{example}
  \clearpage

  Relations among trigonometric functions are useful when we want to simplify trigonometric functions. 

  \begin{example}
    Rewrite the following trigonometric functions in term of \(\sin(\theta)\) and \(\cos(\theta)\).

    \begin{align*}
      \tan(\theta) &= \hspace{3in} && \cot(\theta) &= \hspace{3in} \\[8ex]
      \sec(\theta) &= \hspace{3in} && \csc(\theta) &= \hspace{3in} \\[5ex]
    \end{align*}
    
  \end{example}

  The Pythagorean Theorem gives us fundamental trig identities.
  \blanklines{12}

  Sometimes, we need to solve trig equations. This typically involves some trig identities, some algebra and some trial-and-error. 

  First, we should remember that trig equations typically (but not always) have more than one solutions. 
  \begin{example}
    Find all solutions of \(\sin(\theta + \pi/3) = \frac{\sqrt{2}}{2}\).

    \blanklines{15}
  \end{example}
  \clearpage

  Let's focus on algebra skills.
  \begin{example}[Textbook Example~1.25a on page 68]
    Find all solutions of \(1 + \cos(2 \theta) = \cos(\theta)\).

    {\scriptsize The textbook straight up tells us to use trig identities. Why? What's a \hlmain{general} problem-solving idea that helps us in the long run?}

    \blanklines{45}
  \end{example}

  \faExclamationTriangle{} Most of the suggested exercises in this section challenge your algebra skill --- a key requirement for success in calculus. Please do all of them and discuss your results with others. Talk about how you break up a problem into smaller ones, the purpose of your calculations and problem-solving strategies. 
  \clearpage

  % \begin{example}[Textbook Exercise~148]
  %   Verify that \(\frac{\sec^{2}(\theta)}{\tan(\theta)} = \sec{\theta}\csc{\theta}\) is an identity. 
  % \end{example}
  % \clearpage

  \begin{example}
    Label each graph with their corresponding trig function. Which ones are even functions? Which ones are odd functions?  \newline
    {\footnotesize Recall even functions satisfy \(f(-x) = f(x)\) and has symmetry about the \(y\)-axis. Odd functions satisfy \(f(-x) = -f(x)\) and has symmetric about the origin.}

    \begin{center}
      \includegraphics{../standalones/build/plot_sin}
      \hspace{0.5in}
      \includegraphics{../standalones/build/plot_cos}
    \end{center}
    \vfill{}

    \begin{center}
      \includegraphics{../standalones/build/plot_tan}
      \hspace{0.5in}
      \includegraphics{../standalones/build/plot_cot}
    \end{center}
    \vfill{}

    \begin{center}
      \includegraphics{../standalones/build/plot_sec}
      \hspace{0.5in}
      \includegraphics{../standalones/build/plot_csc}
    \end{center}
    \vfill{}
  \end{example}

  \begin{example}
    Recall special values of sine and cosine functions. Use symmetry properties of trig functions to calculate the rest.

    \begin{tabular}{r|p{1cm}|p{1cm}|p{1cm}|p{1cm}|p{1cm}|p{1cm}|p{1cm}|p{1cm}|p{1cm}}
      \(\theta\) & \(0\) & \(\frac{\pi}{6}\) & \(\frac{\pi}{4}\) & \(\frac{\pi}{3}\) & \(\frac{\pi}{2}\) & \(\frac{2\pi}{3}\) & \(\frac{3\pi}{4}\) & \(\frac{5\pi}{6}\) & \(\pi\) \\\midrule
      \(\sin(\theta)\) &&&&&&&&&\\[3ex]\midrule
      \(\cos(\theta)\) &&&&&&&&&\\[3ex]\midrule
      \(\tan(\theta)\) &&&&&&&&&\\[3ex]
    \end{tabular}

    \begin{tabular}{r|p{1cm}|p{1cm}|p{1cm}|p{1cm}|p{1cm}|p{1cm}|p{1cm}|p{1cm}|p{1cm}}
      \(\theta\) & \(0\) & \(-\frac{\pi}{6}\) & \(-\frac{\pi}{4}\) & \(-\frac{\pi}{3}\) & \(-\frac{\pi}{2}\) & \(-\frac{2\pi}{3}\) & \(-\frac{3\pi}{4}\) & \(-\frac{5\pi}{6}\) & \(-\pi\) \\\midrule
      \(\sin(\theta)\) &&&&&&&&&\\[3ex]\midrule
      \(\cos(\theta)\) &&&&&&&&&\\[3ex]\midrule
      \(\tan(\theta)\) &&&&&&&&&\\[3ex]
    \end{tabular}
  \end{example}
  \vfill{}

  \clearpage
\end{lesson}
\end{document}
