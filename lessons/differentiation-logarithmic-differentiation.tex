%! TeX program = lualatex
\documentclass[../main.tex]{subfiles}
\begin{document}
\begin{lesson}{Logarithmic Differentiation}
  We introduce a method called \emph{logarithmic differentiation}. It works by combining implicit differentiation, Equation~\eqref{eq:derivative-of-ln(g(x))} below and properties of logarithms.
  \begin{mdframed}[style=withref]
    If \(g(x)\) is a differentiable function, then by the Chain Rule
    \begin{equation} \label{eq:derivative-of-ln(g(x))}
      \frac{d}{dx} \ln( g(x) ) = \frac{\frac{d}{dx}g(x)}{g(x)}.
    \end{equation}

    \textbook{Equation~(3.33) specialized to \(b = e\). See page 326.}
  \end{mdframed}
  \blanklines{8}

  Here is the \emph{reference} for the full idea. To differentiate \(y = f(x)\) using logarithmic differentiation, do the following four steps.
  \begin{enumerate}
    \item Take the natural logarithm \(\ln(\cdots{})\) on both sides of the original equation \(y = f(x)\). Simplify the resulting right-hand side using properties of logarithms such as 
      \[
        \ln(ab) = \ln(a) + \ln(b), \quad \ln\left(\frac{a}{b}\right) = \ln(a) - \ln(b), \quad \ln(a^{c}) = c \ln(a). 
      \]
      These properties turn products (of functions) into sums of \(\ln(\cdots)\), quotients into difference of \(\ln(\cdots)\), and powers into coefficients of \(\ln(\cdots)\).

      We will get an equation whose form is 
      \begin{equation} \label{eq:logarithmic-differentiation}
        \color{main}
        \ln(y) = (\text{stuff})\ln( \text{simpler function} ) \pm \cdots \pm (\text{stuff})\ln( \text{simpler function} ),
      \end{equation}
      where all products and quotients in \(f(x)\) are turned into sums and differences of \(\ln(\cdots)\), and all powers in \(f(x)\) are turned into constants in front of \(\ln(\text{simpler function})\).

      \hlmain{Benefit: Equation~\eqref{eq:logarithmic-differentiation} has a simpler structure. Its summands can often (not always) be differentiated in a single step using Equation~\eqref{eq:derivative-of-ln(g(x))}. If the (stuff) in front of a \(ln(\text{simpler function})\) is a non-constant function, then we need the product rule.}

    \item Implicitly differentiate Equation~\eqref{eq:logarithmic-differentiation} with respect to \(x\). Use Equation~\eqref{eq:derivative-of-ln(g(x))} when appropriate.

    \item Isolate for \(\frac{dy}{dx}\) to rewrite your expression as \(\frac{dy}{dx} = \cdots\). 
    \item Replace \(y\) by the original expression \(f(x)\), so that \(dy/dx\) restored as a function of \(x\).
  \end{enumerate}
  \clearpage

  \begin{example} \label{ex:log-diff}
    Differentiate \(y = \frac{\sin(2x)^{4/3} \sqrt{x^{2}+1}}{e^{x} \cos^{5}(x)}\) using logarithmic differentiation.

    \begin{enumerate}[wide, noitemsep]
      \item Take \(\ln(\cdots{})\) of both sides of the equation. Simplify the right-hand side using properties of logarithms.
        \blanklines{30}
      \item Implicitly differentiate Equation~\eqref{eq:logarithmic-differentiation} with respect to \(x\). Use Equation~\eqref{eq:derivative-of-ln(g(x))} when appropriate.
        \blanklines{19}
      \item Isolate for \(\frac{dy}{dx}\) to rewrite your expression as \(\frac{dy}{dx} = \cdots\). 
        \blanklines{10}
      \item Replace \(y\) by the original expression \(f(x)\), so that \(dy/dx\) is restored as function of \(x\).
        \blanklines{10}
    \end{enumerate}
  \end{example}

  Logarithmic differentiation can be used to differentiate exponential-like functions.  However, it is not the only method. Unless a question specifies a method, we are free to use whichever method we like.
  \begin{example}[Example 26 from week 2]
    Differentiate \(x^{x}\) with and without logarithmic differentiation.

    \blanklines{24}
  \end{example}

  \clearpage

  \begin{example}[Example~28 from week 2]
    Differentiate \(\left( \frac{1}{x+2} \right)^{\sin(x)}\) with and without logarithmic differentiation.
    \blanklines{25}
  \end{example}

  Do the next two exercises if you are interested in more theory; otherwise, it is safe skip them.

  \begin{example}
    If you are interested in seeing the benefit of logarithmic differentiation, differentiate \(y = \frac{\sin(2x)^{4/3} \sqrt{x^{2}+1}}{e^{x} \cos^{5}(x)}\) (the function in Example~\ref{ex:log-diff}) without logarithmic differentiation and compare the \emph{length and complexity} of your two solutions. You will probably need a separate page (or two).
  \end{example}

  \begin{example}
    Deduce the power rule \(\frac{d}{dx} x^{n} = nx^{n-1}\) for any \(n \ne 0\) using logarithmic differentiation. We only need basic properties of logarithms and that \(\tfrac{d}{dx} \ln(x) = 1/x\).
  
    \blanklines{15}
  \end{example}
\end{lesson}
\end{document}
