%! TeX program = lualatex
\documentclass[../main.tex]{subfiles}
\begin{document} \section{Exam-writing strategies}
  \faComment{} How do we manage time during an exam?
  \blanklines{50}
  \clearpage

  \faComment{} How do we ``unstuck'' ourselves during an exam?
  \blanklines{50}
  \clearpage

  % \textbf{Change of variable} comes up a few times but is never the focus of any topic. However, a change of variable can quickly turn an unfamiliar problem into a familiar one. 
  %
  % Besides the chain rule (in differentiation) and the substitution rule (in integration), change of variables lands itself very well in evaluating limits. We have four ``magic'' limits.
  % \[
  %   \lim_{x \to 0} \frac{\sin(x)}{x} = \hspace{1cm}, \quad \lim_{x \to 0} \frac{1 - \cos(x)}{x} = \hspace{1cm},\quad \lim_{x \to 0} \frac{e^{x} - 1}{x} = \hspace{1cm},\quad \lim_{x \to \infty} \left( 1 + \frac{1}{x} \right)^{x} = \hspace{1cm}.
  % \]
  %
  % \begin{example} \label{ex:strategy-change-of-variable}
  %   Use a change of variable to evaluate \(\lim_{x \to 0} \frac{\left(e^{x}\right)^{2} - 1}{x/3}\), quickly!
  %   \blanklines{10}
  % \end{example}
  %
  % Seeing lots of examples helps us to understand \hlmain{when} a change of variable works \hlmain{in our favour}. Spend a few minutes to generate examples for yourself. 
  %
  % \faStar{} In each of the four ``magic'' limits, \(x\) appears twice. For each ``magic'' limit, randomly choose two non-zero numbers \(a,b\) and replace one \(x\) by \(ax\) and the other by \(bx\). For the last one, only choose positive numbers. Choose integers as well as fractions. Evaluate the resulting limit by a change of variable. Repeat a few times, and you should see a pattern. Example~\ref{ex:strategy-change-of-variable} is generated by choosing \(a = 2\) and \(b = 1/3\).
  % \blanklines{24}

  \clearpage
  Very few math objects or problems are amendable to an \textbf{universal method}. However, exponential-like functions are. We saw the following pattern in the review of exponential functions, logarithmic differentiation and l'H\^opital's rule.

  To deal with functions or equations in the form \(y = f(x)^{g(x)}\), regardless the context (basic algebraic tasks, limit, differentiation, integration, etc.), rewrite it as a down-to-earth composite function by changing base to \(e\).
  \blanklines{20}

  Alternatively, we can take \(\ln( \cdots{} )\) on both sides\footnote{Our textbook does this most (probably all) of the time.}.
  \blanklines{20}

  \clearpage

  Last, but certainly not least, here are some extremely general \cancel{exam-writing advices} problem-solving strategies.

  \begin{enumerate}
    \item Remember \hlmain{precise the wording} of definitions and theorems!
    \item When in doubt, draw a picture! We talked about so so so so many topics with associated geometric pictures.
    \item Before going into any technical calculation, write down the exact expression you are trying to figure out.
    \item Let notations guide you. 
    \item Keep your writings clean, so it's easy to backtrack.
    \item Not sure about a multiple choice problem? How about start with the answers and see which one fit the question best?
    \item Don't guess. Separate facts from ``I wish this is true.'' Use facts to \hlmain{deduce} a solution.
  \end{enumerate}

\end{document}
