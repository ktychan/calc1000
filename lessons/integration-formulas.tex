%! TeX program = lualatex
\documentclass[../main.tex]{subfiles}
\begin{document} \section{Integration Formulas and Properties} \label{lession:integration-formulas}
  Every \hlmain{integration} formula comes from a \hlsupp{differentiation} formula.

  \[
    \begin{array}{rcclcrclllcl} 
      \multicolumn{5}{c}{\text{\textbf{\hlsupp{Differentiation} Formulas}}} & \multicolumn{7}{c}{\text{\textbf{\hlmain{Integration} Formulas}}} \\
      \frac{d}{dx} & {\color{supp} cx} &=& \phantom{\color{main} c}
                   & \longleftrightarrow 
                   & \int & \phantom{\color{main} c} & dx &=& \phantom{\color{supp} cx} 
                   &+& (\text{constant}) \\[2ex]
      \frac{d}{dx} & {\color{supp} x^{n+1}  } &=& \phantom{\color{main} (n+1) x^{n}}
                   & \overset{\color{warn} n \ne -1}{\longleftrightarrow}
                   & \int & \phantom{\color{main} x^{n}} & dx &=& \phantom{\color{supp} \frac{x^{n+1}}{n+1} } 
                   &+& (\text{constant}) \\[2ex]
      \frac{d}{dx} & {\color{supp} \ln|x| } &=& \phantom{\color{main} x^{-1}}
                   & \longleftrightarrow 
                   & \int & \phantom{\color{main} x^{-1}} & dx &=& \phantom{\color{supp} \ln|x|} 
                   &+& (\text{constant}) \\[2ex]
      \frac{d}{dx} & {\color{supp} e^{x}} &=& \phantom{\color{main} e^{x}}
                   & \longleftrightarrow
                   & \int & \phantom{\color{main} e^{x}      } & dx &=& \phantom{\color{supp} e^{x}} 
                   &+& (\text{constant}) \\[2ex]
      \frac{d}{dx} & {\color{supp} \sin(x)} &=& \phantom{\color{main} \cos(x)}
                   & \longleftrightarrow 
                   & \int & \phantom{\color{main} \cos(x)} & dx &=& \phantom{\color{supp} \sin(x)} 
                   &+& (\text{constant}) \\[2ex]
      \frac{d}{dx} & {\color{supp} \cos(x)} &=& \phantom{\color{main} -\sin(x)}
                   & \longleftrightarrow 
                   & \int & \phantom{\color{main} \sin(x)} & dx &=& \phantom{\color{supp} -\cos(x)} 
                   &+& (\text{constant}) \\[2ex]
      \frac{d}{dx} & {\color{supp} \tan(x)} &=& \phantom{\color{main} \sec^{2}(x)}
                   & \longleftrightarrow 
                   & \int & \phantom{\color{main} \sec^{2}(x)} & dx &=& \phantom{\color{supp} \tan(x)} 
                   &+& (\text{constant}) \\[2ex]
      \frac{d}{dx} & {\color{supp} \sec(x)} &=& \phantom{\color{main} \tan(x)\sec(x)}
                   & \longleftrightarrow 
                   & \int & \phantom{\color{main} \tan(x)\sec(x)} & dx &=& \phantom{\color{supp} \sec(x)} 
                   &+& (\text{constant}) \\[2ex]
      \frac{d}{dx} & {\color{supp} \arctan(x)} &=& \phantom{\color{main} \frac{1}{x^{2}+1}}
                   & \longleftrightarrow 
                   & \int & \phantom{\color{main} \frac{1}{x^{2} + 1}} & dx &=& \phantom{\color{supp} \arctan(x)} 
                   &+& (\text{constant}) \\[2ex]
      \frac{d}{dx} & {\color{supp} \arcsin(x)} &=& \phantom{\color{main} \frac{1}{\sqrt{1 - x^{2}}}}
                   & \longleftrightarrow 
                   & \int & \phantom{\color{main} \frac{1}{\sqrt{1 - x^{2}}}} & dx &=& \phantom{\color{supp} \arcsin(x)} 
                   &+& (\text{constant})
    \end{array}
  \]
  {\footnotesize \faExclamationTriangle{} The restriction \hlwarn{\(n \ne -1\)} on the second formula \hlwarn{applies to the entire line} because \(x^{-1}\) cannot be the derivative of any power function. It is also \hlwarn{not true} that \(\textstyle \int x^{-1} \;dx = \frac{x^{-1 + 1}}{-1 + 1}\) (division by \(0\)). The integral \(\textstyle \int x^{-1} \;dx\) \emph{does} exist (see the third formula).}

  \textbf{Properties of integrals} (all three properties also work for definite integrals).
  \begin{align*}
    \int c f(x) \;dx 
    &= \phantom{c \int f(x) \;dx} && \text{(constant multiple)}\\
    \int f(x) + g(x) \;dx 
    &= \phantom{\int f(x) \;dx + \int g(x) \;dx} && \text{(sum)}\\
    \int f(x) - g(x) \;dx 
    &= \phantom{\int f(x) \;dx - \int g(x) \;dx} && \text{(difference)}
  \end{align*}

  \textbf{Additional properties of definite integrals only}.
  \[
    \int_{a}^{b} f(x) \;dx = \phantom{- \int_{b}^{a} f(x) \;dx}
    \quad\text{and}\quad
    \int_{a}^{b} f(x) \;dx = \phantom{\int_{a}^{c} f(x) \;dx + \int_{c}^{b} f(x) \;dx.}
  \]
  For the last property, the number \(c\) can be \underline{\hspace{4in}}
  \clearpage

  Every \hlmain{integration} formula comes from a \hlsupp{differentiation} formula.

  \[
    \begin{array}{rcclcrclllcl} 
      \multicolumn{5}{c}{\text{\textbf{\hlsupp{Differentiation} Formulas}}} & \multicolumn{7}{c}{\text{\textbf{\hlmain{Integration} Formulas}}} \\
      \frac{d}{dx} & {\color{supp} cx} &=& {\color{main} c}
                   & \longleftrightarrow 
                   & \int & {\color{main} c} & dx &=& {\color{supp} cx} 
                   &+& (\text{constant}) \\[2ex]
      \frac{d}{dx} & {\color{supp} x^{n+1}  } &=& {\color{main} (n+1) x^{n}}
                   & \overset{\color{warn} n \ne -1}{\longleftrightarrow}
                   & \int & {\color{main} x^{n}} & dx &=& {\color{supp} \frac{x^{n+1}}{n+1} } 
                   &+& (\text{constant}) \\[2ex]
      \frac{d}{dx} & {\color{supp} \ln|x| } &=& {\color{main} x^{-1}}
                   & \longleftrightarrow 
                   & \int & {\color{main} x^{-1}} & dx &=& {\color{supp} \ln|x|} 
                   &+& (\text{constant}) \\[2ex]
      \frac{d}{dx} & {\color{supp} e^{x}} &=& {\color{main} e^{x}}
                   & \longleftrightarrow
                   & \int & {\color{main} e^{x}      } & dx &=& {\color{supp} e^{x}} 
                   &+& (\text{constant}) \\[2ex]
      \frac{d}{dx} & {\color{supp} \sin(x)} &=& {\color{main} \cos(x)}
                   & \longleftrightarrow 
                   & \int & {\color{main} \cos(x)} & dx &=& {\color{supp} \sin(x)} 
                   &+& (\text{constant}) \\[2ex]
      \frac{d}{dx} & {\color{supp} \cos(x)} &=& {\color{main} -\sin(x)}
                   & \longleftrightarrow 
                   & \int & {\color{main} \sin(x)} & dx &=& {\color{supp} -\cos(x)} 
                   &+& (\text{constant}) \\[2ex]
      \frac{d}{dx} & {\color{supp} \tan(x)} &=& {\color{main} \sec^{2}(x)}
                   & \longleftrightarrow 
                   & \int & {\color{main} \sec^{2}(x)} & dx &=& {\color{supp} \tan(x)} 
                   &+& (\text{constant}) \\[2ex]
      \frac{d}{dx} & {\color{supp} \sec(x)} &=& {\color{main} \tan(x)\sec(x)}
                   & \longleftrightarrow 
                   & \int & {\color{main} \tan(x)\sec(x)} & dx &=& {\color{supp} \sec(x)} 
                   &+& (\text{constant}) \\[2ex]
      \frac{d}{dx} & {\color{supp} \arctan(x)} &=& {\color{main} \frac{1}{x^{2}+1}}
                   & \longleftrightarrow 
                   & \int & {\color{main} \frac{1}{x^{2} + 1}} & dx &=& {\color{supp} \arctan(x)} 
                   &+& (\text{constant}) \\[2ex]
      \frac{d}{dx} & {\color{supp} \arcsin(x)} &=& {\color{main} \frac{1}{\sqrt{1 - x^{2}}}}
                   & \longleftrightarrow 
                   & \int & {\color{main} \frac{1}{\sqrt{1 - x^{2}}}} & dx &=& {\color{supp} \arcsin(x)} 
                   &+& (\text{constant})
    \end{array}
  \]
  {\footnotesize \faExclamationTriangle{} The restriction \hlwarn{\(n \ne -1\)} on the second formula \hlwarn{applies to the entire line} because \(x^{-1}\) cannot be the derivative of any power function. It is also \hlwarn{not true} that \(\textstyle \int x^{-1} \;dx = \frac{x^{-1 + 1}}{-1 + 1}\) (division by \(0\)). The integral \(\textstyle \int x^{-1} \;dx\) \emph{does} exist (see the third formula).}

  \textbf{Properties of integrals} (all three properties also work for definite integrals).
  \begin{align*}
    \int c f(x) \;dx 
    &= c \int f(x) \;dx && \text{(constant multiple)}\\
    \int f(x) + g(x) \;dx 
    &= \int f(x) \;dx + \int g(x) \;dx && \text{(sum)}\\
    \int f(x) - g(x) \;dx 
    &= \int f(x) \;dx - \int g(x) \;dx && \text{(difference)}
  \end{align*}

  \textbf{Additional properties of definite integrals only}.
  \[
    \int_{a}^{b} f(x) \;dx = - \int_{b}^{a} f(x) \;dx 
    \quad\text{and}\quad
    \int_{a}^{b} f(x) \;dx = \int_{a}^{c} f(x) \;dx + \int_{c}^{b} f(x) \;dx.
  \]
  For the last property, the number \(c\) can be any number, even outside \([a,b]\).



\end{document}
