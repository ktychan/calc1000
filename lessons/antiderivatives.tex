%! TeX program = lualatex
\documentclass[../main.tex]{subfiles}
\begin{document}
\begin{lesson}{Antiderivatives}
  We set up terminologies and notations for integration.

  \begin{mdframed}[style=withref-compact]
    If \(f(x)\) and \(F(x)\) are related by
    \[
      f(x) = F'(x) \text{ for every } x \text{ in the domain of } f(x),
    \]
    then we \emph{say} \(F(x)\) is \emph{an} antiderivative of \(f(x)\).
  \end{mdframed}

  Let's make sense of the above definition.
  \begin{example}
    Consider \(f(x) = 2xe^{x^{2}}\) and \(g(x) = e^{x^{2}}\). Is \(f\) an antiderivative of \(g\)? Is \(g\) an antiderivative of \(f\)?
    \blanklines{10}
  \end{example}

  \begin{example}
    Is \(\ln(x)\) an antiderivative of \(1/x\)? Justify your answer using the definition above.

    \blanklines{10}
  \end{example}

  \begin{example}
    Consider \(f(x) = 3x^{2}\).

    \begin{enumerate}
      \item Find an antiderivative of \(3x^{2}\) using the definition.
      \item List infinitely many antiderivatives of \(3x^{2}\).
    \end{enumerate}
    \blanklines{10}
  \end{example}
  \clearpage

  \begin{mdframed}[style=withref-compact]
    The \hlmain{indefinite integral} of \(f\) is the most general antiderivative of \(f(x)\), denoted by 
    \[
      \int f(x) \;dx.
    \]
    If \(F\) is \emph{an} antiderivative of \(f\), then we get an explicit expression \(\int f(x) \;dx = F(x) + C\) where \(C\) represents a general but unknown constant.

    \textbook{Definition, page 487}
  \end{mdframed}
  The definition says if we have expressions for \(F(x)\) and \(f(x)\) and want to know if \(F(x) = \int f(x) \;dx\), then we must check two things:
  \blanklines{10}

  Let's make sure we can parse relevant notations. 
  \begin{example}
    Which of the following is (or are) true statement(s).
    \begin{enumerate}[label=(\alph*)]
      \item \(\int e^{x} dx = e^{x}\).
      \item \(\int e^{x} dx = e^{x} + C\).
      \item \(\int e^{x} dx = e^{x} + 1 + C\).
      \item \(\int e^{x} dx = e^{x} + 2C\).
    \end{enumerate}
    \blanklines{10}
  \end{example}

  % \begin{example}[Graphical relations among antiderivatives]
  %   Suppose \(F(x)\) is \emph{an} antiderivative of a differentiable function \(f(x)\). The graph of \(F(x)\) is sketched below. 
  %
  %   Identify another antiderivative of \(f\). Is it \(G(x)\)? Is it \(H(x)\)?
  %   \begin{center}
  %     \begin{tikzpicture}
  %       \begin{axis}[enlargelimits, grid=both, minor tick num=1, ymin=-3, ymax=2, xmin=-2, xmax=3]
  %         \addplot[very thick, domain=-1:2, smooth, samples=100, main] {x*(x-2)*(x+1)};
  %         \node[below, main] at (axis cs:-1,0) {\(F(x)\)};
  %         \addplot[very thick, domain=-1:2, smooth, samples=100, supp, dotted] {x*(x-2)*(x+1) + 1};
  %         \node[above, supp] at (axis cs:2,1) {\(G(x)\)};
  %         \addplot[very thick, domain=-0:3, smooth, samples=100, blue, dashed] {(x-1)*(x-3)*(x)};
  %         \node[above, blue] at (axis cs:3,0) {\(H(x)\)};
  %       \end{axis}
  %     \end{tikzpicture}
  %   \end{center}
  % \end{example}

  % \begin{example}
  %   Suppose \(F(x)\) and \(G(x)\) are antiderivatives of a function \(f(x)\) on an interval \(I\). Which of the following is/are true statement(s)?
  %
  %   \begin{enumerate}[label=(\alph*)]
  %     \item \(F(x) = G(x)\) on \(I\).
  %     \item \(F'(x) = G'(x)\) on \(I\).
  %     \item \(\int F'(x) \;dx = \int G'(x) \;dx\) on \(I\).
  %     \item \(F'(x) = f(x)\) on \(I\).
  %     \item \(G'(x) = f(x)\) on \(I\).
  %   \end{enumerate}
  % \end{example}

  \clearpage
  Many property of indefinite integrals comes \emph{directly} from properties of derivatives.

  \begin{mdframed}[style=withref-compact]
    \begin{align*}
      \int c f(x) \;dx 
    &= c \int f(x) \;dx && \text{(constant multiple)}\\
    \int f(x) + g(x) \;dx 
    &= \int f(x) \;dx + \int g(x) \;dx && \text{(sum)}\\
    \int f(x) - g(x) \;dx 
    &= \int f(x) \;dx - \int g(x) \;dx && \text{(difference)}
    \end{align*}

    \textbook{Theorem~4.16, page 491}
  \end{mdframed}

  \begin{example}
    Suppose \(f(x) = \sin(x) - \frac{1}{x} + \sec^{2}(\pi/4)\). Evaluate \(\int f(x) \;dx\). 

    Suppose \(F'(x) = f(x)\) and \(F(1) = 3\). Find an explicit expression for \(F(x)\).

    \blanklines{27}
  \end{example}

  \hlmain{Insights}. The notions of antiderivatives and indefinite integrals is literally just a fancy names and not much else. ``Evaluate \(\textstyle \int f(x) \;dx\)'' means ``find the most general antiderivative of \(f(x)\).'' How do we do that? We find \emph{an} antiderivative of \(f(x)\) by \hlmain{recognizing derivatives} and then add ``\(+ C\)'' at the end.

  Recognizing derivatives typically requires a change of thinking pattern. Instead of applying differentiation rules, we need to read differentiation formulas \emph{backwards} and ask ``is a given \(f(x)\) the derivative of something I know?'' Adapt your thinking as quickly as you can because this skill is essential when we learn to use the substitution rule in two weeks.

  \clearpage
\end{lesson}
\end{document}
