%! TeX program = lualatex
\documentclass[../main.tex]{subfiles}
\begin{document}
\begin{lesson}{Continuity}

  Functions that model nature typically don't ``break.'' Conversely, ``breaks'' in functions are really horrible to work with. We make such notion precise. 

  \begin{mdframed}[style=withref]
    \textbf{Definition}. A function \(f(x)\) is \emph{continuous at a number \(a\)} if 
    \[
      {\lim_{x \to a} f(x) = f(a).}
    \]
    \textbook{Page 180}
  \end{mdframed}

  To conclude that \(f(x)\) is continuous at a number \(a\), we must successfully complete all three tasks:
  \begin{enumerate}[label=(C\arabic*), itemsep={2ex}]
    \item Calculate \underline{\hspace{3in}}.
    \item Calculate \underline{\hspace{3in}}.
    \item Verify \underline{\hspace{3.23in}}.
  \end{enumerate}

  \begin{example}
    Is \(f(x) = \sqrt{x}\) continuous at \(1\)? 

    \blanklines{10}
  \end{example}

  \begin{example}
    Is \(f(x) = \frac{x^{2}-1}{x+1}\) continuous at \(-1\)? 

    \blanklines{10}
  \end{example}

  \begin{mdframed}[style=withref-compact]
    Polynomials, rational functions and trigonometric functions are all continuous at every number \underline{\hspace{2in}}. 
    
    \textbook{Theorem~2.8 on page 183 and Theorem~2.10 on page 188}
  \end{mdframed}
  \clearpage


  If \(f\) is \emph{not} continuous at \(a\), then we say
  \begin{center}
    ``\(f\) is \hlmain{discontinuous} at \(a\)'' \quad or \quad ``\(f\) has a \hlmain{discontinuity} at \(a\).''
  \end{center}

  What do discontinuities look like? There are three \emph{types} of discontinuities.

  \begin{figure}[h]  % [h] for here, [ht] for here top, [hb] for here bottom
    \centering
    \includegraphics{../standalones/build/plot_discontinuity_examples}
    \label{fig:discontinuities}
  \end{figure}

  \begin{enumerate}[itemsep={3ex}]
    \item If \underline{\hspace{3in}}, then \(f\) has a \hlmain{removable} discontinuity at \(a\).
    \item If \underline{\hspace{3in}}, then \(f\) has a \hlmain{jump} discontinuity at \(a\).
    \item If \underline{\hspace{3in}}, then \(f\) has an \hlmain{infinite} discontinuity at \(a\).
  \end{enumerate}

  Discontinuities naturally ``break'' a function into pieces.
  A function is called \hlmain{continuous over an interval} if it is continuous at every number in that interval. At the endpoints of each interval, we only require a function to be continuous from one-side.

  We say a function \(f\) is 
  \begin{itemize}
    \item \hlmain{continuous from the \underline{\hspace{1in}}} at a number \(a\) if \(\lim_{{x \to a^{-}}} f(x) = f(a)\).
    \item \hlmain{continuous from the \underline{\hspace{1in}}} at a number \(a\) if \(\lim_{{x \to a^{+}}} f(x) = f(a)\).
  \end{itemize}

  Intuitively, if a pen can trace the graph of a function without leaving the paper over an interval \(I\), then the function is continuous over \(I\).

  For example, 
  \begin{itemize}[itemsep={3ex}]
    \item \(f_{1}(x)\) is continuous over \underline{\hspace{2in}}.
    \item \(f_{2}(x)\) is continuous over \underline{\hspace{2in}}.
    \item \(f_{3}(x)\) is continuous over \underline{\hspace{2in}}.
  \end{itemize}
  \vfill{}
  \clearpage

  To \hlmain{classify} a discontinuity is to determine if the discontinuity is removable, a jump or an infinite discontinuity.

  \faComments{} How can we find discontinuities?
  \blanklines{15}

  \begin{example}
    Classify all discontinuities of \(f(x) = \frac{x^{2}-x}{(x-1)(x+2)}\), if any.  Find the intervals over which \(f(x)\) is continuous.

    \blanklines{15}
  \end{example}

  \begin{example}
    Classify all discontinuities of \(f(x) = \begin{cases} \sqrt{x} - 2,& x \ge 4 \\ 3x - 1, &x < 4\end{cases}\), if any.  Find the intervals over which \(f(x)\) is continuous.

    \blanklines{10}
  \end{example}
  \clearpage

  \begin{example}
    Is \(f(x) = \frac{x}{\sin(x)}\) continuous over \((-2\pi, 2\pi)\)? If not, find the intervals over which \(f(x)\) is continuous.
    \blanklines{15}
  \end{example}

  
  Continuities and limits are closely related as we can see from the definition of continuity: Whenever a function \(f(x)\) is {continuous} at \(a\), then we can evaluate \(\lim_{x \to a} f(x)\) by direct substitution. 

  But there is more!

  \begin{mdframed}[style=withref]
    \textbf{Theorem}. If \(f\) is continuous at \(b\) and \(\lim_{x \to a} g(x) = b\), then 
    \[
      {\color{main} \lim_{x \to a}} {\color{attn}f({\color{black}g(x)})} = {\color{attn}f \left( {\color{main} \lim_{x \to a}} {\color{black} g(x) }\right)}.
    \]

    \textbook{Theorem~2.9 on page 187}
  \end{mdframed}

  Notice the ``inside function'' \(g(x)\) is allowed to be discontinuous at \(a\).

  \begin{example}
    Evaluate \(\lim_{x \to 2} \sin( - x^{3} + 7  )\).
  \end{example}
  \clearpage

  \begin{example}
    Suppose \(f\) is continuous, \(f(1) = 3\) and \(f(2) = 5\).  Find \(\lim_{x \to \pi/4} f( \tan(x) + 1 )\).

    \blanklines{10}
  \end{example}

  \begin{example}
    Find the mistake(s) in the following \emph{incorrect} reasoning about the continuity of \[f(x) = \frac{(3x-1)(x-2)}{(x-2)}.\]
    \bigskip

    \begin{quote}
      \itshape
      We cancel the common factor to get
      \[
        f(x) = \frac{(3x - 1)\cancel{(x - 2)}}{\cancel{(x-2)}} = 3x - 1.
      \]
      Because \(3x - 1\) is a polynomial, we can conclude \(f(x)\) is continuous at every number, including \(x=2\).
    \end{quote}

    \blanklines{15}
  \end{example}
\end{lesson}
\end{document}

