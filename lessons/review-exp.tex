%! TeX program = lualatex
\documentclass[../main.tex]{subfiles}
\begin{document} \section{Review of exponential functions and logarithms}
A function \(b^{x}\), where \(b\) is a constant so that \(b > 0\) and \(b \ne 1\) is called an \hlmain{exponential function with base \(b\)} and exponent \(x\). Within the context of our textbook, every exponential function \(b^{x}\) has an \hlsupp{inverse} called the \hlmain{logarithmic function with base \(b\)}, denoted by \(\log_{b}(x)\). \newline
{\scriptsize Some authors allow \(1^{x}\) to be an exponential function. However, \(1^{x}\) is a constant function and does not have an inverse.}

\begin{minipage}{0.6\textwidth}
  \centering
  \begin{tabular}{l|p{1in}|p{1in}}
    & Domain & Range \\\midrule
    \(b^{x}\) & & \\[2ex]\midrule
    \(\log_{b}(x)\) & & \\[2ex]
  \end{tabular}
\end{minipage}
\begin{minipage}{0.4\textwidth}
  \centering
  \includestandalone[page=1]{../standalones/plot-exp-log}
\end{minipage}

Let's quickly recall exponents rules. \(e^{a}e^{b} = \underline{\hspace{1in}}\) and \((e^{a})^{b} = \underline{\hspace{1in}}\) for any \(a,b\).
\blanklines{5}

The number \(e\) is an important constant whose definition is \(e = \lim_{x \to \infty} \left( 1 + \frac{1}{x} \right)^{x}\). Whenever you see the limit on the right-hand side, you can write down \(e\) as the answer with justification ``by definition.'' 
\bigskip

The function \(e^{x}\) is important for a variety of reasons. One of them is \((e^{x})' = e^{x}\). 
\begin{mdframed}[style=simple-compact]
  The \emph{natural} exponential function is \(e^{x}\) and the \emph{natural} logarithm is \(\ln(x) = \log_{e}(x)\).
  \begin{equation} \label{eq:exp-log}
    e^{\ln(x)} = x \text{ on } \underline{\hspace{1in}} \quad\text{and}\quad \ln(e^{x}) = x \text{ on } \underline{\hspace{1in}}.
  \end{equation}

  The goal of this lesson is to consolidate all of our knowledge of exponential and exponential-like functions to Equation~\eqref{eq:exp-log} and the ``\hlmain{change base to \(e\)}'' trick
  \begin{equation} \label{eq:exp-change-of-base}
    b = \underline{\hspace{1in}} \quad\text{and}\quad b^{x} = \underline{\hspace{1in}}, \quad\text{ but need } b > 0.
  \end{equation}
\end{mdframed}
\faStar{} Equation~\eqref{eq:exp-change-of-base} offers a \hlattn{universal method}, called \hlmain{change base to \(e\)}, to deal with all exponential or exponential-like functions (such as \(x^{x}\)). 

\blanklines{5}

\clearpage

% \begin{example} \label{ex:change-base-to-e}
%   Let \(b\) be a constant so that \(b > 0\) and \(b \ne 1\).  Change \(b^{x}\) and \(\log_{b}(x)\) to base \(e\). 
%
%   \blanklines{15}
% \end{example}

\begin{example}
  Solve the equation \(\frac{e^{2x}}{2^{x-1}} = 3\). 

  \blanklines{20}
\end{example}

\begin{example}
  Simplify \(\frac{\log_{6}(3)}{\log_{7}(2)}\).

  \blanklines{10}
\end{example}

\begin{exercise}
  Solve the equation \(2^{3x - 1} = e\).

  \blanklines{10}
\end{exercise}

\begin{exercise}
  Solve the equation \(\ln \sqrt{x - 1} = 0\). 

  \blanklines{5}
\end{exercise}


\clearpage

Formulas in Equation~\eqref{eq:log-formulas} can be deduced from Equation~\eqref{eq:exp-change-of-base} and can help solve problems \emph{quickly}. 
\begin{equation} \label{eq:log-formulas}
  \log_{b}(a) = \frac{\ln(a)}{\ln(b)},
  \quad
  \ln(a^{x}) = x \ln(a),
  \quad
  \ln(ab) = \ln(a) + \ln(b),
  \quad
  \ln(a/b) = \ln(a) - \ln(b).
\end{equation}
\blanklines{5}

We demonstrate solving problems \hlmain{with and without} Equation~\eqref{eq:log-formulas}.
\begin{example}
  Solve for \(y\) given that \(\log_{2}\left(\frac{y}{\sqrt{x}}\right) = 3\).

  \blanklines{20}
\end{example}

\begin{example}
  Simplify \(e^{\log_{7}(x)}\).

  \blanklines{15}
\end{example}
\clearpage

These exercises help you get a feel for the usefulness of change-base-to-\(e\), i.e., Equation~\eqref{eq:exp-change-of-base}. 

\begin{exercise} \label{ex:x-to-x}
  Rewrite the function \(x^{x}\) so that it is in base \(e\), i.e., \hlmain{change base to \(e\)}. 

  \blanklines{5}

  (Preview). Find the derivative of \(x^{x}\). Requires the chain rule.

  \blanklines{5}
\end{exercise}

\begin{exercise}
  Rewrite the function \(\left(1 + \frac{1}{2x}\right)^{3x}\) so that it is in base \(e\), i.e., \hlmain{change base to \(e\)}. 

  \blanklines{5}

  (Preview). Evaluate \(\lim_{x \to \infty} \left(1 + \frac{1}{2x}\right)^{3x}\). Requires l'H\^opital's rule.

  \blanklines{5}
\end{exercise}

\begin{exercise}
  Rewrite the function \(\left(\frac{1}{x+2}\right)^{\sin(x)}\) so that it is in base \(e\), i.e., \hlmain{change base to \(e\)}. 

  \blanklines{5}

  (Preview) Find the derivative of \(\left(\frac{1}{x+2}\right)^{\sin(x)}\). 

  \blanklines{5}
\end{exercise}
\clearpage

\begin{exercise}[Summary of exponential and logarithms rules]
  First, remember the universal method, \hlmain{change base to \(e\)}, for dealing with exponential or exponential-like functions.
  \[
    \begin{array}{rclcrcl}
      b^{x} &=& \underline{\hspace{1in}} 
            &\hspace{1em} & 
      \log_{b}(x) &=& \underline{\hspace{1in}} \\[2ex]
    \end{array}
  \]

  Once everything is in base \(e\), then use the following rules.
  \[
    \begin{array}{rclcrclcrcl}
      e^{\ln(x)} &=& \underline{\hspace{1in}} 
                 &\hspace{1em} & 
      \ln(e^{x}) &=& \underline{\hspace{1in}} 
                 &\hspace{1em} & 
      \ln(b^{x}) &=& \underline{\hspace{1in}} 
      \\[2ex]
      e^{a} e^{b} &=& \underline{\hspace{1in}}
                  &\hspace{1em} & 
      \ln(ab) &=& \underline{\hspace{1in}} \\[2ex]
      \frac{e^{a}}{e^{b}} &=& \underline{\hspace{1in}}
                          &\hspace{1em} & 
      \ln(a/b) &=& \underline{\hspace{1in}} \\[2ex]
    \end{array}
  \]
\end{exercise}
\end{document}

