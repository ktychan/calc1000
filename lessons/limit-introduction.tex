%! TeX program = lualatex
\documentclass[../main.tex]{subfiles}
\begin{document} \section{The limit of a function (at a number)}
A limit describes the pattern of a function as its \emph{independent variable} moves closer to a fixed number.

\begin{mdframed}[style=withref-compact]
  Let \(f(x)\) be a function defined at all values in an open interval containing \(a\), with the possible exception of \(a\) itself, and \(L\) be a real number. 

  If all values of the function \(f(x)\) approach the real number \(L\) as the values of \(x\) (\(\ne a\)) approach the number \(a\), then we say that \hlmain{the limit of \(f(x)\), as \(x\) approaches \(a\), is \(L\)} and denote
  \[
    L = \lim_{x \to a} f(x).
  \]

  If such number \(L\) cannot be found, then we say \hlwarn{\(\lim_{x \to a} f(x)\) does not exist}.

  \textbook{(The Intuitive) definition (of two-sided limits) on page 136}
\end{mdframed}

\faLightbulb{} To estimate \(\lim_{x \to a} f(x)\), we move \(x\) closer and closer to the constant \(a\) \underline{\hspace{6cm}} and look for a pattern: Does \(f(x)\) eventually get \underline{\hspace{3cm}} closer to some constant \(L\)? If such \(L\) exist, then \(L = \lim_{x \to a} f(x)\); otherwise, the limit \(\lim_{x \to a}f(x)\) does not exist.

\begin{example}
  Consider the following functions.

  \centerline{
    \includestandalone[page=1]{../standalones/plot-limit-intro}
    \qquad
    \includestandalone[page=2]{../standalones/plot-limit-intro}
    \qquad
    \includestandalone[page=3]{../standalones/plot-limit-intro}
  }

  Estimate the following limits graphically:
  \begin{align*}
    \lim_{x \to 1} g_{1}(x)  
      &= \underline{\hspace{2cm}}  
      & \lim_{x \to 1} g_{2}(x) 
      &= \underline{\hspace{2cm}}
      & \lim_{x \to 1} g_{3}(x) 
      &= \underline{\hspace{2cm}} \\[2ex]
      g_{1}(1)  
      &= \underline{\hspace{2cm}}  
      & g_{2}(1) 
      &= \underline{\hspace{2cm}}
      & g_{3}(1) 
      &= \underline{\hspace{2cm}}
  \end{align*}
\end{example}

\faComments{} Suppose we have a function \(f(x)\), and we know \(\lim_{x \to a} f(x)\) exists. What do we know about \(f(a)\)?
\blanklines{5}
\clearpage

To evaluate \(\lim_{x \to a} f(x)\) we have to approach the constant \(a\) \hlwarn{from both sides}, why not just look at it \hlsupp{one side at a time}?  Let's define \hlmain{one-sided limits}.
\begin{itemize}
  \item If we can make \(f(x)\) arbitrarily close to a constant \(L\) by moving \(x\) closer to \(a\) \hlmain{from the left}, then we say \(\lim_{x \to a^{-}} f(x) = L\); otherwise, \(\lim_{x \to a^{-}} f(x)\) does not exist.

  \item If we can make \(f(x)\) arbitrarily close to a constant \(L\) by moving \(x\) closer to \(a\) \hlmain{from the right}, then we say \(\lim_{x \to a^{+}} f(x) = L\); otherwise, \(\lim_{x \to a^{+}} f(x)\) does not exist.
\end{itemize}

\begin{example}
  Let's look at \(g_{1}(x)\) and \(g_{2}(x)\) again.

  \begin{center}
    \includestandalone[page=1]{../standalones/plot-limit-intro}
    \qquad
    \includestandalone[page=2]{../standalones/plot-limit-intro}
  \end{center}

  Estimate the following one-sided limits.
  \begin{align*}
    \lim_{x \to 1^{-}} g_{1}(x) 
      &= \underline{\hspace{2cm}}
      & \lim_{x \to 1^{-}} g_{2}(x) 
      &= \underline{\hspace{2cm}} \\[2ex]
      \lim_{x \to 1^{+}} g_{1}(x) 
      &= \underline{\hspace{2cm}}
      & \lim_{x \to 1^{+}} g_{2}(x) 
      &= \underline{\hspace{2cm}}
  \end{align*}
\end{example}

\begin{mdframed}[style=withref-compact]
  Supppse \(f(x)\) is defined at all values in an open interval containing some constant \(a\) with the possible exception of \(a\) itself, and let \(L\) be a real number. Then
  \[
    \lim_{x \to a} f(x) = L \qquad\text{if and only if}\qquad \lim_{x \to a^{-}} f(x) = L \text{ and } \lim_{x \to a^{+}} f(x) = L.
  \]
  \textbook{Theorem 2.2 on page 145}
\end{mdframed}

\begin{itemize}[topsep={1ex}]
  \item If we know \underline{\hspace{2in}}, then conclude \underline{\hspace{3in}}.
    \vfill{}

  \item If we know \underline{\hspace{2in}}, then conclude \underline{\hspace{3in}}.
    \vfill{}
  \item If we know \underline{\hspace{2in}} or one of them does not exist, then conclude that 
    \vfill{}
\end{itemize}
\clearpage


Textbook Theorem~2.2 shows one way that a limit fails to exists: When its left and right limits don't agree. How else can a limit fail to exist?

\begin{example} \label{ex:limit-sin-1/x}
  Does \(\lim_{x \to 0} \sin \left(\frac{1}{x}\right) \) exist? Support your reasoning using a graph or a table of values.

  \blanklines{5}
\end{example}

\begin{example}
  Come up with a function \(f(x)\) whose \( \lim_{x \to 0^{+}} f(x)\) does not exist. Explain why your answer is correct. 

  \blanklines{5}
\end{example}

A limit can fail to exist if a function ``goes to infinity'' at \(a\). This is called an \hlmain{infinite limit}. 

There are four possibilities.
\begin{center}
  \includestandalone[page=3]{../standalones/plot-infinite-limits}
  \includestandalone[page=4]{../standalones/plot-infinite-limits}
  \includestandalone[page=5]{../standalones/plot-infinite-limits}
  \includestandalone[page=6]{../standalones/plot-infinite-limits}
\end{center}
\begin{align*}
  \lim_{x \to 1^{-}} h_{1}(x) 
    &= \underline{\hspace{1cm}} 
    & \lim_{x \to 1^{-}} h_{2}(x) 
    &= \underline{\hspace{1cm}} 
    & \lim_{x \to 1^{-}} h_{3}(x) 
    &= \underline{\hspace{1cm}} 
    & \lim_{x \to 1^{-}} h_{4}(x) 
    &= \underline{\hspace{1cm}} \\[2ex]
    \lim_{x \to 1^{+}} h_{1}(x) 
    &= \underline{\hspace{1cm}} 
    & \lim_{x \to 1^{+}} h_{2}(x) 
    &= \underline{\hspace{1cm}} 
    & \lim_{x \to 1^{+}} h_{3}(x) 
    &= \underline{\hspace{1cm}} 
    & \lim_{x \to 1^{+}} h_{4}(x) 
    &= \underline{\hspace{1cm}} \\[2ex]
    \lim_{x \to 1} h_{1}(x) 
    &= \underline{\hspace{1cm}} 
    & \lim_{x \to 1} h_{2}(x) 
    &= \underline{\hspace{1cm}} 
    & \lim_{x \to 1} h_{3}(x) 
    &= \underline{\hspace{1cm}} 
    & \lim_{x \to 1} h_{4}(x) 
    &= \underline{\hspace{1cm}}
\end{align*}

A vertical line \(x = a\) is called a \hlmain{vertical asymptote} of \(f(x)\) if
\[
  \lim_{x \to a^{-}} f(x) = \pm \infty \qquad\text{ or }\qquad \lim_{x \to a^{+}} f(x) = \pm \infty.
\]
\blanklines{7}
\clearpage

Theorem~2.3 boils down to just two pictures.
\begin{center}
  \includestandalone[page=1]{../standalones/plot-infinite-limits}
  \qquad
  \includestandalone[page=2]{../standalones/plot-infinite-limits}
\end{center}

\blanklines{25}

  % We will skip Theorem~2.3 in the textbook for now. There is no need to memorize such table. We will learn a very simple technique to identify vertical asymptotes algebraically after we learn limit laws. 
  %
  % Here is the technique for the impatient. We know \(\frac{1}{(x-a)^{n}}\) has a vertical asymptote from our review of sketching graphs. The only thing left is to decide the sign of the infinity in these limits:
  % \[
  %   \lim_{x \to a^{-}} \frac{1}{(x-a)^{n}} = (\hlsupp{\text{pos or neg?}})\,\infty 
  %   \hspace{1in}
  %   \lim_{x \to a^{+}} \frac{1}{(x-a)^{n}} = (\hlsupp{\text{pos or neg?}})\,\infty.
  % \]
  %
  % \begin{itemize}
  %   \item For \(\lim_{x \to a^{-}}\), evaluate the function at a number \(b\) very close to \(a\) but to the left of \(a\) (meaning smaller than \(a\)), say \(b = a - 0.001\). The one-sided limit has the same sign as \(f(b)\).
  %   \item For \(\lim_{x \to a^{+}}\), evaluate the function at a number \(b\) very close to \(a\) but to the right of \(a\) (meaning larger than \(a\)), say \(b = a + 0.001\). The one-sided limit has the same sign as \(f(b)\).
  % \end{itemize}


  % Vertical asymptotes have a very subtle trap!
  % \begin{example}
  %   Let \(f(x) = \frac{6x^{2} + 19 x + 10}{2x + 5}\).  Is \(x = -5/2\) a vertical asymptote of \(f(x)\)?
  % \end{example}

Here is a variation-on-the-theme exercise.
\begin{example}
  A function \(f(t)\) is represented by a table of values below. Estimate \(\lim_{t \to 0} f(x)\).

  \hspace{2in} \includestandalone{../standalones/table_t_squared}
\end{example}

\end{document}
