%! TeX program = lualatex
\documentclass[../main.tex]{subfiles}
\begin{document}
\begin{lesson}{Implicit Differentiation}

  \begin{example}
    Let \(g = g(x)\) be a function. Differentiate \(g(x)^{2}\) with respect to \(x\).
  \end{example}


  The \href{https://www.wolframalpha.com/input?i=implicit+plot+%28P+%2B+1%2FV%5E2%29%28V+-+1%29+%3D+1}{equation} relating the pressure \(P\) and the volume \(V\) of a made-up gas is
    \[
      \left( P + \frac{1}{V^{2}} \right)(V - 1) = 1.
    \]
    How can we mathematically \emph{formulate} and \emph{make sense} of the question 
    \begin{center}
      ``\emph{Find the rate of change of volume with respect to pressure?}''
    \end{center}

    \begin{mdframed}[style=simple]
      The \emph{method of implicit differentiation} says to find \hlattn{the slope \(\frac{dy}{dx}\)} at \hlmain{a point \((x,y)\)} on the curve defined by an equation in which \hlattn{both  \(x,y\) are variables}, differentiate both sides of the equation \hlattn{with respect to \(x\)}, using the chain rule to differentiate \hlattn{expressions in \(y\)}, then finally solve for \hlattn{\(\frac{dy}{dx}\)}.
    \end{mdframed}

    \begin{example}
      Use implicit differentiation to find \(dy/dx\) given \(x^{2} + y^{2} = 25\). 
    \end{example}


    \begin{example}
      Find the slope of \(x^{2} + y^{2} = 25\) at the point \((3,4)\).
    \end{example}


    \begin{example}
      Find \(x'\) given \(\cos(t) = \sin(x)\). Assume \(x\) is a function of \(t\).
    \end{example}


    \begin{example}
      Consider the following attempt to find functions implicitly defined by an equation.
      \begin{align*}
    && \sin(y^{2}) + \frac{x^{2}}{3} &= 1 \\[3ex]
    &\implies& \sin(y^{2}) &= 1 - \frac{x^{2}}{3} &&\text{(add \(-x^{3}/3\) to both sides)}\\[3ex]
    &\implies& y^{2} &= \arcsin\left( 1 - \frac{x^{2}}{3} \right) &&\text{(apply \(\arcsin\) to both sides)} \\[3ex]
    &\implies& y &= \pm \sqrt{\arcsin\left(1 - \frac{x^{2}}{3}\right)}.
      \end{align*}

      If the above calculations were correct, then the graphs of \(y = \pm \sqrt{\arcsin\left(1 - \frac{x^{2}}{3}\right)}\) should decompose the curve \(\sin(y^{2}) + \frac{x^{2}}{3} = 1\).
    \end{example}


    \clearpage

    We are going to study the derivatives of \(\ln(x)\) and \(\arctan(x)\) together. 

    \begin{example}
      Find the derivative of \(\ln(x)\).  
    \end{example}

    \begin{example}
      Find the derivative of \(\arctan(x)\).
    \end{example}

    \faComment{} Why should we think about \texorpdfstring{\(\frac{d}{dx}\ln(x)\)}{ln'(x)} and \texorpdfstring{\(\frac{d}{dx}\arctan(x)\)}{arctan'(x)} together?

  \end{lesson}
\end{document}
