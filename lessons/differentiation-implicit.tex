%! TeX program = lualatex
\documentclass[../main.tex]{subfiles}
\begin{document} \section{Implicit differentiation}
We learn the full picture of implicit differentiation in two steps. Step one is about the procedure. Step two is about the geometry.

\begin{mdframed}[style=simple-compact]
  \textbf{The procedure}. To apply implicit differentiation to an \hlwarn{equation involving \(x,y\)} to find \(\tfrac{dy}{dx}\), we \hlwarn{assume} \hlmain{\(y\) is implicitly defined as a function of \(x\)} and do the following:
  \begin{enumerate}
    \item Differentiate \hlmain{both sides} of the equation \hlmain{with respect to \(x\)}, then
    \item \hlmain{Isolate} for \(\tfrac{dy}{dx}\) by treating \(\tfrac{dy}{dx}\) as a single symbol.
  \end{enumerate}

  The resulting \(\tfrac{dy}{dx}\) is a function in two variables \(x\) and \(y\). 
\end{mdframed}

\begin{example} \label{ex:implicit-circle}
  Use implicit differentiation to find \(\tfrac{dy}{dx}\) given \(x^{2} + y^{2} = 25\). 

  \begin{enumerate}[wide]
    \item Differentiate both sides with respect to \(x\) to get an equation in which \(\tfrac{dy}{dx}\) appears.

      \blanklines{10}

    \item Isolate for \(\tfrac{dy}{dx}\) to get \(\tfrac{dy}{dx} = \left( \cdots \text{ no } \tfrac{dy}{dx} \text{ here } \cdots \right)\).

      \blanklines{10}
  \end{enumerate}
\end{example}

\begin{example}
  Use implicit differentiation to find \(dy/dx\) given \(x = e^{y}\). 

  \blanklines{10}
\end{example}

\clearpage
The Leibniz notation tells us which symbols is the function and which is the variable.
\begin{example}
  Use implicit differentiation to find \(dx/dt\) given \(x^{3} - xt = 1\).  

  \blanklines{20}
\end{example}

Here is a more complicated example. Keep your writing well-organized to avoid little mistakes.
\begin{example}
  Use implicit differentiation to find \(dy/dx\) given \(x^{2}y - \sin(xy) = 0\).

  \blanklines{25}
\end{example}

\clearpage
\textbf{The geometric meaning}. Implicit differentiation allows us to find slopes of lines tangent to \hlwarn{curves} that are not functions. 

Let's continue from Example~\ref{ex:implicit-circle}. Notice \(x^{2} + y^{2} = 25\) defines a curve. If we ``zoom in enough'' at a point on the curve and see a function, then \(\tfrac{dy}{dx}\) is the slope of the tangent line at that point. 

\begin{center}
  \includestandalone[page=1]{../standalones/plot-implicit-differentiation}
  \qquad
  \includestandalone[page=2]{../standalones/plot-implicit-differentiation}
  \qquad
  \includestandalone[page=3]{../standalones/plot-implicit-differentiation}
\end{center}

\begin{example}
  Find an equation of the tangent line to the curve \(x^{2} + y^{2} = 25\) at \((3,4)\).

  \blanklines{20}
\end{example}

\faComment{} Can we use find the slope of the tangent line to the curve \(x^{2} + y^{2} = 25\) at \((5,5)\).

\blanklines{10}

\clearpage
Here is an example tying together the procedure and the geometric meaning of implicit differentiation.
\begin{example}
  Find an equation of the line tangent to the curve \(\tan(xy) = -x\) at the point \((1,3\pi/4)\).

  \blanklines{45}
\end{example}
\end{document}
