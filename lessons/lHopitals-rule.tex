%! TeX program = lualatex
\documentclass[../main.tex]{subfiles}
\begin{document} \section{L'H\^opital's rule}
How can we evaluate \(\lim_{x \to 1} \frac{\ln(x)}{x - 1}\)?

\blanklines{5}

A limit \(\lim_{x \to a} \frac{f(x)}{g(x)}\) is an indeterminate form 
\begin{itemize}
  \item of type \underline{\hspace{1in}} if both \(\lim_{x \to a} f(x)\) and \(\lim_{x \to a} g(x)\) are \underline{\hspace{1in}}.
  \item of type \underline{\hspace{1in}} if both \(\lim_{x \to a} f(x)\) and \(\lim_{x \to a} g(x)\) are \underline{\hspace{1in}}.
\end{itemize}
\hlmain{L'H\^opital's rule} can be used to evaluate indeterminate forms. We demonstrate is application and defer to the textbook (see Theorems 4.12 on page 455 and 4.13 on page 457) for its precise statement.

There are seven types of indeterminate forms grouped (strongly recommended) as such. 
\[
  \frac{0}{0},\; \frac{\infty}{\infty}, \hspace{1in} 0 \cdot \infty,\; \infty - \infty, \hspace{1in} 0^{0}, \; \infty^{0}, \; 1^{\infty}.
\]

\blanklines{3}

Read \(0\) as \underline{\hspace{2in}} and \(\infty\) as \underline{\hspace{2in}}. 

\blanklines{10}

Why aren't these indeterminate forms?
\[
  \frac{\text{constant}}{\infty},\; \frac{\infty}{\text{constant}}, \hspace{0.5in} \infty \cdot \infty,\; (\text{constant}) \cdot (\text{constant}),\; \infty + \infty, \hspace{0.5in} (\text{constant} \ne 1 \text{ or } \infty)^{\infty}.
\]

\blanklines{10}

\begin{example}
  Which of the following is an indeterminate form? If it is, identify its type. 

  \[
    L_{1} = \lim_{x \to 0} x \ln(x),
    \qquad
    L_{2} = \lim_{x \to 1} (1-x)^{x},
    \qquad
    L_{3} = \lim_{x \to -\infty} 2^{2^{x}},
    \qquad
    L_{4} = \lim_{x \to 0} \frac{2x^{2}+1}{\cos(x)}.
  \]

  \blanklines{30}
\end{example}
\clearpage

L'H\^opital's rule directly applies to indeterminate types of \(0/0\) or \(\infty/\infty\).
\begin{example}
  Evaluate \(\lim_{x \to \infty} \frac{x^{2}}{e^{x}}\).

  \blanklines{30}
\end{example}

\begin{example}
  What is wrong with the following calculation?

  \[
    \lim_{x \to 1} \frac{\ln(x)}{2} = \lim_{x \to 1} \frac{\frac{d}{dx} \ln(x)}{\frac{d}{dx} 2} = \lim_{x \to 1} \frac{1/x}{0} = \text{does not exist ?!}
  \]
\end{example}

\clearpage

We can use our algebra skills to rewrite \(0 \cdot \infty\) and \(\infty - \infty\) indeterminate forms as \(0/0\) or \(\infty/\infty\).

\blanklines{10}

\faStar{} Apply the tricks \(x = \frac{1}{1/x}\) and \(f(x) = \frac{1}{1/f(x)}\) to the \(\infty\) in the indeterminate form \(0 \cdot \infty\).

\begin{example}
  Evaluate \(\lim_{x \to 0^{+}} x \ln(x)\).

  \blanklines{15}
\end{example}

\begin{example}
  Evaluate \(\lim_{x \to 1^{+}} \left(\frac{1}{x-1} - \frac{1}{\ln(x)}\right)\).

  \blanklines{20}
\end{example}
\clearpage

Evaluating indeterminate forms \(0^{0}, \infty^{0}, 1^{\infty}\) requires the \emph{universal method} for \(f(x)^{g(x)}\): \faStar{} Change base to \(e\).

\blanklines{10}

\begin{example}
  Evaluate \(\lim_{x \to \infty} x^{1/x}\).

  \blanklines{30}
\end{example}
\clearpage

\begin{example}
  Evaluate \(\lim_{x \to \infty} \left(1 + \frac{1}{4x}\right)^{2x}\).

  \blanklines{30}
\end{example}
\clearpage

\begin{example}
  Evaluate \(\lim_{x \to (\pi/2)^{+}} (x - \pi/2)^{\cos(x)}\).

  \blanklines{50}
\end{example}

\end{document}
