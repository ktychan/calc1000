%! TeX program = lualatex
\documentclass[../main.tex]{subfiles}
\begin{document} \section{Reading week}
The reading week is roughly the midpoint of the term, an opportunity to pause and reflect on our learning. Here are some reflection prompts.

\begin{enumerate}
  \item \hlmain{Do you remember all definitions and theorem statements?} A popular strategy to make sure we don't miss anything is to \hlwarn{make a list} of all statements taught in the course for easy reference. Flash cards are also quite popular.

  \item \hlmain{Do you remember examples?} Definitions and theorem statements can be very abstract and difficult to remember. For each theorem statement in your reference, \hlwarn{list a few examples} to make abstract statements concrete.

  \item \hlmain{Do you allow yourself to forget, recall and grow?} Forgetting an idea fresh off a lecture is a natural part of learning, and growth takes time. Space out your practice sessions and maintain \hlwarn{a positive and growth mindset}.

  \item \hlmain{Do you avoid certain classes of functions?} In calculus, we study limits, derivatives and, later on, integration. These are operations --- verbs of mathematics. At the very minimum, we are expected to perform these operations on different classes of functions correctly (polynomials, rational, algebraic, trigonometric and their inverses, exponential and logarithmic functions). \hlwarn{Practise applying operations to the classes of functions you tend to avoid (or be afraid of)} to improve your basic technical competency.

  \item \hlmain{Can you mix and match techniques to solve complex problems?} Some examples in the lecture notes demonstrate \emph{a single technique} in isolation so that we can understand the basic ideas. Some textbook exercises help us gain fundamental technical competency \hlsupp{one technique at a time}. Make sure to work on complex problems that force us to see the \hlwarn{connections among different ideas and techniques}.

  \item \hlmain{Can you explicitly describe relevant problem-solving principles?} Some examples in the lecture notes discuss problem-solving principles and strategies. If you do a lot of exercises but cannot really describe the purpose of doing these exercises in terms of the ideas taught in class, then you might be slipping into \hlsupp{rote practice}. Knowledge acquired through rote practice gives us a \hlsupp{false sense of competency} and is often challenging to recall when encountering similar problems presented in different forms. Sometimes, less is more. Review the practice problems and explicitly \hlwarn{related them to the ideas taught in class}. 

  \item \hlmain{Are you experiencing information overload?} We go through a lot of examples in the course. If you feel like every example is a new trick waiting to be memorized, then you might be missing the big picture and need to organize your knowledge using a process called \hlwarn{chunking}. Take a look at examples and exercises presented under similar topics and describe for yourself the common problem-solving strategy.
\end{enumerate}

  % We, as learners, can benefit from knowing a little bit of education theory. 
  %
  % \includestandalone[width=\textwidth]{../standalones/revised_bloom}
  %
  % The above is diagram is the 30-second version of a well-known education theory, called the Revised Bloom's Taxonomy (of Educational Objectives) that describes different stages of education . The theory was first developed by Bloom (Bloom, Engelhart, Furst, \& Krathwohl, 1956 and Anderson & Krathwohl, 2001)This theory influences many educators around the world. 

\bigskip
Enjoy the fall colours and a slower week! 

  \end{document}
