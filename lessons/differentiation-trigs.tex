%! TeX program = lualatex
\documentclass[../main.tex]{subfiles}
\begin{document} \section{Derivatives of Trigonometric Functions}
  The derivatives \(\frac{d}{dx}\sin(x) = \cos(x)\) and \(\frac{d}{dx} \cos(x) = -\sin(x)\) comes from these two limits
  \begin{equation} \label{eq:limits-trigs}
    \lim_{h \to 0} \frac{\sin(h) - \sin(0)}{h} = 1
    \qquad\text{and}\qquad
    \lim_{h \to 0} \frac{\cos(h) - \cos(0)}{h} = 0
  \end{equation}
  \begin{center}
    \begin{tikzpicture}
      \tikzstyle{every node}=[font=\footnotesize];

      \begin{axis}[
        no markers,
        xtick={0}, ytick={0},
        enlargelimits=true, 
        smooth, samples=100
        ]
        \addplot[domain=-pi:pi] { sin(x) };
        \addplot[magenta, thick] coordinates { (-1,-1) (1,1) };
        \fill (axis cs:0,0) circle (2pt);
        \node[below left] at (axis cs:pi,0) {\(\sin(x)\)};
      \end{axis}
    \end{tikzpicture}
    \qquad
    \begin{tikzpicture}
      \tikzstyle{every node}=[font=\footnotesize];

      \begin{axis}[
        no markers,
        xtick={0}, ytick={0},
        enlargelimits=true, 
        smooth, samples=100
        ]
        \addplot[domain=-pi:pi] { cos(x) };
        \addplot[magenta, thick] coordinates { (-1,1) (1,1) };
        \fill (axis cs:0,1) circle (2pt);
        \node[above left] at (axis cs:pi,0) {\(\cos(x)\)};
      \end{axis}
    \end{tikzpicture}
  \end{center}


  Derivatives of all other trigonometric functions \(\tan(x), \cot(x), \sec(x), \csc(x)\) can be calculated from the derivatives of \(\sin(x)\) and \(\cos(x)\) using basic differentiation rules.
  \begin{example}
    Differentiate \(\tan(x)\).

    \blanklines{15}
  \end{example}

  \begin{example}
    Suppose \(y = \sec(x)\). Find \(dy/dx\). 

    \blanklines{15}
  \end{example}

\end{document}
