%! TeX program = lualatex
\documentclass[../main.tex]{subfiles}
\begin{document} Solving applied optimization problems is a landmark application of differential calculus.  
To optimize something means to find the largest/smallest, most/least, etc., of something. 

Optimization problems are not ``cookie-cutter problems'' but doable. Suggested problems in textbook sections 4.3, 4.5 and 4.7 offer excellent opportunities to improve our problem-solving skills. In particular, finding extrema is a subtask of solving applied optimization problems.

\section{Maxima and minima}

Visually speaking, an \hlmain{absolute extremum} of a function \(y = f(x)\) is the highest or the lowest \(y\)-value on the graph of a function.   The \hlsupp{absolute} in absolute extrema should be interpreted as \hlsupp{for certain}, not as taking absolute values.

\begin{example} \label{ex:abs-extrema}
  Identify absolute extrema for the functions below, or explain why they do not exist.

  \begin{center}
    \includestandalone[page=1]{../standalones/plot-extrema-intro}
    \includestandalone[page=2]{../standalones/plot-extrema-intro}
    \includestandalone[page=3]{../standalones/plot-extrema-intro}
    
    % \includestandalone[page=4]{../standalones/plot-extrema-intro}
    % \includestandalone[page=5]{../standalones/plot-extrema-intro}
    % \includestandalone[page=6]{../standalones/plot-extrema-intro}
  \end{center}

  \blanklines{5}
\end{example}

We use inequalities to precisely capture the idea of absolute extrema.
\begin{mdframed}[style=withref-compact]
  Let \(f\) be a function defined over an interval \(I\) and let \(c \in I\).  We say
  \begin{itemize}
    \item \(f\) has an \hlmain{absolute maximum on \(I\) at \(c\)} if \underline{\hspace{2.5in}}.
    \item \(f\) has an \hlmain{absolute minimum on \(I\) at \(c\)} if \underline{\hspace{2.5in}}.
  \end{itemize}
  \textbook{Definition on page 366}
\end{mdframed}

An absolute extremum is an absolute maximum or an absolute minimum. An The plural of maxim\hlmain{um} is maxim\hlsupp{a}, minim\hlmain{um} is minim\hlsupp{a}, and extrem\hlmain{um} is extrem\hlsupp{a}.  Some textbooks use \emph{global} instead of \emph{absolute}. For example, global maximum and absolute maximum have exactly the same meaning.

\clearpage

% local = the peak for a local village
% bring meeples

A problem-solving strategy (not math-specific) is to break up a large problem into smaller ones.  If we want to find absolute extrema, we can look for extrema over smaller intervals.

\begin{mdframed}[style=withref-compact]
  Let \(f\) be a function defined over an interval \(I\).

  \begin{itemize}[noitemsep]
    \item \(f\) has a \hlmain{local maximum at \(c\)} if \(f(c) \ge f(x)\) for every \(x\) \hlmain{near} \(c\) but \(x\) \hlwarn{cannot be an endpoint} of \(I\).
    \item \(f\) has a \hlmain{local minimum at \(c\)} if \(f(c) \le f(x)\) for every \(x\) \hlmain{near} \(c\) but \(x\) \hlwarn{cannot be an endpoint} of \(I\).
  \end{itemize}

  \textbook{Definition on page 368, slightly simplified}
\end{mdframed}

Critical numbers are \(x\)-values at which local extrema \emph{could} occur.   Example~\ref{ex:critical-numbers} and Exercise~\ref{ex:critical-number-is-candidate} warn us about the subtle relation between critical numbers and local extrema.
\begin{mdframed}[style=withref-compact]
  A \hlmain{critical number} \(c\) is a constant in the domain of \(f(x)\) (but NOT an endpoint) if 
  \[
    f'(c) = 0 \quad\text{or}\quad f'(c) \text{ does not exist}.
  \]
  If \hlmain{\(c\) is a critical number} of \(f(x)\), then \hlsupp{\((c, f(c))\)} is called a \hlsupp{critical point}.

  \textbook{Definition on page 369}
\end{mdframed}

\begin{example} \label{ex:critical-numbers}
  Find its critical numbers, local extrema and absolute extrema of the given \(f(x)\).

  \begin{center}
    \includestandalone[height=3in]{../standalones/extrema_example_blank}
  \end{center}
\end{example}

\begin{exercise} \label{ex:critical-number-is-candidate}
  Is it true that \(f(\text{critical number})\) must be a local extrema?

  \begin{center}
    \includestandalone[page=7]{../standalones/plot-extrema-intro}
    \hfill
    \includestandalone[page=7]{../standalones/plot-extrema-intro}
    \hfill
    \includestandalone[page=7]{../standalones/plot-extrema-intro}
    \hfill
    \includestandalone[page=7]{../standalones/plot-extrema-intro}
  \end{center}
\end{exercise}
\clearpage

We need to be careful with endpoints when looking for critical numbers.
\begin{center}
  \begin{tabular}{c|l|l}
    Domain of \(f\) & Where to find critical numbers? & Critical number \emph{cannot} be \ldots{} \\\midrule
    \([a,b]\) or \((a,b]\) or \([a,b)\) or \((a,b)\) & & \\[1ex]\midrule
    \((-\infty, b]\) or \((-\infty, b)\) & & \\[1ex]\midrule
    \([a, \infty)\) or \((a, \infty)\) & & \\[1ex]\midrule
  \end{tabular}
\end{center}

\begin{example} \label{ex:critical-number-polynomial}
  Find critical numbers of \(f(x) = \frac{x^{3}}{3} - \frac{3}{2}x^{2} + 2x\) over \([0, 5/2]\).

  \blanklines{20}
\end{example}

% \begin{exercise} \label{ex:critical-number-trig}
%   Find critical numbers of \(f(x) = \cos(x)\) on \([0, 2\pi]\).
%
%   \blanklines{10}
% \end{exercise}

\begin{example} \label{ex:critical-number-rational}
  Find critical numbers of \(f(x) = \frac{3}{2} x^{2/3} - \frac{x^{2}}{2}\) over \([-2,2]\). 

  \blanklines{20}
\end{example}

\clearpage


\begin{mdframed}[style=simple-compact]
  \textbf{The general strategy for finding absolute extrema}. Let \(f(x)\) be a function defined over an interval \(I\).

  Absolute extrema can only occur at \(x\)-values \(c\) in exactly one of two ways:
  \begin{center}
    either \underline{\hspace{2in}} or \underline{\hspace{2in}}.
  \end{center}
  
  Note that the strategy applies to all functions.
\end{mdframed}

\blanklines{10}

{\footnotesize \faExclamationTriangle{} Some textbooks (e.g. UBC's CLP Calculus) do not include endpoints as a special case because they allow endpoints to be local extrema. \hlwarn{For grading purposes, we will stick to the course textbook's definition} that local extrema and critical numbers cannot occur at endpoints of the domain of \(f\).}

Because an arbitrary function can have vertical asymptotes, one or both absolute extrema do not always exist. In general, we have to make sure vertical asymptotes does not eliminate the existence of absolute extrema (one or both).

\blanklines{5}

% vertical asymptote and both absolute extrema cannot coexist.

Thanks to a theoretical result called the Extreme Value Theorem (Theorem 4.1 on textbook page 367), \emph{both} absolute extrema for \hlmain{continuous} functions defined over \hlmain{closed and bounded} intervals \([a,b]\) can always be calculated. We do not have to worry about vertical asymptotes for such a function.

\blanklines{15}


\clearpage

\begin{example} \label{ex:closed-interval-method-intro}
  Find absolute extrema and the \(x\)-values at which they occur for 
  \[
    f(x) = \frac{x^{3}}{3} - \frac{3}{2}x^{2} + 2x \qquad\text{defined over }[0,5/2].
  \]

  This is the function from Example~\ref{ex:critical-number-polynomial}. Note \(f(5/2) = 5/6\). 

  \blanklines{20}
\end{example}

\begin{example} 
  Find absolute extrema and the \(x\)-values at which they occur for 
  \[
    f(x) = \frac{3}{2} x^{2/3} - x^{2}/2 \qquad\text{defined over } [-2,2].
  \]

  This is the function from Example~\ref{ex:critical-number-rational}. Note that \(\frac{3}{\sqrt[3]{2}}-2 \approx 0.381\).

  \blanklines{20}
\end{example}
\clearpage

We eluded at the beginning of this section that minimizing or maximizing functions over a domain is a subtask in (applied) optimization problems.
Exercise~\ref{ex:extremum-swim} is a subtask in Exercise~\ref{ex:optimization-swim}.
\begin{exercise} \label{ex:extremum-swim}
  Minimize \(T\) given that \(T(s) = \frac{2 - 6 \sqrt{s^{2} - 1}}{6} + \frac{s}{3}\) on the domain \(1 \le s \le \sqrt{5}\).

  To minimize means to find the absolute minimum.  The function look a bit spicy. It just means that we have to be careful with our algebra. 

  \blanklines{40}
\end{exercise}
\clearpage
\end{document}
