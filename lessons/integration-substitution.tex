%! TeX program = lualatex
\documentclass[../main.tex]{subfiles}
\begin{document}
\begin{lesson}{The Substitution Rule}
  We have seen throughout the course that a change of variable can \emph{transform} a complex problem into a simpler one. The substitution rule performs such a change of variable in integration.
  \begin{mdframed}[style=withref]
    {If {\color{main} \(u = g(x)\)} is differentiable and \(f\) is continuous on the range of \(g\), then}
    \begin{equation} \label{eq:sub-definite}
      \int_{a}^{b} f( {\color{main} g(x)} ) \, g'(x) \;dx = \int_{u(a)}^{u(b)} f({\color{main} u}) \;d {\color{main} u}.
    \end{equation}

    For indefinite integrals, simply do nothing for the bounds. 
    \begin{equation} \label{eq:sub-indefinite}
      \phantom{\int f( {\color{main} g(x)} ) \, g'(x) \;dx = \int f(u) \;du}
    \end{equation}

    \textbook{Theorem 5.7 on page 584 and Theorem 5.8 on page 589}
  \end{mdframed}
  The substitution rule works by undoing the chain rule. To use the substitution rule, we need to 
  \begin{enumerate}
    \item \underline{\hspace{1in}} a substitution $u$ as a function of $x$, \hfill{} \makebox[2in][l]{\footnotesize \color{attn} (seek out useful information)}
    \item apply Equation~\eqref{eq:sub-definite}~or~\eqref{eq:sub-indefinite} to transform the given integral, \hfill{} \makebox[2in][l]{\footnotesize \color{attn} (use a mathematical tool)}
    \item evaluating the new integral in $u$, \hfill{} \makebox[2in][l]{\footnotesize \color{attn} (solve a simpler problem)}
    \item put back the original independent variable.
  \end{enumerate}

  \medskip
  Let's make sure we can apply Equation~\eqref{eq:sub-definite} properly.
  \begin{example} \label{ex:u-sub-2}
    Evaluate \(\int_{1}^{2} x e^{x^{2}} \;dx\) by substituting \(u = x^{2}\).
    \blanklines{20}
  \end{example}
  \clearpage

  In general, we need to choose a substitution \(\color{main} u = (\text{something})\) before applying Equation~\eqref{eq:sub-definite}~or~\eqref{eq:sub-indefinite}.

  Strategy for choosing \(u\): \hlattn{Recognize} a \hlmain{pair} of functions related by differentiations and choose \(u\) to be the derivative.  Sometimes, there are more than one choices for \(u\). Be brave and go with some trial and error!
  \begin{example}
    Evaluate \(\int x^{2} \sqrt{1 + x^{3}} \;dx\).
    \blanklines{48}
  \end{example}

  Sometimes, the choice \(u = ax + b\), where \(a,b\) are constants, works well. 
  \begin{example}
    Evaluate $\int \frac{x^{2} + 1}{\phantom{(} x - 1 \phantom{)^{1}}} \;dx$.
    \blanklines{20}
  \end{example}

  When there is no obvious substitution, we can try reorganizing the integrand first.
  \begin{example}
    Evaluate \(\int_{0}^{\pi/4} \tan(\theta) \;d\theta\).
    \blanklines{25}
  \end{example}

  We can apply the substitution rule in more abstract situations. 
  \begin{example}
    Suppose \(\int_{0}^{1} f(x) \;dx = -5\). Evaluate \(\int_{0}^{\pi/4} f(\tan(x)) \sec^{2}(x) \;dx\).
    \blanklines{40}
  \end{example}
  
  \faComment{} Does the substitution \(u = \sec(x)\) work? Try it out yourself.
  \blanklines{8}
\end{lesson}
\end{document}
