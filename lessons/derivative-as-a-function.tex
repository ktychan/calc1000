%! TeX program = lualatex
\documentclass[../main.tex]{subfiles}
\begin{document} \section{Derivatives (as functions)}
% Do change of variable.

\begin{mdframed}[style=withref]
  \textbf{Definition}. The \emph{derivative} of a \(f(x)\) as a function, denoted by \(f'(x)\), is
  \begin{equation} \label{eq:derivative}
    f'(x) = {\lim_{h \to 0} \frac{f(x+h) - f(x)}{h}}
  \end{equation}
  The function \(f'(x)\) is defined wherever the limit in Equation~\eqref{eq:derivative} exists.

  \textbook{Page 232}
\end{mdframed}
\textbf{Terminologies}. Let \(f(x)\) be a function and \(a\) be a number.
\begin{itemize}
  \item ``Differentiate \(f\) (with respect to \(x\))'' means \underline{\hspace{3in}}
  \item ``\(f(x)\) is differentiable at \(a\)'' means \underline{\hspace{3.62in}}
  \item ``\(f(x)\) is differentiable on \((a,b)\)'' means \underline{\hspace{3.36in}}
\end{itemize}

\bigskip
\begin{example}
  Let \(f(x) = \sqrt{x + 1}\). Find a formula for \(f'\) from definition (meaning use Equation~\eqref{eq:derivation}) and determine the domain of \(f'\).

  \blanklines{32}
\end{example}

\clearpage

\textbf{Notations}. If we write \(y = f(x)\) to denote a function, then \emph{all} of the following denote the function that is the derivative of \(f(x)\).
\[
  f'(x) 
  \;=\; f' 
  \;=\; y' 
  \qquad =\qquad  
  {\color{main} \frac{d}{dx}} \, f(x) \;=\; \frac{df}{dx} \;=\; \frac{dy}{dx}
  \qquad=\quad 
  Df(x) 
  \;=\; D_{x} f(x)
\]

\emph{All} of the following denote the derivative of \(f(x)\) at a number \(a\).
\begin{align*}
  & f'(a) \quad=\quad \frac{dy}{dx} \bigg|_{x = a} \quad=\quad \frac{dy}{dx} \bigg]_{x = a}
\end{align*}

\label{page:higher-derivatives}
In general, we define \(f^{(0)} = f\) and the \(n\)-th derivative to be
\[
  \overbrace{f^{' \cdots '}}^{\text{\(n\) primes}} \quad=\quad f^{(n)} \quad=\quad \frac{d^{n}}{dx^{n}} f \quad=\quad \frac{d}{dx} f^{(n-1)} \quad\text{for integers } n \ge 1.
\]

\begin{example}
  Find the second derivative of \(x^{2}\) from the definition (use Equation~\eqref{eq:derivative}).

  \blanklines{25}
\end{example}

\begin{example}
  Suppose \(s(t)\) is the displacement function of some object. What is the physical interpretation of \(s''(t)\)?

  \blanklines{5}
\end{example}
\clearpage

\begin{example}
  The derivative of a function \(f\) is \(x+1\) and \(f\) passes through \((1, 2)\). Sketch the tangent line of \(f\) at \(x = 1\). 

  \includestandalone[page=3]{../standalones/plot-exercise-sketch-derivative}
\end{example}

\begin{example}
  Use the graph of \(f(x)\) to sketch its derivative. 

  \begin{center}
    \includestandalone[page=1]{../standalones/plot-exercise-sketch-derivative}

    \includestandalone[page=2]{../standalones/plot-exercise-sketch-derivative}
  \end{center}
\end{example}
\end{document}

