%! TeX program = lualatex
\documentclass[../main.tex]{subfiles}
\begin{document} \section{Approximating Area}
  The geometric idea of \hlmain{area} is very flexible and widely applicable in science.  

  \begin{enumerate}[wide]
    \item If the velocity of an object (travelling in a straight line) is a constant function \(v(t) = 5\) km per minute, then the displacement of time \(t = 1\) minute to \(t = 3\) minute is \underline{\hspace{2in}}.
      \blanklines{5}

    \item If a medicine is administered intravenously at the rate of \(r(t) = 2\) ml per minute, then, after \(20\) minutes, a patient would have received \underline{\hspace{3in}} ml of medicine.
      \blanklines{5}
  \end{enumerate}

  It seems desirable to ask: What if \(v(t)\) or \(r(t)\) not a constant function? Can we still figure out the displacement and the amount of medicine administered?

  Both scenarios, and countlessly more, are asking the same fundamental \hlmain{geometric} question:
  \blanklines{5}

  
  Before we look for an exact answer, consider a practical idea: Approximation. \\
  \url{https://www.geogebra.org/calculator/bsdvzx3f}
  \blanklines{20}

  It turns out that approximation is the theoretically advantageous starting point for finding an exact answer: If we approximate an area using \underline{\hspace{2in}} many rectangles, then we get the exact area.
  
  The following definition describes a major concept in integration. \hlmain{Make sure you can evaluate such expressions (for functions to which formulas in Equation~\eqref{formula:summations} on page~\pageref{formula:summations} apply) and use its geometric interpretation to approximate area under the curve.}

  \begin{mdframed}[style=withref-compact]
    Let \(f(x)\) be a \emph{continuous} and nonnegative function on interval \([a,b]\). \hlmain{The area under the curve \(y = f(x)\) on \([a,b]\)} is defined to be
    
    \begin{equation} \label{eq:area-under-the-curve} \huge
      \lim_{{\color{main}n} \to \infty} \sum_{{\color{main}i}=1}^{{\color{main}n}} f(x_{{\color{main}i}}^{{\color{main}*}}) \; {\color{supp} \Delta x}.
    \end{equation}

    \textbook{The definitions on page 519}
  \end{mdframed}
  \blanklines{30}

  Side note (not examinable). Why is continuity important? Because there are strange discontinuous and nonnegative functions for which the limit in Equation~\eqref{eq:area-under-the-curve} does not exist.  Such a function \(f\) is defined by \(f(x) = 1\) if \(x\) is rational but \(f(x) = 0\) if \(x\) is irrational. 
 
  \clearpage
  Continuity guarantees the choice of \(x_{i}^{*}\) does not matter at all. We are free to choose anything we want from \([x_{i-1}, x_{i}]\) to be our \(x_{i}^{*}\). From a practical point of view, it means that as long as we approximate with enough rectangles we will eventually get more and more accurate answers, regardless of how we choose \(x_{i}^{*}\) (could just be a random choice).

  The freedom to choose \(x_{i}^{*}\) give us useful approximation techniques. Here are five simple methods.
  \begin{enumerate}
    \item \textbf{Left-endpoint approximation}: Choose \(x_{i}^{*}\) to be the left endpoint of the \(i\)-th interval.
    \item \textbf{Right-endpoint approximation}: Choose \(x_{i}^{*}\) to be the right endpoint of the \(i\)-th interval.
    \item \textbf{Mid-endpoint approximation}: Choose \(x_{i}^{*}\) to be the midpoint of the \(i\)-th interval.
    \item \textbf{Upper sum}: Choose \(x_{i}^{*}\) to be the number in the \(i\)-th interval at which \(f(x)\) is maximized.
    \item \textbf{Lower sum}: Choose \(x_{i}^{*}\) to be the number in the \(i\)-th interval at which \(f(x)\) is minimized.
  \end{enumerate}
  For the upper and lower sums, the Extreme Value Theorem guarantees such choices exist.

  \begin{example}
    For the function \(f(x)\) below, find and sketch the left-endpoint, right-endpoint and midpoint approximations using \(3\) subintervals. Find \(\Delta x\) and subintervals and write the corresponding Riemann sums in Sigma notation.

    \hfill{}
    \includegraphics{../standalones/build/plot_riemann_sum_draw_rectangles}
    \hfill{}
    \includegraphics{../standalones/build/plot_riemann_sum_draw_rectangles}
    \hfill{}
    \includegraphics{../standalones/build/plot_riemann_sum_draw_rectangles}
    \hfill{}

    \blanklines{20}
  \end{example}
  \clearpage

  \begin{example}
    For the function \(f(x)\) below, sketch the upper and lower sums using \(3\) subintervals.

    \hfill{}
    \includegraphics{../standalones/build/plot_riemann_sum_draw_rectangles}
    \hfill{}
    \includegraphics{../standalones/build/plot_riemann_sum_draw_rectangles}
    \hfill{}

    \blanklines{3}
  \end{example}

  \begin{example}
    In Sigma notation, write down the right-endpoint approximation for \(f(x) = \frac{1}{x^{2} - 1}\) on \([-2, 5]\) using \(n = 20\) subintervals.
    \blanklines{30}
  \end{example}
  \clearpage

  \begin{example}
    Use the definition, i.e., Equation~\eqref{eq:area-under-the-curve}, to find the area under the curve \(y = x^{2}\) on \([0,1]\).

    \begin{tikzpicture}[scale=3]
      \begin{scope}
        \clip (-0.5,0) rectangle (1.5, 1.5);
        \draw[thick, dashed] plot[domain=-0.5:0] (\x, \x*\x);
        \draw[thick, teal] plot[domain=0:1] (\x, \x*\x);
        \draw[thick, dashed] plot[domain=1:1.5] (\x, \x*\x);
      \end{scope}
      \draw[very thin, ->] (-0.5,0) -- (1.2,0) node[right] {\footnotesize \(x\)};
      \draw[very thin, ->] (0,-0.2) -- (0,1.2) node[above] {\footnotesize \(y\)};
      \foreach \x in {1} {
        \draw (\x, 0.05) -- (\x, -0.05) node[below] {\footnotesize \(\x\)};
      }
      \foreach \y in {1} {
        \draw (0.05, \y) -- (-0.05, \y) node[left] {\footnotesize \(\y\)};
      }
    \end{tikzpicture}
    \blanklines{40}
  \end{example}

\end{document}
