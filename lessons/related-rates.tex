%! TeX program = lualatex
\documentclass[../main.tex]{subfiles}
\begin{document} \section{Related rates}
Given a \hlmain{relation among \emph{two} quantities}, we can use implicit differentiation \hlwarn{with respect to time \(t\)} to find the \emph{relation} among rates of changes with respect to time \(t\).

\begin{mdframed}[style=simple-compact]
  \textbf{Key principle}. \hlmain{Related given quantities} (form an equation) and do calculus.
\end{mdframed}

\begin{example}[Similar to textbook Example~4.3] \label{ex:related-rates-intro}
  A student walks along a straight path at \(1\) metre per second.  A spotlight is located on the ground (to be measured) \underline{\hspace{1in}} away from the path and is kept focused on the student. At what rate is the spotlight rotating when the student is (to be measured) \underline{\hspace{1in}} away from the point on the path closest to the spotlight?
\end{example}

\blanklines{42}
\clearpage

\begin{example}[Textbook Example~4.2] \label{ex:related-rates-plane}
  An airplane is flying overhead at a constant elevation of \(4000\) ft. An observer is viewing the plane from a position of \(3000\) ft from the base of a radio tower. The plane is flying horizontally away from the observer. If the plane is flying at the rate of \(600\) ft/sec, at what rate is the distance between the observer and the plane increasing when the plane passes over the radio tower?

  \blanklines{45}
\end{example}
\clearpage

Example~\ref{ex:related-rates-plane} is fundamentally about \underline{\hspace{4.4in}}.  Studying its variations helps you see patterns and develop skills to tackle \hlmain{conceptually challenging parts (to you)} of related rates problems.

\begin{exercise} \label{ex:related-rates-plane-variations}
  Solve the following two problems whose solutions are very similar to that of Example~\ref{ex:related-rates-plane}. You are encouraged to \hlmain{reuse parts of the solution} of Example~\ref{ex:related-rates-plane} as much as possible.

  \begin{minipage}[t]{0.49\textwidth}
    \textit{Variation 1}.
    The \((x, L)\)-coordinates of a moving particle satisfy \(x^{2} + 4000^{2} = L^{2}\) as time \(t\) increases. Find \(\frac{dL}{dt}\) at \(x = 3000\) given that \(\frac{dx}{dt} = x/5\).
  \end{minipage}
  \hfill\vline\hfill
  \begin{minipage}[t]{0.49\textwidth}
    \textit{Variation 2}. 
    A particle travels along the curve \(y = 4x/3\).  Suppose its \(x\)-coordinate is changing at a constant rate of \(600\) units per second. At what rate is the distance between the origin and the particle changing when \(x = 3000\)?
  \end{minipage}

  \faStar{} Compare and contrast their thinking processes to get the most out of this exercise.

  \blanklines{40}
\end{exercise}
\clearpage

\faExclamationTriangle{} The examples and exercises in this section (up to now) \hlwarn{could mislead} us to believe that related rates problems are all about right-angle triangles. NOT always. Exercise~\ref{ex:related-rates-pyramids} does not involve any right-angle triangles. However, the \hlmain{key principle} to solve related rates still applies: \hlmain{Related given quantities and do calculus}.

\begin{example} \label{ex:related-rates-pyramids}
  An empty water tank is shaped like an upside-down pyramid, with base of \(3\) metres by \(3\) metres and a height of \(12\) metres. Water enters the tank at a rate of \(2\,\text{m}^{3}/\text{s}\) (cubic metres per second). How fast does the height of the water increase when the water is \(2\) metre deep?

  Recall that the volume of a pyramid is \(\frac{(\text{area of the base}) \times (\text{height})}{3}\).

  \blanklines{45}
\end{example}

\end{document}
