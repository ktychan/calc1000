%! TeX program = lualatex
\documentclass[../main.tex]{subfiles}
\begin{document} \section{Problem-solving using differentiation and derivatives}
  Let's use differentiation to solve some real problems. The learning objective is to \hlmain{seek useful information} to formulate problems into \hlmain{something for which we have tools to solve}.

  \label{page:diff-problem-solving}
  We can \hlmain{get into the habit} of
  \begin{enumerate}
    \item visualizing a given scenario if possible, 
    \item identifying information suggesting that differentiation is relevant, 
    \item drawing on our knowledge of derivatives of known functions, 
    \item formulating problems about differentiation using the language of derivatives, and
    \item being okay with exploring solutions \emph{one step at a time} even if we don't know how to solve the whole thing right away.
  \end{enumerate}

  \bigskip
  \begin{example}
    Let \(f(x) = x^{3} + 4x + 1\).  Determine the values of \(x\) for which \(f(x)\) has a horizontal tangent line, or explain why such \(x\) does not exist. What if \(f(x) = x 2^{x}\)?

    \blanklines{35}
  \end{example}
  \vfill{}
  \clearpage


  \begin{example}
    Let \(f(x) = \begin{cases} \sqrt[4]{x} &\text{ if } 0 \le x < 2 \\ g(x) &\text{ if } x \ge 2 \end{cases}\). Find a linear function \(g(x)\) defined on \(x \ge 2\) such that \(f(x)\) is differentiable everywhere.

    \blanklines{30}
  \end{example}


  \begin{example}
    Evaluate \(\lim_{x \to 8} \frac{\sqrt[3]{x} - 2}{x - 8}\).

    \blanklines{15}
  \end{example}
  \clearpage

  \begin{example}
    Let \(f(x) = 2^{x}\). Find an integer \(n\) such that \(f^{(n)}(x) = \ln(2)^{30} 2^{x}\).

    \blanklines{20}
  \end{example}

  \begin{example}
    Suppose \(f(x)\) is a polynomial such that \(f^{(5)}(x) = 0\), \(f^{(4)}(0) = 0\) and \(f'''(1) \ne 0\). What are the possible degrees of \(f\)? Explain your answer.

    \blanklines{30}
  \end{example}
  \clearpage

  \begin{example}
    Find the slope of line tangent to \(f(x) = \cos(x)\cot(x)\) at \(x = \pi/4\). 

    \blanklines{25}
  \end{example}

  \begin{example}
    Find \(\frac{d^{2025}}{dx^{2025}} \sin(x)\). Review the Leibniz notation for higher derivatives on page~\pageref{page:higher-derivatives}.

    \blanklines{20}
  \end{example}

\end{document}
