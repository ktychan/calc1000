%! TeX program = lualatex
\documentclass[../main.tex]{subfiles}
\begin{document} \section{The Net Change Theorem}
  The Net Change theorem provides the physical meaning of FTC part 2.

  \begin{mdframed}[style=withref-compact]
    If \(F(x)\) measures some physical quantity, then 
    \[
      F(b) - F(a) = \int_{a}^{b} F'(x) \;dx.
    \]

    \textbook{Theorem: 5.6: Net Change Theorem on page 567}
  \end{mdframed}
  \blanklines{5}

  For example, if \(s\) is the displacement of an object, and \(v\) measures its velocity, then the Net Change theorem says 
  \[
    s(b) - s(a) = \int_{a}^{b} v(t) \;dt.
  \]
  \blanklines{10}

  \begin{example}
    If \(v(t) = -t^{2} + 100\) is the velocity of some particle, find the \hlwarn{net displacement} from time \(t = 0\) to \(t = 30\).
    \blanklines{15}
  \end{example}
  \clearpage

  Let's change the question ever so slightly from \hlwarn{net} displacement to \hlwarn{total} displacement. The Net Change Theorem still applies, but it requires a small adjustment.
  \begin{example}
    If \(v(t) = -t^{2} + 100\) is the velocity of some particle, find the \hlwarn{total displacement} from time \(t = 0\) to \(t = 30\).
    \blanklines{40}
  \end{example}

\end{document}
