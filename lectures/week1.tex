%! TeX program = lualatex
\documentclass[../main.tex]{subfiles}

\ifcsname preamble@file\endcsname
  \externaldocument[main-]{../../build/main}
  \setcounter{page}{\getpagerefnumber{main-week1}}
\fi

\begin{document} \makelectureweek{1}

\subfile{../lessons/review-function-basics.tex}

\begin{example}[Horizontal compression and stretch] \label{ex:ecg}
  The graph of a function \(h(t)\) that models an electrocardiography (ECG) is shown below. Identify the graphs of \(h(2t)\) and \(h(t/3)\).

  \begin{tikzpicture}[scale=1]
    \begin{axis}[
      axis lines = middle, % boxed, middle
      axis on top,
      axis equal image,
      width={7in},
      %
      % domain and range
      %
      % xmin={}, xmax={},
      % ymin={}, ymax={},
      enlargelimits=true,
      %
      % axis labels
      %
      xlabel={time $t$ (ms)}, xlabel style={anchor=west},
      ylabel={$y$}, ylabel style={anchor=south},
      label style={at={(ticklabel* cs:1)}, font=\footnotesize},
      %
      % ticks
      %
      xtick={0, 15, ..., 60}, % xticklabels={},
      % ytick={}, yticklabels={},
      ticklabel style={font=\footnotesize},
      %
      % grid
      % none, major, minor, both
      grid=none, grid style={gray!20},
      % minor tick num=1, 
      % minor grid style={gray!20},
      % 
      % plot parameters
      %
      smooth, samples=100, no markers,
      title={The graph of \(h(t)\)}
      ]
      % \usepackage{pgfplots}
      % \pgfplotsset{compat=newest, trig format=rad} 
      % \pgfplotsset{label style={font=\footnotesize}}
      %
      % python:
      % L = lambda n: [(n + 0,0), (n + 1,0), (n +2,1), (n +3,0), (n +4.5,0), (n +5,-0.25), (n +6,10), (n +7,-3), (n +8,0), (n +10,0), (n +12,2), (n +13,0), (n +15,0)]
      % " ".join(map(str, L(0)))
      % " ".join(map(str, L(15)))
      % " ".join(map(str, L(30)))
      % " ".join(map(str, L(45)))
      %
      \addplot[thick,supp] coordinates {
          (0,0) (1,0) (2,1) (3,0) (4.5,0) (5,-0.25) (6,10) (7,-3) (8,0) (10,0) (12,2) (13,0) (15,0)
          (15, 0) (16, 0) (17, 1) (18, 0) (19.5, 0) (20, -0.25) (21, 10) (22, -3) (23, 0) (25, 0) (27, 2) (28, 0) (30, 0)
          (30, 0) (31, 0) (32, 1) (33, 0) (34.5, 0) (35, -0.25) (36, 10) (37, -3) (38, 0) (40, 0) (42, 2) (43, 0) (45, 0)
          (45, 0) (46, 0) (47, 1) (48, 0) (49.5, 0) (50, -0.25) (51, 10) (52, -3) (53, 0) (55, 0) (57, 2) (58, 0) (60, 0)
        };
    \end{axis}
  \end{tikzpicture}

\end{example}

\begin{example}
  Sketch the graph of \(y = 3 \sin\left(\frac{x}{2\pi} + \frac{\pi}{4}\right)\) and label the axes.
\end{example}

\begin{example}
  Let's consider the graph of \(h(t)\) in Example~\ref{ex:ecg} again. 
\end{example}

\section{Trigonometric Functions}

The sine and cosine functions, denoted by \(\sin(x)\) and \(\cos(x)\), are fundamental to all trig functions. 

How do you think about \(\sin(x)\) and \(\cos(x)\)? For example, how do you calculate \(\sin(\pi/3)\)?

\blanklines{10}


What identity is central to \hlmain{all} trig calculations?
\blanklines{10}

Listed below are all commonly used trig functions. Write each function in terms of \(\sin(x)\) and \(\cos(x)\).
\begin{align*}
  \tan(x) &= \hspace{3in} \\[2ex]
  \cot(x) &= \hspace{3in} \\[2ex]
  \sec(x) &= \hspace{3in} \\[2ex]
  \csc(x) &= \hspace{3in} \\[2ex]
\end{align*}


\section{Algebraic operations on functions}

Algebraic operations \[+, \quad -, \quad \times, \quad \div, \quad \text{exponentiation} \quad\text{and}\quad \text{composition}\] are ``verbs'' of the mathematical language. By \emph{legally} combining verbs (operations) with nouns (functions) we can build sophisticated mathematical functions to describe natural phenomena. 

We review algebraic operations, so we can correctly \hlmain{parse} mathematical expressions and \hlmain{apply formulas}. 

\begin{example}
  To calculate \(3 \times 4 + 2\), we perform the multiplication before the addition. 
  
  \blanklines{6}
\end{example}

\begin{mdframed}[style=simple-compact]
  Algebraic operations must be applied in the following order
  \[ \underbrace{( \cdots )}_{\text{first}} \quad\longrightarrow\quad \text{exponentiation} \quad\longrightarrow\quad \times, \div \quad\longrightarrow\quad \underbrace{+, -}_{\text{last}}. \]

  \begin{enumerate}[itemsep=0pt]
    \item Parentheses are always applied first. If there are multiple parentheses, the innermost ones should be applied first.
    \item Exponentiation are applied next, before \(\times\) and \(\div\).
    \item Multiplication and division have the third highest precedence.
    \item Addition and subtraction division have the lowest precedence.
  \end{enumerate}

  Operations with the same precedence can be performed \hlsupp{in parallel}.
\end{mdframed}

Know the order of operations well. Applying operations out of order is a major but avoidable mistake.

\begin{example}
  Describe the order of operations in the function \( f(x) = (x^{3/2}+1)^{2} + x - \frac{3 (x + 1)}{x}\).  
\end{example}

\section{Exponentials and Logarithms}

A function \(b^{x}\), where \(b\) is a \emph{positive} constant and \(x\) is the independent variable, is called \emph{an} exponential function. 
\begin{example}
  Which of the following is an exponential function?

  \begin{enumerate*}
    \item \(2^{-x}\)
    \item \((1/2)^{x}\)
    \item \(x^{3}\)
    \item \(e^{x}\)
    \item \(x^{x}\)
  \end{enumerate*}
\end{example}


Let's recall some basic properties of an exponential function.  Assume \(b\) is a positive constant. 
\begin{enumerate}[itemsep={1ex}]
  \item The domain of \(b^{x}\) is 
  \item The range of \(b^{x}\) is 
\end{enumerate}



Take it for granted, for now, that \(b^{x}\) has an inverse\footnote{The only exponential function without an inverse is \(1^{x}\).} as long as \(b > 0\) and \(b \ne 1\).  The inverse of \(b^{x}\) is called an logarithm of base \(b\), denoted as \(\log_{b}(x)\).  

\faComment{} Consider the logarithm \(\log_{2}(x)\). How can we \emph{verify} (without using any rules) that \(\log_{2}(8) = 3\)?
\blanklines{4}

The Euler's number is a \hlmain{very special} constant, denoted simply as \(e\). The exponential function \(e^{x}\) is called the \emph{natural exponential function}. The inverse of \(e^{x}\) is given a special symbol \(\ln(x)\). In other words, \(\ln(x) = \log_{e}(x)\).


You have probably memorized various logarithmic identities such as \(\ln(ab) = \ln(a) + \ln(b)\) and \(\ln(a/b) = \ln(a) - \ln(b)\) provided that \(a > 0\) and \(b > 0\). However, students often \hlwarn{misremember} the identity leading to preventable mistakes in tests and exams (hopefully not in rocket science).

Let's \hlmain{consolidate our understanding} of logarithm laws using the Basis Change Trick. 

\begin{mdframed}[style=simple]
  \textbf{The Basis Change Trick}. Given any positive constant \(b\), we have an identity
  \[
    b = e^{\ln(b)}.
  \]

  \faStar{} This identity provides a \hlmain{uniform method} to understand all logarithms in term of the easier base \(e\).
\end{mdframed}

Although \(b = \ln(e^{b})\) is true for any constant \(b > 0\), but it is admittedly less useful in computations. 


\begin{example}[Logarithm Laws]
  Let \(a,b\) be positive constants. Use the Basis Change Trick to help you remember the three most frequently used logarithm laws.
  \begin{enumerate}[topsep={0in}, itemsep={1ex}]
    \item \(\ln(ab) = \ln(a) + \ln(b) \)
    \item \(\ln(a/b) = \ln(a) - \ln(b) \)
    \item \(\ln(a^{x}) = x \ln(a) \)
  \end{enumerate}
\end{example}
\blanklines{10}


\faStar{} The Basis Change Trick can be used to rewrite ``weird'' exponential-like functions into a familiar form. We demonstrate this technique in the following three examples.

\begin{example} \label{ex:x-to-x}
  Find a function \(f(x)\) so that \(x^{x} = e^{f(x)}\). Assume \(x > 0\).

  {\footnotesize Preview: Example~\ref{ex:x-to-x} is an intermediate step in calculating the derivative of \(x^{x}\).}
  \blanklines{5}
\end{example}



\begin{example}
  Simplify the function \(x^{\arctan(x)}\) into a more familiar exponential function. Assume \(x > 0\). 
  \blanklines{5}
\end{example}


\begin{example}
  Simplify the function \(\left( 1 + \frac{1}{x} \right)^{x}\) into a more familiar exponential function. Assume \(x > 0\). 
  \blanklines{5}
\end{example}


\begin{example} \label{ex:b(x)-to-f(x)}
  Notice the calculations in the previous three examples follow a pattern! In this example, we summarize this pattern in a more general language. 

  Assume \(b(x)\) and \(f(x)\) are both functions. 

  \begin{enumerate}
    \item Rewrite \(b(x)^{f(x)}\) as a more familiar exponential function. 
    \item What restrictions are needed on the domains and ranges of \(b(x)\) and \(f(x)\) to make part 1 work?
  \end{enumerate}
\end{example}

Example~\ref{ex:b(x)-to-f(x)} is a more general version of Example~\ref{ex:x-to-x} and describes a \hlsupp{reusable} pattern of calculations that typically come up in calculations of limits and derivatives. 

\begin{example}
  Try the following exercises if you are familiar with l'H\^opital's rule and the Chain Rule.

  \begin{enumerate}
    \item Find the derivative of \(\left(\frac{1}{x}\right)^{\sin(x)}\) by first applying Example~\ref{ex:b(x)-to-f(x)} then the Chain Rule. 

      % {\footnotesize The answer can be written many forms. One of them is \(-\left( \frac{\sin(x)}{x} + \ln(x)\cos(x) \right) e^{-\ln(x)\sin(x)}\).}

    \item Evaluate \(\lim_{x \to \infty} \left( 1 + \frac{1}{x} \right)^{x}\). 

      % {\footnotesize The answer is \(e\).}

  \end{enumerate}
\end{example}

\end{document}
