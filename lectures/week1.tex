%! TeX program = lualatex
\documentclass[../main.tex]{subfiles}

\ifcsname preamble@file\endcsname
  \externaldocument[main-]{../../build/main}
  \setcounter{page}{\getpagerefnumber{main-week1}}
\fi

\begin{document} \makelectureweek{1}

\subfile{../lessons/review-function-basics.tex}

% \begin{example}[Horizontal compression and stretch] \label{ex:ecg}
%   The graph of a function \(h(t)\) that models an electrocardiography (ECG) is shown below. Identify the graphs of \(h(2t)\) and \(h(t/3)\).
%
%   \begin{tikzpicture}[scale=1]
%     \begin{axis}[
%       axis lines = middle, % boxed, middle
%       axis on top,
%       axis equal image,
%       width={7in},
%       %
%       % domain and range
%       %
%       % xmin={}, xmax={},
%       % ymin={}, ymax={},
%       enlargelimits=true,
%       %
%       % axis labels
%       %
%       xlabel={time $t$ (ms)}, xlabel style={anchor=west},
%       ylabel={$y$}, ylabel style={anchor=south},
%       label style={at={(ticklabel* cs:1)}, font=\footnotesize},
%       %
%       % ticks
%       %
%       xtick={0, 15, ..., 60}, % xticklabels={},
%       % ytick={}, yticklabels={},
%       ticklabel style={font=\footnotesize},
%       %
%       % grid
%       % none, major, minor, both
%       grid=none, grid style={gray!20},
%       % minor tick num=1, 
%       % minor grid style={gray!20},
%       % 
%       % plot parameters
%       %
%       smooth, samples=100, no markers,
%       title={The graph of \(h(t)\)}
%       ]
%       % \usepackage{pgfplots}
%       % \pgfplotsset{compat=newest, trig format=rad} 
%       % \pgfplotsset{label style={font=\footnotesize}}
%       %
%       % python:
%       % L = lambda n: [(n + 0,0), (n + 1,0), (n +2,1), (n +3,0), (n +4.5,0), (n +5,-0.25), (n +6,10), (n +7,-3), (n +8,0), (n +10,0), (n +12,2), (n +13,0), (n +15,0)]
%       % " ".join(map(str, L(0)))
%       % " ".join(map(str, L(15)))
%       % " ".join(map(str, L(30)))
%       % " ".join(map(str, L(45)))
%       %
%       \addplot[thick,supp] coordinates {
%           (0,0) (1,0) (2,1) (3,0) (4.5,0) (5,-0.25) (6,10) (7,-3) (8,0) (10,0) (12,2) (13,0) (15,0)
%           (15, 0) (16, 0) (17, 1) (18, 0) (19.5, 0) (20, -0.25) (21, 10) (22, -3) (23, 0) (25, 0) (27, 2) (28, 0) (30, 0)
%           (30, 0) (31, 0) (32, 1) (33, 0) (34.5, 0) (35, -0.25) (36, 10) (37, -3) (38, 0) (40, 0) (42, 2) (43, 0) (45, 0)
%           (45, 0) (46, 0) (47, 1) (48, 0) (49.5, 0) (50, -0.25) (51, 10) (52, -3) (53, 0) (55, 0) (57, 2) (58, 0) (60, 0)
%         };
%     \end{axis}
%   \end{tikzpicture}
%
% \end{example}

% \begin{example}
%   Sketch the graph of \(y = 3 \sin\left(\frac{x}{2\pi} + \frac{\pi}{4}\right)\) and label the axes.
% \end{example}

\end{document}
