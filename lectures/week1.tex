%! TeX program = lualatex
\documentclass[../main.tex]{subfiles}

\begin{document}
\makelectureweek{1}{Week 1}

\section{Overview of the course syllabus}


\begin{itemize}
  \item Learn ``new'' operations such as limits \(\lim_{x \to a} f(x)\), differentiation \(\frac{d}{dx} f(x)\) and integration \(\int_{a}^{b}f(x)\;dx\).
  \item Use these operations to gain \emph{new} insights of functions you learned in high school.
  \item Apply calculus to solve problems.
\end{itemize}

Teach student technical skills. Ask students to identify transferable skills (where to do you math?).

Treat exams a little bit like a job interview. Are you ready to take on a bigger challenge?

\section{A Brief Review of Functions}

\begin{example}
  Which one of the following is \emph{not} a function?
  \begin{enumerate}[label=(\alph*)]
    \item \(h = \frac{7t^{3} + 2t^{2} + 5}{t - 1}\).
    \item \(y = x^{2} - 1\).
    \item \(y = x^{2} - 1\) restricted to \(0 \le x < 1\).
    \item \(H(\theta) = (e^{\theta} - e^{-\theta})/2\).
    \item \(x^{2} + y^{2} = 1\).
    \item \(g(t) = \frac{f(a + t) - f(a)}{t}\) where \(f\) is a function and \(a\) is a constant.
    \item \(f(t) = \begin{cases} 1, &\text{if } t < 2 \\ 2, &\text{otherwise} \end{cases}\).
  \end{enumerate}
\end{example}
\vfill

A function, often denoted abstractly as \(y = f(x)\) or just \(f(x)\), expresses the relationship between two quantities: the {independent variable} (or input) \(x\) and the {dependent variable} (or output) \(y\).

\begin{mdframed}[style=withref]
  Review. A function \(f\) is a rule that assigns to {each} element \(x\) in a set \(D\) {exactly one} element called \(f(x)\). The {domain} of a function is the set \(D\).

  \textbook{\stewart[black]{8}{Section 1.1}}
\end{mdframed}

\section{One-sided Limits}
We introduce limits of a function. We will learn to \hlsupp{make sense} of an abstract definition by \hlmain{estimating limits numerically and graphically}.

%--------------------
% left-hand limit
%--------------------
\begin{example}
  The table below samples the function \(y = t^{2}\) \emph{near} the number \(t = 1\).
  What pattern of \(t^{2}\) do you observe as \(t\) gets closer to \(1\) from the left \hlattn{but is not allowed to be \(1\)}?

  \includegraphics{../standalones/build/table_t_squared}
\end{example}
\includegraphics{../standalones/build/paper-dotted-small}

\begin{mdframed}[style=withref]
  \textbf{Definition} (left-hand limit). Suppose \(f(x)\) is defined near the number \(a\).

  We write {\(\lim_{x \to a^{-}} f(x) = L\)} and say that
  \begin{center}
    "\emph{the {left-hand limit} of \(f(x)\) as \(x\) approaches \(a\) is equal to \(L\)}"
  \end{center}
  or
  \begin{center}
    "\emph{the {limit} of \(f(x)\) as \(x\) approaches \(a\) {from the left} is equal to \(L\)}"
  \end{center}
  if we can make the value of \(f(x)\) arbitrarily close to \(L\) by restricting \(x\) to be sufficiently close to \(a\) with \(x < a\). If \(L\) does not exists, then we say the limit \emph{does not exist}.

  \textbook{\stewart{86}{\fbox{2} Intuitive Definition of One-Sided Limits}}
\end{mdframed}
\vfill

\begin{example}
  Use the graph below to estimate the limit of \(p(x)\) as \(x\) approaches \(\frac{\pi}{2}\) from the left.

  \includegraphics{../standalones/build/plot_piecewise}
\end{example}

\includegraphics{../standalones/build/paper-dotted-small}

%--------------------
% right-hand limit
%--------------------
\begin{mdframed}[style=withref]
  \textbf{Definition} (right-hand limit). Suppose \(f(x)\) is defined {near} the number \(a\).

  We write \({\lim_{x \to a^{+}} f(x)} = L\) and say that
  \begin{center}
    "\emph{the {right-hand limit} of \(f(x)\) as \(x\) {approaches} \(a\) is equal to \(L\)}"
  \end{center}
  or
  \begin{center}
    "\emph{the {limit} of \(f(x)\) as \(x\) {approaches} \(a\) {from the right} is equal to \(L\)}"
  \end{center}
  if we can make the value of \(f(x)\) {arbitrarily} close to \(L\) by {restricting} \(x\) to be {sufficiently close} to \(a\) with {\(x > a\)}. If \(L\) does not exists, then we say the limit does not exist.

  \textbook{\stewart{86}{\fbox{2} Intuitive Definition of One-Sided Limits}}
\end{mdframed}
\includegraphics{../standalones/build/paper-dotted-small}

\begin{example} \label{activity:one-sided}
  Use the GeoGebra worksheet \url{https://www.geogebra.org/calculator/xzawxtcy}.
  \begin{enumerate}[label=(\alph*)]
    \item Estimate \(\lim_{t \to 1^{-}} t^{2}\) and \(\lim_{t \to 1^{+}} t^{2}\).
          \vfill{}

    \item \label{part:piecewise} Estimate the left-hand and right-hand limits of \(f(t) =
          \begin{cases}
            1, & \text{if } t < 2 \\
            2, & \text{otherwise}
          \end{cases}\) as \(t\) approaches \(2\).
          \vfill{}
  \end{enumerate}
\end{example}

\includegraphics{../standalones/build/paper-dotted-large}

\section{Two-sided Limits}
\begin{mdframed}[style=withref]
  \textbf{Definition} (Two-sided limit). Suppose \(f(x)\) is {defined when \(x\) is near the number \(a\).}

  We write \({\lim_{x \to a} f(x)} = L\) and say
  \begin{center}
    "the {limit} of \(f(x)\), as {\(x\) approaches \(a\)}, equals to \(L\)"
  \end{center}
  if we can {make the value of \(f(x)\) arbitrarily close to \(L\) by restricting \(x\) to be sufficiently close to \(a\) (on either side of \(a\))} {but not equal to \(a\).}

  If \(L\) does not exists, then we say the limit {does not exist.}

  \textbook{\stewart{83}{\fbox{1} Intuitive Definition of a Limit}}
\end{mdframed}
\includegraphics{../standalones/build/paper-dotted-small}

\begin{example}
  Use the GeoGebra worksheet \url{https://www.geogebra.org/calculator/xzawxtcy} to estimate \(\lim_{t \to \pi/2} \sin(t)\).
\end{example}
\includegraphics{../standalones/build/paper-dotted-small}

\begin{mdframed}[style=withref]
  \textbf{Theorem}.
  \[
    \lim_{x \to a} f(x) = L \qquad\text{if and only if}\qquad {\lim_{x \to a^{-}} f(x) = L \quad\text{and}\quad \lim_{x \to a^{+}} f(x) = L.}
  \]

  \textbook{\stewart{87}{\fbox{3}}}
\end{mdframed}
\includegraphics{../standalones/build/paper-dotted-small}

\begin{example}
  Let \(f(t)\) be the function in Example~\ref{activity:one-sided}\ref{part:piecewise}. Does \(\lim_{t \to 2} f(t)\) exist?
\end{example}
\includegraphics{../standalones/build/paper-dotted-small}

\begin{example}
  Use the GeoGebra worksheet \url{https://www.geogebra.org/calculator/xzawxtcy} to explore limits of other functions.
\end{example}
\clearpage

\begin{example}
  Find \(\lim_{x \to 0} \frac{1}{x^{2}}\) if it exists.
\end{example}
\includegraphics{../standalones/build/paper-dotted-medium}

\begin{mdframed}[style=withref]
  \textbf{Definition}. Let \(f\) be a function defined on both sides of \(a\), except possibly at \(a\) itself. Then
  \[
    \lim_{x \to a} f(x) = \infty
    \quad\text{or}\quad
    \lim_{x \to a} f(x) = -\infty
  \]
  means that the value of \(f(x)\) can be made arbitrarily large (or small respectively) by taking \(x\) sufficiently close to \(a\), but not equal to \(a\).

  \textbook{\stewart{89}{\fbox{4} Intuitive Definition of an Infinite Limit} and \stewart{90}{\fbox{5} Definition}}
\end{mdframed}
\includegraphics{../standalones/build/paper-dotted-small}

\begin{example}
  Find \(\lim_{t \to (\pi/2)^{-}} \tan(t)\) if it exists.
\end{example}
\includegraphics{../standalones/build/paper-dotted-small}

\begin{mdframed}[style=withref]
  \textbf{Definition}. A vertical line \(t = a\) is a {vertical asymptote} of a function \(f(t)\) if one of its \emph{one-sided} limits, as \(t\) approaches \(a\), goes to an infinity.

  \textbook{Page 90, \textcolor{red}{\fbox{6} Definition}}
\end{mdframed}
\vfill

\begin{example}
  Identify vertical asymptote(s) of \(\tan(\theta)\).
\end{example}
\includegraphics{../standalones/build/paper-dotted-small}
\clearpage

\section{Review of polynomials, roots, sine, cosine, and exponentials}

\begin{example}
  Label the function with its type: polynomial, root function, sine, cosine, or exponential. The independent variables are \(x\) and \(t\). % If a function is none of the above, then leave its label blank.
  % Anticipate the question: Can a function have multiple types?

  \begin{figure}[h!]  % [h] for here, [ht] for here top, [hb] for here bottom
    \centering
    \begin{tikzpicture}
      \node at ( 0, 4) {\large \(x^{2} + 1\)};
      \node at ( 2,-2) {\large \(\frac{1}{2} x^{5} + x^{3} - \frac{2}{3} + \sqrt{x}\)};
      \node at (-4,-1) {\large \(- \frac{3}{7} t^{359} - 2 t^{5} + \pi t^{3}\)};
      \node at ( 0,-5) {\large \(x^{\pi}\)};
      \node at ( 5,-3) {\large \(x^{-1}\)};
      \node at (-6,-3) {\large \(\sqrt{x}\)};
      \node at (-5, 4) {\large \(\sqrt[3]{x}\)};
      \node at (-7, 3) {\large \(x^{1/5}\)};
      \node at (-5, 6) {\large \(x^{3/2}\)};
      \node at ( 2,-6) {\large \(\sin(x)\)};
      \node at (-2, 7) {\large \(\cos(t)\)};
      \node at ( 6,-2) {\large \(\frac{\sin(\theta)}{\cos(\theta)}\)};
      \node at ( 6, 6) {\large \(\sin(x) + \cos(x)\)};
      \node at (-4, 2) {\large \(2^{x}\)};
      \node at (-5,-6) {\large \(2^{2x}\)};
      \node at ( 2, 5) {\large \(\left(\frac{1}{2}\right)^{x}\)};
      \node at (-2, 1) {\large \(e^{x}\)};
      \node at ( 5,-7) {\large \((-2)^{x}\)};
      \node at ( 7, 2) {\large \(x^{x}\)};
    \end{tikzpicture}
  \end{figure}
\end{example}
If a function does not match one of the types above, then leave it blank.

\includegraphics{../standalones/build/paper-dotted-small}

Use ALEKS (Google "YorkU Bethune ALEKS" or use link below) to review background material.

Link: \url{https://www.yorku.ca/colleges/bethune/orientation/aleks-math-help-and-diagnostic/}

\clearpage

\begin{table}[h]  % [h] for here, [ht] for here top, [hb] for here bottom
  \centering
  \includegraphics{../standalones/build/table_basic_functions}
  \caption{Basic functions}
  \label{table:basic_functions}
\end{table}

\section{Limit Laws}

Last week, we learned how to estimate limits of functions. We now learn how to \hlsupp{evaluate} limits \emph{exactly}.
\faExclamationTriangle{} This section covers a bit more than the corresponding textbook section and is organized differently to minimize redundancy. See the last page for a short comparison.

\begin{mdframed}[style=simple]
  \textbf{Theorem \faStar{}}. If a function \(f(x)\) is one of the functions in Table~\ref{table:basic_functions} and \(a\) is a number in the {domain} of \(f(x)\), then
  \[
    \lim_{x \to a} f(x) = {f(a).}
  \]
\end{mdframed}
\includegraphics{../standalones/build/paper-dotted-small}

\begin{example}
  Evaluate \(\lim_{x \to -2} (x^{2} + 5 x^{3})\).

  \includegraphics{../standalones/build/paper-dotted-medium}
\end{example}
\clearpage


\begin{mdframed}[style=withref]
  \textbf{Theorem} (Limit Laws). Suppose \(\lim_{x \to a} f(x)\) and \(\lim_{x \to a} g(x)\) {both exist}. Then
  \begin{enumerate}[label=(\arabic*)]
    \item \(\lim_{x \to a} [f(x) + g(x)] = \lim_{x \to a} f(x) + \lim_{x \to a} g(x)\) \hfill (sum)
    \item \(\lim_{x \to a} [f(x) - g(x)] = \lim_{t \to a} f(x) - \lim_{x \to a} g(x)\) \hfill (difference)
    \item \(\lim_{x \to a} [c \; f(x)] = c \lim_{t \to a} f(x)\), where \(c\) is a constant \hfill (constant multiple)
    \item \(\lim_{x \to a} [f(x) \; g(x)] = \lim_{x \to a} f(x) \cdot \lim_{x \to a} g(x)\) \hfill (product)
    \item \(\lim_{x \to a} \frac{f(x)}{g(x)} = \frac{\lim_{x \to a} f(x)}{\lim_{x \to a} g(x)}\), \quad if {\(\lim_{x \to a}g(x) \ne 0\)} \hfill (quotient)
  \end{enumerate}
  \textbook{\stewart{95}{Limit Laws}}
\end{mdframed}
% Before applying the limit laws, make sure the limit exists!

\begin{example}
  Evaluate \(\lim_{x \to 2} \left(\sin(x)\cos(x) - 2 e^{x} + \sqrt[3]{x}\right)\), if it exists.
\end{example}
\includegraphics{../standalones/build/paper-dotted-large}
\clearpage

\begin{mdframed}[style=withref]
  \textbf{Theorem} (Limit Laws continued).
  \begin{enumerate}[label=(\arabic*), start=6]
    \item \(\lim_{x \to a} [f(x)]^{n} = \left[ \lim_{x \to a} f(x) \right]^{n}\) \hfill (power)
    \item \(\lim_{x \to a} \sqrt[n]{f(x)} = \sqrt[n]{\lim_{x \to a} f(x)}\), \quad if \(n\) is even, we assume {\(\lim_{x \to a} f(x) > 0\)} \hfill (root)
  \end{enumerate}

  \textbook{\stewart[ForestGreen]{96}{Power Law} and \stewart[ForestGreen]{96}{Root Law}}
\end{mdframed}

\begin{example}
  Suppose \(\lim_{x \to -3}f(x) = 4\). Evaluate \(\lim_{x \to 2} (f(x))^{3/2}\), if it exists.
\end{example}
\includegraphics{../standalones/build/paper-dotted-medium}

\faLightbulb{} Limit laws also apply to piecewise functions and one-sided limits.
\begin{example}
  Evaluate \(\lim_{\theta \to (\pi/2)} F(x)\) where \(F(x) = \begin{cases} \cos(\theta), & \text{if } \theta > 0 \\ e^{\theta}, &\text{if } \theta \le 0 \end{cases}\).
\end{example}
\includegraphics{../standalones/build/paper-dotted-medium}
\begin{example}
  Recall \(|x| = \begin{cases} x, &\text{if } x \ge 0 \\ -x, &\text{if } x < 0 \end{cases}\). Evaluate \(\lim_{x \to 0} |x|\).
\end{example}
\includegraphics{../standalones/build/paper-dotted-medium}
\clearpage

\section{More on Quotients}\label{sec:quotients}

If \(f(x)\) and \(g(x)\) are functions, then \(\frac{f(x)}{g(x)}\) is a {quotient} function. The domain of \(\frac{f(x)}{g(x)}\) is the domain of \(f(x)\) excluding the numbers \(x\) such that {\(g(x) = 0\).}

A {rational function} is a quotient where both the numerator and the denominator are {polynomials.}

\begin{example}
  Evaluate \(\lim_{t \to 3} \frac{t^{2} - 3t + 2}{t - 2}\), if it exists.
\end{example}
\includegraphics{../standalones/build/paper-dotted-medium}

\begin{example}
  Evaluate \(\lim_{t \to 2} \frac{t^{2} - 3t + 2}{t - 2}\), if it exists.
\end{example}
\includegraphics{../standalones/build/paper-dotted-medium}
\clearpage

\begin{example}
  Evaluate \(\lim_{t \to 0} \frac{\sqrt{t^{2}+1}-1}{t^{2}}\), if it exists.
\end{example}
\includegraphics{../standalones/build/paper-dotted-medium}

\faExclamationTriangle{} Theorem~\faStar{} on page 2 does not appear in the textbook. Limit laws 8, 9, 10, 11 on page 96 and 97 of the textbook are special cases of Theorem~1. Theorem~1 also overlaps with the box entitled \textcolor{red}{Direct Substitution Property} on page 97.

\clearpage

\section{The Squeeze Theorem}
\begin{mdframed}[style=withref]
  \textbf{The Squeeze Theorem}. Let {\(f(x) \le g(x) \le h(x)\)} when \(x\) is near \(a\) (except possibly at \(a\)) and
  \includegraphics{../standalones/build/paper-dotted-small}
  % \[ { \lim_{x \to a} f(x) = \lim_{x \to a} h(x) = L,} \]
  then
  \includegraphics{../standalones/build/paper-dotted-small}
  % {\(\lim_{x \to a} g(x) = L\).}

  \textbook{\stewart{101}{\fbox{3} The Squeeze Theorem}}
\end{mdframed}
\includegraphics{../standalones/build/paper-dotted-small}

\begin{example}
  Evaluate \(\lim_{x \to 0} x \sin\left(\frac{1}{x}\right)\).

  \hfill\includegraphics[width=2in]{../standalones/build/plot_squeeze}
\end{example}
\clearpage

\begin{example}
  Suppose \(x+1 \le h(x) \le x^{3} - 2x + 3\). Find \(\lim_{x \to 1} h(x)\).
  
\includegraphics{../standalones/build/paper-dotted-medium}
\end{example}

\section{The Precise Definition of a limit}

\begin{example} \label{example:delta}
  Let \(f(t) = t^{2}\) be a function. Find a number \(\delta > 0\) such that
  \begin{center}
    if \(0 < | t - 1 | < \delta\), then \(|f(t) - 1| < \fbox{{1/2}}\).
  \end{center}

  \begin{enumerate}
    \item ``\textit{Find a number \(\delta > 0\)}'' means \\
          \includegraphics{../standalones/build/paper-dotted-small}
          % {you have control of \(\delta\).}
    \item ``\textit{such that if ... then ...}'' means \\
          \includegraphics{../standalones/build/paper-dotted-small}
          % {your choice of \(\delta\) should make the above ``\textit{if... then...}'' statement true.}
          \includegraphics{../standalones/build/paper-dotted-small}
          \begin{enumerate}
            \item ``\textit{if \(0 < |t - 1| < \delta\)}'' means \\
                  \includegraphics{../standalones/build/paper-dotted-small}
                  % {you restrict the domain of \(f\) to the interval \(1 - \delta < t < 1 + \delta\) but \(t \ne 1\).} 

            \item ``\textit{then \(|f(t) - 1| < \fbox{{1/2}}\)}'' means \\
                  \includegraphics{../standalones/build/paper-dotted-small}
                  % {you must show the values of \(y = f(t)\) is at most \(\fbox{{1/2}}\) away from \(y = 1\) when the domain of \(t\) is restricted according to your choice of \(\delta\).}
          \end{enumerate}
  \end{enumerate}

  Use the GeoGebra worksheet \url{https://www.geogebra.org/calculator/jskb97hf} to help you answer the example. (1) Choose the function \(f(t)\) to be \(t^2\). (2) Choose the limit point \(t_{0}\) to be \(1\). (3) Enter \(1\) for the would-be limit \(L\). (4) Set the \(\epsilon\) slider to value specified above.

  Which value(s) below are suitable choice(s) for \(\delta\)?
  \begin{center}
    (a) \(\delta = 1/3\), \hspace{1em}
    (b) \(\delta = 1/4\), \hspace{1em}
    (c) \(\delta = 1/5\), \hspace{1em}
    (d) \(\delta < 1/5\).
  \end{center}
\end{example}
\clearpage

\begin{example}
  Use the given graph of \(g\) to find a number \(\delta > 0\) such that
  \begin{center}
    if \(0 < |x - 5| < \delta\), then \(|g(x) - 4| < 1\).
  \end{center}

  \begin{figure}[h]  % [h] for here, [ht] for here top, [hb] for here bottom
    \centering
    \includegraphics{../standalones/build/plot_delta_epsilon}
    \label{fig:plot_delta_epsilon}
  \end{figure}
\end{example}
\includegraphics{../standalones/build/paper-dotted-medium}

\begin{mdframed}[style=withref]
  Let \(f(t)\) be a function and \(a\) be a real number. A candidate limit \(L\), also a real number, is the limit of \(f(t)\) as \(t\) approaches \(a\) if
  \begin{center}
    for every \(\epsilon > 0\), there exists a \(\delta > 0\) such that if \(0 < |t - a| < \delta\), then \(|f(t) - L| < \epsilon\).
  \end{center}

  \textbook{\stewart{106}{\fbox{2} Precise Definition of a Limit}}
\end{mdframed}

To {prove} \(\lim_{t \to a} f(t) = L\) using the precise definition of limit,
\includegraphics{../standalones/build/paper-dotted-medium}
\clearpage

\begin{example}
  Prove that \(\lim_{t \to 3} (4t - 5) = 7\). \hfill \textbook{\stewart[blue]{108}{Example 2}}
\end{example}
\includegraphics{../standalones/build/paper-dotted-huge}
\clearpage

\begin{example}
  Prove that \(\lim_{x \to 2} x^{2} = 4\).

    {\footnotesize See \textcolor{blue}{Example 3} on page 109 of the textbook for a similar example.}
\end{example}
\includegraphics{../standalones/build/paper-dotted-huge}
\end{document}
