%! TeX program = xelatex
\documentclass[../main.tex]{subfiles}

\begin{document}
\makelectureweek{2}{Week 2}

\section{Continuity}

\begin{mdframed}[style=withref]
  \textbf{Definition}. A function \(f(x)\) is \emph{continuous at a number \(a\)} if 
  \[
    {\lim_{x \to a} f(x) = f(a).}
  \]
  \textbook{\stewart{115}{\fbox{1} Definition}}
\end{mdframed}
This definition says if you know \(f(x)\) is {continuous} at \(a\), then you can use {direct substitution} to evaluate \(\lim_{x \to a} f(x)\). 

This definition says to check whether a function \(f(x)\) is continuous at a number \(a\), we need to make sure we successfully complete the following three tasks:
\begin{enumerate}[label=(C\arabic*)]
  \item calculate the number {\(f(a)\),}
  \item calculate the number {\(L = \lim_{x \to a} f(x)\), and} 
  \item verify {\(f(a)\) and \(L\) both exist and are the same.}
\end{enumerate}
\bigskip

\begin{minipage}{.6\textwidth}
\begin{example}
  Is \(\ln(x)\) continuous at \(1\)? 
\end{example}
\vspace{1.5in}

\begin{example}
  Is \(\ln(x)\) continuous at \(0\)? 
\end{example}
\vspace{1in}
\end{minipage}
\hfill\vline\hfill
\begin{minipage}{.35\textwidth}
  \includegraphics[width=\textwidth]{standalones/build/plot_ln}
\end{minipage}

If \(f\) is \emph{not} continuous at \(a\), then we say
\begin{center}
  ``\(f\) is {discontinuous} at \(a\)'' \quad or \quad ``\(f\) has a {discontinuity} at \(a\).''
\end{center}

If \(f\) is continuous at every number on an interval \(I\), then we say 
\begin{center}
  ``\(f\) is continuous on \(I\).''
\end{center}
\bigskip

\begin{example}
  Is \(\ln(x)\) continuous on \((-1,1)\)?
\end{example}
\clearpage

In practice, we want to build continuous functions from a collection of known continuous functions. We start with familiar functions.
\begin{mdframed}[style=withref]
  \textbf{Theorem}. The following \emph{types} of functions are continuous at every number in their domain:
  \begin{center}
  \begin{itemize*}
    \item polynomials \phantom{------}
    \item rational functions \phantom{------}
    \item root functions
  \end{itemize*}
  \end{center}

  \begin{minipage}{.45\textwidth}
  \begin{itemize}
    \item trigonometric functions
    \item exponential functions
  \end{itemize}
  \end{minipage}
  \begin{minipage}{.45\textwidth}
  \begin{itemize}
    \item inverse trigonometric functions
    \item logarithmic functions
  \end{itemize}
  \end{minipage}

  \textbook{\stewart{120}{\fbox{7} Theorem}}
\end{mdframed}

Another way to say ``continuous at every number in their domain'' is 
\begin{center}
  ``continuous {everywhere} in their domain.''
\end{center}

\section{More on discontinuity}

There are three \emph{types} of discontinuities.

\begin{figure}[h]  % [h] for here, [ht] for here top, [hb] for here bottom
\centering
\includegraphics{../standalones/build/plot_discontinuity_examples}
\caption{Three types of discontinuity}
\label{fig:discontinuities}
\end{figure}
\clearpage

\begin{mdframed}[style=withref]
  \textbf{Definition}. We say a function \(f\) is 
  \begin{itemize}
    \item continuous from the {MMMMM} at a number \(a\) if \(\lim_{{x \to a^{+}}} f(x) = f(a)\).
    \item continuous from the {MMMMM} at a number \(a\) if \(\lim_{{x \to a^{-}}} f(x) = f(a)\).
  \end{itemize}

  \textbook{\stewart{117}{\fbox{2} Definition}}
\end{mdframed}

\begin{example}
  Which function in Figure~\ref{fig:discontinuities} is continuous from the left at a number but discontinuous from the right at the same number?
\end{example}
\vspace{2in}

\begin{example}
  If a function \(f\) is continuous at \(a\), is it
  \begin{itemize}
  \item continuous from the left at \(a\)?
  \item continuous from the right at \(a\)?
  \end{itemize}
\end{example}
\vfill

\section{Properties of Continuous functions}
We now study the effect of algebraic operations \(+, -, \times, \div, \circ\) on continuous functions.

\begin{mdframed}[style=withref]
  \textbf{Theorem}. If \(f\) and \(g\) are {both continuous at a number \(a\)}, then the following are all {continuous at \(a\)}
  \[
    f + g, \qquad f - g, \qquad cf \text{ where \(c\) is a constant}, \qquad fg, \qquad \frac{f}{g} \text{ if } g(a) \ne 0.
  \]
  
  \textbook{\stewart{118}{\fbox{4} Theorem}}
\end{mdframed}
This theorem says if you ``build'' functions using \(+, -, \times, \div\) from continuous functions, then you are guaranteed to get a continuous functions. 

\clearpage
\begin{example}
  Is \(\sin(x) + 2e^{x}\) continuous at \(0\)?
\end{example}
\vspace{1.5in}

\begin{example}
  Is \(\sqrt{u}\ln(u)\) continuous on the interval \((0,1)\)?
\end{example}
\vspace{1in}

\begin{example}
  Is \(g_{1}(x) = \frac{x^{2} - 1}{x-1}\) continuous on the interval \((1,2]\)?
\end{example}
\vspace{1.5in}

In high school, we learned to interpret expressions such as \((x-1)^{2}\) as shifting the graph of \(f(x) = x^{2}\) to the right by \(1\) because
\[
  (x-1)^{2} = f(x-1).
\]

This is an example of a general operation called {function composition}, denoted by \(\circ\). 

The symbol \(f \circ g\) means ``plug \(g\) into \(f\).'' In symbols, 
\[
  f \circ g = {f(g).}
\]
\begin{example}
  Let \(\alpha(x) = \sin(x)\) and let \(\beta(x) = x^{2} + \sqrt[3]{x - 1} + 1\). What is \(\alpha \circ \beta\)? What is \(\beta \circ \alpha\)?
\end{example}
\clearpage

\begin{mdframed}[style=withref]
  \textbf{Theorem}. If \(g\) is continuous at \(a\) and \(f\) is continuous at \(g(a)\), then \(f \circ g\) is continuous at \(a\).

  \textbook{\stewart{121}{\fbox{9} Theorem}}
\end{mdframed}

\begin{example}
  Is \(\ln(\sin(x))\) continuous at \(0\)?
\end{example}
\vspace{1.5in}

\begin{example}
  Is \(\ln(\sin(x))\) continuous on \((0,\pi)\)?
\end{example}
\vspace{1.5in}

\begin{example}
  Is \(\ln(\sin(x))\) continuous on \((\pi,2 \pi)\)?
\end{example}
\vspace{1.5in}

\section{Limits and Continuity}
Whenever a function \(f(x)\) is {continuous} at \(a\), then you can evaluate \(\lim_{x \to a} f(x)\) by direct substitution.

\begin{example}
  Evaluate \(\lim_{x \to 2} \left( - x^{3} + 7  \right)^{100}\) if it exists.
\end{example}
\clearpage

The next theorem says being able to recognize function composition is useful.
\begin{mdframed}[style=withref]
  \textbf{Theorem}. If \(f\) is continuous at \(b\) and \(\lim_{x \to a} g(x) = b\), then 
  \[
    {\color{main} \lim_{x \to a}} {\color{attn}f({\color{black}g(x)})} = {\color{attn}f \left( {\color{main} \lim_{x \to a}} {\color{black} g(x) }\right)}.
  \]

  \textbook{\stewart{120}{\fbox{8} Theorem}}
\end{mdframed}
Notice the ``inside function'' \(g(x)\) does not need to be continuous at \(a\).

\begin{example}
  Evaluate \(\lim_{x \to 1} \sin\left( \frac{x^{2}-1}{x-1} \right)\) if it exists.
\end{example}
\vfill
\clearpage

\begin{example}
  Use \textcolor{red}{\fbox{8} Theorem} to prove the power and root limit laws. 

  Suppose \(\lim_{x \to a} f(x)\) exists. Then
  \begin{enumerate}[label=(\arabic*), start=6]
    \item \(\lim_{x \to a} [f(x)]^{n} = \left[ \lim_{x \to a} f(x) \right]^{n}\) \hfill (power)
    \item \(\lim_{x \to a} \sqrt[n]{f(x)} = \sqrt[n]{\lim_{x \to a} f(x)}\), \quad if \(n\) is even, we assume {\(\lim_{x \to a} f(x) > 0\)} \hfill (root)
  \end{enumerate}
\end{example}

\clearpage

\section{The Intermediate Value Theorem}

(Work in small groups) Suppose you went to Montreal for a day trip. If your thermometer reads \(5^{\circ}\)C at \(7 \text{ am}\) and \({10}^{\circ}\)C at \(5\text{ pm}\), does it \underline{have to} be \({8}^{\circ}\)C \emph{sometime} between 7 am and 5 pm?

\vspace{1in}

Imagine and draw the graph of a temperature function \(H(t)\) that fits the above data. Use the graph to support your reasoning. Compare graphs within your group. 

\begin{figure}[!h]  % [h] for here, [ht] for here top, [hb] for here bottom
  \centering
  \includegraphics[width=\textwidth]{standalones/build/plot_ivt_motivation}
  \label{fig:label}
\end{figure}

What \underline{mathematical property} of the temperature function \(H(t)\) \emph{guides} you to think this way?
\vspace{1in}

Can you tell \underline{exactly} when (down to seconds) the temperature reached \(8^{\circ}\)C?
\clearpage

\begin{mdframed}[style=withref]
  \textbf{Intermediate Value Theorem} (IVT). Suppose \(f\) is continuous on a closed interval \([a,b]\) and let \(N\) be any number between \(f(a)\) and \(f(b)\), where \(f(a) \ne f(b)\). Then there exists a number \(c\) in \((a,b)\) such that \(f(c) = N\).

  \textbook{\stewart{122}{\fbox{3} The Intermediate Value Theorem}}
\end{mdframed}
When is IVT useful? You wish to {solve an equation \(f(x) = N\).}
\vspace{1cm}

How do you use IVT? You need to be able to successfully complete all three tasks: 
\begin{enumerate}[label=(IVT \arabic*)]
  \item Find or guess {a \emph{closed} interval \([a,b]\)}
  \item Verify {\(f\) is continuous on \([a,b]\)}
  \item Verify {\(N\) is between \(f(a)\) and \(f(b)\)}
\end{enumerate}
\bigskip

\begin{example}
  Show the function \(f(t) = t^{3} - t^{2} + 1\) has a root between \(-1\) and \(1\).
\end{example}
\vspace{2.5in}

\begin{example}
  Show the function \(f(t) = \sin(t) - t^{3}\) satisfy \(f(c) = -100\) for some number \(c\). 
\end{example}
\clearpage

\section{Limits at infinity and horizontal asymptotes}

This is the \emph{intuition} of a \textbf{limit at infinity}. There is an apple a metre away from you. You start by taking a step \(1/2\) metre away towards the apple. Then each subsequent step is half the length of the previous one. 

\begin{center}
  \begin{tikzpicture}[scale=0.8]
    \draw[dotted, thin] (0,0) grid (15,9);
    \begin{scope}[shift={(1,1)}]
      \draw[->] (-1,0) -- (14.2,0) node[right] {\footnotesize steps};
      \draw[->] (0,-1) -- (0,8.2) node[above right] {\footnotesize distance to apple};
      \node[left] at (0,8) {\footnotesize \(1\)};
      \node[left] at (0,4) {\footnotesize \(1/2\)};
      \node[left] at (0,2) {\footnotesize \(1/4\)};
      \node[left] at (0,1) {\footnotesize \(1/8\)};
      \node[below] at (3,0) {\footnotesize \(1\)};
      \node[below] at (6,0) {\footnotesize \(2\)};
      \node[below] at (9,0) {\footnotesize \(3\)};
      \node[below] at (12,0) {\footnotesize \(4\)};
    \end{scope}
  \end{tikzpicture}
\end{center}
\begin{enumerate}[label=(\alph*)]
  \item Can you get as close to the apple as you want?
    \vspace{1em}
  \item If you are allowed to take ``infinitely'' many steps, do you \emph{expect} to reach the apple?
    \vspace{1em}
\end{enumerate}
\vfill

\begin{mdframed}[style=withref]
  \textbf{Definition}. When a limit at infinity exists and is equal to a number \(L\), we give the \emph{horizontal} line \(y=L\) a special name, called a {horizontal asymptote.}

  \textbook{\stewart{128}{\fbox{3} Definition}}
\end{mdframed}
\clearpage

\begin{example}
  Recall the inverse tangent function \(\arctan(x)\).
  % \[ y = \arctan(x) \qquad \text{if }\qquad \tan(y) = x \quad \text{and} \quad -\frac{\pi}{2} < y < \frac{\pi}{2}.\]

  \includegraphics{../standalones/build/plot_arctan}
\end{example}
\vspace{1in}

\begin{example}
  Different types of limits at infinite that do not exist.

  \begin{minipage}{.4\textwidth}
    \centering
    \includegraphics{../standalones/build/plot_sin}
  \end{minipage}
  \bigskip

  \begin{minipage}{.4\textwidth}
    \centering
    \includegraphics{../standalones/build/plot_cos}
  \end{minipage}
  \bigskip

  \begin{minipage}{.4\textwidth}
    \centering
    \includegraphics{../standalones/build/plot_polynomials}
  \end{minipage}
  \bigskip
\end{example}

\begin{example}
  Evaluate \(\lim_{x \to \infty} (x^{2} - x)\) if it exists.
\end{example}
\clearpage

\begin{mdframed}[style=withref]
  \textbf{Definition}. Let \(f\) be a function defined on some interval \((a,\infty)\). If \(L\) is a number such that \(f(x)\) can be made arbitrarily close to \(L\) by requiring \(x\) to be sufficiently large, then we write
  \[ 
    \lim_{x \to \infty} f(x) = L.
  \]
  Otherwise, the limit of \(f(x)\), as \(x\) approaches (\emph{positive}) infinity, does not exists.

  \textbook{\stewart{127}{\fbox{1} Intuitive Definition of a Limit at Infinity}}
\end{mdframed}

Our main tool to calculate limits at infinity is the following theorem.
\begin{mdframed}[style=withref]
  \textbf{Theorem}. Suppose \(r > 0\) is a rational number. Then
  \[
    \lim_{x \to \infty} \frac{1}{x^{r}} = 0 \quad\text{and}\quad \lim_{x \to -\infty} \frac{1}{x^{r}} = 0.
  \]

  The second part only applies if \(x^{r}\) is well-defined for all number \(x\).

  \textbook{\stewart{130}{\fbox{5} Theorem}}
\end{mdframed}

\begin{example}
  Let \(f(x) = \frac{x^{2}-1}{x^{2} + 1}\). 
  Evaluate \(\lim_{x \to \infty} f(x)\) and \(\lim_{x \to -\infty} f(x)\) if they exist. 

  What is (or are) the horizontal asymptote(s) of \(f(x)\)?
\end{example}
\vfill

\begin{example}
  Evaluate \(\lim_{x \to -\infty} \frac{3x - x^{7}}{- x^{5} + 1}\) if it exists.
\end{example}
\vfill
\clearpage

\begin{example}
  Evaluate \(\lim_{x \to -\infty} \frac{\sqrt[3]{x^{3} + x}}{-x^{2} - 1}\) if it exists.
\end{example}
\vspace{6in}

\begin{example}
  Evaluate \(\lim_{x \to -\infty} \sqrt{e^{x}}\) if it exists.
\end{example}
\vfill
\clearpage

\begin{example}
  Evaluate \(\lim_{x \to \infty} (\sqrt{4x^{2} - 5} - 2x)\) if it exists.
\end{example}
\clearpage


\section{A Common Mistake}
When we \emph{simplify} expressions in quotients, we must be careful \textbf{not to change the domain} of the function. 


\begin{example}
  Find the mistake(s) in the following \emph{incorrect} reasoning about the continuity of \(f(x) = \frac{(3x-1)(x-2)}{(x-2)}\).
  \bigskip

  \begin{quote}
    \itshape
    We can cancel the common factor. So
    \[
      f(x) = \frac{(3x - 1)\cancel{(x - 2)}}{\cancel{(x-2)}} = 3x - 1.
    \]
    Because \(3x - 1\) is a polynomial, we can conclude \(f(x)\) is continuous at every number, including \(x=2\).
  \end{quote}
\end{example}

\end{document}
