%! TeX program = xelatex
\documentclass[../main.tex]{subfiles}

\begin{document}
\makelectureweek{11}{Week 11}


\section{Indefinite Integrals}

% {{{1
Let \(f(x)\) be a {continuous} function. An \emph{indefinite integral} of \(f(x)\) is 
\vspace{3in}

\begin{example}
  Verify that \(\int \sin(\theta) \cos(\theta) \; d\theta = \frac{1}{2} \cos^{2}(\theta) + C\).
\end{example}
\vspace{2in}
\bigskip

\faComments{} State the Fundamental Theorem of Calculus (Part 2) in terms of \emph{indefinite integrals}. 
\vspace{2in}

\clearpage

% properties of the indefinite integral {{{2
\begin{mdframed}[style=simple]
  Properties of indefinite integrals. Let \(f,g\) be continuous functions and \(c\) be a constant.
  \begin{enumerate}[label=(\alph*)]
    \item  \(\int c f(x) \;dx = \)
    \item  \(\int f(x) + g(x) \;dx = \)
    \item  \(\int f(x) - g(x) \;dx = \)
  \end{enumerate}
\end{mdframed}
\faStar{} Make sure you know Table~\ref{table:indefinite-integrals} well.

\begin{example}
  Evaluate \(\int 6 \cdot 2^{x} + \frac{1}{10} \sqrt[5]{x} + \frac{\sqrt{x}}{x^{2}} \; dx\).
\end{example}
% 2}}}
% 1}}}

\clearpage

Fill in the following table.

\begin{table}[h!]  % [h] for here, [ht] for here top, [hb] for here bottom 
\centering
\includegraphics{../standalones/build/table_indefinite_integrals}
\caption{Table of Indefinite Integrals}
\label{table:indefinite-integrals}
\end{table}

\clearpage

\section{The Net Change Theorem}
% {{{1
\begin{mdframed}[style=withref]
  The integral of a rate of the change is the \emph{net change}:
  \[
    \int_{a}^{b} F'(x) \; dx = F(b) - F(a).
  \]

  \textbook{\stewart{412}{Net Change Theorem}}
\end{mdframed}
\vspace{1in}
% integration is the accumulation of change.

% examples {{{2
\begin{example}
  The linear density of a rod of length \(4\) metres is given by \(\rho(x) = 9 + \sqrt{2} x\) measured in kilograms per metre, where \(x\) is measure in metres from one end of the rod. Find the total mass of the rod. 
\end{example}
\clearpage

\begin{example}
  Suppose an object moves along a straight line with position function \(s(t)\). 
 
  \begin{enumerate}[label=(\alph*)]
  \item The velocity function is \(v(t) = s'(t)\). Interpret
    \[
      \int_{t_{1}}^{t_{2}} v(t) \;dt = s(t_{2}) - s(t_{1})
    \]
    in terms of displacement.
    \vspace{1in}

  \item Let \(v(t)\) be defined as in part (a). What is the physical interpretation of the following integral?
    \[
      \int_{t_{1}}^{t_{2}} |v(t)| \; dt.
    \]
    \vspace{2in}
  \end{enumerate}
\end{example}

\begin{example}
  Let \(x\) be a physical quantity whose unit is \emph{foot}. Let \(a\) be a quantity whose unit is \emph{pound per foot}.
  \begin{enumerate}
  \item What is the unit for \(\frac{da}{dx}\)?
    \vspace{1in}
  \item What is the unit for \(\int_{0}^{1} a(x) \;dx\)?
    \vspace{1in}
  \end{enumerate}
\end{example}
% 2}}}

% 1}}}
\clearpage

\section{The Substitution Rule}

\begin{mdframed}[style=withref]
  If \(u = g(x)\) is a differentiable function whose range is an interval \(I\) and \(f\) is continuous on \(I\), then
  \[
    {\int f(g(x)) g'(x) \;dx = \int f(u) \;du}
  \]
  \textbook{\stewart{420}{\fbox{4} The Substitution Rule}}
\end{mdframed}
\vspace{2in}

\begin{example}
  Evaluate \(\int \frac{\ln(x)}{x} \;dx\).

\end{example}

\clearpage
\begin{mdframed}[style=withref]
  If \(g'\) is continuous on \([a,b]\) and \(f\) is continuous on the range of \(u = g(x)\), then
  \[
    {\int_{a}^{b} f(g(x)) g'(x) \;dx = \int_{u(a)}^{u(b)} f(u) \;du}
  \]
  \textbook{\stewart{423}{The Substitution Rule for Definite Integrals}}
\end{mdframed}
\vspace{1in}

\begin{example}
  Evaluate \(\int_{1}^{e} \frac{\ln(x)}{x} \;dx\) in two different ways.

  \begin{enumerate}
  \item Use the Fundamental Theorem of Calculus.
    \vspace{2in}

  \item Use the Substitution Rule for Definite Integral directly.

  \end{enumerate}
\end{example}

\clearpage
\begin{example}
  Evaluate \(\int \frac{x^{2}}{x^{3} + 1} \;dx\).
\end{example}
\clearpage

\begin{example}
  Evaluate \(\int_{-1}^{-2} \frac{(x+1)^{2}}{(x+1)^{3} + 12} \;dx\).
\end{example}
\clearpage

\end{document}
