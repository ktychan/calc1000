%! TeX program = xelatex
\documentclass[../main.tex]{subfiles}

\begin{document}
\makelectureweek{13}{Review Discussions}

You can answer the questions below to get an \emph{overview} of the course. Discuss the answer of these questions with your study partners. This document is just a guide, \emph{not} an exam outline. \emph{Not} every single concept taught in the course is included in here. 

To effectively review for the final exam, you should still do the practice problems in the weekly contents. 

Remember to ground your understanding in definitions and theorems.

\section{Limits}

\begin{enumerate}[label={\thesection.\arabic*.}, align=left]
  \item How do we graphically estimate the limit of a function?

  \item What is an one-sided limit? What are the notations for one-sided limits?

  \item What is the relation between one-sided limits and two-sided limits?

  \item What does it mean for \(\lim_{x \to a} f(x)\) to exist?

  \item How can \(\lim_{x \to a} f(x)\) fail to exist? Repeat the question for one-sided limits.

  \item How do we evaluate \(\lim_{x \to a} f(x)\) if \(f(x)\) is continuous?

  \item What is an infinite limit? 

  \item Find is a function \(f(x)\) and a number \(a\) such that \(\lim_{x \to a} f(x)\) is an infinite limit?

  \item What is the \emph{definition} of a vertical asymptote? 

  \item Given the graph of a function \(f(x)\), how do you find a vertical asymptote of \(f(x)\)?

  \item Given an algebraic expression of a function \(f(x)\), how do you find a vertical asymptote of \(f(x)\)?

  \item State the limit laws for addition, subtraction, multiplication by a constant, multiplication, and quotient. Remember to include all conditions of each limit law. 

  \item How can a quotient law fail to be useful?

  \item If we cannot use the quotient law to evaluate a function, what can we do? 

  \item What is the statement of the Squeeze Theorem?

  \item What is the delta-epsilon definition of limits? 

  \item What is a limit \emph{at} infinity? 

  \item What is the \emph{definition} of a vertical asymptote? 

  \item Given the graph of a function \(f(x)\), how do you find a vertical asymptote of \(f(x)\)?

  \item Given an algebraic definition of a function \(f(x)\), how do you find a vertical asymptote of \(f(x)\)?

  \item Give an example of a function \(f(x)\) such that \(f(x)\) has no horizontal asymptote.

  \item How many distinct horizontal asymptotes can a function \(f(x)\) have?

  \item State L'H\^opital's rule. 
\end{enumerate}

\section{Continuity}

\begin{enumerate}[label=\thesection.\arabic*, align=left]
  \item What is the definition of ``a function \(f(x)\) is continuous at a number \(a\)?''

  \item How can we check if a function \(f(x)\) is continuous at a number \(a\) from the definition?

  \item What is the definition of ``a function \(f(x)\) is continuous on an interval \(I\)?''

  \item What is a discontinuity?
    
  \item What are the three types of discontinuities? 

  \item If you know both \(f,g\) are both continuous at a number \(a\), what can you say about the continuity of \(f+g\), \(f-g\), \(cf\) where \(c\) is a constant, \(fg\), and \(f/g\) at \(a\)?
    \begin{enumerate}
    \item Compare properties of continuous function to limit laws. What do you observe?
    \end{enumerate}

  \item Fill in the blanks. If \(g\) is \underline{\phantom{XXXXXXX}} at \(a\) and \(f\) is \underline{\phantom{XXXXXXX}} at \underline{\phantom{XXX}}, then \(f \circ g\) is continuous at \(a\).

  \item Fill in the blanks. If \(\lim_{x \to a}g(x) = b\) and \(f\) is continuous at \(b\), then \underline{\phantom{XXXXXXXXXXXX}}.

  \item What is the statement of the Intermediate Value Theorem?
\end{enumerate}

\section{Derivatives}
\begin{enumerate}[label=\thesection.\arabic*, align=left]
  \item What is the definition of the derivative of a function \(f(x)\) at \(a\)? 

  \item What is the geometric interpretation of \(f'(a)\) where \(a\) is a constant?

  \item Given the graph of a function \(f(x)\) and a number \(a\), how can you determine if \(f'(a)\) exist?

  \item Given an algebraic expression of a function \(f(x)\) and a number \(a\), how can you determine if \(f'(a)\) exist?

  \item If \(x\) and \(y\) are physical quantities related by \(y = f(x)\) where \(f\) is a function, then what is the physical interpretation of \(f'(x)\)?

  \item How can a function fail to be differentiable at a number \(a\)?

  \item What does \(f^{(n)}\) mean? If you are given \(f^{(n)}\), how can you calculate \(f^{(n+m)}\) where \(m\) is a positive integer.

  \item State differentiation rules. 
    \begin{enumerate}
      \item Compare differentiation rules to limit laws. 
    \end{enumerate}

  \item State the product rule.

  \item State the quotient rule.

  \item State the chain rule in both the prime notation and in Leibniz notation. 

  \item What is \(\lim_{x \to 0} \frac{\sin(x)}{x}\)? \(\lim_{x \to 0} \frac{\cos(x) - 1}{x}\)?

  \item Prove \(\frac{d}{dx} \sin(x) = \cos(x)\) and \(\frac{d}{dx} \cos(x) = -\sin(x)\).

  \item How can we differentiate an \emph{equation}? For example, given \(x^{2} + y^{2} = 1\), how can we find \(\frac{dy}{dx}\)?

  \item Assume \(b > 0\) and \(b \ne 1\). Find \(\frac{d}{dx} \log_{b}(x)\) by implicitly differentiating \(b^{y} = x\) with respect to \(x\).

  \item Assume \(y\) is a differentiable function of \(x\). Given \(f(y) = x\), find \(y'\).

  \item Assume \(f\) is differentiable at \(a\). What does ``linearize \(f(x)\) at a number \(a\)'' mean?

  \item Assume \(y\) is mathematical model of some quantity at time \(t\). What is the solution to \(\frac{dy}{dt} = kt\) where \(k\) is a constant? Is the solution a number, a function, or a family of functions?

  \item What is the general problem solving strategy for exponential growth or decay problems?

  \item If two quantities are related by an equation, what differentiation technique can be used to express the rate of change of one quantity in terms of the rate of change of the other?

  \item What is the general problem solving strategy for related rates problems?

  \item What are local extreme values of a function?

  \item What are absolute (or global) extreme values of a function?

  \item True or false? 
    \begin{enumerate}
    \item An extreme value is a number in the domain of a function.
    \item An extreme value is a number in the range of a function.
    \end{enumerate}

  \item Can local extreme values be attained at the endpoints of a closed interval?

  \item Describe a function \(f\) such that \(f\) has no local extreme values but has global extreme values.

  \item Describe a function \(f\) such that \(f\) has local extreme values but no global extreme values.

  \item How many absolute maximum values can a function have if it exists? How many absolute minimum values can a function have if it exists? 

  \item True or false? If a function \(f\) has an absolute extreme value \(m\), then there is \emph{exactly one} number \(c\) such that \(f(c) = m\). 

  \item What is a critical number of a function.

  \item True or false? Let \(f\) be a function. To find \emph{all} critical numbers of \(f\), we just need to solve for \(x\) in \(f'(x) = 0\). 

  \item What the definition of ``\(f\) is concave upward on an interval \(I\)?''

  \item What the definition of ``\(f\) is concave downward on an interval \(I\)?''

  \item What is an inflection point?

  \item What is the Increasing/Decreasing Test?

  \item What is the Concavity Test?

  \item What is the Closed Interval Method?

  \item What is the First Derivative Test?

  \item What is the Second Derivative Test?

  \item Compare the Closed Interval Method, the First Derivative Test, and the Second Derivative Test. If you are asked to find (local or global) extreme values of a function \(f(x)\), how do you decide which method to use?

  \item Suppose a function \(f\) is defined on a closed interval \([1,5]\). You are asked to find \emph{the global minimum}. You use the first derivative test to find two local minimums attained at \(x=3\) and \(x=4\). You check that \(f(3) > f(4)\). Can you, or can you not, conclude that \(f(3)\) is the global minimum? If not, what values of \(f\) do you have to compare \(f(3)\) with?

  \item True or false? If \(P = (x,y)\) is an inflection point on the graph of a function \(f\), then \(f'(x) = 0\).

  \item State the Mean Value Theorem and Rolle's Theorem.

  \item Assume the Rolle's Theorem is true. Prove the Mean Value Theorem from Rolle's Theorem.

  \item What information of a function \(f(x)\) do you need to sketch its graph? What are the steps to sketch the graph of a function \(f(x)\)?

  \item How can symmetries (even, odd, periodic) help you simplify the task of sketching a curve?

  \item What is the general problem solving strategy for optimization problems?
\end{enumerate}

\section{Integration}
\begin{enumerate}[label=\thesection.\arabic*, align=left]
  \item What is an antiderivative of a function \(f(x)\)?
  \item What is the most general antiderivative of a function \(f(x)\)?
  \item The domain of \(1/x\) consists of all non-zero real numbers. What is an antiderivative of \(1/x\)?

  \item What is a Riemann sum? What does \(x^{*}_{i}\) mean in a Riemann sum?

  \item What is a definite integral?

  \item What is the geometric interpretation of a definite integral?

  \item True or false? A definite integral is always a non-negative number. 

  \item True or false? If \(f\) is integrable from \(a\) to \(b\), then \(\int_{a}^{b} f(x) dx = \int_{a}^{b} |f(x)| dx\).

  \item What are the properties of definite integrals?
    \begin{enumerate}
    \item Compare properties of definite integrals to limit laws. 
    \end{enumerate}

  \item Find a non-zero function \(f(x)\) such that \(\int_{0}^{1} f(x) dx = 0\).
  \item Find a function \(f(x)\) such that \(\int_{0}^{1} f(x) dx = 1\).
  \item Find a function \(f(x)\) such that \(\int_{0}^{1} f(x) dx = -1\).

  \item Is it true that \(\int_{a}^{b} f(x)g(x) dx = \int_{a}^{b} f(x) dx \cdot \int_{a}^{b} g(x) dx\)?
  \item Is it true that \(\int_{a}^{b} \frac{f(x)}{g(x)} dx = \frac{\int_{a}^{b} g(x) dx}{\int_{a}^{b} f(x) dx}\)?

  \item List all \(9\) properties of the definite integral.

  \item One of the properties of the definite integral states that \(\int_{a}^{b} f(x) \;dx + \int_{b}^{c} f(x) \;dx = \int_{a}^{c} f(x) \;dx\). Is it required that \(a \le b \le c\)? 

  \item State the Fundamental Theorem of Calculus (both parts). 
      
  \item What is an indefinite integral? 

  \item What do the notations \([F(x)]_{a}^{b}\) and \(F(x)]_{a}^{b}\) mean?

  \item Compare properties of the indefinite integral to the properties of the definite integrals. 

  \item State the Net Change Theorem.

  \item State the Substitution Rule and the Substitution Rule for definite integrals.
\end{enumerate}
\end{document}
