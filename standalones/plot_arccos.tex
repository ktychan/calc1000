\documentclass{standalone}
\usepackage[dvipsnames]{xcolor}
\usepackage{pgfplots}
\pgfplotsset{
  compat=1.18, 
  trig format=rad, 
  ticklabel style = {font=\tiny},
  axis equal image,
}

% see https://tex.stackexchange.com/questions/18359/plotting-an-implicit-function-using-pgfplots
% for implicit plots using gnuplot
\begin{document}
\begin{tikzpicture}
  \begin{axis}[
    xlabel={\footnotesize \(x\)},
    ylabel={\footnotesize \(y\)},
    ymin={-pi/8},
    ymax={pi+pi/8},
    xmin={-1.25},
    xmax={1.25},
    axis x line={middle},
    axis y line={middle},
    xlabel style={at={(ticklabel* cs:1)}, anchor=west},
    ylabel style={at={(ticklabel* cs:1)}, anchor=south},
    xtick={-1,0,1},
    ytick={-pi/4, 0, pi/4, pi/2, 3*pi/4, pi, 5*pi/4},
    % yticklabels={,,},
    yticklabels={,,,\(\frac{\pi}{2}\),,\(\pi\),,},
    ]

    \addplot[ForestGreen, thick, smooth, samples=1000, domain=-1:1] {acos(x)};
    % \addplot[ForestGreen, thick, smooth, samples=1000, domain={-pi/2}:{pi/2}] {sin(x)};
    % \addplot[ForestGreen, thick, dotted, smooth, samples=1000, domain={-pi-pi/4}:{-pi/2}] {sin(x)};
    % \addplot[ForestGreen, thick, dotted, smooth, samples=1000, domain={pi/2}:{pi+pi/4}] {sin(x)};
    \node[right] at (-1+0.25, {pi/2+1}) {\footnotesize \(y = \arccos(x)\)};
  \end{axis}
\end{tikzpicture}
\end{document}
