\documentclass[tikz]{standalone}
\input{../typography.tex.preamble}
\input{../tikz.tex.preamble}
\usepackage{cancel}
\usetikzlibrary{decorations.pathreplacing}

\begin{document}

\begin{tikzpicture}[declare function = {
  f(\x) = \x^3/12 - \x^2/6 - \x + 1;
  a = -2/3*(sqrt(10) - 1);
  b = +2/3*(sqrt(10) + 1);
  c = 4;
}]
  \begin{axis}[
    axis lines = middle, % boxed, middle
    axis equal image,
    %
    % domain and range
    %
    % xmin={}, xmax={},
    % ymin={}, ymax={},
    enlargelimits=true,
    %
    % axis labels
    %
    xlabel={\(x\)}, xlabel style={anchor=west},
    ylabel={\(y\)}, ylabel style={anchor=south},
    label style={at={(ticklabel* cs:1)}, font={\tiny}},
    %
    % ticks
    %
    xtick=\empty, % xticklabels={},
    ytick=\empty, % yticklabels={},
    ticklabel style={font=\tiny},
    %
    % grid
    % none, major, minor, both
    grid=none, grid style={gray!20},
    % minor tick num=1, 
    % minor grid style={gray!20},
    % 
    % plot parameters
    %
    smooth, samples=100, no markers,
    title={Graph of \(f(x)\)}
    ]

    % attention
    % \draw[main, dotted, fill=attn!10] (-4, 4) rectangle (6, 3);
    % \node at (axis cs:1,3.5) {\sffamily\footnotesize because \(f'(c)\) does not exist};
    % \node[below] (C) at (axis cs:a,1) {\sffamily\footnotesize \(c\)};
    % \draw[attn, ->] (axis cs:1,3) -- (C);

    % \addplot[thick, domain=-5:4, dotted] {f(x)};
    \addplot[thick, domain=-4:5] {f(x)};
    \addplot[thick] coordinates {(5, {f(5)}) (7, {f(5)})};
    \draw[fill=white] (7, {f(5)}) circle[radius=2pt];
    \draw[fill=black] (7, 3) circle[radius=2pt];
    \addplot[thick] coordinates {(7, 3) (9, 4)};
    \addplot[thick] coordinates {(9, 4) (10, 2)};
    \draw[fill=black] (10, 2) circle[radius=2pt];

    % % smooth local max
    % \addplot[main] coordinates {(a,0.2) (a,-0.2)};
    % \node[main, below] at (axis cs:a,-0.2) {\footnotesize\sffamily CN};
    % \fill[supp] ({a}, {f(a)}) circle[radius=0.2];
    % \addplot[supp, dashed] coordinates { (a-2, {f(a)}) ({a+2}, {f(a)}) };
    %
    % % smooth local min
    % \addplot[main] coordinates {(b,-0.2) (b,0.2)};
    % \node[main, above] at (axis cs:b,0.2) {\footnotesize\sffamily CN};
    % \fill[supp] (b, {f(b)}) 
    %   circle[radius=0.2];
    % \addplot[supp, dashed] coordinates { (b-2, {f(b)}) ({b+2}, {f(b)}) };
    %
    % % flat
    % \addplot[main] coordinates {(5,0.2) (5,-0.2)};
    % \addplot[main] coordinates {(7,0.2) (7,-0.2)};
    %
    % \node[main, above] at (axis cs:5,0) {\tiny\sffamily CN};
    % \node[main, above] at (axis cs:7,0) {\tiny\sffamily CN};
    % \draw[main, decorate, decoration={brace, raise=0.2cm, mirror}] (5,0) -- (7,0);
    % \node[main, below] at (axis cs:6,-0.6) {\footnotesize\sffamily all CNs};
    %
    % % absolute max
    % \addplot[main] coordinates {(9,0.2) (9,-0.2)};
    % \node[main, below] at (axis cs:9,-0.2) {\footnotesize\sffamily CN};
    % % \draw[supp] (axis cs:9,4) circle[radius=0.5];
    % \fill[supp] (axis cs:9,4) circle[radius=0.2];
    % \addplot[supp, dotted] coordinates { (9,4) (0,4) } node[above right]  {\tiny\sffamily its \(y\)-value is the abs max};
    %
    % \addplot[warn] coordinates {(10,0.2) (10,-0.2)};
    % \node[warn, above] at (axis cs:10,0.2) {\footnotesize\sffamily \xcancel{CN}};
  \end{axis}
\end{tikzpicture}
\end{document}
