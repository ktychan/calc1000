%! Tex Program = lualatex
\documentclass[14pt, draft]{beamer} 
% \usetheme{metropolis} 
\everymath{\displaystyle} 
\setbeamersize{text margin left=.5cm} 
\setbeamersize{text margin right=.5cm} 
\beamertemplatenavigationsymbolsempty{}

\usepackage{xcolor, ifdraft}

\newcommand{\toinstructors}[1]{{
    \scriptsize \color{magenta} {#1} \newline 
  Remove this comment by removing the draft option from \textbackslash{}documentclass.}
}

\ifdraft{}{
  \renewcommand{\toinstructors}[1]{}
}

\begin{document} 

% \begin{frame}
%   Question
%
%   \medskip
%   \begin{itemize} \setlength\itemsep{2ex}
%     \item[(a)] 
%     \item[(b)] 
%     \item[(c)] 
%     \item[(d)] There is not enough information.
%     \item[(e)] I am not sure.
%   \end{itemize} 
% \end{frame}

\section{Icebreaker}

\begin{frame}
  Is a hot dog a sandwich?
  
  \medskip
  \begin{itemize} \setlength\itemsep{2ex}
    \item[(a)] Yes!
    \item[(b)] Yes?
    \item[(c)] No!
    \item[(d)] No?
    \item[(e)] I no longer know what a sandwich is.
  \end{itemize} 
\end{frame}


\section{Review}
\begin{frame}
  How can we calculate \(y = \log_{2}(8)\)?
  \medskip
  \begin{itemize} \setlength\itemsep{2ex}
    \item[(a)] Solve \(2^{y} = 8\).
    \item[(b)] Solve \(2^{8} = y\).
    \item[(c)] Calculate \(\frac{\ln(8)}{\ln(2)}\). 
    \item[(e)] 
    \item[(d)] I am not sure.
  \end{itemize} 
\end{frame} 

\section{Limits}
\begin{frame}
  Suppose \(\lim_{x \to 3} f(x)\) exists. Can we find \(f(3)\)?
  \medskip
  \begin{itemize} \setlength\itemsep{2ex}
    \item[(a)] \(f(3) = \lim_{x \to 3} f(x)\).
    \item[(b)] \(f(3)\) is undefined.
    \item[(c)] There is not enough information.
    \item[(d)] 
    \item[(d)] I am not sure.
  \end{itemize} 
\end{frame} 

\begin{frame}
  True or false? We can \emph{always} apply the quotient law to evaluate \(\lim_{x \to a} \frac{f(x)}{g(x)}\).

  \medskip
  \begin{itemize} \setlength\itemsep{2ex}
    \item[(a)] True.
    \item[(b)] False
    \item[(c)] 
    \item[(e)] 
    \item[(d)] I am not sure.
  \end{itemize} 
\end{frame}

\begin{frame}
  If \(f(x)\) is discontinuous on \([1,3]\), then we know \(f(x)\) has a discontinuity at \underline{\hspace{1in}} number in \([1,3]\). 
  \medskip
  \begin{itemize} \setlength\itemsep{2ex}
    \item[(a)] every
    \item[(b)] exactly one
    \item[(c)] at least one
    \item[(d)] 
    \item[(e)] I am not sure.
  \end{itemize} 

  \toinstructors{Could be used as a lead-in to discuss continuity over intervals.}
\end{frame} 


\begin{frame}
  True or false? Every piecewise function has \emph{at least one} discontinuity on its domain.

  \medskip
  \begin{itemize} \setlength\itemsep{2ex}
    \item[(a)] True.
    \item[(b)] False
    \item[(c)] 
    \item[(e)] 
    \item[(d)] I am not sure.
  \end{itemize} 
\end{frame} 

% template
% \section{}
% \begin{frame}
%
%
%   \medskip
%   \begin{itemize} \setlength\itemsep{2ex}
%     \item[(a)] 
%     \item[(b)] 
%     \item[(c)] 
%     \item[(d)] 
%     \item[(3)] 
%   \end{itemize} 
% \end{frame}

\end{document}
