%! Tex Program = lualatex
\documentclass[14pt]{beamer} 
\usetheme{metropolis} 
\everymath{\displaystyle} 
\setbeamersize{text margin left=.5cm} 
\setbeamersize{text margin right=.5cm} 
\beamertemplatenavigationsymbolsempty{}
\usepackage[normalem]{ulem}

\input{./colours.tex.preamble}

%
% uncomment the whole \setbeamertemplate command to add an iClicker watermark.
%
\usepackage{tikz, graphicx}
\setbeamertemplate{background}{
  \begin{tikzpicture}[overlay, remember picture]
    \node[teal!10, anchor=south] at (current page.south) {\resizebox{\textwidth}{!}{iClicker}};
  \end{tikzpicture}    
}

%
% instructor notes
%
\usepackage{ifdraft}
\newcommand{\toinstructors}[1]{
  \ifdraft{ \scriptsize \color{magenta} {#1} \newline 
      Remove this comment by removing the draft option from \textbackslash{}documentclass.
    }{
      \renewcommand{\toinstructors}[1]{}
    }
  }

\begin{document} 

% \begin{frame}
%   Question
%
%   \medskip
%   \begin{itemize} \setlength\itemsep{2ex}
%     \item[(a)] 
%     \item[(b)] 
%     \item[(c)] 
%     \item[(d)] There is not enough information.
%     \item[(e)] I am not sure.
%   \end{itemize} 
% \end{frame}

\section{Icebreaker}

\begin{frame}
  Is a hot dog a sandwich?
  
  \medskip
  \begin{itemize} \setlength\itemsep{2ex}
    \item[(a)] Yes!
    \item[(b)] Yes?
    \item[(c)] No!
    \item[(d)] No?
    \item[(e)] I no longer know what a sandwich is.
  \end{itemize} 
\end{frame}


\section{Review}
\begin{frame}
  How can we calculate \(y = \log_{2}(8)\)?
  \medskip
  \begin{itemize} \setlength\itemsep{2ex}
    \item[(a)] Solve \(2^{y} = 8\).
    \item[(b)] Solve \(2^{8} = y\).
    \item[(c)] Calculate \(\frac{\ln(8)}{\ln(2)}\). 
    \item[(d)] 
    \item[(e)] I am not sure.
  \end{itemize} 
\end{frame} 


\section{Limits}
\begin{frame}
  Suppose \(\lim_{x \to 3} f(x)\) exists. Can we find \(f(3)\)?
  \medskip
  \begin{itemize} \setlength\itemsep{2ex}
    \item[(a)] \(f(3) = \lim_{x \to 3} f(x)\).
    \item[(b)] \(f(3)\) is undefined.
    \item[(c)] There is not enough information.
    \item[(d)] 
    \item[(d)] I am not sure.
  \end{itemize} 
\end{frame} 


\begin{frame}
  True or false? We can \emph{always} apply the quotient law to evaluate \(\lim_{x \to a} \frac{f(x)}{g(x)}\).

  \medskip
  \begin{itemize} \setlength\itemsep{2ex}
    \item[(a)] True.
    \item[(b)] False
    \item[(c)] 
    \item[(d)] 
    \item[(e)] I am not sure.
  \end{itemize} 
\end{frame}


\begin{frame}
  If \(f(x)\) is discontinuous on \([1,3]\), then we know \(f(x)\) has a discontinuity at \underline{\hspace{1in}} number in \([1,3]\). 
  \medskip
  \begin{itemize} \setlength\itemsep{2ex}
    \item[(a)] every
    \item[(b)] exactly one
    \item[(c)] at least one
    \item[(d)] 
    \item[(e)] I am not sure.
  \end{itemize} 

  \toinstructors{Could be used as a lead-in to discuss continuity over intervals.}
\end{frame} 


\begin{frame}
  Which of the following tell us a function \(f(x)\) is NOT continuous at \(a\)?
  \medskip
  \begin{itemize} \setlength\itemsep{2ex}
    \item[(a)] \(f\) is not defined at \(a\).
    \item[(b)] \(f\) is defined at \(a\) but \(\lim_{x \to a} f(x) \ne f(a)\).
    \item[(c)] \(\lim_{x \to a^{-}} f(x) = \pm \infty\) or \(\lim_{x \to a^{+}} f(x) = \pm \infty\).
    \item[(d)] \(\lim_{x \to a^{-}} f(x) \ne \lim_{x \to a^{+}} f(x)\).
    \item[(e)] I am not sure.
  \end{itemize} 
\end{frame}


\begin{frame}
  True or false? Every piecewise function has \emph{at least one} discontinuity in its domain.

  \medskip
  \begin{itemize} \setlength\itemsep{2ex}
    \item[(a)] True.
    \item[(b)] False
    \item[(c)] 
    \item[(d)] 
    \item[(e)] I am not sure.
  \end{itemize} 
\end{frame} 



\section{Limits at infinity}
\begin{frame}
  How many horizontal asymptotes can a function have?

  \medskip
  \begin{itemize} \setlength\itemsep{2ex}
    \item[(a)] One.
    \item[(b)] Two.
    \item[(c)] At most two.
    \item[(d)] At most infinitely many.
    \item[(e)] I am not sure.
  \end{itemize} 
\end{frame}


\begin{frame}
  Can a function cross its horizontal asymptotes?

  \medskip
  \begin{itemize} \setlength\itemsep{2ex}
    \item[(a)] Yes.
    \item[(b)] No.
    \item[(c)] 
    \item[(d)] 
    \item[(e)] I am not sure.
  \end{itemize} 
\end{frame}


\begin{frame}
  What can we deduce from knowing \(\lim_{x \to \infty} f(x)\) exists? 

  \medskip
  \begin{itemize} \setlength\itemsep{2ex}
    \item[(a)] \(f\) is not increasing on some interval \((a, \infty)\)
    \item[(b)] \(f\) is not decreasing on some interval \((b, \infty)\)
    \item[(c)] \(f\) is not oscillating on some interval \((c, \infty)\)
    \item[(d)] 
    \item[(e)] I am not sure.
  \end{itemize} 
\end{frame}


\begin{frame}
  Analyze the dominant terms of \(\lim_{x \to \infty} \frac{x^{2}+3x}{\sqrt{x+3}}\). With only a quick calculations, the limit should be

  \medskip
  \begin{itemize} \setlength\itemsep{2ex}
    \item[(a)] \(\infty\) or \(-\infty\).
    \item[(b)] a finite number.
    \item[(c)] 
    \item[(d)] 
    \item[(e)] I am not sure.
  \end{itemize} 
\end{frame}



\section{Derivatives}
\begin{frame}
  % Let's pop a balloon by dropping it on a needle. 
  What is the balloon's velocity when it popped?

  \medskip
  \begin{itemize} \setlength\itemsep{2ex}
    \item[(a)] Does not exist because there is no balloon.
    \item[(b)] Non-zero because the balloon has to be moving. 
    \item[(c)] Zero because time freezes at that very instant. 
    \item[(d)] 
    \item[(e)] I am not sure.
  \end{itemize} 
  \toinstructors{Philosophical open question. Pop a balloon in class. Can be used as a lead-in to define the derivative as a model for instantaneous rate of change.}
\end{frame}


\begin{frame}
  True or false? If a function is continuous at \(a\), then it must be differentiable at \(a\).

  \medskip
  \begin{itemize} \setlength\itemsep{2ex}
    \item[(a)] True.
    \item[(b)] False.
    \item[(c)] 
    \item[(d)] 
    \item[(e)] I am not sure.
  \end{itemize} 
\end{frame}


\begin{frame}
  Compare the domains of \(f\) and \(f'\).

  \medskip
  \begin{itemize} \setlength\itemsep{2ex}
    \item[(a)] \(f'\) is defined wherever \(f\) is defined.
    \item[(b)] The domain of \(f'\) can be larger than the domain of \(f\).
    \item[(c)] The domain of \(f'\) can be smaller than the domain of \(f\).
    \item[(d)] 
    \item[(e)] I am not sure.
  \end{itemize} 
\end{frame}


\begin{frame}
  If a function is \emph{not} differentiable on \((0,1)\), then 

  \medskip
  \begin{itemize} \setlength\itemsep{2ex}
    \item[(a)] \(f\) is not continuous on \((0,1)\).
    \item[(b)] \(f\) is a piecewise function.
    \item[(c)] 
    \item[(d)] 
    \item[(e)] I am not sure.
  \end{itemize} 
\end{frame}


\begin{frame}
  If a function is \emph{not} differentiable on \((0,1)\), then 

  \medskip
  \begin{itemize} \setlength\itemsep{2ex}
    \item[(a)] \(f'(a)\) does not exists at any number in \((0,1)\).
    \item[(b)] \(f'(a)\) does not exists for at least one number in \((0,1)\).
    \item[(c)] 
    \item[(d)] 
    \item[(e)] I am not sure.
  \end{itemize} 
\end{frame}


\begin{frame}
  If a function is \emph{not} differentiable on \((0,1)\), then \(f'\) is not differentiable at \underline{\hspace{1in}} number in \((0,1)\).

  \medskip
  \begin{itemize} \setlength\itemsep{2ex}
    \item[(a)] every 
    \item[(b)] exactly one
    \item[(c)] at least one
    \item[(d)] 
    \item[(e)] I am not sure.
  \end{itemize} 
\end{frame}


\begin{frame}
  (True or false?) If \(f(x)\) is not differentiable at \(a\), then \(f\) is not continuous at \(a\).

  \medskip
  \begin{itemize} \setlength\itemsep{2ex}
    \item[(a)] True.
    \item[(b)] False.
    \item[(c)] 
    \item[(d)] 
    \item[(e)] I am not sure.
  \end{itemize} 
\end{frame}


\section{Differentiation Rules}
\begin{frame}
  (True or false?) By the power rule, \(\frac{d}{dx} \frac{1}{\sqrt{x}} = \frac{1}{\frac{1}{2}x^{-1/2}}\).

  \medskip
  \begin{itemize} \setlength\itemsep{2ex}
    \item[(a)] True.
    \item[(b)] False.
    \item[(c)] 
    \item[(d)] 
    \item[(e)] I am not sure.
  \end{itemize} 
\end{frame}


\begin{frame}
  (True or false?) By the power rule, \(\frac{d}{dx} 2^{x} = x \cdot 2^{x-1}\).

  \medskip
  \begin{itemize} \setlength\itemsep{2ex}
    \item[(a)] True.
    \item[(b)] False.
    \item[(c)] 
    \item[(d)] 
    \item[(e)] I am not sure.
  \end{itemize} 
\end{frame}


\begin{frame}
  (True or false?) By the power rule, \(\frac{d}{dx} 2^{10} = 10 \cdot 2^{9}\).

  \medskip
  \begin{itemize} \setlength\itemsep{2ex}
    \item[(a)] True.
    \item[(b)] False.
    \item[(c)] 
    \item[(d)] 
    \item[(e)] I am not sure.
  \end{itemize} 
\end{frame}


\begin{frame}
  Which of the following words suggest a question is about derivatives?

  \medskip
  \begin{itemize} \setlength\itemsep{2ex}
    \item[(a)] asymptote
    \item[(b)] secant line
    \item[(c)] tangent line
    \item[(d)] slope
    \item[(e)] I am not sure.
  \end{itemize} 
\end{frame}


\section{The Chain Rule}
\begin{frame}
  Which is the chain rule formula?

  \medskip
  \begin{itemize} \setlength\itemsep{2ex}
    \item[(a)] \(\tfrac{d}{dx} f(g(x)) = f'(g)g'\)
    \item[(b)] \(\tfrac{d}{dx} f(g(x)) = f(g')g'\)
    \item[(c)] \(\tfrac{d}{dx} f(g(x)) = f'(g')g\)
    \item[(d)] 
    \item[(e)] I am not sure.
  \end{itemize} 
\end{frame}

\begin{frame}
  Suppose \(h(x) = f(g(x))\). What is \(h'(1)\) given
  \begin{align*}
    f(1) &= 1,& f'(1) &= -1, &f(2) &= 3, & f'(2) &= -3, \\
    g(1) &= 2,& g'(1) &= -2, &g(2) &= 4, & g'(4) &= -4?
  \end{align*}

  \medskip
  \begin{itemize} \setlength\itemsep{2ex}
    \item[(a)] \(h'(1) = -2\)
    \item[(b)] \(h'(1) = -1\)
    \item[(c)] \(h'(1) = 6\)
    \item[(d)] \(h'(1) = -6\)
    \item[(e)] I am not sure.
  \end{itemize} 
\end{frame}

\begin{frame}
  Suppose \(h(x) = f(g(x))\). What is \(h'(1)\) given
  \begin{align*}
    f(1) &= 1,& f'(1) &= -1, &f(2) &= 3, & f'(2) &= -3, \\
    {\color{teal}g(1)} &{\color{teal}= 2},& {\color{magenta}g'(1)} &{\color{magenta}= -2}, &g(2) &= 4, & g'(4) &= -4?
  \end{align*}

  \medskip
  \begin{itemize} \setlength\itemsep{2ex}
    \item[(a)] 
    \item[(b)] 
    \item[(c)] \(h'(1) = f'({\color{teal}g(1)}){\color{magenta}g'(1)} \onslide<2->{= f'(2)(-2)} \onslide<3->{= (-3)(-2) = 6}\)
    \item[(d)] 
    \item[(e)] 
  \end{itemize} 
\end{frame}

\begin{frame}
  Differentiate \(\sin(x)^{2}\).

  \medskip
  \begin{itemize} \setlength\itemsep{2ex}
    \item[(a)] \(\cos(x)^{2}\)
    \item[(b)] \(\sin(x)\cos(x)\)
    \item[(c)] \(2\sin(x)\cos(x)\)
    \item[(d)] \(2\sin(x)\)
    \item[(e)] I am not sure.
  \end{itemize} 
\end{frame}


\section{Implicit Differentiation}
\begin{frame}
  Given \(\sin(y) = x\), what is \(dy/dx\)? Choose all correct options.

  \medskip
  \begin{itemize} \setlength\itemsep{2ex}
    \item[(a)] \(0 = 1\), so \(y'\) does not exist.
    \item[(b)] \(dy/dx = \tfrac{1}{\cos(y)}\).
    \item[(c)] \(dy/dx = \cos(y)\).
    \item[(d)] \(dy/dx = \tfrac{d}{dx} \arcsin(x)\).
    \item[(e)] I am not sure.
  \end{itemize} 
\end{frame}


\begin{frame}
  Given \(e^{y} = x\), what is \(dy/dx\)? Choose all correct options.

  \medskip
  \begin{itemize} \setlength\itemsep{2ex}
    \item[(a)] \(0 = 1\), so \(y'\) does not exist.
    \item[(b)] \(dy/dx = \tfrac{1}{e^{y}}\).
    \item[(c)] \(dy/dx = e^{y}\).
    \item[(d)] \(dy/dx = \tfrac{d}{dx} \ln(x)\).
    \item[(e)] I am not sure.
  \end{itemize} 
\end{frame}


\begin{frame}
  Find \(dx/dt\) given \(x^{3} - xt = 1\).

  \medskip
  \begin{itemize} \setlength\itemsep{2ex}
    \item[(a)] \(\frac{dx}{dt} = \frac{t}{3x^{2}}\).
    \item[(b)] \(\frac{dx}{dt} = \frac{x}{3x^{2} - t}\).
    \item[(c)] \(\frac{dx}{dt} = \frac{3x^{2} - t}{x}\).
    \item[(d)] None of the above.
    \item[(e)] I am not sure.
  \end{itemize} 
\end{frame}

\begin{frame}
  Can we use implicit differentiation to find the slope of \(y^{2} + x^{2} = 25\) at the point \((5,5)\)?

  \medskip
  \begin{itemize} \setlength\itemsep{2ex}
    \item[(a)] Yes, we plug \(x = 5\) and \(y = 5\) into \(dy/dx\).
    \item[(b)] No, it's not a function and has no \(dy/dx\).
    \item[(c)] No, we cannot plug \(x = 5\) and \(y = 5\) into \(dy/dx\).
    \item[(d)] I don't understand the question.
    \item[(e)] I am not sure.
  \end{itemize} 
\end{frame}

\begin{frame}
  What is the slope of the tangent line to \(\tan(xy) = -x\) at the point \((1, 3\pi/4)\)?

  \medskip
  \begin{itemize} \setlength\itemsep{2ex}
    \item[(a)] \(-1 - \frac{3\pi}{2}\). 
      \onslide<2>{{\color{magenta}Probably forgot about brackets.}}
    \item[(b)] \(-\frac{1}{2} - \frac{3\pi}{4}\). 
      \onslide<2>{{\color{teal}The correct answer.}}
    \item[(c)] \(\frac{1}{\sqrt{2}} - \frac{3\pi}{4}\). 
      \onslide<2>{{\color{magenta} Probably forgot the square or made the mistake \(\tfrac{d}{dx}\tan(x) = \sec(x)\)}}
    \item[(d)] 
    \item[(e)] I am not sure.
  \end{itemize} 
\end{frame}


\begin{frame}
  How many horizontal tangent lines does the curve \(y^{2}/2 - xy = 1\) have? 

  \medskip
  \begin{itemize} \setlength\itemsep{2ex}
    \item[(a)] Zero.
    \item[(b)] Finitely many.
    \item[(c)] Infinitely many.
    \item[(d)] I don't know \emph{WHAT} to solve for in \(\tfrac{y}{x - y} = 0\).
    \item[(e)] I don't know how to start.
  \end{itemize} 
\end{frame}


\begin{frame}
  Assume \(y = f^{-1}(x)\). To which equations do we apply the chain rule to calculate \(y'\)? Choose all correct options.

  \medskip
  \begin{itemize} \setlength\itemsep{2ex}
    \item[(a)] \(f(f^{-1}(x)) = x\)
    \item[(b)] \(f^{-1}(f(x)) = x\)
    \item[(c)] \(y = f(x)\)
    \item[(d)] \(x = f(y)\)
    \item[(e)] I don't recognize equations (a) and (b).
  \end{itemize} 
\end{frame}


\begin{frame}
  Can the inverse function theorem be deduced from implicit differentiation?

  \medskip
  \begin{itemize} \setlength\itemsep{2ex}
    \item[(a)] Yes.
    \item[(b)] No.
    \item[(c)] I think so, but not sure how.
    \item[(d)] 
    \item[(e)] I am not sure.
  \end{itemize} 
\end{frame}


\section{Logarithmic differentiation}
\begin{frame}
  Differentiate \(x^{x}\).

  \medskip
  \begin{itemize} \setlength\itemsep{2ex}
    \item[(a)] \(x x^{x-1}\).
    \item[(b)] \(x^{x}(1 + \ln(x)\)
    \item[(c)] \(x^{x}\)
    \item[(d)] \(x^{x}\ln(x)\)
    \item[(e)] I am not sure.
  \end{itemize} 
\end{frame}

\section{Related rates}
\begin{frame}[t]
  What are we asked to find in Example~9?

  \medskip
  \begin{itemize} \setlength\itemsep{2ex}
    \item[(a)] Find \(d\theta/dt\).
    \item[(b)] Find \(dx/dt\).
    \item[(c)] Find \(d\theta/dt\) when \(x = 4\).
    \item[(d)] Find \(dx/dt\) when \(x = 4\).
    \item[(e)] I am not sure.
  \end{itemize} 
\end{frame}

\section{Extrema}

\begin{frame}[t]
  If there are constants \(a,b\) such that \(a \le f(x) \le b\) for every \(x\) on the domain of \(f\), does it mean that \(f(x)\) must have an absolute extrema?

  \medskip
  \begin{itemize} \setlength\itemsep{2ex}
    \item[(a)] No.
    \item[(b)] \only<1>{Yes.}\only<2>{\sout{Yes.}} \onslide<2>{This is wrong because of functions like \(x \sin(1/x)\).}
    \item[(c)] 
    \item[(d)] 
    \item[(e)] I am not sure.
  \end{itemize} 
\end{frame}


\begin{frame}[t]
  Is it true that every absolute extrema is also a local extrema?

  \medskip
  \begin{itemize} \setlength\itemsep{2ex}
    \item[(a)] No.
    \item[(b)] \only<1>{Yes.}\only<2>{\sout{Yes.}} \onslide<2>{This is wrong because absolute extrema can occur at endpoints of the domain of a function but local extrema cannot.}
    \item[(c)] 
    \item[(d)] 
    \item[(e)] I am not sure.
  \end{itemize} 
\end{frame}

\begin{frame}[t]
  What is the relation between local extrema and critical points?

  \medskip
  \begin{itemize} \setlength\itemsep{1ex}
    \item[(a)] Some, but not all, critical points are local extrema.
    \item[(b)] A local extremum must correspond to a critical point.
    \item[(c)] Every critical point correspond to a local extremum.
    \item[(d)] \(\{ \text{critical points} \} = \{ \text{local extrema} \}\).
    \item[(e)] I am not sure.
  \end{itemize} 
\end{frame}

\section{Derivative Tests}

\begin{frame}[t]
  Consider the graph in Example~10. Using only the first derivative test, where does \(f(x)\) have a local maximum? 

  \medskip
  \begin{itemize} \setlength\itemsep{1ex}
    \item[(a)] \(x = a\). \onslide<2>{\hlmain{\(f'(a) = 0\) (\(x = a\) is a critical number of \(f\)), and \(f'\) changes from positive (\(f\) is decreasing) to negative (\(f\) is increasing) as \(x\) goes from left to right passing \(a\).}}
    \item[(b)] \(x = b\). 
    \item[(c)] \(x = c\).
    \item[(d)] \(x = d\). 
    \item[(e)] I am not sure.
  \end{itemize} 
\end{frame}

\begin{frame}[t]
  Can the First Derivative Test be applied to find the local min of \(f(x) = |x|\) (and, more generally, any continuous piecewise function)? 

  \medskip
  \begin{itemize} \setlength\itemsep{1ex}
    \item[(a)] Yes, because \(f\) does not need to be differentiable at the critical number \(0\).
    \item[(b)] No, because \(f\) is not differentiable at the critical number \(0\).
    \item[(c)] 
    \item[(d)] 
    \item[(e)] I am not sure.
  \end{itemize} 
\end{frame}

\begin{frame}[t]
  Is \(|x|\) concave up on its domain? 

  Can the Second Derivative Test be used to test its concavity?

  \medskip
  \begin{itemize} \setlength\itemsep{1ex}
    \item[(a)] Yes and yes.
    \item[(b)] Yes and no.
    \item[(c)] No and yes.
    \item[(d)] No and no.
    \item[(e)] I am not sure.
  \end{itemize} 
\end{frame}

\section{Optimization}
\begin{frame}[c]
  In Example~1, is the domain of \(A(x)\) really \(0 \le x \le 100\)?
\end{frame}

\begin{frame}[c]
  In Example~2, why do we measure the walking distance \(w\) and the swimming distance \(s\)? Why not something other quantities?
\end{frame}

\begin{frame}[t]
  What is wrong with the (proposed) solution of Example~2?

  \medskip
  \begin{itemize} \setlength\itemsep{1ex}
    \item[(a)] There is an algebra mistake.
    \item[(b)] The domain of \(T(s)\) is wrong.
    \item[(c)] It's not possible to minimize travel time.
    \item[(d)] 
    \item[(e)] I am not sure.
  \end{itemize} 
\end{frame}

\begin{frame}[t]
  What is the domain of \(T(s)\) from Example~2?

  \medskip
  \begin{itemize} \setlength\itemsep{1ex}
    \item[(a)] \(s \ge 1\). \onslide<2>{\hlsupp{We don't need to swim forever.}}
    \item[(b)] \(1 \le s \le \sqrt{5}\). \onslide<2>{\hlmain{Note \(\sqrt{5}\) is the length of the full diagonal.}}
    \item[(c)] 
    \item[(d)] 
    \item[(e)] I am not sure.
  \end{itemize} 
\end{frame}

\section{L'H\^opital's Rule}
\begin{frame}[t]
  Which of the following is an indeterminate form? Choose all correct options.

  \medskip
  \begin{itemize} \setlength\itemsep{1ex}
    \item[(a)] \(L_{1} = \lim_{x \to 0} x \ln(x)\).
    \item[(b)] \(L_{2} = \lim_{x \to 1} (1 - x)^{x}\).
    \item[(c)] \(L_{3} = \lim_{x \to -\infty} 2^{2x}\).
    \item[(d)] \(L_{4} = \lim_{x \to 0} \frac{2x^{1}+1}{\cos(x)}\).
    \item[(e)] I am not sure.
  \end{itemize} 
\end{frame}

\begin{frame}
  What's wrong with the following calculation?
  \[
    \lim_{x \to 1} \frac{\ln(x)}{2} = \lim_{x \to 1} \frac{\frac{d}{dx} \ln(x)}{\frac{d}{dx} 2} = \lim_{x \to 1} \frac{1/x}{0} = \text{does not exists?!?!}
  \]

  \pause{} 

  L'H\^opital's rule \emph{does not apply} because the limit is not an indeterminate form.

  ONLY apply l'H\^opital's rule to indeterminate forms of \(0/0\) or \(\infty/\infty\) and nothing else.
\end{frame}

\section{Sigma Notations}

\begin{frame}[t]
Do \(\sum_{i=1}^{3} f(a + i/2)\) and \(\sum_{i=0}^{2} f(a + (i+1)/2)\) represent different objects?

  \medskip
  \begin{itemize} \setlength\itemsep{1ex}
    \item[(a)] The two summations are equal.
    \item[(b)] The two summations are not equal.
    \item[(c)] 
    \item[(d)] 
    \item[(e)] I am not sure.
  \end{itemize} 
\end{frame}

\begin{frame}[t]
  Assume \(m < n\), both positive integers. Is it true that 
  \[
    \sum_{i=1}^{m} a_{i} + \sum_{i=1}^{n} b_{i} = \sum_{i=1}^{n} (a_{i}+ b_{i})?
  \]

  \medskip
  \begin{itemize} \setlength\itemsep{1ex}
    \item[(a)] The equality is true.
    \item[(b)] The equality is not always true.
    \item[(c)] 
    \item[(d)] 
    \item[(e)] I am not sure.
  \end{itemize} 
\end{frame}

\section{Antiderivatives}

\begin{frame}[t]
  Consider \(f(x) = 2x e^{x^{2}}\) and \(g(x) = e^{x^{2}}\). Which of the following statement is true.

  \medskip
  \begin{itemize} \setlength\itemsep{1ex}
    \item[(a)] \(f(x)\) is an antiderivative of \(g(x)\).
    \item[(b)] \(f(x)\) is the antiderivative of \(g(x)\).
    \item[(c)] \(g(x)\) is an antiderivative of \(f(x)\).
    \item[(d)] \(g(x)\) is the antiderivative of \(f(x)\).
    \item[(e)] I am not sure.
  \end{itemize} 
\end{frame}

\begin{frame}
  Why is \(\ln(x)\) not an antiderivative of \(1/x\)?
\end{frame}

\begin{frame}[t]
  Which of the following are true statements?

  \medskip
  \begin{itemize} \setlength\itemsep{1ex}
    \item[(a)] \(\int e^{x} \;dx = e^{x}\).
    \item[(b)] \(\int e^{x} \;dx = e^{x} + C\).
    \item[(c)] \(\int e^{x} \;dx = e^{x} + 1 + C\).
    \item[(d)] \(\int e^{x} \;dx = e^{x} + 2C\).
    \item[(e)] I am not sure.
  \end{itemize} 
\end{frame}

\begin{frame}[t]
  Evaluate \(\int \sec^{2}(\pi/4) \;dx\).

  \begin{itemize} \setlength\itemsep{1ex}
    \item[(a)] \(\tan(\pi/4) + C\).
    \item[(b)] \(\sec^{2}(\pi/4)x + C\).
    \item[(c)] We don't know enough integration techniques to evaluate this yet. 
    \item[(d)] 
    \item[(e)] I am not sure.
  \end{itemize} 
\end{frame}

\section{Integration formulas}
\begin{frame}[t]
  Recall that differentiating a {\color{red} non-constant polynomial} lowers its degree by one. Does integrating any {\color{red} non-zero polynomial} (including constant ones) raise its degree by one?

  \medskip
  \begin{itemize} \setlength\itemsep{1ex}
    \item[(a)] Yes, because \(\frac{d}{dx} x^{n} = n x^{n-1}\) for any integer \(n \ge 0\).
    \item[(b)] No, because \(\frac{d}{dx} (\text{any constant}) = 0\).
    \item[(c)] 
    \item[(d)] 
    \item[(e)] I am not sure.
  \end{itemize} 
\end{frame}

\begin{frame}[t]
  Does integrating a {\color{red} power function}, i.e., \(x^{n}\) with NO restriction on \(n\), always raise its degree by one? Are there any exceptions?

  \medskip
  \begin{itemize} \setlength\itemsep{1ex}
    \item[(a)] Yes. There are no exceptions.
    \item[(b)] No, an exception is \(x^{-1}\).
    \item[(c)] 
    \item[(d)] 
    \item[(e)] I am not sure.
  \end{itemize} 
\end{frame}

\section{Definite integrals}

\begin{frame}
  Evaluate \(\int_{a}^{a} f(x) \;dx\). 

  \begin{itemize} \setlength\itemsep{1ex}
    \item[(a)] It is not possible since we don't know what \(f(x)\) is.
    \item[(b)] It is not possible since we don't know what \(a\) is.
    \item[(c)] \(0\).
    \item[(d)] \(> 0\)
    \item[(e)] I am not sure.
  \end{itemize}
\end{frame}

\begin{frame}[t]
  The two endpoints of an interval are \(a,b\), and we don't know if \(a < b\) or \(a > b\). However, we know that \(\int_{b}^{a} f(x) \;dx < 0\) and \(f\) is nonnegative. It follows that 

  \begin{itemize} \setlength\itemsep{1ex}
    \item[(a)] \(a < b\).
    \item[(b)] \(a > b\).
    \item[(c)] \(a = b\).
    \item[(d)] 
    \item[(e)] I am not sure.
  \end{itemize} 
\end{frame}

\begin{frame}[t]
  Let \(f(x) = (x+2)(x-1)\) and \(A = \int_{-3}^{1} \big|f(x)\big| \;dx\).  Which of the following is a true statement?

  \begin{itemize} \setlength\itemsep{1ex}
    \item[(a)] \(A = \phantom{-}\int_{-3}^{-2} f(x) \;dx + \int_{-2}^{1} f(x) \;dx\).
    \item[(b)] \(A = \phantom{-}\int_{-3}^{-2} f(x) \;dx - \int_{-2}^{1} f(x) \;dx\).
    \item[(c)] \(A = -\int_{-3}^{-2} f(x) \;dx + \int_{-2}^{1} f(x) \;dx\).
    \item[(d)] \(A = -\int_{-3}^{-2} f(x) \;dx - \int_{-2}^{1} f(x) \;dx\).
    \item[(e)] I am not sure.
  \end{itemize} 
\end{frame}

\begin{frame}[t]
  Let \(A = \int_{a}^{b} f(x) \;dx\). Which of the following is a true statement?

  \begin{itemize} \setlength\itemsep{1ex}
    \item[(a)] \(A = \int_{0}^{a} f(x) \;dx + \int_{0}^{b} f(x) \;dx\).
    \item[(b)] \(A = -\int_{0}^{a} f(x) \;dx + \int_{0}^{b} f(x) \;dx\).
    \item[(c)] 
    \item[(d)] 
    \item[(e)] I am not sure.
  \end{itemize} 
\end{frame}

\begin{frame}[t]
  If \(f(x) \le g(x)\) over \([a,b]\), then ...

  \begin{itemize} \setlength\itemsep{1ex}
    \item[(a)] \(\int_{a}^{b} f(x) \;dx \le \int_{a}^{b} g(x) \;dx\).
    \item[(b)] \(\int_{a}^{b} f(x) \;dx = \int_{a}^{b} g(x) \;dx\).
    \item[(c)] \(\int_{a}^{b} f(x) \;dx \ge \int_{a}^{b} g(x) \;dx\).
    \item[(d)] 
    \item[(e)] I am not sure.
  \end{itemize} 
\end{frame}

\section{Fundamental Theorem of Calculus}
\begin{frame}[t]
  Which of the following are good starting points for our solutions?

  \begin{enumerate} \setlength\itemsep{1ex}
    \item[(a)] Write \(F(x) = \int_{0}^{x^{3}} e^{t^{2}} \;dt + \int_{0}^{x^{2}+1} e^{t^{2}} \;dt \).
    \item[(b)] Write \(F(x) = \int_{x^{3}}^{0} e^{t^{2}} \;dt + \int_{0}^{x^{2}+1} e^{t^{2}} \;dt \).
    \item[(c)] Write \(F(x) = \int_{0}^{x^{3}} e^{t^{2}} \;dt + \int_{0}^{1} e^{t^{2}} \;dt + \int_{1}^{x^{2}} e^{t^{2}} \;dt\).
    \item[(d)] Write \(F(x) = \int_{x^{3}}^{1} e^{t^{2}} \;dt + \int_{1}^{x^{2}+1} e^{t^{2}} \;dt \).
    \item[(e)] I am not sure.
  \end{enumerate}
\end{frame}

\begin{frame}
  Evaluate \(\int_{-\pi}^{\pi/2} 2 \cos(x) \;dx\).
\end{frame}

\section{The Substitution Rule}
\begin{frame}[t]
  What if we didn't see the ``convenient'' choice for \(u\)? 

  Transform \(\int x^{2} \sqrt{ 1 + x^{3} } \;dx\) by subbing \(u = \sqrt{ 1 + x^{3} }\).

  \begin{enumerate} \setlength\itemsep{1ex}
    \item[(a)] This substitution does not work.
    \item[(b)] We get \(\int \frac{2}{3} u^{2} \;du\). \only<2->{\hfill{}\hlmain{Correct.}}
    \item[(c)] We get \(\int \frac{2}{3} \frac{u}{\sqrt{ 1 + x^{3} }} \;du\). \only<2->{\hfill\hlwarn{ Algebra mistake.}}
  \item[(d)] We get \(\int \frac{2}{3} u \sqrt{1 + x^{3}} \;du\). \only<2->{\hfill\hlsupp{Incomplete.}}
    \item[(e)] None of the above, or I am not sure.
  \end{enumerate}
\end{frame}

\section{Applications of integration}

\begin{frame}[t]
  What are expressions (plural!) for the area bounded between \(\sin(x)\) and \(\cos(x)\) over \([0, \pi]\)?

  \begin{enumerate} \setlength\itemsep{1ex}
    \item[(a)] \(\int_{0}^{\pi} \sin(x) - \cos(x) \;dx\).
    \item[(b)] \(\int_{0}^{\pi} |\sin(x) - \cos(x)| \;dx\).
    \item[(c)] \(\int_{0}^{\pi/4} \cos(x) - \sin(x) \;dx + \int_{\pi/4}^{\pi} \sin(x) - \cos(x) \;dx\).
    \item[(d)] \(\int_{0}^{\pi/4} \cos(x) - \sin(x) \;dx - \int_{\pi/4}^{\pi} \cos(x) - \sin(x) \;dx\).
    \item[(e)] I am not sure. 
  \end{enumerate}
\end{frame}

\begin{frame}[t]
  Suppose \(R\) is the region enclosed between \(y = \sin(x)\) on \([pi/6, pi/3]\) and the \(x\)-axis.  Suppose a solid of revolution \(S\) is obtained by rotating \(R\) about the \(x\)-axis. 

  Sketch the region \(R\) and the solid \(S\).

  What is the inner radius of the cross-section at \(x\)? 
\end{frame}

\section{Problem-solving}
\begin{frame}[t]
  Suppose \(\sin(f(x)) = 0\), \(f(1) = \pi\). Find \(f'(1)\).

  What question did you ask yourself to discover a problem-solving idea?
\end{frame}

\begin{frame}[t]
  How can we determine the continuity of a piecewise function \(f(x)\) at a constant \(c\)?

  \medskip
  \begin{itemize} \setlength\itemsep{2ex}
    \item[(a)] Apply limit laws.
    \item[(b)] Calculate and compare left and right limits at \(c\).
    \item[(c)] Check if \(\lim_{x \to c}f(x) = f(c)\). May have to calculate left and right limits separately.
    \item[(d)] Check if \(f(x)\) is defined at \(c\).
    \item[(e)] I am not sure.
  \end{itemize} 
\end{frame}

\begin{frame}[t]
  Suppose \(f(x) = \begin{cases} ax^{2} &\text{if } 0 \le x < 1 \\ -1 &\text{if } x = 1 \end{cases}\) is continuous on \([0,1]\) and you are asked to find \(a\). What information seems useful?

  \medskip
  \begin{itemize} \setlength\itemsep{2ex}
    \item[(a)] Limit laws.
    \item[(b)] Squeeze Theorem.
    \item[(c)] \(f(x)\) is continuous on \([0,1]\).
    \item[(d)] Intermediate Value Theorem.
    \item[(e)] I am not sure.
  \end{itemize} 
\end{frame}

\begin{frame}[t]
  Suppose \(f(x) = \begin{cases} ax^{2} &\text{if } 0 \le x < 1 \\ -1 &\text{if } x = 1 \end{cases}\) is continuous on \([0,1]\) and you are asked to find \(a\). Continuity seems important. What else is useful?

  \medskip
  \begin{itemize} \setlength\itemsep{2ex}
    \item[(a)] \(f\) must be continuous at \(0\).
    \item[(b)] \(f\) must be continuous at \(1\).
    \item[(c)] 
    \item[(d)] 
    \item[(e)] I am not sure.
  \end{itemize} 
\end{frame}

\begin{frame}[t]
  Suppose \(f(x) = \begin{cases} ax^{2} &\text{if } 0 \le x < 1 \\ -1 &\text{if } x = 1 \end{cases}\) is continuous on \([0,1]\) and you are asked to find \(a\). What can we do with the continuity at \(1\)?

  \medskip
  \begin{itemize} \setlength\itemsep{2ex}
    \item[(a)] Deduce \(\lim_{x \to 1^{-}} f(x) = \lim_{x \to 1^{+}} f(x)\).
    \item[(b)] Deduce \(\lim_{x \to 1^{+}} f(x) = f(1)\).
    \item[(c)] Deduce \(\lim_{x \to 1^{-}} f(x) = f(1)\).
    \item[(d)] 
    \item[(e)] I am not sure.
  \end{itemize} 
\end{frame}

\begin{frame}[t]
  Suppose \(f(x) = \begin{cases} ax^{2} &\text{if } 0 \le x < 1 \\ -1 &\text{if } x = 1 \end{cases}\) is continuous on \([0,1]\) and you are asked to find \(a\). What can we do with the continuity at \(1\)?

  \medskip
  \begin{itemize} \setlength\itemsep{2ex}
    \item[(a)] Deduce \(\lim_{x \to 1^{-}} f(x) = \lim_{x \to 1^{+}} f(x)\).
    \item[(b)] Deduce \(\lim_{x \to 1^{+}} f(x) = f(1)\).
    \item[(c)] Deduce \(\lim_{x \to 1^{-}} f(x) = f(1)\).
    \item[(d)] 
    \item[(e)] I am not sure.
  \end{itemize} 
\end{frame}

\begin{frame}[t]
  Suppose \(x f(x)^{2} = 3\) and \(f(3) = -1\). Find \(f'(3)\).

  \medskip
  \begin{itemize} \setlength\itemsep{2ex}
    \item[(a)] \(1/6\)
    \item[(b)] \(-1/6\)
    \item[(c)]
    \item[(d)]
    \item[(e)] I am not sure.
  \end{itemize}
\end{frame}

\section{Study Skills}
\begin{frame}
  Describe what motivates you to study calculus using a single word.
\end{frame}

\begin{frame}
  How do you assess your learning progress?

  \medskip
  \begin{itemize} \setlength\itemsep{2ex}
    \item[(a)] Do suggested problems.
    \item[(b)] Do do-at-home problems.
    \item[(c)] Reflect on problem-solving ideas.
    \item[(d)] Summarize to reduce my mental load.
    \item[(e)] I am not sure.
  \end{itemize} 
\end{frame}


\section{Review}
\begin{frame}
  What topic should we review?

  \medskip
  \begin{itemize} \setlength\itemsep{2ex}
    \item[(a)] Shapes of graphs (min/max and derivative tests).
    \item[(b)] Related rates or optimization.
    \item[(c)] Fundamental of integration.
    \item[(d)] Integration techniques.
    \item[(e)] Go through past exams.
  \end{itemize} 
\end{frame}

% template
% \section{}
% \begin{frame}
%
%
%   \medskip
%   \begin{itemize} \setlength\itemsep{2ex}
%     \item[(a)] 
%     \item[(b)] 
%     \item[(c)] 
%     \item[(d)] 
%     \item[(e)] I am not sure.
%   \end{itemize} 
% \end{frame}
\end{document}

