%! Tex Program = lualatex
\documentclass[14pt]{beamer} 
\usetheme{metropolis} 
\everymath{\displaystyle} 
\setbeamersize{text margin left=.5cm} 
\setbeamersize{text margin right=.5cm} 
\beamertemplatenavigationsymbolsempty{}

\usepackage{xcolor}

%
% uncomment the whole \setbeamertemplate command to add an iClicker watermark.
%
\usepackage{tikz, graphicx}
\setbeamertemplate{background}{
  \begin{tikzpicture}[overlay, remember picture]
    \node[teal!10, anchor=south] at (current page.south) {\resizebox{\textwidth}{!}{iClicker}};
  \end{tikzpicture}    
}

%
% instructor notes
%
\usepackage{ifdraft}
\newcommand{\toinstructors}[1]{
  \ifdraft{ \scriptsize \color{magenta} {#1} \newline 
      Remove this comment by removing the draft option from \textbackslash{}documentclass.
    }{
      \renewcommand{\toinstructors}[1]{}
    }
  }

\begin{document} 

% \begin{frame}
%   Question
%
%   \medskip
%   \begin{itemize} \setlength\itemsep{2ex}
%     \item[(a)] 
%     \item[(b)] 
%     \item[(c)] 
%     \item[(d)] There is not enough information.
%     \item[(e)] I am not sure.
%   \end{itemize} 
% \end{frame}

\section{Icebreaker}

\begin{frame}
  Is a hot dog a sandwich?
  
  \medskip
  \begin{itemize} \setlength\itemsep{2ex}
    \item[(a)] Yes!
    \item[(b)] Yes?
    \item[(c)] No!
    \item[(d)] No?
    \item[(e)] I no longer know what a sandwich is.
  \end{itemize} 
\end{frame}


\section{Review}
\begin{frame}
  How can we calculate \(y = \log_{2}(8)\)?
  \medskip
  \begin{itemize} \setlength\itemsep{2ex}
    \item[(a)] Solve \(2^{y} = 8\).
    \item[(b)] Solve \(2^{8} = y\).
    \item[(c)] Calculate \(\frac{\ln(8)}{\ln(2)}\). 
    \item[(d)] 
    \item[(e)] I am not sure.
  \end{itemize} 
\end{frame} 


\section{Limits}
\begin{frame}
  Suppose \(\lim_{x \to 3} f(x)\) exists. Can we find \(f(3)\)?
  \medskip
  \begin{itemize} \setlength\itemsep{2ex}
    \item[(a)] \(f(3) = \lim_{x \to 3} f(x)\).
    \item[(b)] \(f(3)\) is undefined.
    \item[(c)] There is not enough information.
    \item[(d)] 
    \item[(d)] I am not sure.
  \end{itemize} 
\end{frame} 


\begin{frame}
  True or false? We can \emph{always} apply the quotient law to evaluate \(\lim_{x \to a} \frac{f(x)}{g(x)}\).

  \medskip
  \begin{itemize} \setlength\itemsep{2ex}
    \item[(a)] True.
    \item[(b)] False
    \item[(c)] 
    \item[(d)] 
    \item[(e)] I am not sure.
  \end{itemize} 
\end{frame}


\begin{frame}
  If \(f(x)\) is discontinuous on \([1,3]\), then we know \(f(x)\) has a discontinuity at \underline{\hspace{1in}} number in \([1,3]\). 
  \medskip
  \begin{itemize} \setlength\itemsep{2ex}
    \item[(a)] every
    \item[(b)] exactly one
    \item[(c)] at least one
    \item[(d)] 
    \item[(e)] I am not sure.
  \end{itemize} 

  \toinstructors{Could be used as a lead-in to discuss continuity over intervals.}
\end{frame} 


\begin{frame}
  Which of the following tell us a function \(f(x)\) is NOT continuous at \(a\)?
  \medskip
  \begin{itemize} \setlength\itemsep{2ex}
    \item[(a)] \(f\) is not defined at \(a\).
    \item[(b)] \(f\) is defined at \(a\) but \(\lim_{x \to a} f(x) \ne f(a)\).
    \item[(c)] \(\lim_{x \to a^{-}} f(x) = \pm \infty\) or \(\lim_{x \to a^{+}} f(x) = \pm \infty\).
    \item[(d)] \(\lim_{x \to a^{-}} f(x) \ne \lim_{x \to a^{+}} f(x)\).
    \item[(e)] I am not sure.
  \end{itemize} 
\end{frame}


\begin{frame}
  True or false? Every piecewise function has \emph{at least one} discontinuity in its domain.

  \medskip
  \begin{itemize} \setlength\itemsep{2ex}
    \item[(a)] True.
    \item[(b)] False
    \item[(c)] 
    \item[(d)] 
    \item[(e)] I am not sure.
  \end{itemize} 
\end{frame} 



\section{Limits at infinity}
\begin{frame}
  How many horizontal asymptotes can a function have?

  \medskip
  \begin{itemize} \setlength\itemsep{2ex}
    \item[(a)] One.
    \item[(b)] Two.
    \item[(c)] At most two.
    \item[(d)] At most infinitely many.
    \item[(e)] I am not sure.
  \end{itemize} 
\end{frame}


\begin{frame}
  Can a function cross its horizontal asymptotes?

  \medskip
  \begin{itemize} \setlength\itemsep{2ex}
    \item[(a)] Yes.
    \item[(b)] No.
    \item[(c)] 
    \item[(d)] 
    \item[(e)] I am not sure.
  \end{itemize} 
\end{frame}


\begin{frame}
  What can we deduce from knowing \(\lim_{x \to \infty} f(x)\) exists? 

  \medskip
  \begin{itemize} \setlength\itemsep{2ex}
    \item[(a)] \(f\) is not increasing on some interval \((a, \infty)\)
    \item[(b)] \(f\) is not decreasing on some interval \((b, \infty)\)
    \item[(c)] \(f\) is not oscillating on some interval \((c, \infty)\)
    \item[(d)] 
    \item[(e)] I am not sure.
  \end{itemize} 
\end{frame}


\begin{frame}
  Analyze the dominant terms of \(\lim_{x \to \infty} \frac{x^{2}+3x}{\sqrt{x+3}}\). With only a quick calculations, the limit should be

  \medskip
  \begin{itemize} \setlength\itemsep{2ex}
    \item[(a)] \(\infty\) or \(-\infty\).
    \item[(b)] a finite number.
    \item[(c)] 
    \item[(d)] 
    \item[(e)] I am not sure.
  \end{itemize} 
\end{frame}



\section{Derivatives}
\begin{frame}
  % Let's pop a balloon by dropping it on a needle. 
  What is the balloon's velocity when it popped?

  \medskip
  \begin{itemize} \setlength\itemsep{2ex}
    \item[(a)] Does not exist because there is no balloon.
    \item[(b)] Non-zero because the balloon has to be moving. 
    \item[(c)] Zero because time freezes at that very instant. 
    \item[(d)] 
    \item[(e)] I am not sure.
  \end{itemize} 
  \toinstructors{Philosophical open question. Pop a balloon in class. Can be used as a lead-in to define the derivative as a model for instantaneous rate of change.}
\end{frame}


\begin{frame}
  True or false? If a function is continuous at \(a\), then it must be differentiable at \(a\).

  \medskip
  \begin{itemize} \setlength\itemsep{2ex}
    \item[(a)] True.
    \item[(b)] False.
    \item[(c)] 
    \item[(d)] 
    \item[(e)] I am not sure.
  \end{itemize} 
\end{frame}


\begin{frame}
  Compare the domains of \(f\) and \(f'\).

  \medskip
  \begin{itemize} \setlength\itemsep{2ex}
    \item[(a)] \(f'\) is defined wherever \(f\) is defined.
    \item[(b)] The domain of \(f'\) can be larger than the domain of \(f\).
    \item[(c)] The domain of \(f'\) can be smaller than the domain of \(f\).
    \item[(d)] 
    \item[(e)] I am not sure.
  \end{itemize} 
\end{frame}


\begin{frame}
  If a function is \emph{not} differentiable on \((0,1)\), then 

  \medskip
  \begin{itemize} \setlength\itemsep{2ex}
    \item[(a)] \(f\) is not continuous on \((0,1)\).
    \item[(b)] \(f\) is a piecewise function.
    \item[(c)] 
    \item[(d)] 
    \item[(e)] I am not sure.
  \end{itemize} 
\end{frame}


\begin{frame}
  If a function is \emph{not} differentiable on \((0,1)\), then 

  \medskip
  \begin{itemize} \setlength\itemsep{2ex}
    \item[(a)] \(f'(a)\) does not exists at any number in \((0,1)\).
    \item[(b)] \(f'(a)\) does not exists for at least one number in \((0,1)\).
    \item[(c)] 
    \item[(d)] 
    \item[(e)] I am not sure.
  \end{itemize} 
\end{frame}


\begin{frame}
  If a function is \emph{not} differentiable on \((0,1)\), then \(f'\) is not differentiable at \underline{\hspace{1in}} number in \((0,1)\).

  \medskip
  \begin{itemize} \setlength\itemsep{2ex}
    \item[(a)] every 
    \item[(b)] exactly one
    \item[(c)] at least one
    \item[(d)] 
    \item[(e)] I am not sure.
  \end{itemize} 
\end{frame}


\begin{frame}
  (True or false?) If \(f(x)\) is not differentiable at \(a\), then \(f\) is not continuous at \(a\).

  \medskip
  \begin{itemize} \setlength\itemsep{2ex}
    \item[(a)] True.
    \item[(b)] False.
    \item[(c)] 
    \item[(d)] 
    \item[(e)] I am not sure.
  \end{itemize} 
\end{frame}


\section{Differentiation Rules}
\begin{frame}
  (True or false?) By the power rule, \(\frac{d}{dx} \frac{1}{\sqrt{x}} = \frac{1}{\frac{1}{2}x^{-1/2}}\).

  \medskip
  \begin{itemize} \setlength\itemsep{2ex}
    \item[(a)] True.
    \item[(b)] False.
    \item[(c)] 
    \item[(d)] 
    \item[(e)] I am not sure.
  \end{itemize} 
\end{frame}


\begin{frame}
  (True or false?) By the power rule, \(\frac{d}{dx} 2^{x} = x \cdot 2^{x-1}\).

  \medskip
  \begin{itemize} \setlength\itemsep{2ex}
    \item[(a)] True.
    \item[(b)] False.
    \item[(c)] 
    \item[(d)] 
    \item[(e)] I am not sure.
  \end{itemize} 
\end{frame}


\begin{frame}
  (True or false?) By the power rule, \(\frac{d}{dx} 2^{10} = 10 \cdot 2^{9}\).

  \medskip
  \begin{itemize} \setlength\itemsep{2ex}
    \item[(a)] True.
    \item[(b)] False.
    \item[(c)] 
    \item[(d)] 
    \item[(e)] I am not sure.
  \end{itemize} 
\end{frame}


\begin{frame}

  \medskip
  \begin{itemize} \setlength\itemsep{2ex}
    \item[(a)] 
    \item[(b)] 
    \item[(c)] 
    \item[(d)] 
    \item[(e)] I am not sure.
  \end{itemize} 
\end{frame}


\section{The Chain Rule}
\begin{frame}
  (True or false?) 

  \medskip
  \begin{itemize} \setlength\itemsep{2ex}
    \item[(a)] True.
    \item[(b)] False.
    \item[(c)] 
    \item[(d)] 
    \item[(e)] I am not sure.
  \end{itemize} 
\end{frame}


\begin{frame}
  Can we use use implicit differentiation to find the derivative of \(y^{2} + x^{2} = 1\) at the point \((5,5)\)?

  \medskip
  \begin{itemize} \setlength\itemsep{2ex}
    \item[(a)] Yes, we plug \(x = 5\) and \(y = 5\) into \(dy/dx\).
    \item[(b)] Yes? I'm not sure.
    \item[(c)] No.
    \item[(d)] No? I'm not sure.
    \item[(e)] I don't understand the question.
  \end{itemize} 
\end{frame}



% template
% \section{}
% \begin{frame}
%
%
%   \medskip
%   \begin{itemize} \setlength\itemsep{2ex}
%     \item[(a)] 
%     \item[(b)] 
%     \item[(c)] 
%     \item[(d)] 
%     \item[(e)] I am not sure.
%   \end{itemize} 
% \end{frame}

\end{document}
