\documentclass[../main]{subfiles} 

% \title{Introduction to derivatives}
% \topic{Introduction to derivatives}
% \date{September 25 to 29, 2023}
% Stewart (9E). Sections 2.7 and 2.8.

\begin{document}

%--------------------------------------------------
%
% instructor notes
%
%--------------------------------------------------
Notes.
  \begin{enumerate}
    \item See the textbook page 141 for the point-slope form of the equation of a line. It is stated in the sidenote just below Figure 1.
    \item See the textbook page 22 for the slope-intercept form of the equation of a line.
  \end{enumerate}
%
% 

%--------------------------------------------------
%
% learning objectives
%
%--------------------------------------------------
Learning objectives.
  At the end of these lectures, students should be able to
  \begin{itemize}
    \item compute and graph the tangent line of a curve \(y = f(x)\) at a point \((a, f(a))\),
    \item compute the derivative of a function \(f(x)\) at a number \(a\),
    \item interpret a derivative as a rate of change of given quantities,
    \item compute the derivative of simple functions from the definition,
    \item recognize a derivative from various notations,
    \item determine where a function is differentiable, and
    \item use the continuity of a function \(f(x)\) to determine where a function is not differentiable.
  \end{itemize}


%--------------------------------------------------
%
% activities
%
%--------------------------------------------------
\begin{outline}{sec}{The tangent line of a curve}{page} \label{act:tangent}
  \begin{enumerate}
    \item Review the \emph{slope} of a line and the \emph{point-slope form} \(y - y_{1} = m(x - x_{1})\) of the equation of a line.
    \item {The tangent line to a curve is defined as follows.}
      \begin{mdframed}[style=simple]
        Let \(y = f(x)\) be a curve and fix a point \(P = (a, f(a))\) on the curve.  The \emph{tangent line to the curve \(y = f(x)\) through \(P\)} is a line that passes through \(P\) with slope 
        \begin{equation} \label{eq:m}
          m = \lim_{x \to a} \frac{f(x) - f(a)}{x - a} 
        \end{equation}
        provided that this limit \emph{exists}.
      \end{mdframed}
        
    \item \label{part:tangent1} (Recommended as a group exercise) We study the slope of the tangent line to the curve \(y = x^{3}\) near the point \(P = (1,1)\).
      \begin{enumerate}
        \item Identify the function \(f(x)\) and the number \(a\) required by Equation~\eqref{eq:m}.  
        \item Evaluate the limit \(m = {\lim_{x \to a}} \frac{f(x) - f(a)}{x - a}\). Hint: Factor \(x^{3} - a^{3} = (x - a) (x^{2} + ax + a^{2})\).
        \item Draw the tangent line on Figure~\ref{fig:cubic}. Write down the point-slope form of the equations of the tangent line. By looking at the graph, write down the slope-intercept form of the equation of the tangent line.
      \end{enumerate}
      \begin{figure}
        \centering
        \includegraphics{../standalones/build/plot_cubic}
        \caption{A plot of \(y = x^{3}\) and the point \((1,1)\).}
        \label{fig:cubic}
      \end{figure}

  \item Consider the curve \(y = |x|\) near the origin \((0,0)\). The graph of this curve is shown in Figure~\ref{fig:abs}. Discuss the existence of the tangent line at the origin \((0,0)\).  Highlight the condition in the definition of a tangent line that \(|x|\) does not satisfy at \(0\). 
      \begin{figure}
        \centering
        \includegraphics{../standalones/build/plot_abs}
        \caption{A plot of \(|x|\).}
        \label{fig:abs}
      \end{figure}
    \item {To compute the tangent line of a curve \(y = f(x)\) at a point \(P = (a, f(a))\), we need to find its slope \(m\) using Equation~\eqref{eq:m}~or~\eqref{eq:m2}, then use the point-slope form to write down its equation. Note Equation~\eqref{eq:m2} appears in Outline~\ref{act:derivative-as-function}.}
    \item {A tangent line to a curve is a \emph{uniquely} defined line passing through a point on the curve. Having a ``corner'' or a ``kink'' at a point is one of a few reasons that the tangent line does not exist at this point. In the next section, we will study more situations where a tangent line does not exist.}
  \end{enumerate}
\end{outline}



\begin{outline}{sec}{The derivative and rate of change}{page} \label{act:derivative}
  \begin{enumerate}
    \item {The \emph{derivative} of a function \emph{at a number} \(a\), denoted by \(f'(a)\), is}
      \begin{mdframed}[style=simple]
        \begin{equation} \label{eq:derivative} 
          f'(a) = \lim_{x \to a} \frac{f(x) - f(a)}{x - a},
        \end{equation}
        provided that this limit \emph{exists}.
      \end{mdframed}
        
    \item \label{part:balloon} (SageMath demonstration \url{https://sagecell.sagemath.org/?q=usjqlb}) Let \(h(t)\) be the displacement in centimeters of a balloon relative to the bottom edge of a blackboard over time \(t\).  If \(t_{1}, t_{2}\) are two time values, then
      \[
        \frac{h(t_{2}) - h(t_{1})}{t_{2} - t_{1}}.
      \]
      represents the \emph{average} rate of change of the displacement of the balloon between times \(t_{1}\) and \(t_{2}\).

      \begin{enumerate}
        \item What is another word for ``rate of change in distance?''
        \item Assume \(t_{1} < t_{2}\).  For each of the condition below, describe the displacement of the balloon at time \(t_{2}\) relative to its displacement at time \(t_{1}\).
          \[
            \frac{h(t_{2}) - h(t_{1})}{t_{2} - t_{1}} > 0, 
            \hspace{2em}
            \frac{h(t_{2}) - h(t_{1})}{t_{2} - t_{1}} = 0, 
            \hspace{2em}
            \frac{h(t_{2}) - h(t_{1})}{t_{2} - t_{1}} < 0.
          \]
        % \item Fix a time \(a\).  Interpret the quantity \(\frac{h(t) - h(a)}{t - a}\) in terms of rate of change.
        % \item Interpret the derivative \(h'(a)\) in terms of the steepness of the tangent line on the graph of \(h\).
      \end{enumerate}
  \item Given a function \(f\) of a quantity \(t\), i.e. \(f(t)\), then the derivative \(f'(a) = {\lim_{t \to a}}\frac{f(t) - f(a)}{t - a}\) is the \emph{instantaneous} rate of change of \(f\) with respect to \(t\) at \(t = a\).
    \item[\faComments{}] (Group discussion) Let \(v(t)\) be the \emph{velocity} at time \(t\) of a car travelling in a straight line on a \emph{frictionless} surface. Assume the driver can change the velocity of the car. Fix a particular time at \(t = a\). Interpret \(v'(a)\) as a rate of change. Think of an action by the driver that causes \(v'(a) > 0\)?  How can the driver achieve \(v'(a) = 0\)?  What can you say about \(v'(a)\) if the car is slowing down at time \(a\)?
    \item {The derivative of a function \(f(x)\) \emph{at a number} \(a\) is another name for the \emph{slope} of the tangent to the curve \(y = f(x)\) at the point \((a, f(a))\).}
    \item {Given a quantifity \(f\) that is a function of \(x\) and is represented as \(f(x)\), then the derivative \(f'(a)\) describes the \emph{instantaneous rate of change} of the quantity \(f\) at \(x = a\).}
  \end{enumerate}
\end{outline}




\begin{outline}{sec}{Derivatives as functions}{page} \label{act:derivative-as-function}
  \begin{enumerate}
    \item With the change of variable \(h = x - a\), the limit in Equation~\eqref{eq:m} can be written as
      \begin{equation} \label{eq:m2}
        m = \lim_{h \to 0} \frac{f(a + h) - f(a)}{h}.
      \end{equation}

      Compute the slope of the curve in Part~\ref{part:tangent1} using Equation~\eqref{eq:m2}.  This can be used as an additional problem if time does not allow an in-class discussion.

    \item {The \emph{derivative} of \(f(x)\) as a function, denoted by \(f'(x)\), is}
      \begin{mdframed}[style=simple]
        \begin{equation} \label{eq:derivative-function} 
          f'(x) = \lim_{h \to 0} \frac{f(x + h) - f(x)}{h}.
        \end{equation}
        The function \(f'(x)\) is defined wherever the limit in Equation~\eqref{eq:derivative-function} exists.
      \end{mdframed}
        
    All of the following mean the function that is the derivative of the function \(y = f(x)\).
      \[
        f'(x) = y' = \frac{dy}{dx} = \frac{df}{dx} = \frac{d}{dx} f(x).
      \]
    \item {The symbol \(\frac{d}{dx}\) is called the \emph{differentiation operator} with respect to \(x\) because it takes a function \(f(x)\) and produces another function \(f'(x)\). To \emph{differentiate} a function \(f(x)\) is to find the function \(f'(x)\).}
    \item {To indicate the derivative of a function \(y = f(x)\) at a specific value \(a\) (as discussed in Outline~\ref{act:derivative}), we use the notation 
        \[
          {\frac{dy}{dx}}\bigg|_{x = a} \quad\text{or}\quad {\frac{dy}{dx}}\bigg]_{x = a}.
      \]}
    \item Differentiate \(y = \sqrt{x}\).  
    \item {If the derivative of \(f(x)\) at a number \(a\) exists, then we say \(f(x)\) is \emph{differentiable} at \(a\). If \(f(x)\) is \emph{differentiable at every number} on an \emph{open} interval \(I\), then we say \(f(x)\) is differentiable on \(I\).}
    \item Where is \(y = \sqrt{x}\) differentiable? How is differentiability related to the domain of \(f'(x)\)?
    \item \textbf{Theorem.} If a function \(f\) is differentiable at \(a\), then \(f\) is continuous at \(a\).
      \begin{enumerate}
        \item Highlight that continuity is \emph{necessary but not sufficient} for differentiability.
        % \item Describe the contrapositive of this statement.
      \end{enumerate}
      
    \item Explain why each of the functions is not differentiable at \(0\). 
      \[
        g_{1}(x) = \frac{x+1}{x}, 
        \hspace{2em}
        g_{2}(x) = \begin{cases} 1 & \text{if } x \ge 0 \\ 0 & \text{if } x < 0 \end{cases},
        \hspace{2em}
        g_{3}(x) = x^{1/3}.
      \]
    \item The absolute value function \(|x|\) is continuous at every number. Discuss its non-differentiability at \(0\).
    \item {For every non-negative integer \(n\), the symbol \(f^{(n)}(x)\) means iterated applications of \(\frac{d}{dx}\) to \(f(x)\) a total of \(n\) times and \(f^{(0)}(x) = f(x)\).}
    \item Describe acceleration as a higher derivative of the position function \(h(x)\).
    \item Compute the second derivative of \(x^{2}\) using the definition of a derivative as a function.
    \item {The derivative of a function becomes a function when we let the symbol \(a\) in Equation~\eqref{eq:derivative} vary.}
    \item {To describe where a function \(f(x)\) is differentiable is to describe the domain of its derivative \(f'(x)\). A function must be continuous at \(a\) for its derivative to exist at \(a\).  If a function is \emph{not continuous} at \(a\), then it is \emph{not differentiable} at \(a\).  But not all continous functions are differentiable.}
  \end{enumerate}
\end{outline}




%--------------------------------------------------
%
% additional problems
%
%--------------------------------------------------
\begin{enumerate}
\item Repeat the computation in Outline~\ref{act:tangent} Part~\ref{part:tangent1} using the function \(x^{3}\) at point \((2,8)\).
\item Consider the function \(g(x) = x^{2} + x + 1\). Compute the derivative \(g'(1)\).
\item Compute the third derivative of \(x^{3}\) using the definition of a derivative as a function.
\item From the textbook, exercises 1, 3, 5-8, 11-13, 16-19, 23-25, 29, 37, 57 in Section 2.7
\item From the textbook, exercises  1-14 in Section 2.8
\item From the textbook, True-False Quiz (questions 1-26) on page 167; there are also questions that review Chapter~2 on pages 168 to 170. 
\end{enumerate}

\end{document}

