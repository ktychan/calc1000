\documentclass[../main]{subfiles} 

% \title{Differentiation Rules, Part 1}
% \topic{Differentiation Rules}
% \date{October 2 to 6, 2023}
% {Stewart (9E). Sections 3.1, 3.2, and 3.3. Pages 174 to 199.}

\begin{document}


%--------------------------------------------------
%
% instructor notes
%
%--------------------------------------------------
Notes.
  \begin{enumerate}
    % \item It is recommended to spend \(30\) minutes introducing differentiation rules and basic examples, \(20\) minutes on group exercises, and \(30\) minutes to review group exercises. \todo{This seems like a very tight schedule.}
    % \item Instructors may invite teaching assistants assigned to their section to help answer questions during group exercises. A teaching assistant has the right to decline such an invitation.
    \item Review Section 1.2, Section 1.4, and Appendix D for essential functions, exponential functions and relevant trigonometric identities, respectively, as necessary.
    % \item See Section 1.4 for a review of (general) exponential functions and their properties.
    % \item See Appendix D for the addition formulas for \(\sin(x + y)\) and \(\cos(x + y)\).
    % \item \todo{Change to ``Review Section 1.4 and Appendix D as necessary.''}
  \end{enumerate}




%--------------------------------------------------
%
% learning objectives
%
%--------------------------------------------------

  At the end of the lecture, students should be able to
  \begin{itemize}
    \item apply the power rule, 
    \item apply the constant multiple rule, the sum rule, and the difference rule, 
    \item differentiate the natural exponential function \(e^{x}\), 
    \item apply the product rule and the quotient rule,
    \item differentiate \(\sin(x)\) and \(\cos(x)\), and
    \item use any combination of differentiation rules to differentiate other trigonometric functions.
  \end{itemize}




%--------------------------------------------------
%
% General instructions
%
%--------------------------------------------------
% \begin{instruction}[Instruction]{How to ...}
%   \begin{enumerate}[label=(\alph*)]
%     \item TODO
%   \end{enumerate}
% \end{instruction}
%



%--------------------------------------------------
%
% activities
%
%--------------------------------------------------
% \begin{outline}{sec}{Motivation}{page} \label{act:motivation}
%   \begin{enumerate}
%     \item Motivate differentiation rules by the problem solving strategy of breaking down a problem into sub-problems.
%   \end{enumerate}
% \end{outline}
%


\begin{outline}{sec}{Derivatives of polynomials and power functions}{page} 
  \label{act:polynomials}
  \begin{enumerate}
    \item \textbf{Theorem} (The Constant Rule). If \(c\) is a constant, then \(\dfrac{d}{dx}(c) = 0\).
    \item Review the convention \(x^{0} = 1\).
    \item \textbf{Theorem} (The Power Rule). Power functions \(x^{n}\) are always differentiable, and
      \begin{mdframed}[style=simple]
        \[
          \frac{d}{dx} x^{n} = n x^{n-1} \text{ for any real number } n.
        \]
      \end{mdframed}
        

      Highlight that the derivative of a constant is a special case of the power rule.
      
    \item Differentiate \(x^{3}\) and \(\sqrt{x}\).

    % \item Discuss functions can be combined using addition, subtraction and constant multiplication (scaling) to build more complicated functions. \todo{Maybe leave it to instructors to motivate linearity?}

    \item \textbf{Theorem} (The Constant Multiple Rule). If \(c\) is a constant and \(f\) is a differentiable function, then
      \begin{mdframed}[style=simple]
        \[
          \frac{d}{dx} c f(x) = c \dfrac{d}{dx} f(x).
        \]
      \end{mdframed}
        

    \item \textbf{Theorem} (The Sum and Difference Rules). If \(f\) and \(g\) are both differentiable, then
      \begin{mdframed}[style=simple]
        \begin{align*}
          \frac{d}{dx} \Big[f(x) + g(x) \Big] &= \frac{d}{dx} f(x) + \frac{d}{dx} g(x), \\
          \frac{d}{dx} \Big[f(x) - g(x) \Big] &= \frac{d}{dx} f(x) - \frac{d}{dx} g(x).
        \end{align*}
      \end{mdframed}
        

    % \item  \textbf{Theorem} (Linearity). If \(c_{1}, c_{2}\) are constants and \(f,g\) are both differentiable, then 
    %   \begin{center}
    %     \setlength\fboxsep{1em}
    %     \fbox{
    %       \begin{minipage}{.8\textwidth}
    %         \begin{align*}
    %           \frac{d}{dx} \bigg( c_{1}f(x) + c_{2}g(x) \bigg) = c_{1} f'(x) + c_{2} g'(x).
    %         \end{align*}
    %       \end{minipage}
    %     }
    %   \end{center}

    \item Differentiate \(2 x^{3} + \sqrt{x} - 1\) and \((x + 2)\left(\frac{1}{x^{1/e}} - 1\right)\).  Highlight that new derivatives are obtained from old derivatives. 

    \item Discuss that neither \(\sqrt{-x^{3} + 1}\) nor \((4x^{2} + x - 1)^{2}\) is a power function. Discuss that \((a + b)^{n} \ne a^{n} + b^{n}\) in general assuming such expressions are well-defined. 

    \item {Use the power rule to compute the derivative of power functions.}

    \item {Many functions are constant multiple, sums, and differences of simpler functions. The constant multiple rule, sum and difference rules can be used to break up complicated derivatives into simpler ones.}
\end{enumerate}
\end{outline}



\begin{outline}{sec}{The derivative of the natural exponential function \texorpdfstring{\(e^{x}\)}{exp(x)}.}{page} 
  \label{act:exp}
  \begin{enumerate}
    \item Distinguish \emph{general} exponential functions from power functions. 
    \item Discuss that \(e^{-x}\) and \(e^{x^{2}}\) are neither exponential functions nor power functions by definition. 
    \item Define \(e\) as the unique number satisfying
      \[
        \lim_{h \to 0} \frac{e^{h} - 1}{h} = 1.
      \]

    \item \textbf{Theorem}. 
      \begin{mdframed}[style=simple]
        \[
          \frac{d}{dx} e^{x} = e^{x}.
        \]
      \end{mdframed}
        

    \item Differentiate \(2x^{\pi} + e^{3} + \frac{2}{3}e^{x}\).
    % \item \todo{This feels out of place. Maybe wait until later?} Review the natural logarithm \(\ln(x)\). For any \(b > 0\), review and the relation \(b = e^{\ln(b)}\) and derive
    %   \[
    %     b^{x} = e^{\ln(b) x}.
    %   \]
    %   Discuss that the constant multiple rule \emph{cannot} be used to \emph{directly} evaluate \(\frac{d}{dx}b^{x}\).  We will learn to differentiate \(b^{x}\) using the Chain Rule from Section 3.4 in an upcoming lecture.
    \item {The derivative of the \emph{natural} exponential function \(e^{x}\) is itself \(e^{x}\).}
    \item {We will learn to differentiate a general exponential function \(b^{x}\) for some \(b > 0\) using the chain rule later.}
\end{enumerate}
\end{outline}



\begin{outline}{sec}{The product and quotient rules}{page} \label{act:products}
  \begin{enumerate}
    \item \textbf{Theorem} (The Product Rule). If \(f(x)\) and \(g(x)\) are \emph{differentiable} functions, then 
      \begin{mdframed}[style=simple]
        \[
          \frac{d}{dx} f(x)g(x) = f'(x) g(x) + f(x)g'(x).
        \]
      \end{mdframed}
        
    \item Differentiate \(x e^{x}\).

    \item \textbf{Theorem} (The Quotient Rule). If \(f(x)\) and \(g(x)\) are \emph{differentiable} functions, then 
      \begin{mdframed}[style=simple]
        \[
          \frac{d}{dx} \frac{f(x)}{g(x)} = \frac{f'(x) g(x) - f(x) g'(x)}{\big[g(x)\big]^{2}}.
        \]
      \end{mdframed}
        
    \item Differentiate \(\dfrac{x^{\frac{3}{2}}}{\pi e^{x}}\).
    \item {Use the product rule to differentiate products of functions. The derivative of a product \(f(x)g(x)\) is \emph{NOT} the product of their individual derivatives.}
    \item {Use the quotient rule to differentiate quotients of functions. The derivative of a quotient \(\frac{f(x)}{g(x)}\) is \emph{NOT} the quotient of their individual derivatives.}
\end{enumerate}
\end{outline}



\begin{outline}{sec}{Trigonometric functions}{page} \label{act:trigs}
  \begin{enumerate}
    \item Introduce two special limits
      \[
        \lim_{x \to 0} \frac{\sin(x)}{x} = 1 \quad\text{and}\quad \lim_{x \to 0} \frac{\cos(x) - 1}{x} = 0.
      \]
      \begin{figure}[ht]
        \centering
        \includegraphics[width=.8\textwidth]{../standalones/build/plot_sin_over_x}
        \includegraphics[width=.8\textwidth]{../standalones/build/plot_cos_minus_one_over_x}
        \caption{The graph of \(\dfrac{\sin(x)}{x}\) and \(\dfrac{\cos(x) - 1}{x}\) near \(x=0\).}
      \end{figure}
      
    \item Prove the derivative of \(\sin(x)\). Review Appendix D for trigonometric identities as necessary.
    \item \textbf{Theorem}.
      \begin{mdframed}[style=simple]
        \[
          \frac{d}{dx} \sin(x) = \cos(x) 
          \quad\text{and}\quad
          \frac{d}{dx} \cos(x) = -\sin(x).
        \]
      \end{mdframed}
        

    \item {The derivative of \(\sin(x)\) is \(\cos(x)\). Know its proof.}
    \item {The derivative of \(\cos(x)\) is \(-\sin(x)\). Know its proof.}
    \item {To find the derivative of other trigonometric functions, rewrite them in terms of \(\sin(x)\) and \(\cos(x)\) according to their definitions then apply differentiation rules.}
\end{enumerate}
\end{outline}



\begin{outline}{sec}{In-class Group Exercises}{page} \label{act:worksheet}
  Depending on time constraints, an instructor may choose some or all of the following exercises for in-class group work. Give students a few minutes to work on the following exercises in small groups or independently. For each of the chosen exercises, optionally ask a student volunteer to present their solution and discuss the solution if time permits. 

  Students should complete these exercises at home if they were not completed in class.

  \begin{enumerate}
    \item Differentiate \(x^{3} + 3x^{2} + 3x + 1\).
    \item Compute the first, second, and third derivatives of \(e^{x}\) and \(\sin(x)\).
    \item Differentiate \(e^{x} (2\sqrt{x} - \sin(x))^{2}\) using the product rule.
    \item Differentiate \(\tan(x) = \dfrac{\sin(x)}{\cos(x)}\) using the quotient rule.  Hint: \(\sin^{2}(x) + \cos^{2}(x) = 1\) and \(\sec(x) = \dfrac{1}{\cos(x)}\).
  \end{enumerate}

\end{outline}


%--------------------------------------------------
%
% additional problems
%
%--------------------------------------------------
Exercises.
\begin{enumerate}
\item Recall that \(f^{(n)}\) is the \(n\)-th derivative of \(f(x)\). If \(f'(x) = f'(0)f(x)\), evaluate \(f^{(1)}, f^{(2)}, f^{(3)}\). What is \(f^{(100)}\)?
\item If \(f(x) = x^{5} + x^{4} + x^{3} + x^{2} + x + 1\), what is \(f^{(6)}\)?
\item If \(f(x) = \sin(x) - \cos(x)\), find a formula for \(f^{(n)}(x)\) for every positive integer \(n\).
\item Using the limit definition of the derivative, prove that \(\dfrac{d}{dx} \cos(x) = -\sin(x)\).
\item From the textbook, exercises 1a,  2-40, 49-50, 53, 55, 59-62, 71, 75, 79-81 in Section 3.1.
\item From the textbook, exercises  1-36, 44-54, 60, 63-66 in Section 3.2.
\item From the textbook, exercises 1-27, 31a, 35-36, 38-41, 43, 45-55, 61-63, 64a,b in Section 3.3.
\end{enumerate}

\end{document}
