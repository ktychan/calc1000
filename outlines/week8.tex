\documentclass[../main]{subfiles} 

% \title{Application of Differentiation 2}
% \topic{Application of Differentiation 2}
% \date{November 6 to 10}
% {Stewart (9E). Section 4.3, 4.4, 4.5, and 4.7. Pages 303 to 329 and 336 to 351.}

\begin{document}


%--------------------------------------------------
%
% instructor notes
%
%--------------------------------------------------
Notes.
  \begin{itemize}
    \item The topic ``slant asymptotes'' in Section 4.5 is skipped.
    \item Curve sketching is split into two parts in the textbook. The first part is in Section 4.3, pages 303 to 305. The second part is in Section 4.5, pages 320 to page 327.
  \end{itemize}




%--------------------------------------------------
%
% learning objectives
%
%--------------------------------------------------

  At the end of the lecture, students should be able to
  \begin{itemize}
    \item identify indeterminate forms,
    \item use l'H\^{o}pital's rule to evaluate indeterminate forms,
    \item formulate an optimization problem from a word problem and use calculus to solve it, and
    \item sketch curves without computer aid.
  \end{itemize}




%--------------------------------------------------
%
% General instructions
%
%--------------------------------------------------
% \begin{instruction}[Instruction]{How to ...}
%   \begin{enumerate}[label=(\alph*)]
%     \item TODO
%   \end{enumerate}
% \end{instruction}
%


%--------------------------------------------------
%
% activities
%
%--------------------------------------------------

\begin{outline}{sec}{L'H\^opital's rule}{page} \label{outline:lhopital}
  \begin{enumerate}
    \item Indeterminate forms. %For the next two definitions, we use a box \(\square\) as a placeholder. The meaning of the placeholder \(\square\) is \emph{consistent in a single sentence}. The placeholder \(\square\) can be a number, the positive infinity \(+\infty\) or the negative infinity \(-\infty\). For example, if \(\square\) means a number \(a\), then all instances of \(\square\) in a sentence mean the number \(a\).
      \begin{itemize}
        \item A limit \(\lim_{x \to a} \frac{f(x)}{g(x)}\) is an \emph{indeterminate form of type} \(\dfrac{0}{0}\) if both \(\lim_{x \to a} f(x)\) and \(\lim_{x \to a} g(x)\) are \(0\).  
        \item A limit \(\lim_{x \to a} \frac{f(x)}{g(x)}\) is an \emph{indeterminate form of type} \(\dfrac{\infty}{\infty}\) if both \(\lim_{x \to a} f(x)\) and \(\lim_{x \to a} g(x)\) are infinities. They do not have to be the same infinity.
        \item The limit of an indeterminate form may or may not exist.
        \item The symbol \(a\) can be a number, the positive infinity \(+ \infty\), or the negative infinity \(-\infty\).
      \end{itemize}
      
    \item \textbf{Theorem} (L'H\^{o}pital's Rule).
      \begin{mdframed}[style=simple]
        Suppose \(f\) and \(g\) are both differentiable and \(g'(x) \ne 0\) on an open interval \(I\) that contains the number \(a\) (except possibly at \(a\)). 

        Suppose \(\lim_{x \to a}\frac{f(x)}{g(x)}\) is an indeterminate form of type \(\frac{0}{0}\) or \(\frac{\infty}{\infty}\).  Then 
        \[ \lim_{x \to a} \frac{f(x)}{g(x)} = \lim_{x \to a} \frac{f'(x)}{g'(x)}, \]
        if the limit on the right side exists or is \(\infty\) or \(-\infty\).

        \medskip
        L'H\^{o}pital's rule is also valid for one-sided limits and for limits at the positive infinity \(+\infty\) or the negative infinity \(-\infty\).
      \end{mdframed}
        

    \item Suggested problems: Evaluate the limits \(\lim_{x \to 1} \frac{\ln(x)}{x - 1}\) and \(\lim_{x \to 0^{+}} x \ln(x)\).
      
    \item {L'H\^{o}pital's rule can be used to evaluate limits of indeterminate forms \(\frac{0}{0}, \frac{\infty}{\infty}\).}
    \item {Sometimes we need to rewrite a product \(f \cdot g\) as a fraction \(\frac{f}{1/g}\) before applying l'H\^{o}pital's rule.}
\end{enumerate}
\end{outline}



\begin{outline}{sec}{Curve Sketching}{page} \label{outline:curve-sketching}
  \begin{enumerate}
    \item Motivate curve sketching by considering the computer-generated plot of the function
      \[
        y = 6 x^{5} - 521 x^{4} + 2122 x^{3} - 3300 x^{2}
      \]
      using popular online software such as \href{https://www.desmos.com/calculator/9njuthdtv5}{\textsf{Desmos}}, \href{https://www.geogebra.org/calculator/cpjqfsen}{\textsf{GeoGebra}}, \href{https://www.symbolab.com/graphing-calculator}{\textsf{Symbolab}} and \href{https://www.wolframalpha.com/input?i=plot+6*x%5E5+-+521*x%5E4+%2B+2122*x%5E3+-+3300*x%5E2}{\textsf{WolframAlpha}}. Note the sans serif font indicates clickable links. For \href{https://www.symbolab.com/graphing-calculator}{\textsf{Symbolab}}, manually enter the following by copy-and-paste: 
        \begin{center}
          \verb!6*x^5 - 521*x^4 + 2122*x^3 - 3300*x^2!
        \end{center}
        Without zooming in and out, the plots by Desmos, GeoGebra and Symbolab do not seem to have a local minimum. Then look for a local minimum by zooming in and panning if necessary. Notice the plots with the default zoom and pan setting reveal \emph{incomplete information} about the function. 

        Motivate the need for curve sketching by discussing why computer generated plots sometimes only reveal \emph{partial} information and cannot always be relied on to find extreme values.
    \item Review symmetry in functions. Let \(f(x)\) be a function.
      \begin{itemize}
        \item If \(f(-x) = f(x)\) for every \(x\) in its domain, then we say \(f\) is an \emph{even} function. See Figure~\ref{fig:even}.
        \item If \(f(x) = -f(x)\) for every \(x\) in its domain, then we say \(f\) is an \emph{odd} function. See Figure~\ref{fig:odd}.
        \item If there exists some fixed number \(p\) such that \(f(x + p) = f(x)\) for all \(x\) on its domain, then we say \(f\) is a \emph{periodic} function with period \(p\). See Figure~\ref{fig:periodic}.
      \end{itemize}
      

    \item Guideline for sketching a curve \(y = f(x)\). First, label the axes.
      \begin{enumerate}[label=(\Alph*)]
        \item \textbf{Domain}. Determine the domain \(D\) of the function \(f(x)\). 
        \item \textbf{Intercepts}. Determine and plot the \(y\)-intercept and the \(x\)-intercept(s) of the function \(f(x)\). Omit this step if the intercepts do not exist or are difficult to find.
        \item \textbf{Symmetry}. 
          \begin{enumerate}
            \item If the function is \emph{even}, then we only need to sketch the curve for \(x \ge 0\) then \emph{reflect} about the vertical axis to obtain the other half. See Figure~\ref{fig:even}.

            \item If the function is \emph{odd}, then we only need to sketch the curve for \(x \ge 0\) then \emph{rotate} \({180}^{\circ}\) to obtain the other half. See Figure~\ref{fig:odd}.

            \item If the function is \emph{periodic}, then we only need to sketch the curve for an \emph{interval} of length \(p\) and obtain the rest by horizontal translation. See Figure~\ref{fig:periodic}.

            \item[\faIcon{exclamation-triangle}] If a function is not even, not odd \emph{and} not periodic, then skip this step. See Figure~\ref{fig:no-symmetry}.
          \end{enumerate}
        \item \textbf{Asymptotes}. Determine and draw vertical \emph{and} horizontal asymptotes of the function, if any.
        \item \textbf{Intervals of Increase or Decrease}. Use the Increasing/Decreasing Test to determine intervals of increase or decrease. Label these intervals on the horizontal axis.
        \item \textbf{Local Extreme Values}. Determine and plot \emph{local} extreme values.
        \item \textbf{Concavity and Points of Inflection}. Determine intervals of positive/negative concavity and plot all inflection points.
        \item \textbf{Sketch the Curve}. Sketch the graph by passing through each plotted point, rising and falling according to intervals of increase or decrease, and showing concavity as computed above. If the curve has a horizontal asymptote at either side, then the curve should also approach that asymptote on the appropriate side. 
      \end{enumerate}
      
    \item Suggested problems. For each problem below, eloborate and discuss the process of curve sketching.
      \begin{enumerate}
        \item Sketch the graph of \(f(x) = \dfrac{x^{2}}{\sqrt{x + 1}}\).
        \item Sketch the graph of \(g(x) = x e^{x}\).
        \item Sketch the graph of \(y = \ln(4 - x^{2})\).
      \end{enumerate}
      
    \item {Curve sketching uses all properties we have learned about a function so far (domain, intercepts, symmetry, periodicity, asymptotes, intervals of increase/decrease, extreme values, concavity and inflection points) to obtain an overview of a function's behaviour.}
    \item {Use the curve sketching guideline to sketch the graph of a function.}
\end{enumerate}
\end{outline}



\begin{outline}{sec}{Optimization Problems}{page} \label{outline:optimization}
  \begin{enumerate}
    \item Consider a ball placed at height \(y = 0\) and that is launched with initial velocity \(v_{0}\) and initial angle \(\theta\). The horizontal range \(d\) is the distance travelled by the ball when it returns to its initial height \(y = 0\). Figure~\ref{fig:projectile} demonstrates this setup. A ball is launched from the origin in the direction of the blue arrow. The angle between the blue line and the horizontal axis is the initial angle \(\theta\).

      Ignore air resistance and assume gravity is the only force acting on the ball during flight. Then it is known that the horizontal range \(d\) is a function of the initial angle
      \[
        d(\theta) = \dfrac{v_{0}^{2}}{g} \sin(2\theta).
      \]
      Find the initial angle \(\theta\) that maximizes the horizontal range \(d\).
      \begin{figure}[h]
        \includegraphics[width=0.3\textwidth]{../standalones/build/demo_projectile_launcher_1}
        \quad
        \includegraphics[width=0.3\textwidth]{../standalones/build/demo_projectile_launcher_2}
        \quad
        \includegraphics[width=0.3\textwidth]{../standalones/build/demo_projectile_launcher_3}
        \caption{A projectile launched at different angles.}
        \label{fig:projectile}
      \end{figure}
      
      % https://www.toronto.ca/community-people/animals-pets/pets-in-the-city/backyard-hens/
    \item Suggested problems.
      \begin{enumerate}
        \item Suppose we want to build a rectangular fence that encloses an area of \(100\) square feet. What dimension (width and height) minimizes the cost of the fence?

        % https://www.cbc.ca/news/business/blue-jays-season-tickets-changes-1.6853575
        \item A haunted house is trying to maximize revenue for the next year.  This year, if the ticket price was \textdollar{20}, then the average attendance was 500 visitors per evening. If the ticket price was \textdollar{15}, the average attendance was 800 visitors per evening.

          Assume the rate of change in attendance as a function of price is constant. Also, assume selling tickets are their only source of revenue. Find the ticket price that maximizes revenue.

      \end{enumerate}
    \item {Formulating and solving optimization problems are practical skills valued by internships, practicums, undergraduate summer ressearch positions (\href{https://www.nserc-crsng.gc.ca/students-etudiants/ug-pc/usra-brpc_eng.asp}{\textsf{USRA}}, \href{https://sfs.yorku.ca/scholarships/award-search?awardID=4405}{\textsf{DURA}}, and \href{https://lassonde.yorku.ca/research/undergraduate-research-at-lassonde}{\textsf{LURA}}), industry jobs, and graduate programs.}
    \item {General steps to solve an optimization problem.
        \begin{enumerate}[label=(\arabic*)]
          \item Identify quantities in the problem and introduce appropriate notations. Drawing a diagram often helps.
          \item Identify the relation among quantities and express dependent quantities as a function of independent quantities.  You may need to eliminate one or more variables using these relations.
          \item Use calculus to maximize or minimize the function as requested. 
          \item Clearly state your answer using a complete sentence.
        \end{enumerate}
      }
\end{enumerate}
\end{outline}



%--------------------------------------------------
%
% additional problems
%
%--------------------------------------------------
Exercises.
\begin{enumerate}
\item Show the zero function is both even and odd. Are there other other functions that are both even and odd?
\item Evaluate \({\lim_{x \to 1^{+}}} \left( \frac{1}{\ln(x)} - \frac{1}{x-1} \right)\).
\item Evaluate \({\lim_{x \to 0^{+}}} x^{x}\).
\item From the textbook, exercises  45, 56, 57, 59-61, 67 in Section 4.3%This are curve sketching exercises, other exercises in Section 4.3 are noted in 09_applications_1
\item From the textbook, exercises  1-70, 75-76, 80-81, 83-84, 89 in Section 4.4
\item From the textbook, exercises  1-3, 9, 12, 33-35, 55, 57, 59, 61  in Section 4.5%No slant asymptote
\item From the textbook, exercises   2-11, 13, 19-20, 25-27, 29-32, 40, 46-49, 65, 66-67, 71 in Section 4.7
\end{enumerate}


%--------------------------------------------------
%
% appendix
%
%--------------------------------------------------


\noindent\textbf{Examples of symmetries and periodicity in functions}

\begin{figure}[h]  % [h] for here, [ht] for here top, [hb] for here bottom
  \centering
  \begin{minipage}{.45\textwidth}
    \includegraphics{../standalones/build/plot_symmetry_x_squared}
  \end{minipage}
  \quad
  \begin{minipage}{.45\textwidth}
    \includegraphics{../standalones/build/plot_symmetry_cos}
  \end{minipage}
  \caption{Examples of even functions. They possess reflectional symmetry about the vertical axis.}
  \label{fig:even}
\end{figure}

\begin{figure}[h]  % [h] for here, [ht] for here top, [hb] for here bottom
  \centering
  \begin{minipage}{.45\textwidth}
    \includegraphics{../standalones/build/plot_symmetry_x_cubed}
  \end{minipage}
  \quad
  \begin{minipage}{.45\textwidth}
    \includegraphics{../standalones/build/plot_symmetry_sin}
  \end{minipage}
  \caption{Examples of odd functions. They possess rotational symmetry.}
  \label{fig:odd}
\end{figure}

\begin{figure}[h]  % [h] for here, [ht] for here top, [hb] for here bottom
  \centering
  \centering
  \begin{minipage}{.3\textwidth}
    \includegraphics[width=\textwidth]{../standalones/build/plot_periodic_sin}
  \end{minipage}
  \quad
  \begin{minipage}{.3\textwidth}
    \includegraphics[width=\textwidth]{../standalones/build/plot_periodic_cos}
  \end{minipage}
  \quad
  \begin{minipage}{.3\textwidth}
    \includegraphics[width=\textwidth]{../standalones/build/plot_periodic_tan}
  \end{minipage}
  \caption{Examples of periodic functions.}
  \label{fig:periodic}
\end{figure}

\begin{figure}[h]  % [h] for here, [ht] for here top, [hb] for here bottom
  \centering
  \begin{minipage}{.45\textwidth}
    \includegraphics{../standalones/build/plot_no_symmetry_poly}
  \end{minipage}
  \quad
  \begin{minipage}{.45\textwidth}
    \includegraphics{../standalones/build/plot_no_symmetry_wavy}
  \end{minipage}
  \caption{Examples of functions with no apparent symmetry.}
  \label{fig:no-symmetry}
\end{figure}

\end{document}

