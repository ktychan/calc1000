\documentclass[../main]{subfiles} 

% \title{Integration 3}
% \date{November 27 to December 1}

% {Stewart (9E). Sections 5.4 and 5.5. Pages 409 to 427.}

\begin{document}


%--------------------------------------------------
%
% instructor notes
%
%--------------------------------------------------
Notes.
  \begin{itemize}
    \item Properties of the indefinite integrals are taken from the first lines of \textcolor{red}{\fbox{2} Table of Indefinite Integrals} on page 410 of the textbook.
    \item The topic ``symmetry'' in Section~5.5 is treated as additional exercises.
  \end{itemize}
  




%--------------------------------------------------
%
% learning objectives
%
%--------------------------------------------------

  At the end of the lecture, students should be able to
  \begin{itemize}
    \item distinguish between definite integrals and indefinite integrals, 
    \item evaluate simple indefinite integrals, 
    \item interpret Fundamental Theorem of Calculus as the Net Change Theorem,
    \item use the Net Change Theorem to interpret definite integrals of physical rates of change,
    \item apply the substitute rule (also known as \(u\)-substitution) to evaluate definite and indefinite integrals.
  \end{itemize}




%--------------------------------------------------
%
% General instructions
%
%--------------------------------------------------
% \begin{instruction}[Instruction]{How to ...}
%   \begin{enumerate}[label=(\alph*)]
%     \item TODO
%   \end{enumerate}
% \end{instruction}
%


%--------------------------------------------------
%
% activities
%
%--------------------------------------------------
\begin{outline}{sec}{Indefinite Integrals}{page} \label{outline:indefinite_integrals}
  \begin{enumerate}
    % \item The most general antiderivative is speical. And we give it a special name.
    \item \textbf{Definition}. Let \(f(x)\) be a continuous function. The indefinite integral of \(f(x)\), denoted as
      \[
        \int f(x) dx
      \]
      is the most general antiderivative of \(f(x)\). So \emph{an indefinite integral is a function} and it satisfies
      \[
        \frac{d}{dx} \int f(x) dx = f(x).
      \]
      Notice that an indefinite integral is distinguished from a definite integral by the lack of limits of integration in its notation. 
      See Table~\ref{table:indefinite_integrals} in Appendix for a list of common indefinite integrals.
      
    \item If \(F\) is an antiderivative of a continuous function \(f\), then we use the notation \(F(x) ]^{b}_{a}\) to mean \(F(b) - F(a)\). Some authors includes the left bracket, e.g., \([ F(x) ]^{b}_{a} = F(b) - F(a)\).
    \item Express the Fundamental Theorem of Calculus Part 2 using indefinite integrals.
    \item Discuss the convention that the domain that an indefinite integral is the same as its integrand.
    \item Properties of indefinite integrals. 
      \begin{enumerate}[label=(\alph*)]
      \item \(\int c f(x) dx = c \int f(x) dx\), where \(c\) is any constant.
      \item \(\int f(x) + g(x) dx = \int f(x) dx + \int g(x) dx\).
      \item \(\int f(x) - g(x) dx = \int f(x) dx - \int g(x) dx\).
      \end{enumerate}

    \item Suggested problems.
      \begin{enumerate}
      \item Verify that \(\int \sin(\theta) \cos(\theta) d \theta = - \frac{1}{2} \cos^{2}(\theta) + C\).
      \item Evaluate \(\int 6 \cdot 2^{x} + \frac{1}{10} \sqrt[5]{x} + \frac{\sqrt{x}}{x^{2}} dx\).
      \end{enumerate}
      
    \item {An indefinite integral is just a name and a special notation for the most general antiderivative of a function. An indefinite integral is not a new concept.}
    \item {An indefinite integral is a \emph{function} while a definite integral is a number.}
\end{enumerate}
\end{outline}



\begin{outline}{sec}{The Net Change Theorem}{page} \label{outline:net_change}
  \begin{enumerate}
    \item \textbf{Theorem}. The integral of a rate of change is the net change
      \begin{mdframed}[style=simple]
        \[ \int_{a}^{b} F'(x) dx = F(b) - F(a). \]
      \end{mdframed}
      
    \item Relate the statement of the Net Change Theorem to the statement of the Fundamental Theorem of Calculus Part 2.
    \item Suggested problems: 
      \begin{enumerate}
        \item Suppose an object moves along a straight line with position function \(s(t)\), then its velocity is \(v(t) = s'(t)\). Interpret 
          \[
            \int_{t_{1}}^{t_{2}} v(t) dt = s(t_{2}) - s(t_{1})
          \]
          in terms of displacement. % Example on page 413.

        \item Let \(v(t)\) be defined as in part (a). What is the physical meaning of the following integral?
          \[
            \int_{t_{1}}^{t_{2}} |v(t)| dt
          \]
        \item If the units for \(x\) are \textit{feet} and the units for \(a(x)\) are \textit{pounds per foot}, what are the units for \(\frac{da}{dx}\)? What units does \(\int_{0}^{1} a(x) dx\) have? % Exercise 66
        \item The linear density of a rod of length \(4\) metres is given by \(\rho(x) = 9 + \sqrt{2}x\) measured in kilograms per metre, where \(x\) is measured in metres from one end of the rod. Find the total mass of the rod. % Exercise 73.
      \end{enumerate}
      
    \item {The Net Change Theorem gives the Fundamental Theorem of Calculus Part 2 a physical interpretation. It translates an definite integral from mathematics to the language of other sciences.}
\end{enumerate}
\end{outline}




\begin{outline}{sec}{The Substitution Rule}{page} \label{outline:u-substitution}
  \begin{enumerate}
    \item \textbf{Theorem} (The Substitution Rule). If \(u = g(x)\) is a differentiable function whose range is an interval \(I\) and \(f\) is continuous on \(I\), then 
      \begin{mdframed}[style=simple]
        \begin{equation} \label{eq:u-subs}
          \int f(g(x)) g'(x) dx = \int f(u) du.
        \end{equation}
      \end{mdframed}

    \item \textbf{Theorem} (The Substitution Rule for definite integrals). If \(g'\) is continuous on \([a,b]\) and \(f\) is continuous on the range of \(u = g(x)\), then 
      \begin{mdframed}[style=simple]
        \begin{equation} \label{eq:u-subs-definite}
          \int_{a}^{b} f(g(x)) g'(x) dx = \int_{u(a)}^{u(b)} f(u) du.
        \end{equation}
      \end{mdframed}
      Notice the limits of integration are different on two sides of the equation.

    \item Evaluate \(\int_{1}^{e} \frac{\ln(x)}{x} dx\) in two different ways. 
      \begin{enumerate}
        \item First compute the indefinite integral \(F(x) = \int \frac{\ln(x)}{x} dx\) using the substitution rule, i.e., Equation~\eqref{eq:u-subs}, then use the Fundamental Theorem of Calculus.
        \item Apply the Substitution rule for definite integrals, i.e., Equation~\eqref{eq:u-subs-definite} directly.
      \end{enumerate}
    \item Suggested problems.
      \begin{enumerate}
      \item Evaluate \(\int \frac{x^{2}}{x^{3}+1} dx\).
      \item Evaluate \(\int \sin(\sin(\theta)) \cos(\theta) d \theta\).
      \item Evaluate \(\int_{1}^{3} \frac{\sin(1 + \ln(x))}{x} dx\).
      \item Evaluate \(\int \frac{x^{2} + 2x + 1}{x^{3} + 3 x^{2} + 3x + 13} dx\).
      \end{enumerate}
      
    \item {The substitution rule can be used to evaluate integrals of composite functions.}%There is \emph{NO} chain rule for integrals.

    \item {When applying the Substitution Rule for \emph{definite} integrals, i.e., Equation~\eqref{eq:u-subs-definite}, \emph{remember} to change the limits of integraion.}

\end{enumerate}
\end{outline}




%--------------------------------------------------
%
% additional problems
%
%--------------------------------------------------
Exercises.
\begin{enumerate}
\item Let \(f(x)\) be an even function. Show that \(\int_{-a}^{a} f(x) dx = 2 \int_{0}^{a} f(x) dx\).
\item Let \(f(x)\) be an odd function. Show that \(\int_{-a}^{a} f(x) dx = 0\).
\item   From the textbook, exercises 1-24, 27-54, 58-62, 68-69, 71, 73, 75, 77, 81 in Section 5.4
\item   From the textbook, exercises  1-54, 59-80, 83-88, 93-100 in Section 5.5
\item From the textbook, True-False Quiz (questions 1-20) on page 428; there are also questions that review
Chapter 5 on pages 429 to 431.
\end{enumerate}

\noindent\textbf{Appendix}.
\begin{table}[h]  % [h] for here, [ht] for here top, [hb] for here bottom
  \centering
  \includegraphics{../standalones/build/table_indefinite_integrals}
  \caption{A table of common indefinite integrals.}
  \label{table:indefinite_integrals}
\end{table}
\end{document}
