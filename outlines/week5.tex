\documentclass[../main]{subfiles} 

% \title{Differentiation Rules, Part 2}
% \topic{Differentiation Rules}
% \date{October 16 to 20, 2023}
% {Stewart (9E). Sections 3.4, 3.5, and 3.6. Pages 199 to 225.}

\begin{document}


%--------------------------------------------------
%
% instructor notes
%
%--------------------------------------------------
Notes.
  \begin{enumerate}
    \item Review the Vertical Line Test for functions, exponential and logarithmic functions, and inverse trigonometric functions as necessary,
    % \item Review Sections 1.4 and 1.5 for properties of exponential and logarithmic functions as necessary.
    % \item Review Section 1.5 for inverse functions as necessary.
    \item We use \(\arcsin(x), \arccos(x), \arctan(x)\) to denote the inverse sine, cosine and tangent functions, respectively. See the textbook pages 62 and 63 for references. 
    % \item Instructors may invite teaching assistants assigned to their section to help answer questions during group exercises. A teaching assistant has the right to decline such an invitation.
  \end{enumerate}




%--------------------------------------------------
%
% learning objectives
%
%--------------------------------------------------

  At the end of the lecture, students should be able to
  \begin{itemize}
    \item apply the chain rule to differentiate composite functions,
    \item apply the method of implicit differentiation,
    \item differentiate general exponential and logarithmic functions, and
    \item differentiate inverse trigonometric functions.
  \end{itemize}




%--------------------------------------------------
%
% General instructions
%
%--------------------------------------------------
% \begin{instruction}[Instruction]{How to ...}
%   \begin{enumerate}[label=(\alph*)]
%     \item TODO
%   \end{enumerate}
% \end{instruction}
%



%--------------------------------------------------
%
% activities
%
%--------------------------------------------------


\begin{outline}{sec}{The Chain Rule}{page} \label{outline:chain}
  \begin{enumerate}

    \item \textbf{Theorem}. If a function \(g\) is differentiable at \(x\) and a function \(f\) is differentiable at \(g(x)\), then the derivative of the composite function \(F = f \circ g\) defined by \(F(x) = f(g(x))\) is given by
      \begin{mdframed}[style=simple]
        \[
          F'(x) = f'(g(x)) \cdot g'(x).
        \]
      \end{mdframed}
        
      In Leibniz notation, if \(y = f(u)\) and \(u = g(x)\) are both differentiable functions, then
      \begin{mdframed}[style=simple]
        \[
          \frac{dy}{dx} = \frac{df}{du} \cdot \frac{du}{dx}.% = \left( \frac{d}{du} f(u) \right) \cdot \left( \frac{d}{dx} g(x) \right)
        \]
      \end{mdframed}
        

    \item Differentiate \((x+1)^{3}\) with respect to \(x\). Recall from last week, we differentiated \(x^{3} + 3 x^{2} + 3x + 1\) and these two expressions are the same polynomial. Discuss the benefit of using the chain rule in this case.
    \item Differentiate \(\cos(\sin(x^{-2}) + \frac{\pi}{2})\) with respect to \(x\).  
    \item Let \(g = g(x)\) be a function.  Use the chain rule to show \(\frac{d}{dx} \big( g(x) \big)^{2} = 2g \frac{dg}{dx}\).

    \item {The Chain Rule can be used to differentiate composite functions.}
  \end{enumerate}
\end{outline}

 

\begin{outline}{sec}{Implicit Differentiation}{page} \label{outline:implicit}
  \begin{enumerate}
    \item Review the equation of a circle \((x-a)^{2} + (y-b)^{2} = r^{2}\).
    \item Review the Vertical Line Test for functions. (See the textbook page 13.)  Review the difference between a function \(y = f(x)\) in which \(y\) is the dependent variable, and \(x\) is the independent variable and an \emph{equation} in which both \(x\) and \(y\) are variables. Review that the solution set to an equation in which both \(x\) and \(y\) are variables forms a curve in the \(xy\)-plane. See the leftmost plots in Figure~\ref{fig:circle}~and~\ref{fig:implicit_curves} for examples.

      Discuss that it is \emph{tricky and not wise} to solve for \(y\) as one or more functions of \(x\). Compare the equation \(x^{2} + y^{2} = 25\) in Figure~\ref{fig:circle} and the equation \(\sin(y^{2}) + \frac{x^{3}}{3} = 1\) in Figure~\ref{fig:implicit_curves}. As shown in Figure~\ref{fig:circle}, we can solve for \(y\) in \(x^{2} + y^{2} = 25\) using relatively simple algebra. However, if we apply a similar technique to solve for \(y\) in \(\sin(y^{2}) + \frac{x^{3}}{3} = 1\), then, as shown in Figure~\ref{fig:implicit_curves}, we \emph{will be mistaken} that \(y = \sqrt{\arcsin(1 - \frac{x^{3}}{3})}\) and \(y = -\sqrt{\arcsin(1 - \frac{x^{3}}{3})}\) are the only solutions.

      \begin{figure}[h]
        \centering
        \centerline{
          \includegraphics{../standalones/build/plot_circle}
          \includegraphics{../standalones/build/plot_circle_upper}
          \includegraphics{../standalones/build/plot_circle_lower}
        }
        \caption{The equation \(x^{2} + y^{2} = 25\) implicitly defines two functions.}
        \label{fig:circle}
      \end{figure}


      \begin{figure}[h]
        \centering
        \centerline{
          \includegraphics{../standalones/build/plot_fun}
          \includegraphics{../standalones/build/plot_fun_tricky_pos}
          \includegraphics{../standalones/build/plot_fun_tricky_neg}
        }
        \caption{It is not a good idea and is tricky to decompose the equation \(\sin(y^{2}) + \frac{x^{3}}{3} = 1\) into functions. }
        \label{fig:implicit_curves}
      \end{figure}

    \item Suppose we have an equation in which both \(x\) and \(y\) are variables. The points \((x,y)\) satisfying the equation form a curve in the \(xy\)-plane. If \((x,y)\) is a point \emph{on this curve} whose tangent line is well-defined, then \(\frac{dy}{dx}\) is the slope of this tangent line.

      \begin{mdframed}[style=simple]
        The \emph{method of implicit differentiation} says to find the slope \(\frac{dy}{dx}\) at a point \((x,y)\) on the curve defined by an equation in which both \(x\) and \(y\) are variables, differentiate both sides of the equation \emph{with respect to \(x\)}, using the chain rule to differentiate expressions in \(y\), then finally \emph{solve} for \(\frac{dy}{dx}\).

        \medskip
        For example, to find the slope \(\frac{dy}{dx}\) at a point \((x,y)\) on the curve defined by \(x^{2} + y^{2} = 25\), we differentiate both sides of the equation with respect to \(x\) to get \(2x + 2y \frac{dy}{dx} = 0\). After solving for \(\frac{dy}{dx}\), we get \(\frac{dy}{dx} = -\frac{x}{y}\).
      \end{mdframed}
        
      Emphasize that \(\frac{dy}{dx}\) should be treated as a single symbol, not a fraction.

    \item Find the slope at the point \((3,4)\) given \(x^{2} + y^{2} = 25\).  Discuss that the slope at the point \((5,0)\) is not well-defined.  Discuss that we cannot plug in \(x = y = \frac{\sqrt{2}}{2}\) to find the slope of the tangent line at the point \((\frac{\sqrt{2}}{2}, \frac{\sqrt{2}}{2})\) because this point is not on the curve \(x^{2} + y^{2} = 25\).
    \item Find \(x'\) given \(\cos(t) = \sin(x)\). Assume \(x\) is a function of \(t\).  
      
    \item {The method of implicit differentiation is an application of the chain rule.  It can be used to find the slope \(\frac{dy}{dx}\), if well-defined, for a point \((x,y)\) on the curve of \(f(x,y) = 0\).}
\end{enumerate}
\end{outline}

 

\begin{outline}{sec}{Logarithmic Functions}{page} \label{outline:logs}
  \begin{enumerate}
    \item For a number \(b > 0\) and \(b \ne 1\), the \emph{logarithmic function with base \(b\)}, denoted by \(\log_{b}(x)\), is the inverse of the exponential function \(b^{x}\). The two functions are related by
      \[
        \log_{b}(x) = y \quad \Longleftrightarrow \quad b^{y} = x.
      \]

      If time permits, let students do the following folding activity to understand the idea of \(\log_{b}(x)\) in base \(b = 2\).  It is recommended to do it in groups of two.
      \begin{enumerate}
        \item Repeatedly fold a triangular origami paper by identifying two vertices as many times as you wish, but do not let your partner see the number of folds you made. Call the number of folds \emph{your} \(k\). Your partner will have their own number \(k\). These two do not have to be the same number. See Figure~\ref{fig:origami} for an example of the first three folds.
          \begin{figure}[h]
            \centering
            \includegraphics{../standalones/build/demo_origami}
            \caption{Fold across the dotted line.}
            \label{fig:origami}
          \end{figure}
          
        \item Unfold completely and count the number of triangles you have. \hlmain{You are computing \(2^{k}\) because each fold doubles the number of triangles.}
        \item Exchange the origami paper with your partner and count the number of triangles \(n\) on your partner's origami paper. 
        \item How many folds did your partner make? You can retrace the folds and keep track of the number of folds you make. \hlmain{You are computing \(\log_{2}(n)\) where \(n\) is the number of triangles on your partner's origami paper.}
      \end{enumerate} 
      {The logarithm \(\log_{b}(x)\) ``undoes'' the exponentiation \(b^{x}\) in the same base and vice versa.  These are expressed mathematically as identities
      \[
        x = \log_{b}(b^{x}) \text{ for any number } x, \quad\text{and}\quad x = b^{\log_{b}(x)} \text{ if } x > 0.
      \]
    }
      
    \item The natural logarithm, denoted by \(\ln(x)\), is the logarithmic function with base \(e\). Review the base change formula \(\log_{b}(x) = \frac{\ln(x)}{\ln(b)}\). See page 60 of the textbook and Figure~\ref{fig:exp_log}.
      \begin{figure}[h]
        \centering
        \includegraphics{../standalones/build/plot_exp_log}
        \caption{Functions \(e^{x}\) and \(\ln(x)\) are inverse to each other.}
        \label{fig:exp_log}
      \end{figure}

    \item Review Laws of Exponents and Laws of Logarithms as necessary. See pages 47 and 58 of the textbook, respectively.        
    \item \textbf{Theorem}. If \(b > 0\) and \(b \ne 1\), then
      \begin{mdframed}[style=simple]
        \[
          \frac{d}{dx} (\log_{b}(x)) = \frac{1}{x \ln(b)}.
        \]
      \end{mdframed}
        
    \item \label{part:ln-trick} By taking logarithms first, differentiate \(y = \dfrac{2x^{3}\big(\tan(x)\big)^{9}}{x^{2} + e^{x}}\). Use the following steps as your guide.
      \begin{enumerate}
        \item Take the natural logarithm \(\ln\) on both sides then rewrite the equation as
          \[
            \ln(y) = \ln(\;\text{numerator}\;) - \ln(\;\text{denominator}\;).
          \]
          Use the Laws of Logarithm to simplify your expression as much as possible.
        \item Implicitly differentiate with respect to \(x\).
        \item Solve for \(\frac{dy}{dx}\) to rewrite your expression as \(\frac{dy}{dx} = \cdots{}\) where \(\frac{dy}{dx}\) does not appear on the right-hand side.
        \item Replace every occurrence of \(y\) on the right-hand side by the original expression \(y = \dfrac{2x^{3}\tan^{9}(x)}{x^{2} + e^{x}}\).
      \end{enumerate}
      Compare this method with the quotient rule. 

    \item {Using implicit differentiation, we find the derivative of a logarithmic function \(\log_{b}(x)\) where \(b > 0\) and \(b \ne 1\).}
    \item {As an application of logarithmic derivatives, we can often quickly evaluate complicated functions that are products, quotients, and powers of simpler functions.}
\end{enumerate}
\end{outline}



\begin{outline}{sec}{Inverse trigonometric functions}{page} \label{act:inverse_trigs}
  \begin{enumerate}
    \item Review inverse trigonometric functions as necessary. See Figure~\ref{fig:arctan} for the graph of \(\arctan(x)\). Also see Figure~\ref{fig:arcsin-arccos} for \(\arcsin(x)\) and \(\arccos(x)\), and pages 62 and 63 of the textbook for reference.
      % \begin{align*}
      %   \arcsin(x) = y &\quad\Longleftrightarrow\quad \sin(y) = x, \quad\text{ for }&  -\pi/2 &\le y \le \pi/2, \\
      %   \arccos(x) = y &\quad\Longleftrightarrow\quad \cos(y) = x, \quad\text{ for }&  0 &\le y \le \pi, \\
      %   \arctan(x) = y &\quad\Longleftrightarrow\quad \tan(y) = x,\quad\text{ for }&  -\pi/2 &< y < \pi/2.
      % \end{align*}
      
      \begin{figure}[h]
        \centering
        \begin{minipage}{.45\textwidth}
          \includegraphics{../standalones/build/plot_tan_as_arctan_preimage}
        \end{minipage}
        \quad
        \begin{minipage}{.45\textwidth}
          \includegraphics{../standalones/build/plot_arctan}
        \end{minipage}

        \caption{Inverse trigonometric function \(\arctan(x)\) and its relation to \(\tan(x)\).}
        \label{fig:arctan}
      \end{figure}
    \item Review trigonometric identities
      \[
        \sin^{2}(x) + \cos^{2}(x) = 1 \quad\text{and}\quad \tan^{2}(x) + 1 = \sec^{2}(x).
      \]
      
    \item Differentiate \(\tan(y) = x\) implicitly with respect to \(x\) to find the derivative of \(\arctan(x)\).
    \item {Combining implicit differentiation and trigonometric identities, we can differentiate inverse trigonometric functions.}
\end{enumerate}
\end{outline}



\begin{outline}{sec}{In-class Group Exercises}{page}\label{outline:worksheet}
  Depending on time constraints, an instructor may choose some or all of the listed exercises for in-class group work. Give students a few minutes to work on the following exercises in small groups or independently. For each of the chosen exercises, optionally ask a student volunteer to present their solution and discuss the solution if time permits. 
  \begin{enumerate}
    \item Differentiate \(e^{ax}\) where \(a\) is a constant. Let \(b > 0\) be a number. Use the relation \(b^{x} = e^{\ln(b)x}\) to conclude that \((b^{x})' = \ln(b) b^{x}\).
    \item Use implicit differentiation to find \(\frac{dy}{dx}\) given \(x^{2} + 2xy + y^{3} = 0\).  Find the equation for the tangent line at \((-1,1)\) on the curve.
    \item Find \(y'\) given \(2^{y} = x\).
    \item Differentiate \(y = \dfrac{x^{3/4}\sqrt{x^{2} + 1}}{(3x + 5)^{5}}\).  Use the example in Part~\ref{part:ln-trick} of Outline~\ref{outline:logs} as your guide.
  \end{enumerate}
\end{outline}



%--------------------------------------------------
%
% additional problems
%
%--------------------------------------------------
Exercises.
\begin{enumerate}
\item Find the slope of the tangent line at the point \((1,\sqrt{\pi})\) given \(\sin(y^{2}) + x^{2} = 1\) in two different ways. First, use implicit differentiation.  Then try again by solving for \(y\) then differentiate the function you obtained.
\item Differentiate \(x^{\sin(\theta)}\) with respect to \(\theta\).
\item Differentiate \(x^{x}\) with respect to \(x\).
\item Differentiate \(\sin(y) = x\) with respect to \(x\) to find the derivative of \(\sin(x)\).
\item Differentiate \(\tan(y) = x\) implicitly with respect to \(x\) to find the derivative of \(\arctan(x)\). % Hint: \(\frac{d}{dx} \tan(x) = \sec^{2}(x)\) and \(\sec^{2}(y) = 1 + \tan^{2}(y)\).
\item From the textbook, exercises 1-60, 65-86, 95, 101 in Section 3.4
\item From the textbook, exercises 1- 28, 39-44, 53-54, 61 in Section 3.5
\item From the textbook, exercises 2-26, 29-32, 37-39, 44-61, 63-66, 83  in Section 3.6
\end{enumerate}

\begin{figure}[h]
        \centering
        \begin{minipage}{.45\textwidth}
          \includegraphics{../standalones/build/plot_sin_as_arcsin_preimage}
        \end{minipage}
        \quad
        \begin{minipage}{.45\textwidth}
          \includegraphics{../standalones/build/plot_arcsin}
        \end{minipage}

        \vspace{2em}
        \begin{minipage}{.45\textwidth}
          \includegraphics{../standalones/build/plot_cos_as_arccos_preimage}
        \end{minipage}
        \quad
        \begin{minipage}{.45\textwidth}
          \includegraphics{../standalones/build/plot_arccos}
        \end{minipage}

        % \vspace{2em}
        % \begin{minipage}{.45\textwidth}
        %   \includegraphics{../standalones/build/plot_tan_as_arctan_preimage}
        % \end{minipage}
        % \quad
        % \begin{minipage}{.45\textwidth}
        %   \includegraphics{../standalones/build/plot_arctan}
        % \end{minipage}

        \caption{Inverse trigonometric functions \(\arcsin(x), \arccos(x)\) and their relation to \(\sin(x), \cos(x)\) respectively.}
        \label{fig:arcsin-arccos}
      \end{figure}
\end{document}

