\documentclass[../main]{subfiles} 

% \title{Differentiation Rules 3}
% \topic{Differentiation Rules 3}
% \date{October 23 to 27.}

% {Stewart (9E). Sections 3.8, 3.9, 3.10. Pages 239 to 261.}

\begin{document}


%--------------------------------------------------
%
% instructor notes
%
%--------------------------------------------------
Notes.
  \begin{enumerate}
    \item The topic ``differentials'' in Section~3.10 is skipped.
  \end{enumerate}




%--------------------------------------------------
%
% learning objectives
%
%--------------------------------------------------

  At the end of the lecture, students should be able to
  \begin{itemize}
    \item linearize a function \(f(x)\) at a number \(x = a\), 
    \item solve \(\frac{dy}{dt} = ky\) where \(y\) is a function of \(t\) and \(k\) is a constant,
    \item use the function \(y(t) = C e^{kt}\) to model exponential growth and decay,
    \item use implicit differentiation to solve related rates problems, and
    \item use calculus to solve real-world problems.
  \end{itemize}




%--------------------------------------------------
%
% General instructions
%
%--------------------------------------------------
% \begin{instruction}[Instruction]{How to ...}
%   \begin{enumerate}[label=(\alph*)]
%   \end{enumerate}
% \end{instruction}
%



%--------------------------------------------------
%
% activities
%
%--------------------------------------------------
\begin{outline}{sec}{Linear Approximation}{page} \label{outline:linear_approx}
  \begin{enumerate}
    \item A function \(f(x)\) can be approximated by a linear function near a point \((a, f(a))\) using the equation
      \begin{equation} \label{eq:linearization}
        L(x) = f(a) + f'(a) (x-a).
      \end{equation}
      The function \(L(x)\) is the tangent line of \(f(x)\) at \((a,f(a))\) and is called the \emph{linearization} of \(f(x)\) at \(a\).
      \begin{figure}[h]
        \centering
        \includegraphics{../standalones/build/plot_linearization}
        \caption{Linear approximation of \(\sqrt{x}\) at \(a = 1\).}
        \label{fig:linearization-of-sqrt}
      \end{figure}
      
    % \item \todo{The notation \(f(x) \approx L(x)\) does not include the point \(a\). But Stewart uses it everywhere. Any suggestions to get around this?}
    % \item Linearize the function \(f(x) = x^{2}\) at \(a = 2\) and at \(a = 3\). Discuss different choices for the number \(a\) Equation~\eqref{eq:linearization} to approximate \(f(2.9)\).
    \item Approximate \(\sqrt{2}\) by linearizing the square root function \(\sqrt{x}\) at \(a = 1\) and \(a = 4\).  Compare the two approximations.  Discuss a graphical explanation for the error of approximation.
    \item Approximate \(\sin(3)\) and \(\sin(10)\) by linearizing \(\sin(x)\).  
    \item {A function \(f(x)\) can be approximated by the linear function \(L(x)\) at a number \(a\).}
    % \item {It is often desirable in real-world applications to control the error arising from linearization by restricting the domain of \(L(x)\). }
  \end{enumerate}
\end{outline}



\begin{outline}{sec}{The Exponential Growth and Decay}{page} \label{outline:exp-model}
  \begin{enumerate}
    \item Let \(y(t)\) be a mathematical model of some quantity \(y\) at time \(t\). If the \emph{rate of change} of \(y\) with respect to \(t\) is \emph{proportional} to \(y\) at any time, then we have an equation
      \begin{equation} \label{eq:exp-diff-eq} 
        \frac{dy}{dt} = ky, \quad\text{where } k \text{ is a constant.}
      \end{equation} 

      \begin{itemize}
        \item Discuss that \(y(t) = t^{2}\) is not a solution of Equation~\eqref{eq:exp-diff-eq} despite \(y'(1) = 2 y(1)\).
        \item Emphasize a solution to Equation~\eqref{eq:exp-diff-eq} must satisfy \(y'(t) = ky(t)\) \emph{for every} \(t\) in the domain of \(y(t)\).
      \end{itemize}

    \item Check that \(y = C e^{kt}\), where \(C\) is some constant, is a solution to Equation~\eqref{eq:exp-diff-eq} and show that \(C = y(0)\).
      \begin{figure}[ht]
        \centering
        \includegraphics{../standalones/build/plot_exp_growth}
        \quad
        \includegraphics{../standalones/build/plot_exp_decay}
        \label{fig:exp-models}
        \caption{The general graphs of exponential models.}
      \end{figure}

    \item \textbf{Theorem}. Given a constant \(k\). The only solution to \(\frac{dy}{dt} = ky\) is \(y(t) = y(0) e^{kt}\). In the context of Equation~\eqref{eq:exp-diff-eq}, the number \(y(0)\) is called \emph{the initial value} of the function \(y(t)\).  See Figure~\ref{fig:exp-models} for their graphs.
      \begin{itemize}
        \item If we were to find \(y(t)\), then we must determine the numbers \(k\) and \(y(0)\).
      \end{itemize}
      
    % \item When \(k > 0\), the number \(k\) is called \emph{the relative growth rate}.  
    % \item When \(k < 0\), its absolute value \(|k|\) is called \emph{the half-life}.  
    \item Discuss similar examples to Example~1 on page 240 and Example~2 on page 241 of the textbook.
    \item {A function \(y(t) = y(0) e^{kt}\) is a suitable model for real-life application when the rate of change \(y\) is proportional to \(y\).}
    \item {To find a solution to Equation~\eqref{eq:exp-diff-eq}, we must determine the number \(k\) and an initial value \(y(0)\) for the function \(y(t) = y(0) e^{kt}\). }
\end{enumerate}
\end{outline}



\begin{outline}{sec}{Related Rates}{page} \label{outline:related-rates}
  \begin{enumerate}
    \item If two quantities \(x\) and \(y\) are related by an equation, then their rates of change with respect to a variable \(t\), denoted as \(\frac{dx}{dt}\) and \(\frac{dy}{dt}\), are also related. We call such a relation of \(\frac{dx}{dt}\) and \(\frac{dy}{dt}\) \emph{related rates}. Related rates can be computed by implicit differentiation of the equation involving both \(x\) and \(y\) with respect to \(t\). It must be emphasised that both \(x\) and \(y\) are treated as functions of \(t\).
    \item We work through Example~5 on page~250 of the textbook as follows. We set up a scenario where a related rates calculation can be verified by a real-life experiment. 
      \begin{figure}[ht]
        \centering
        \includegraphics{../standalones/build/demo_related_rates}
        \label{fig:related-rates}
        \caption{The related rates experiment setup.}
      \end{figure}

      \begin{enumerate}
        \item Mark out two points \(A\) and \(B\) in front of the black board. Mark out another point \(O\) (for the observer) some distance away from the blackboard. See Figure~\ref{fig:related-rates}.  Let \(d\) be the distance between \(A\) and \(O\).
        \item Ask one student to volunteer as the walker and one as the observer. The walker should be positioned at \(A\) and the observer at \(O\).
        \item The observer should remain stationary throughout this experiment and always point their phone at the walker. Use the Phyphox app to record its rotational speed (in radians). If available, the observer can point a flashlight at the walk to keep measurements accurate. 
        \item The walker should move at a constant speed \(v = 1\) metre per second\footnote{\(1\) metre per second is just below the natural walking speed of an average adult.}. Walking to the beat of 60 beats per minute using \href{https://www.metronomeonline.com/}{an online metronome} or having the class clap together is sufficient.
        \item Let \(w\) be the distance between the walker and their starting point \(A\).  On the diagram, the number \(w\) is the length of the solid line. Let \(\theta\) be the angle measured as shown on the diagram.  The quantities \(\theta\) and \(t\) is related by
          \[
            \tan(\theta) = \frac{w}{d}.
          \]
        \item Implicitly differentiate \(\tan(\theta) = \frac{w}{d}\) with respect to \(t\) to find
          \begin{equation} \label{eq:related-rates-demo}
            \frac{d\theta}{dt} = \cdots.
          \end{equation}
          Remember \(d\) is a constant, and \(\theta\) and \(w\) are functions of \(t\).
        \item Because we assumed that \(v = \frac{dw}{dt}\) is constant, we can replace \(\frac{dw}{dt}\) by the constant \(v\) in Equation~\eqref{eq:related-rates-demo} to find the rotational rate \(\frac{d\theta}{dt}\) as a function of \(t\).
        \item Using Equation~\eqref{eq:related-rates-demo}, predict that the rotational rate is faster when the walker is close to \(B\) compared when they are close to \(A\).  Compare this prediction against the measured rotational rate from the observer.
      \end{enumerate}

    \item Discuss another example of related rates. 
    % \item Discuss similar examples to Example~1 and Example~3 in the textbook on page 247 and 249 respectively.
    % \item Work through Example~3 in the textbook on page 249.
    % \item Work through Example~3 in the textbook on page 249.
    \item {Implicit differentiation can be used to solve related rates problems.}
\end{enumerate}
\end{outline}



%--------------------------------------------------
%
% additional problems
%
%--------------------------------------------------
Exercises.
\begin{enumerate}
    \item From the textbook, exercises 1-4, 7-14,  in Section 3.8 % No newton's cooling, and no compound interest
    \item From the textbook, exercises 1-17, 22, 25-26, 32-33, 37, 39, 46 in Section 3.9
    \item From the textbook, exercises  1-4. 31, 40a, 50a,b, 51-52 in Section 3.10 % No differentials
    \item From the textbook, True-False Quiz (questions 1-15) on page 269; there are also questions that review
Chapter 3 on pages 270 to 273.
\end{enumerate}

\end{document}