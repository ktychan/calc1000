%! Tex Program = lualatex
\documentclass[12pt]{beamer} 
\usetheme{metropolis} 
\everymath{\displaystyle} 
\setbeamersize{text margin left=.5cm} 
\setbeamersize{text margin right=.5cm} 
\beamertemplatenavigationsymbolsempty{}

\usepackage{fontawesome5}
\usepackage{graphicx}
\input{../colours.tex.preamble}
\input{../tikz.tex.preamble}

\begin{document} 
\begin{frame}[t]
  The obvious routes \emph{do not} minimize travel time. Why?
  \begin{center}
    \begin{tikzpicture}[scale=1.75]
      \node at (0,1) {\(\bullet\)};
      \node[left] at (0,1) {\footnotesize start};
      \node at (2,0) {\(\bullet\)};
      \node[right] at (2,0) {\footnotesize end};
      \draw[very thick] (0,0) -- (2,0);
      \onslide<1>{
        \draw[very thick] (0,1) -- (2,1);
      }
      \draw[decorate, decoration={brace, raise=5pt}] (2,0) -- (0,0)
        node[midway, below=5pt] {\footnotesize \(2\) km};
      \draw[decorate, decoration={brace, raise=5pt, mirror}] (2,0) -- (2,1)
        node[midway, right=5pt] {\footnotesize \(1\) km};
      \onslide<1>{
        \draw[Aquamarine, domain=0.2:1.8, smooth, samples=100] plot ({\x},{sin(4*pi*\x)*0.05+0.75});
        \draw[Aquamarine, domain=0.2:1.8, smooth, samples=100] plot ({\x},{sin(4*pi*\x)*0.051+0.50});
        \draw[Aquamarine, domain=0.2:1.8, smooth, samples=100] plot ({\x},{sin(4*pi*\x)*0.051+0.25});
      }
      \onslide<2>{
        \draw[main, dashed, very thick] (0,1) -- (2,0) node[midway, above right] {\footnotesize swim};
      }
      \onslide<3>{
        \draw[main, dashed, very thick] (0,1) -- (2,1) node[midway, above] {\footnotesize walk};
        \draw[main, dashed, very thick] (2,1) -- (2,0) node[midway, left] {\footnotesize swim};
      }
      \onslide<4>{
        \draw[main, dashed, very thick] (0,1) 
          -- (1,1) 
          node[midway, above] {\footnotesize walk \(1\) km}
          -- (2,0) 
          node[midway, left] {\footnotesize swim};
      }
    \end{tikzpicture}
  \end{center}

  \onslide<2->{\footnotesize
    If one swims only, then it takes
    \(
    \frac{\sqrt{2^{2} + 1^{2}} \text{ km}}{3 \text{ kph}} \approx 0.745 \text{ hour}.
    \)
  }

  \onslide<3->{\footnotesize 
    If one walks the whole length and swims across, then it takes 
    \[
      \frac{2 \text{ km}}{6 \text{ kph}} + \frac{1 \text{ km}}{3 \text{ kph}} \approx 0.666 \text{ hour}.
    \]
  }

  \onslide<4>{\footnotesize
    However, if one walks 1 km and swims across, then it takes even less time
    \[
      \frac{1 \text{ km}}{6 \text{ kph}} + \frac{\sqrt{2}}{3 \text{ kph}} \approx 0.638 \text{ hour}.
    \]
  }
\end{frame}

\begin{frame}[t]
  Some combination of walk+swim minimizes travel time. 

  \begin{center}
    \begin{tikzpicture}[scale=1.75]
      \node at (0,1) {\(\bullet\)};
      \node[left] at (0,1) {\footnotesize start};
      \node at (2,0) {\(\bullet\)};
      \node[right] at (2,0) {\footnotesize end};
      \draw[very thick] (0,0) -- (2,0);
      % \draw[very thick] (0,1) -- (2,1);
      \draw[decorate, decoration={brace, raise=5pt}] (2,0) -- (0,0)
        node[midway, below=5pt] {\footnotesize \(2\) km};
      \draw[decorate, decoration={brace, raise=5pt, mirror}] (2,0) -- (2,1)
        node[midway, right=5pt] {\footnotesize \(1\) km};
      \draw[main, dashed, very thick] (0,1) 
        -- (0.75,1) 
        node[midway, above] {\footnotesize \(w\)} 
        -- (2,0) 
        node[midway, left] {\footnotesize \(s\)};
    \end{tikzpicture}
  \end{center}

  Let \(w\) denote the \emph{walking distance} and \(s\) the swimming distance as shown. 

  Let \(T\) denote travel time. \hlattn{Goal: We want to minimize \(T\).}
\end{frame}

\begin{frame}[t]
  \hlattn{Goal: We want to minimize \(T\).}

  \begin{center}
    \begin{tikzpicture}[scale=1.75]
      \node at (0,1) {\(\bullet\)};
      \node[left] at (0,1) {\footnotesize start};
      \node at (2,0) {\(\bullet\)};
      \node[right] at (2,0) {\footnotesize end};
      \draw[very thick] (0,0) -- (2,0);
      % \draw[very thick] (0,1) -- (2,1);
      \draw[decorate, decoration={brace, raise=5pt}] (2,0) -- (0,0)
        node[midway, below=5pt] {\footnotesize \(2\) km};
      \draw[decorate, decoration={brace, raise=5pt, mirror}] (2,0) -- (2,1)
        node[midway, right=5pt] {\footnotesize \(1\) km};
      \onslide<3->{
        \fill[supp!25] (0.75,1) -- (0.75,0) -- (2,0) -- (0.75,1);
      }

      \draw[main, dashed, very thick] (0,1) 
        -- (0.75,1) 
        node[midway, above] {\footnotesize \(w\)} 
        -- (2,0) 
        node[midway, left] {\footnotesize \(s\)};
    \end{tikzpicture}
  \end{center}

  We have relations
  \[
    \onslide<2->{T = \frac{w}{6} + \frac{s}{3}} 
    \quad\text{and}\quad
    \onslide<3->{w + {\color{supp}\sqrt{s^{2} - 1^{2}}} = 2}.
  \]

  \onslide<4->{
    Eliminate one independent variable. We get
    \[
      T = \frac{\sqrt{s^{2} - 1} - 2}{6} + \frac{s}{3}.
    \]

    We can now use tools from previous sections to minimize \(T\).
  }
\end{frame}

\begin{frame}[t]
  \hlattn{Goal: We want to minimize \(T\).}

  \begin{center}
    \begin{tikzpicture}[scale=1.75]
      \node at (0,1) {\(\bullet\)};
      \node[left] at (0,1) {\footnotesize start};
      \node at (2,0) {\(\bullet\)};
      \node[right] at (2,0) {\footnotesize end};
      \draw[very thick] (0,0) -- (2,0);
      % \draw[very thick] (0,1) -- (2,1);
      \draw[decorate, decoration={brace, raise=5pt}] (2,0) -- (0,0)
        node[midway, below=5pt] {\footnotesize \(2\) km};
      \draw[decorate, decoration={brace, raise=5pt, mirror}] (2,0) -- (2,1)
        node[midway, right=5pt] {\footnotesize \(1\) km};
      \fill[supp!25] (0.75,1) -- (0.75,0) -- (2,0) -- (0.75,1);

      \draw[main, dashed, very thick] (0,1) 
        -- (0.75,1) 
        node[midway, above] {\footnotesize \(w\)} 
        -- (2,0) 
        node[midway, left] {\footnotesize \(s\)};
    \end{tikzpicture}
  \end{center}

  We have relations
  \[
    T = \frac{w}{6} + \frac{s}{3}
    \quad\text{and}\quad
    w + {\color{supp}\sqrt{s^{2} - 1^{2}}} = 2.
  \]

  Eliminate one independent variable. \hlwarn{There was an algebra mistake}. We get
  \[
    T = \frac{\color{warn} 2 - \sqrt{s^{2} - 1}}{6} + \frac{s}{3}.
  \]

  We can now use tools from previous sections to minimize \(T\).
\end{frame}
\begin{frame}[t]
  \onslide<2->{
    If you choose to eliminate \(s\), then you should get
    \[
      T(w) = \frac{w}{6} + \frac{\sqrt{(2-w)^{2} + 1}}{3},
    \]
    because \(s = \sqrt{(2-w)^{2}+1}\).
  }
  
  \bigskip
  
  We can now use tools from previous sections to minimize \(T\) using \onslide<2->{either}
  \[
    T(s) = \frac{2 - \sqrt{s^{2} - 1}}{6} + \frac{s}{3}
  \]
  \onslide<2->{or
    \[
      T(w) = \frac{w}{6} + \frac{\sqrt{(2-w)^{2} + 1}}{3}.
    \]
  }
\end{frame}
\end{document}


