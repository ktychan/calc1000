%! Tex Program = lualatex
\documentclass[12pt, handout]{beamer} 
\usetheme{metropolis} 
\everymath{\displaystyle} 
\setbeamersize{text margin left=.5cm} 
\setbeamersize{text margin right=.5cm} 
\beamertemplatenavigationsymbolsempty{}

\usepackage{graphicx}
\input{../colours.tex.preamble}
\usepackage{fontawesome5}

\begin{document} 
\begin{frame}[t]
  \includegraphics[width=\textwidth]{../standalones/build/extrema_example_blank.pdf}

  Let's scan the graph from left to right to find critical numbers.
\end{frame}

\begin{frame}[t]
  \includegraphics[width=\textwidth]{../standalones/build/extrema_example_1_1.pdf}

  Recall \(f'(c) = 0\) means the graph of \(y = f(x)\) has a horizontal tangent line at \(c\).
\end{frame}

\begin{frame}[t]
  \includegraphics[width=\textwidth]{../standalones/build/extrema_example_1.pdf}
  
  \pause
  The constant \hlmain{\(c\) is a critical number}, but the point \hlsupp{\((c, f(c))\)} is a critical point.

  Also notice that \(f(c)\) is a \hlsupp{local maximum}.
\end{frame}

\begin{frame}[t]
  \includegraphics[width=\textwidth]{../standalones/build/extrema_example_2_1.pdf}

  Similar to the previous critical number, we have a horizontal tangent line here.
\end{frame}

\begin{frame}[t]
  \includegraphics[width=\textwidth]{../standalones/build/extrema_example_2.pdf}
  
  Similar to the previous critical number, we have a horizontal tangent line here.
  \pause

  Unlike the previous critical number, \(f(c)\) is a \hlsupp{local minimum} at this particular critical number.
\end{frame}


\begin{frame}[t]
  \includegraphics[width=\textwidth]{../standalones/build/extrema_example_3_1.pdf}

  Such situation often, but not always, occurs in piecewise functions.
\end{frame}

\begin{frame}[t]
  \includegraphics[width=\textwidth]{../standalones/build/extrema_example_3.pdf}
  
  Such situation often, but not always, occurs in piecewise functions.

  \pause 
  Notice that \(f(c)\) is a \hlsupp{local maximum}.
\end{frame}



\begin{frame}[t]
  \includegraphics[width=\textwidth]{../standalones/build/extrema_example_4_1.pdf}

  Notice \(f'(c)\) does not exist because \(f\) is \hlwarn{discontinuous} at \(c\). 
\end{frame}

\begin{frame}[t]
  \includegraphics[width=\textwidth]{../standalones/build/extrema_example_4.pdf}
  
  Notice \(f'(c)\) does not exist because \(f\) is \hlwarn{discontinuous} at \(c\). 
  \pause 

  Notice that \(f(c)\) is NEITHER a \hlsupp{local minimum} NOR a local maximum.
\end{frame}


\begin{frame}[t]
  \includegraphics[width=\textwidth]{../standalones/build/extrema_example_5_1.pdf}

  Notice \(f'(c)\) does not exist.
\end{frame}

\begin{frame}[t]
  \includegraphics[width=\textwidth]{../standalones/build/extrema_example_5.pdf}
  
  \pause 
  Notice that \(f(c)\) is a NOT \hlsupp{local minimum} at this particular critical number.
\end{frame}


\begin{frame}[t]
  \includegraphics[width=\textwidth]{../standalones/build/extrema_example_6_1.pdf}

  Notice \(f'(c)\) does not exist because \(c\) is the right \hlwarn{endpoint} of the domain of \(f\).
\end{frame}

\begin{frame}[t]
  \includegraphics[width=\textwidth]{../standalones/build/extrema_example_6.pdf}
  
  Notice \(f'(c)\) does not exist because \(c\) is the right \hlwarn{endpoint} of the domain of \(f\).

  \pause        
  \hlwarn{\faExclamationTriangle{}} We \hlwarn{must exclude endpoints} of the given function when searching for critical numbers.
\end{frame}



\begin{frame}[t]
  \includegraphics[width=\textwidth]{../standalones/build/extrema_example_7_1.pdf}
  
  \faComment{} Have we found \emph{all} critical points?
\end{frame}

\begin{frame}[t]
  \includegraphics[width=\textwidth]{../standalones/build/extrema_example_7.pdf}
  
  Nope. Don't forget the flat part. 

  Every point on this flat part is a critical point because all excpet the endpoints have horizontal tangent lines.
\end{frame}


\begin{frame}[t]
  \includegraphics[width=\textwidth]{../standalones/build/extrema_example.pdf}

  Lastly, \(f(x)\) has an absolute maximum but no absolute minimum because \(\lim_{x \to -\infty} f(x) = -\infty\) assuming the trend continues.
\end{frame}
\end{document}


