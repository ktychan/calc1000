%! TeX Program = lualatex
\documentclass[../main.tex]{subfiles}
\begin{document}
\pagestyle{empty}
\section*{Suggested exercises progressions}

The following is a suggested progression of skill development in \thecoursesubject{}~\thecoursenumb{}.  The suggested progression takes you through the list of Recommended Practice Problems, from basic concept checks to routine calculations to more advanced problem-solving.  Remember that this is an \emph{opinionated} suggestion only, not a guarantee. Feel free to find whatever study path that works the best for you.

Skill development varies among students and depends on many factors. A minimum progression of skill development involves, on average, eight to ten exercises per textbook section, typically less than half of all recommended exercises. Do more than the minimum if you find specific a group of exercises challenging. Lastly, we promote studying in groups.

Recommend practice exercises are divided into three categories: 
\begin{itemize}
  \item (Level 1) Basic concept checks or fundamental technical skills.
  \item (Level 2) Routine calculations or reasoning.
  \item (Level 3) Problem solving.
\end{itemize}

You need to catch up immediately if you find level 1 exercises challenging.
\begin{itemize}
  \item Do level 1 exercises \hlwarn{within two days of} the corresponding lecture, ideally the day of. Catch up, visit your instructor's office hours (see BrightSpace) or the Math Help Centre (PAB 48/49, the Math-Physics Accelerator) \hlwarn{as soon as possible}.
\end{itemize}

Level 2 exercises check if you can apply basic definitions, theorems and fundamental problem-solving techniques to solve problems \hlmain{similar to those discussed in lectures}. They can feel a bit challenging at first. However, routine calculations and reasoning get easier with practice as you gain confidence with unfamiliar functions, weed out misconceptions and develop intuition. 
\begin{itemize}
  \item Start working on level 2 exercises \hlmain{as soon as} you can easily and correctly complete the minimum suggested level 1 exercises.
  \item Space out your study sessions. A few days of practice is typically required to develop proficiency.
  \item Each group of exercises (the ones that say \emph{do at least one of}) usually target just \hlmain{one} basic theorem or fundamental problem-solving skill. Focus on applying the principles you learned in lectures. 
  \item For each group of exercises, start with one that looks approachable to gain some confidence, then do the ones that look unfamiliar until you can articulate the \hlmain{common idea} behind them. Such ideas are often taught \emph{explicitly} during lectures.
\end{itemize}

While level 2 exercises help you develop individual skills \emph{in isolation}, level 3 exercises challenge your all-round problem-solving skills. These exercises could involve more challenging algebra work, ask familiar questions in unfamiliar forms, involve more complicated functions, require some trial-and-error before a clear idea emerges, and so on.
\begin{itemize}
  \item Start working on level 3 exercises when you feel confident about routine calculations. 
  \item Discuss problem-solving ideas with your study group. 
  \item As the term goes on, some level 3 exercises from earlier topics become routine calculations for later topics.
\end{itemize}
\clearpage

\subsection*{Textbook Section 2.2} 
\subsubsection{Two-sided and one-sided limits and their relations}
\begin{enumerate}
  \item (basic concept checks) Do at least one group of (46 to 49), (50 to 52), (59 to 64). 
  \item (problem-solving) Do at least one of 77 or 79.
\end{enumerate}

\subsection*{Textbook Section 2.3} 

\subsubsection*{Limit Laws}
\begin{enumerate}
  \item (fundamental technical skills) Do at least one of 83, 85, 87, 89, 91.
  \item (routine calculations) Do at least one of 103, 105.
  \item (routine calculations) Do 97 and at least one of 93, 95, 101.
    \begin{itemize}
      \item Do 99 (don't skip it) after we have covered section 2.4 in lectures.
    \end{itemize}
  \item (problem-solving) Do at least two of 107, 109, 111, 113. 
\end{enumerate}

\subsubsection{The Squeeze Theorem}
\begin{enumerate}
  \item (routine calculations) Do both 126 and 127.
    \begin{itemize}
      \item Do 128 only if time allows. 
    \end{itemize}
\end{enumerate}

\subsection*{Textbook Section 2.4}

\subsubsection*{Continuity}
\begin{enumerate}
  \item (basic concept checks) Do 154 and 157.
  \item (routine calculations) Do 139 and at least one of 141, 143.
  \item (routine calculations) Do at least one of 131, 133 and at least one of 135, 137.
    \begin{itemize}
      \item Do not plot graphs for 131 and 133. 
      \item If 135 or 137 feels challenging, look at their graphs to get an idea. 
    \end{itemize}
  \item (problem-solving) Do at least one of 145, 147, 155.
    \begin{itemize}
      \item Do 149 if time allows.
    \end{itemize}
  \item (subtleties of continuity) Do at least one of 161, 163, 165.
\end{enumerate}

\subsection*{Intermediate Value Theorem}
\begin{enumerate}
  \item (routine calculations) Do both 150 and 153.
\end{enumerate}

\clearpage
\subsection*{Textbook Section 4.6}

Limits at infinity and horizontal asymptotes.
\begin{enumerate}
  \item[*] Exercises 251, 253, 255, 257, 259 are about vertical asymptotes which were taught in Section 2.2 The Limit of a Function. These are good exercises. However, if you wish to practise evaluating limits at infinity or finding \emph{horizontal} asymptotes, you can safely skip these.
  \item (basic concept checks) Do all of 261, 263, 271.
  \item (routine calculations) Do 267. Do at least one of 265, 269. 
  \item (routine calculations) Do at least one of 273, 279, 281.
  \item (routine calculations) Do at least one of 275, 277, 283. Require a bit more algebra.
  \item (problem-solving) Evaluate \(\lim_{x \to \infty} \frac{3 + 2^{x}}{2 + 3^{x}}\) and \(\lim_{x \to -\infty} \frac{3 + 2^{x}}{2 + 3^{x}}\).
\end{enumerate}

\subsection*{Textbook Section 3.1}

The derivative at a number. 
\begin{enumerate}
  \item(basic concept checks) Do 39. Do at least one of 7, 9. Do 43.
  \item(routine calculations) Do at least one of 11, 13, 15, 17, 19.
  \item(routine calculations) Do at least one of 21, 23, 25, 27, 29.
  \item(routine calculations) Do at least one of 41, 44.
\end{enumerate}

\subsection*{Textbook Section 3.2}

The derivative as a function.
\begin{enumerate}
  \item (basic concept checks) Do 79. Do at least one of 65, 67. 
  \item (basic concept checks) Do 75, 81.
  \item (routine calculations) Do at least one of 61, 63. Do 77 if time allows.
  \item (routine calculations) Do 73.
  \item (problem-solving) Do at least one of 69, 71. Do 83 if you are comfortable with 81 and want more practise on algebra skills.
\end{enumerate}

\clearpage
\subsection*{Textbook Section 3.3}

Differentiation rules.
\begin{enumerate}
  \item (basic concept checks) Do at least one of 107, 109. Do at least one of 115, 117.
  \item (routine calculations) Do at least one of 111, 113. 
  \item (routine calculations) Do at least one of 127, 129.
  \item (routine calculations) Do at least one of 137, 139. Do 141.
  \item (problem-solving) Do at least one of 123, 125.
  \item (problem-solving) Do 131.
  \item (problem-solving) Do 143. 
\end{enumerate}

\subsection*{Textbook Section 3.5}

Derivative of trigonometric functions.
\begin{enumerate}
  \item (routine calculations) Do at least two of 173, 177, 179, 181, 183.
  \item (routine calculations) Do at least one of 191, 193, 19
  \item (problem-solving) Do at least one of 197, 199.
  \item (problem-solving) Do 201.
  \item (problem-solving) Do 208.
\end{enumerate}

\clearpage
\subsection*{Textbook Section 3.6}

Basic applications of the chain rule.
\begin{enumerate}
  \item (basic concept checks) Do at least one of 215, 217, 219.
  \item (basic concept checks) Do 221, 229. Do not expand the polynomials nor use the quotient rule. The point is to practise using the chain rule.
  \item (basic concept checks) Do 231. 
  \item (routine calculations) Do 225.  Notice that finding the derivative of \(\csc(\theta)\) is no longer a challenging exercise at this stage of the course. It becomes a routine calculation.
  \item (routine calculations) Do at least one of 223, 227. 
    \begin{itemize}
      \item Clarification for Exercise 227: \(\sin^{-3}(x)\) is the sine function raised to the \(-3\) power, i.e., \((\sin(x))^{-3}\). It is NOT the inverse sine function raised to the third power.
    \end{itemize}
  \item (routine calculations) Do 245.
  \item (problem-solving) Do at least one of 247, 249, 251. 
  \item (problem-solving) Do at least one of 239, 243.
  \item (problem-solving) Do 255.
\end{enumerate}

\subsection*{Textbook Section 3.7}

Using the chain rule to find derivative of inverse functions.
\begin{enumerate}
  \item (basic concept checks) Do 261. 
  \item (routine calculations) Do 265 and 267.
    \begin{itemize}
      \item (routine calculations) Do 279 after completing 267.
    \end{itemize}
  \item (problem-solving) Do at least one of 269, 271. 
  \item (routine calculations) Do at least one of 275, 277.
  \item (problem-solving) Do at least one of 281, 283.
  \item (problem-solving) Do 296.
\end{enumerate}

\subsection*{Textbook Section 3.8}

Using the chain rule to perform implicit differentiation.
\begin{enumerate}
  \item (basic concept checks) Do 301. Do at least one of 303, 309.
  \item (routine calculations) Do at least one of 305, 307. 
  \item (routine calculations) Do at least one of 311, 313, 316.
  \item (routine calculations) Do 316 (b).
  \item (routine calculations) Do 322.
  \item (problem-solving) Do both 318 and 319.
  \item (problem-solving) Do at least one of 325, 327.
\end{enumerate}
\clearpage

\subsection*{Textbook Section 3.9}

Use the derivatives of \(e^{x}\) and \(\ln(x)\) to find derivatives of general exponential and logarithmic functions.  You should also know how to deduce the derivative of \(\ln(x)\) using implicit differentiation or using the Inverse Function Theorem.
\begin{enumerate}
  \item (basic concept checks) Do 331.
  \item (routine calculations) Do at least one of 333, 335, 337.
  \item (routine calculations) Do at least one of 341, 343, 345.
  \item (problem-solving) Do at least one of 356 and 357.
\end{enumerate}

Logarithmic differentiation
\begin{enumerate}
  \item (routine calculations) Do 353.
  \item (routine calculations) Do at least one of 347, 349 and 351. All three can be solved \emph{with and without} logarithmic differentiation. You are encouraged to use both methods. 
\end{enumerate}

\subsection*{Textbook Section 4.1}

Given an equation involving \(x\) and \(y\), use implicit differentiation to find their derivative\underline{s} with respect to a third variable \(t\).
\begin{enumerate}
  \item (routine calculations) Do at least one of 1, 3.
\end{enumerate}

Given a scenario involving two variables, use geometry to formulate a related rates problem, and solve it.  The geometry varies among problems, but the idea stays the same. It takes some practice to get used to describing different types of geometry. 
\begin{enumerate}
  \item (basic concept checks) Do 5. 
  \item (problem-solving) Do at least one of 7, 9.
  \item (problem-solving) Do both 10 and 11.
  \item (problem-solving) Do all of 16, 17, 18, 31, 33, 34.
  \item (problem-solving) Do at least one of 42, 43.
\end{enumerate}
\clearpage

\subsection*{Textbook Section 4.3}

Use definitions, extreme value theorem, or Fermat's Theorem (Theorem 4.2 in the textbook) to understand extreme values of a function (which can be described abstractly, as concrete expressions or graphically).
\begin{enumerate}
  \item (basic concept checks) Do all of 90, 91, 92, 94, 95.
  \item (basic concept checks) Do at least one of 129 and 131.
  \item (routine calculations) Do all of 105, 107.
  \item (routine calculations) Do at least two of 109, 111, 113, 115, 117.
  \item (routine calculations) Do 125.
  \item (routine calculations) Do at least two of 119, 121 123, 127.
  \item (routine calculations) Do 133.
  \item (problem-solving) Do at least one of 144, 145.
\end{enumerate}

\subsection*{Textbook Section 4.5}

Use derivative tests to find local extrema and determine the shape of graphcs.
\begin{enumerate}
  \item (basic concept checks) Do all of 194 to 200.
  \item (basic concept checks) Do at least two of 201, 203, 205.
  \item (basic concept checks) Do at least one of 207, 209.
  \item (basic concept checks) Do at least one of 211, 213, 215. 
  \item (routine calculations) Do all of 221, 223.
  \item (routine calculations) Do at least one of 225, 227, 229.
  \item (routine calculations) Do at least one of 237, 238, 239, 240.
  \item (problem-solving) Do all of 246, 247, 248, 249, 250.
\end{enumerate}

\clearpage
\subsection*{Textbook Section 4.7}

Use calculus (any techniques) to solve applied optimization problems.  There are no routine calculations in this type of problems.
\begin{enumerate}
  \item (basic concept checks) Do all of 313, 314.
  \item (problem-solving) Do at least five of 315, 317, 318, 320, 322, 327, 328, 335, 336, 341, 344.
\end{enumerate}

\subsection*{Textbook Section 4.8}

Recognize indeterminate forms. Use l'H\^opital's to evaluate indeterminate forms.
\begin{enumerate}
  \item (basic concept checks) Do all of 356, 358, 361.
  \item (basic concept checks) Do all of 363, 365.
  \item (routine calculations) Do as many of 367, 369, 371, 373, 375, 377, 379, 381, 383, 385, 387, 389, 391, 393 as you can until you have encountered all types of indeterminate forms, i.e., \(0/0, \infty/\infty, 0 \cdot \infty, \infty - \infty, \infty^{0}, 0^{0}\).
\end{enumerate}
\end{document}
