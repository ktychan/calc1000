%! Tex Program = lualatex
\documentclass[12pt]{beamer} 
\usetheme{metropolis} 
\everymath{\displaystyle} 
\setbeamersize{text margin left=.5cm} 
\setbeamersize{text margin right=.5cm} 
\beamertemplatenavigationsymbolsempty{}

\usepackage{fontawesome5}
\usepackage{graphicx}
\input{../colours.tex.preamble}
\input{../tikz.tex.preamble}

\begin{document} 
\begin{frame}[t]
  On page 127 of the lecture notes, we introduced three summation formulas. One of them is
  \[
    \sum_{i=1}^{n} i = \frac{n(n+1)}{2}.
  \]

  Here is a quick way to make sense of the formula. 
  The next few slides provide some details.
  \medskip

  \[
    \begin{array}{cccccccccc}
          & 1 &+& 2 &+& 3 &+& 4 &+& 5 \\
      + & 5 &+& 4 &+& 3 &+& 2 &+& 1 \\
      = & 6 &+& 6 &+& 6 &+& 6 &+& 6
    \end{array}
  \]

  Similar ideas are used for the other two formulas, however they are a lot more complicated. 
\end{frame}

\begin{frame}[t]
  To perform \(1 + 2 + 3 + 4 + 5\), we do it twice with a small twist.
  \[
    \begin{array}{cccccccccc}
        & 1 &+& 2 &+& 3 &+& 4 &+& 5 \\
      \pause + & 5 \pause &+& 4 \pause &+& 3 \pause &+& 2 \pause &+& 1 \\
      \pause = & 6 \pause &+& 6 \pause &+& 6 \pause &+& 6 \pause &+& 6
    \end{array}
  \]

  \pause
  We get \(\sum_{i=1}^{5} i = \frac{5(5+1)}{2}\) where \(5\) is the number of columns, and \(5+1\) is the bottom number of each column above.  Lastly, we have to divide by \(2\) because we added \(1 + 2 + 3 + 4 + 5\) twice.

  \pause
  Such a pattern of calculation works for any integer \(n \ge 1\), hence
  \[
    \sum_{i=1}^{n} i = \frac{n(n+1)}{2}.
  \]
\end{frame}
\end{document}
