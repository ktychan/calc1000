%! Tex Program = lualatex
\documentclass[aspectratio=169, 14pt]{beamer}  % 12, 14, 17, 20

\usetheme{metropolis} 
\everymath{\displaystyle} 
\setbeamersize{text margin left=.5cm} 
\setbeamersize{text margin right=.5cm} 
\beamertemplatenavigationsymbolsempty{}

\renewcommand{\arcsin}{\sin^{-1}}
\renewcommand{\arccos}{\cos^{-1}}
\renewcommand{\arctan}{\tan^{-1}}

\usepackage[normalem]{ulem}
\usepackage{qrcode}
\usepackage{fontawesome5}

\usepackage{enumitem}
\setlist[enumerate]{label={(\alph*)}, itemsep={1ex}}

\input{./tikz.tex.preamble}
\input{./typography.tex.preamble}

%
% uncomment the whole \setbeamertemplate command to add an iClicker watermark.
%
\usepackage{tikz, graphicx}


\newenvironment{poll}{
  \begingroup
  \setbeamertemplate{background}{
    \begin{tikzpicture}[overlay, remember picture]
      \node[teal!10, anchor=south] at (current page.south) {\resizebox{\textwidth}{!}{iClicker}};
    \end{tikzpicture}    
  }
  \begin{frame}
}{
  \end{frame}
  \endgroup
}


\begin{document} 
\section{First day of class}

\begin{frame}
  \frametitle{Welcome to Calc~1000A!}
  \bigskip

  \normalsize
  Check the course website (\href{https://owl.uwo.ca}{\texttt{owl.uwo.ca}}) frequently!

  \begin{center}
    \url{https://join.iclicker.com/ABCD} 

    \qrcode[height=1.5in]{https://join.iclicker.com/ABCD}

    (make sure you join iClicker)
  \end{center}
\end{frame}

\begin{poll}
  Is a hot dog a sandwich?

  \medskip
  \begin{enumerate} 
    \item Yes!
    \item Yes?
    \item No!
    \item No?
    \item I no longer know what a sandwich is.
  \end{enumerate} 
\end{poll}


\section{Fundamentals and Reviews}
\begin{poll}
  Which sentences do not make sense to you?

  \begin{enumerate}
    \item Evaluate \(x^{2} + x + 1\) at \(3\).
    \item Find \(x^{2} + x + 1\) at \(3\).
    \item Solve for the derivative of \(x^{2} + x + 1\).
  \end{enumerate}
\end{poll}

\section{Trigonometry}

\begin{poll}
  What does \(\arctan(x)\) represent?  

  The output of \(\arctan(x)\) is ...
  \begin{enumerate}
    \item an angle.
    \item a side length in a right triangle.
    \item a ratio of side lengths in a right triangle.
    \item 
    \item I am not sure.
  \end{enumerate}
\end{poll}

\begin{poll}
  What does \(x\) represent in \(\arctan(x)\)?  

  The symbol \(x\) represents ...

  \begin{enumerate}
    \item an angle.
    \item a side length in a right triangle.
    \item a ratio of side lengths in a right triangle.
    \item 
    \item I am not sure.
  \end{enumerate}
\end{poll}

\begin{poll}
  Let \(y = \sec(\arctan(x))\). What does \(y\) represent? 

  The symbol \(y\) represents ...
  \begin{enumerate}
    \item an angle.
    \item a side length in a right triangle.
    \item a ratio of side lengths in a right triangle.
    \item 
    \item I am not sure.
  \end{enumerate}
\end{poll}

\begin{poll}
  Which of the following is a correct statement?

  \begin{minipage}{0.4\textwidth}
    \begin{enumerate}
      \item \(\theta = \arcsin(x)\)
      \item \(\theta = \arccos(x)\)
      \item \(\theta = \arctan(x)\)
      \item \(\theta = \sec^{-1}(x)\)
      \item I am not sure.
    \end{enumerate}
  \end{minipage}
  \begin{minipage}{0.5\textwidth}
    \includegraphics[page=1]{./standalones/plot-triangles-inverse-trigs}
  \end{minipage}
\end{poll}

\section{Exponential and logarithmic functions}

\begin{frame}[t]
  \frametitle{Exponential and logarithmic rules}

  \large

  \[
    \begin{array}{rclcrcl}
      e^{a}e^{b} &=& e^{a + b} 
                 &&
      (e^{a})^{b} &=& e^{ab} \\[1ex]
      \ln(ab) &=& \ln(a) + \ln(b) 
              &&
      \ln(a {\color{red}/} b) &=& \ln(a) {\,\color{red}-\,} \ln(b) \\[3ex]\pause
      b &=& e^{\ln(b)}
        &&
      b^{x} &=& e^{\ln(b)x}
    \end{array}
  \]
\end{frame}

\begin{poll}
  How can we calculate \(y = \log_{2}(8)\)?
  \medskip
  \begin{enumerate} 
    \item Solve \(2^{y} = 8\).
    \item Solve \(2^{8} = y\).
    \item Calculate \(\frac{\ln(8)}{\ln(2)}\). 
    \item 
    \item I am not sure.
  \end{enumerate} 
\end{poll} 


\section{Limits}
\begin{poll}
  Suppose \(\lim_{x \to 3} f(x)\) exists. Can we find \(f(3)\)?

  \medskip
  \begin{enumerate} 
    \item \(f(3) = \lim_{x \to 3} f(x)\).
    \item \(f(3)\) is undefined.
    \item There is not enough information.
    \item 
    \item I am not sure.
  \end{enumerate} 
\end{poll} 


\begin{poll}
  True or false? We can \emph{always} apply the quotient law to evaluate \(\lim_{x \to a} \frac{f(x)}{g(x)}\).

  \medskip
  \begin{enumerate} 
    \item True.
    \item False
    \item 
    \item 
    \item I am not sure.
  \end{enumerate} 
\end{poll}


\begin{poll}
  If \(f(x)\) is discontinuous on \([1,3]\), then we know \(f(x)\) has a discontinuity at \underline{\hspace{1in}} number in \([1,3]\). 
  \medskip
  \begin{enumerate} 
    \item every
    \item exactly one
    \item at least one
    \item 
    \item I am not sure.
  \end{enumerate} 

\end{poll} 


\begin{poll}
  Which of the following tell us a function \(f(x)\) is NOT continuous at \(a\)?
  \medskip
  \begin{enumerate} 
    \item \(f\) is not defined at \(a\).
    \item \(f\) is defined at \(a\) but \(\lim_{x \to a} f(x) \ne f(a)\).
    \item \(\lim_{x \to a^{-}} f(x) = \pm \infty\) or \(\lim_{x \to a^{+}} f(x) = \pm \infty\).
    \item \(\lim_{x \to a^{-}} f(x) \ne \lim_{x \to a^{+}} f(x)\).
    \item I am not sure.
  \end{enumerate} 
\end{poll}


\begin{poll}
  True or false? Every piecewise function has \emph{at least one} discontinuity in its domain.

  \medskip
  \begin{enumerate} 
    \item True.
    \item False
    \item 
    \item 
    \item I am not sure.
  \end{enumerate} 
\end{poll} 



\section{Limits at infinity}
\begin{poll}
  How many horizontal asymptotes can a function have?

  \medskip
  \begin{enumerate} 
    \item One.
    \item Two.
    \item At most two.
    \item At most infinitely many.
    \item I am not sure.
  \end{enumerate} 
\end{poll}


\begin{poll}
  Can a function cross its horizontal asymptotes?

  \medskip
  \begin{enumerate} 
    \item Yes.
    \item No.
    \item 
    \item 
    \item I am not sure.
  \end{enumerate} 
\end{poll}


\begin{poll}
  What can we deduce from knowing \(\lim_{x \to \infty} f(x)\) exists? 

  \medskip
  \begin{enumerate} 
    \item \(f\) is not increasing on some interval \((a, \infty)\)
    \item \(f\) is not decreasing on some interval \((b, \infty)\)
    \item \(f\) is not oscillating on some interval \((c, \infty)\)
    \item 
    \item I am not sure.
  \end{enumerate} 
\end{poll}


\begin{poll}
  Analyze the dominant terms of \(\lim_{x \to \infty} \frac{x^{2}+3x}{\sqrt{x+3}}\). With only a quick calculations, the limit should be

  \medskip
  \begin{enumerate} 
    \item \(\infty\) or \(-\infty\).
    \item a finite number.
    \item 
    \item 
    \item I am not sure.
  \end{enumerate} 
\end{poll}



\section{Derivatives}
\begin{poll}
  % Let's pop a balloon by dropping it on a needle. 
  What is the balloon's velocity when it popped?

  \medskip
  \begin{enumerate} 
    \item Does not exist because there is no balloon.
    \item Non-zero because the balloon has to be moving. 
    \item Zero because time freezes at that very instant. 
    \item 
    \item I am not sure.
  \end{enumerate} 
\end{poll}

\begin{frame}
  We know that \(|x|\) is not differentiable at \(0\).  Now consider \(f(x) = x|x|\) at \(x = 00\).

  \begin{enumerate}
    \item \(f'(0) = 0\).
    \item \(f'(0)\) does not exist, because \(|x|\) is not differentiable at \(0\).
    \item \(f'(0)\) does not exist, because \(|x|\) is a piecewise function.
    \item \(f'(0)\) does not exist, because its associated left and right limits are not equal.
    \item I am not sure.
  \end{enumerate}
\end{frame}


\begin{poll}
  True or false? If a function is continuous at \(a\), then it must be differentiable at \(a\).

  \medskip
  \begin{enumerate} 
    \item True.
    \item False.
    \item 
    \item 
    \item I am not sure.
  \end{enumerate} 
\end{poll}


\begin{poll}
  Compare the domains of \(f\) and \(f'\).

  \medskip
  \begin{enumerate} 
    \item \(f'\) is defined wherever \(f\) is defined.
    \item The domain of \(f'\) can be larger than the domain of \(f\).
    \item The domain of \(f'\) can be smaller than the domain of \(f\).
    \item 
    \item I am not sure.
  \end{enumerate} 
\end{poll}


\begin{poll}
  If a function is \emph{not} differentiable on \((0,1)\), then 

  \medskip
  \begin{enumerate} 
    \item \(f\) is not continuous on \((0,1)\).
    \item \(f\) is a piecewise function.
    \item 
    \item 
    \item I am not sure.
  \end{enumerate} 
\end{poll}


\begin{poll}
  If a function is \emph{not} differentiable on \((0,1)\), then 

  \medskip
  \begin{enumerate} 
    \item \(f'(a)\) does not exists at any number in \((0,1)\).
    \item \(f'(a)\) does not exists for at least one number in \((0,1)\).
    \item 
    \item 
    \item I am not sure.
  \end{enumerate} 
\end{poll}


\begin{poll}
  If a function is \emph{not} differentiable on \((0,1)\), then \(f'\) is not differentiable at \underline{\hspace{1in}} number in \((0,1)\).

  \medskip
  \begin{enumerate} 
    \item every 
    \item exactly one
    \item at least one
    \item 
    \item I am not sure.
  \end{enumerate} 
\end{poll}


\begin{poll}
  (True or false?) If \(f(x)\) is not differentiable at \(a\), then \(f\) is not continuous at \(a\).

  \medskip
  \begin{enumerate} 
    \item True.
    \item False.
    \item 
    \item 
    \item I am not sure.
  \end{enumerate} 
\end{poll}


\section{Differentiation Rules}
\begin{poll}
  (True or false?) By the power rule, \(\frac{d}{dx} \frac{1}{\sqrt{x}} = \frac{1}{\frac{1}{2}x^{-1/2}}\).

  \medskip
  \begin{enumerate} 
    \item True.
    \item False.
    \item 
    \item 
    \item I am not sure.
  \end{enumerate} 
\end{poll}


\begin{poll}
  (True or false?) By the power rule, \(\frac{d}{dx} 2^{x} = x \cdot 2^{x-1}\).

  \medskip
  \begin{enumerate} 
    \item True.
    \item False.
    \item 
    \item 
    \item I am not sure.
  \end{enumerate} 
\end{poll}


\begin{poll}
  (True or false?) By the power rule, \(\frac{d}{dx} 2^{10} = 10 \cdot 2^{9}\).

  \medskip
  \begin{enumerate} 
    \item True.
    \item False.
    \item 
    \item 
    \item I am not sure.
  \end{enumerate} 
\end{poll}


\begin{poll}
  Which of the following words suggest a question is about derivatives?

  \medskip
  \begin{enumerate} 
    \item asymptote
    \item secant line
    \item tangent line
    \item slope
    \item I am not sure.
  \end{enumerate} 
\end{poll}


\section{The Chain Rule}
\begin{poll}
  Which is the chain rule formula?

  \medskip
  \begin{enumerate} 
    \item \(\tfrac{d}{dx} f(g(x)) = f'(g)g'\)
    \item \(\tfrac{d}{dx} f(g(x)) = f(g')g'\)
    \item \(\tfrac{d}{dx} f(g(x)) = f'(g')g\)
    \item 
    \item I am not sure.
  \end{enumerate} 
\end{poll}

\begin{poll}
  Suppose \(h(x) = f(g(x))\). What is \(h'(1)\) given
  \begin{align*}
    f(1) &= 1,& f'(1) &= -1, &f(2) &= 3, & f'(2) &= -3, \\
    g(1) &= 2,& g'(1) &= -2, &g(2) &= 4, & g'(4) &= -4?
  \end{align*}

  \medskip
  \begin{enumerate} 
    \item \(h'(1) = -2\)
    \item \(h'(1) = -1\)
    \item \(h'(1) = 6\)
    \item \(h'(1) = -6\)
    \item I am not sure.
  \end{enumerate} 
\end{poll}

\begin{poll}
  Suppose \(h(x) = f(g(x))\). What is \(h'(1)\) given
  \begin{align*}
    f(1) &= 1,& f'(1) &= -1, &f(2) &= 3, & f'(2) &= -3, \\
    {\color{teal}g(1)} &{\color{teal}= 2},& {\color{magenta}g'(1)} &{\color{magenta}= -2}, &g(2) &= 4, & g'(4) &= -4?
  \end{align*}

  \medskip
  \begin{enumerate} 
    \item 
    \item 
    \item \(h'(1) = f'({\color{teal}g(1)}){\color{magenta}g'(1)} \onslide<2->{= f'(2)(-2)} \onslide<3->{= (-3)(-2) = 6}\)
    \item 
    \item 
  \end{enumerate} 
\end{poll}

\begin{poll}
  Differentiate \(\sin(x)^{2}\).

  \medskip
  \begin{enumerate} 
    \item \(\cos(x)^{2}\)
    \item \(\sin(x)\cos(x)\)
    \item \(2\sin(x)\cos(x)\)
    \item \(2\sin(x)\)
    \item I am not sure.
  \end{enumerate} 
\end{poll}


\section{Implicit Differentiation}
\begin{poll}
  Given \(\sin(y) = x\), what is \(dy/dx\)? Choose all correct options.

  \medskip
  \begin{enumerate} 
    \item \(0 = 1\), so \(y'\) does not exist.
    \item \(dy/dx = \tfrac{1}{\cos(y)}\).
    \item \(dy/dx = \cos(y)\).
    \item \(dy/dx = \tfrac{d}{dx} \arcsin(x)\).
    \item I am not sure.
  \end{enumerate} 
\end{poll}


\begin{poll}
  Given \(e^{y} = x\), what is \(dy/dx\)? Choose all correct options.

  \medskip
  \begin{enumerate} 
    \item \(0 = 1\), so \(y'\) does not exist.
    \item \(dy/dx = \tfrac{1}{e^{y}}\).
    \item \(dy/dx = e^{y}\).
    \item \(dy/dx = \tfrac{d}{dx} \ln(x)\).
    \item I am not sure.
  \end{enumerate} 
\end{poll}


\begin{poll}
  Find \(dx/dt\) given \(x^{3} - xt = 1\).

  \medskip
  \begin{enumerate} 
    \item \(\frac{dx}{dt} = \frac{t}{3x^{2}}\).
    \item \(\frac{dx}{dt} = \frac{x}{3x^{2} - t}\).
    \item \(\frac{dx}{dt} = \frac{3x^{2} - t}{x}\).
    \item None of the above.
    \item I am not sure.
  \end{enumerate} 
\end{poll}

\begin{poll}
  Can we use implicit differentiation to find the slope of \(y^{2} + x^{2} = 25\) at the point \((5,5)\)?

  \medskip
  \begin{enumerate} 
    \item Yes, we plug \(x = 5\) and \(y = 5\) into \(dy/dx\).
    \item No, it's not a function and has no \(dy/dx\).
    \item No, we cannot plug \(x = 5\) and \(y = 5\) into \(dy/dx\).
    \item I don't understand the question.
    \item I am not sure.
  \end{enumerate} 
\end{poll}

\begin{poll}
  What is the slope of the tangent line to \(\tan(xy) = -x\) at the point \((1, 3\pi/4)\)?

  \medskip
  \begin{enumerate} 
    \item \(-1 - \frac{3\pi}{2}\). 
      \onslide<2>{{\color{magenta}Probably forgot about brackets.}}
    \item \(-\frac{1}{2} - \frac{3\pi}{4}\). 
      \onslide<2>{{\color{teal}The correct answer.}}
    \item \(\frac{1}{\sqrt{2}} - \frac{3\pi}{4}\). 
      \onslide<2>{{\color{magenta} Probably forgot the square or made the mistake \(\tfrac{d}{dx}\tan(x) = \sec(x)\)}}
    \item 
    \item I am not sure.
  \end{enumerate} 
\end{poll}


\begin{poll}
  How many horizontal tangent lines does the curve \(y^{2}/2 - xy = 1\) have? 

  \medskip
  \begin{enumerate} 
    \item Zero.
    \item Finitely many.
    \item Infinitely many.
    \item I don't know \emph{WHAT} to solve for in \(\tfrac{y}{x - y} = 0\).
    \item I don't know how to start.
  \end{enumerate} 
\end{poll}


\begin{poll}
  Assume \(y = f^{-1}(x)\). To which equations do we apply the chain rule to calculate \(y'\)? Choose all correct options.

  \medskip
  \begin{enumerate} 
    \item \(f(f^{-1}(x)) = x\)
    \item \(f^{-1}(f(x)) = x\)
    \item \(y = f(x)\)
    \item \(x = f(y)\)
    \item I don't recognize equations (a) and (b).
  \end{enumerate} 
\end{poll}


\begin{poll}
  Can the inverse function theorem be deduced from implicit differentiation?

  \medskip
  \begin{enumerate} 
    \item Yes.
    \item No.
    \item I think so, but not sure how.
    \item 
    \item I am not sure.
  \end{enumerate} 
\end{poll}


\section{Logarithmic differentiation}
\begin{poll}
  Differentiate \(x^{x}\).

  \medskip
  \begin{enumerate} 
    \item \(x x^{x-1}\).
    \item \(x^{x}(1 + \ln(x)\)
    \item \(x^{x}\)
    \item \(x^{x}\ln(x)\)
    \item I am not sure.
  \end{enumerate} 
\end{poll}

\section{Related rates}
\begin{poll}[t]
  What are we asked to find in Example~9?

  \medskip
  \begin{enumerate} 
    \item Find \(d\theta/dt\).
    \item Find \(dx/dt\).
    \item Find \(d\theta/dt\) when \(x = 4\).
    \item Find \(dx/dt\) when \(x = 4\).
    \item I am not sure.
  \end{enumerate} 
\end{poll}

\section{Extrema}

\begin{poll}[t]
  If there are constants \(a,b\) such that \(a \le f(x) \le b\) for every \(x\) on the domain of \(f\), does it mean that \(f(x)\) must have an absolute extrema?

  \medskip
  \begin{enumerate} 
    \item No.
    \item \only<1>{Yes.}\only<2>{\sout{Yes.}} \onslide<2>{This is wrong because of functions like \(x \sin(1/x)\).}
    \item 
    \item 
    \item I am not sure.
  \end{enumerate} 
\end{poll}


\begin{poll}[t]
  Is it true that every absolute extrema is also a local extrema?

  \medskip
  \begin{enumerate} 
    \item No.
    \item \only<1>{Yes.}\only<2>{\sout{Yes.}} \onslide<2>{This is wrong because absolute extrema can occur at endpoints of the domain of a function but local extrema cannot.}
    \item 
    \item 
    \item I am not sure.
  \end{enumerate} 
\end{poll}

\begin{poll}[t]
  What is the relation between local extrema and critical points?

  \medskip
  \begin{enumerate} \setlength\itemsep{1ex}
    \item Some, but not all, critical points are local extrema.
    \item A local extremum must correspond to a critical point.
    \item Every critical point correspond to a local extremum.
    \item \(\{ \text{critical points} \} = \{ \text{local extrema} \}\).
    \item I am not sure.
  \end{enumerate} 
\end{poll}

\section{Derivative Tests}

\begin{poll}[t]
  Consider the graph in Example~10. Using only the first derivative test, where does \(f(x)\) have a local maximum? 

  \medskip
  \begin{enumerate} \setlength\itemsep{1ex}
    \item \(x = a\). \onslide<2>{\hlmain{\(f'(a) = 0\) (\(x = a\) is a critical number of \(f\)), and \(f'\) changes from positive (\(f\) is decreasing) to negative (\(f\) is increasing) as \(x\) goes from left to right passing \(a\).}}
    \item \(x = b\). 
    \item \(x = c\).
    \item \(x = d\). 
    \item I am not sure.
  \end{enumerate} 
\end{poll}

\begin{poll}[t]
  Can the First Derivative Test be applied to find the local min of \(f(x) = |x|\) (and, more generally, any continuous piecewise function)? 

  \medskip
  \begin{enumerate} \setlength\itemsep{1ex}
    \item Yes, because \(f\) does not need to be differentiable at the critical number \(0\).
    \item No, because \(f\) is not differentiable at the critical number \(0\).
    \item 
    \item 
    \item I am not sure.
  \end{enumerate} 
\end{poll}

\begin{poll}[t]
  Is \(|x|\) concave up on its domain? 

  Can the Second Derivative Test be used to test its concavity?

  \medskip
  \begin{enumerate} \setlength\itemsep{1ex}
    \item Yes and yes.
    \item Yes and no.
    \item No and yes.
    \item No and no.
    \item I am not sure.
  \end{enumerate} 
\end{poll}

\section{Optimization}
\begin{poll}[c]
  In Example~1, is the domain of \(A(x)\) really \(0 \le x \le 100\)?
\end{poll}

\begin{poll}[c]
  In Example~2, why do we measure the walking distance \(w\) and the swimming distance \(s\)? Why not something other quantities?
\end{poll}

\begin{poll}[t]
  What is wrong with the (proposed) solution of Example~2?

  \medskip
  \begin{enumerate} \setlength\itemsep{1ex}
    \item There is an algebra mistake.
    \item The domain of \(T(s)\) is wrong.
    \item It's not possible to minimize travel time.
    \item 
    \item I am not sure.
  \end{enumerate} 
\end{poll}

\begin{poll}[t]
  What is the domain of \(T(s)\) from Example~2?

  \medskip
  \begin{enumerate} \setlength\itemsep{1ex}
    \item \(s \ge 1\). \onslide<2>{\hlsupp{We don't need to swim forever.}}
    \item \(1 \le s \le \sqrt{5}\). \onslide<2>{\hlmain{Note \(\sqrt{5}\) is the length of the full diagonal.}}
    \item 
    \item 
    \item I am not sure.
  \end{enumerate} 
\end{poll}

\section{L'H\^opital's Rule}
\begin{poll}[t]
  Which of the following is an indeterminate form? Choose all correct options.

  \medskip
  \begin{enumerate} \setlength\itemsep{1ex}
    \item \(L_{1} = \lim_{x \to 0} x \ln(x)\).
    \item \(L_{2} = \lim_{x \to 1} (1 - x)^{x}\).
    \item \(L_{3} = \lim_{x \to -\infty} 2^{2x}\).
    \item \(L_{4} = \lim_{x \to 0} \frac{2x^{1}+1}{\cos(x)}\).
    \item I am not sure.
  \end{enumerate} 
\end{poll}

\begin{poll}
  What's wrong with the following calculation?
  \[
    \lim_{x \to 1} \frac{\ln(x)}{2} = \lim_{x \to 1} \frac{\frac{d}{dx} \ln(x)}{\frac{d}{dx} 2} = \lim_{x \to 1} \frac{1/x}{0} = \text{does not exists?!?!}
  \]

  \pause{} 

  L'H\^opital's rule \emph{does not apply} because the limit is not an indeterminate form.

  ONLY apply l'H\^opital's rule to indeterminate forms of \(0/0\) or \(\infty/\infty\) and nothing else.
\end{poll}

\section{Sigma Notations}

\begin{poll}[t]
  Do \(\sum_{i=1}^{3} f(a + i/2)\) and \(\sum_{i=0}^{2} f(a + (i+1)/2)\) represent different objects?

  \medskip
  \begin{enumerate} \setlength\itemsep{1ex}
    \item The two summations are equal.
    \item The two summations are not equal.
    \item 
    \item 
    \item I am not sure.
  \end{enumerate} 
\end{poll}

\begin{poll}[t]
  Assume \(m < n\), both positive integers. Is it true that 
  \[
    \sum_{i=1}^{m} a_{i} + \sum_{i=1}^{n} b_{i} = \sum_{i=1}^{n} (a_{i}+ b_{i})?
  \]

  \medskip
  \begin{enumerate} \setlength\itemsep{1ex}
    \item The equality is true.
    \item The equality is not always true.
    \item 
    \item 
    \item I am not sure.
  \end{enumerate} 
\end{poll}

\section{Antiderivatives}

\begin{poll}[t]
  Consider \(f(x) = 2x e^{x^{2}}\) and \(g(x) = e^{x^{2}}\). Which of the following statement is true.

  \medskip
  \begin{enumerate} \setlength\itemsep{1ex}
    \item \(f(x)\) is an antiderivative of \(g(x)\).
    \item \(f(x)\) is the antiderivative of \(g(x)\).
    \item \(g(x)\) is an antiderivative of \(f(x)\).
    \item \(g(x)\) is the antiderivative of \(f(x)\).
    \item I am not sure.
  \end{enumerate} 
\end{poll}

\begin{poll}
  Why is \(\ln(x)\) not an antiderivative of \(1/x\)?
\end{poll}

\begin{poll}[t]
  Which of the following are true statements?

  \medskip
  \begin{enumerate} \setlength\itemsep{1ex}
    \item \(\int e^{x} \;dx = e^{x}\).
    \item \(\int e^{x} \;dx = e^{x} + C\).
    \item \(\int e^{x} \;dx = e^{x} + 1 + C\).
    \item \(\int e^{x} \;dx = e^{x} + 2C\).
    \item I am not sure.
  \end{enumerate} 
\end{poll}

\begin{poll}[t]
  Evaluate \(\int \sec^{2}(\pi/4) \;dx\).

  \begin{enumerate} \setlength\itemsep{1ex}
    \item \(\tan(\pi/4) + C\).
    \item \(\sec^{2}(\pi/4)x + C\).
    \item We don't know enough integration techniques to evaluate this yet. 
    \item 
    \item I am not sure.
  \end{enumerate} 
\end{poll}

\section{Integration formulas}
\begin{poll}[t]
  Recall that differentiating a {\color{red} non-constant polynomial} lowers its degree by one. Does integrating any {\color{red} non-zero polynomial} (including constant ones) raise its degree by one?

  \medskip
  \begin{enumerate} \setlength\itemsep{1ex}
    \item Yes, because \(\frac{d}{dx} x^{n} = n x^{n-1}\) for any integer \(n \ge 0\).
    \item No, because \(\frac{d}{dx} (\text{any constant}) = 0\).
    \item 
    \item 
    \item I am not sure.
  \end{enumerate} 
\end{poll}

\begin{poll}[t]
  Does integrating a {\color{red} power function}, i.e., \(x^{n}\) with NO restriction on \(n\), always raise its degree by one? Are there any exceptions?

  \medskip
  \begin{enumerate} \setlength\itemsep{1ex}
    \item Yes. There are no exceptions.
    \item No, an exception is \(x^{-1}\).
    \item 
    \item 
    \item I am not sure.
  \end{enumerate} 
\end{poll}

\section{Definite integrals}

\begin{poll}
  Evaluate \(\int_{a}^{a} f(x) \;dx\). 

  \begin{enumerate} \setlength\itemsep{1ex}
    \item It is not possible since we don't know what \(f(x)\) is.
    \item It is not possible since we don't know what \(a\) is.
    \item \(0\).
    \item \(> 0\)
    \item I am not sure.
  \end{enumerate}
\end{poll}

\begin{poll}[t]
  The two endpoints of an interval are \(a,b\), and we don't know if \(a < b\) or \(a > b\). However, we know that \(\int_{b}^{a} f(x) \;dx < 0\) and \(f\) is nonnegative. It follows that 

  \begin{enumerate} \setlength\itemsep{1ex}
    \item \(a < b\).
    \item \(a > b\).
    \item \(a = b\).
    \item 
    \item I am not sure.
  \end{enumerate} 
\end{poll}

\begin{poll}[t]
  Let \(f(x) = (x+2)(x-1)\) and \(A = \int_{-3}^{1} \big|f(x)\big| \;dx\).  Which of the following is a true statement?

  \begin{enumerate} \setlength\itemsep{1ex}
    \item \(A = \phantom{-}\int_{-3}^{-2} f(x) \;dx + \int_{-2}^{1} f(x) \;dx\).
    \item \(A = \phantom{-}\int_{-3}^{-2} f(x) \;dx - \int_{-2}^{1} f(x) \;dx\).
    \item \(A = -\int_{-3}^{-2} f(x) \;dx + \int_{-2}^{1} f(x) \;dx\).
    \item \(A = -\int_{-3}^{-2} f(x) \;dx - \int_{-2}^{1} f(x) \;dx\).
    \item I am not sure.
  \end{enumerate} 
\end{poll}

\begin{poll}[t]
  Let \(A = \int_{a}^{b} f(x) \;dx\). Which of the following is a true statement?

  \begin{enumerate} \setlength\itemsep{1ex}
    \item \(A = \int_{0}^{a} f(x) \;dx + \int_{0}^{b} f(x) \;dx\).
    \item \(A = -\int_{0}^{a} f(x) \;dx + \int_{0}^{b} f(x) \;dx\).
    \item 
    \item 
    \item I am not sure.
  \end{enumerate} 
\end{poll}

\begin{poll}[t]
  If \(f(x) \le g(x)\) over \([a,b]\), then ...

  \begin{enumerate} \setlength\itemsep{1ex}
    \item \(\int_{a}^{b} f(x) \;dx \le \int_{a}^{b} g(x) \;dx\).
    \item \(\int_{a}^{b} f(x) \;dx = \int_{a}^{b} g(x) \;dx\).
    \item \(\int_{a}^{b} f(x) \;dx \ge \int_{a}^{b} g(x) \;dx\).
    \item 
    \item I am not sure.
  \end{enumerate} 
\end{poll}

\section{Fundamental Theorem of Calculus}
\begin{poll}[t]
  Which of the following are good starting points for our solutions?

  \begin{enumerate} \setlength\itemsep{1ex}
    \item Write \(F(x) = \int_{0}^{x^{3}} e^{t^{2}} \;dt + \int_{0}^{x^{2}+1} e^{t^{2}} \;dt \).
    \item Write \(F(x) = \int_{x^{3}}^{0} e^{t^{2}} \;dt + \int_{0}^{x^{2}+1} e^{t^{2}} \;dt \).
    \item Write \(F(x) = \int_{0}^{x^{3}} e^{t^{2}} \;dt + \int_{0}^{1} e^{t^{2}} \;dt + \int_{1}^{x^{2}} e^{t^{2}} \;dt\).
    \item Write \(F(x) = \int_{x^{3}}^{1} e^{t^{2}} \;dt + \int_{1}^{x^{2}+1} e^{t^{2}} \;dt \).
    \item I am not sure.
  \end{enumerate}
\end{poll}

\begin{poll}
  Evaluate \(\int_{-\pi}^{\pi/2} 2 \cos(x) \;dx\).
\end{poll}

\section{The Substitution Rule}
\begin{poll}[t]
  What if we didn't see the ``convenient'' choice for \(u\)? 

  Transform \(\int x^{2} \sqrt{ 1 + x^{3} } \;dx\) by subbing \(u = \sqrt{ 1 + x^{3} }\).

  \begin{enumerate} \setlength\itemsep{1ex}
    \item This substitution does not work.
    \item We get \(\int \frac{2}{3} u^{2} \;du\). \only<2->{\hfill{}\hlmain{Correct.}}
    \item We get \(\int \frac{2}{3} \frac{u}{\sqrt{ 1 + x^{3} }} \;du\). \only<2->{\hfill\hlwarn{ Algebra mistake.}}
    \item We get \(\int \frac{2}{3} u \sqrt{1 + x^{3}} \;du\). \only<2->{\hfill\hlsupp{Incomplete.}}
    \item None of the above, or I am not sure.
  \end{enumerate}
\end{poll}

\section{Applications of integration}

\begin{poll}[t]
  What are expressions (plural!) for the area bounded between \(\sin(x)\) and \(\cos(x)\) over \([0, \pi]\)?

  \begin{enumerate} \setlength\itemsep{1ex}
    \item \(\int_{0}^{\pi} \sin(x) - \cos(x) \;dx\).
    \item \(\int_{0}^{\pi} |\sin(x) - \cos(x)| \;dx\).
    \item \(\int_{0}^{\pi/4} \cos(x) - \sin(x) \;dx + \int_{\pi/4}^{\pi} \sin(x) - \cos(x) \;dx\).
    \item \(\int_{0}^{\pi/4} \cos(x) - \sin(x) \;dx - \int_{\pi/4}^{\pi} \cos(x) - \sin(x) \;dx\).
    \item I am not sure. 
  \end{enumerate}
\end{poll}

\begin{poll}[t]
  Suppose \(R\) is the region enclosed between \(y = \sin(x)\) on \([pi/6, pi/3]\) and the \(x\)-axis.  Suppose a solid of revolution \(S\) is obtained by rotating \(R\) about the \(x\)-axis. 

  Sketch the region \(R\) and the solid \(S\).

  What is the inner radius of the cross-section at \(x\)? 
\end{poll}

\section{Problem-solving}
\begin{poll}[t]
  Suppose \(\sin(f(x)) = 0\), \(f(1) = \pi\). Find \(f'(1)\).

  What question did you ask yourself to discover a problem-solving idea?
\end{poll}

\begin{poll}[t]
  How can we determine the continuity of a piecewise function \(f(x)\) at a constant \(c\)?

  \medskip
  \begin{enumerate} 
    \item Apply limit laws.
    \item Calculate and compare left and right limits at \(c\).
    \item Check if \(\lim_{x \to c}f(x) = f(c)\). May have to calculate left and right limits separately.
    \item Check if \(f(x)\) is defined at \(c\).
    \item I am not sure.
  \end{enumerate} 
\end{poll}

\begin{poll}[t]
  Suppose \(f(x) = \begin{cases} ax^{2} &\text{if } 0 \le x < 1 \\ -1 &\text{if } x = 1 \end{cases}\) is continuous on \([0,1]\) and you are asked to find \(a\). What information seems useful?

  \medskip
  \begin{enumerate} 
    \item Limit laws.
    \item Squeeze Theorem.
    \item \(f(x)\) is continuous on \([0,1]\).
    \item Intermediate Value Theorem.
    \item I am not sure.
  \end{enumerate} 
\end{poll}

\begin{poll}[t]
  Suppose \(f(x) = \begin{cases} ax^{2} &\text{if } 0 \le x < 1 \\ -1 &\text{if } x = 1 \end{cases}\) is continuous on \([0,1]\) and you are asked to find \(a\). Continuity seems important. What else is useful?

  \medskip
  \begin{enumerate} 
    \item \(f\) must be continuous at \(0\).
    \item \(f\) must be continuous at \(1\).
    \item 
    \item 
    \item I am not sure.
  \end{enumerate} 
\end{poll}

\begin{poll}[t]
  Suppose \(f(x) = \begin{cases} ax^{2} &\text{if } 0 \le x < 1 \\ -1 &\text{if } x = 1 \end{cases}\) is continuous on \([0,1]\) and you are asked to find \(a\). What can we do with the continuity at \(1\)?

  \medskip
  \begin{enumerate} 
    \item Deduce \(\lim_{x \to 1^{-}} f(x) = \lim_{x \to 1^{+}} f(x)\).
    \item Deduce \(\lim_{x \to 1^{+}} f(x) = f(1)\).
    \item Deduce \(\lim_{x \to 1^{-}} f(x) = f(1)\).
    \item 
    \item I am not sure.
  \end{enumerate} 
\end{poll}

\begin{poll}[t]
  Suppose \(f(x) = \begin{cases} ax^{2} &\text{if } 0 \le x < 1 \\ -1 &\text{if } x = 1 \end{cases}\) is continuous on \([0,1]\) and you are asked to find \(a\). What can we do with the continuity at \(1\)?

  \medskip
  \begin{enumerate} 
    \item Deduce \(\lim_{x \to 1^{-}} f(x) = \lim_{x \to 1^{+}} f(x)\).
    \item Deduce \(\lim_{x \to 1^{+}} f(x) = f(1)\).
    \item Deduce \(\lim_{x \to 1^{-}} f(x) = f(1)\).
    \item 
    \item I am not sure.
  \end{enumerate} 
\end{poll}

\begin{poll}[t]
  Suppose \(x f(x)^{2} = 3\) and \(f(3) = -1\). Find \(f'(3)\).

  \medskip
  \begin{enumerate} 
    \item \(1/6\)
    \item \(-1/6\)
    \item
    \item
    \item I am not sure.
  \end{enumerate}
\end{poll}

\section{Study Skills}
\begin{poll}
  Describe what motivates you to study calculus using a single word.
\end{poll}

\begin{poll}
  How do you assess your learning progress?

  \medskip
  \begin{enumerate} 
    \item Do suggested problems.
    \item Do do-at-home problems.
    \item Reflect on problem-solving ideas.
    \item Summarize to reduce my mental load.
    \item I am not sure.
  \end{enumerate} 
\end{poll}


\section{Review}
\begin{poll}
  What topic should we review?

  \medskip
  \begin{enumerate} 
    \item Shapes of graphs (min/max and derivative tests).
    \item Related rates or optimization.
    \item Fundamental of integration.
    \item Integration techniques.
    \item Go through past exams.
  \end{enumerate} 
\end{poll}

\end{document}
